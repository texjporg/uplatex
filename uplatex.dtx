% \iffalse meta-comment
%% File: uplatex.dtx
%
%    pLaTeX base file:
%       Copyright 1995,1996 ASCII Corporation.
%    and modified for upLaTeX
%
%  Copyright (c) 2010 ASCII MEDIA WORKS
%  Copyright (c) 2016 Takuji Tanaka
%  Copyright (c) 2016-2021 Japanese TeX Development Community
%
%  This file is part of the upLaTeX2e system (community edition).
%  --------------------------------------------------------------
%
% \fi
%
% \iffalse
%<*driver|pldoc>
\ifx\JAPANESEtrue\undefined
  \expandafter\newif\csname ifJAPANESE\endcsname
  \JAPANESEtrue
\fi
%</driver|pldoc>
% \fi
%
% \setcounter{StandardModuleDepth}{1}
% \makeatletter
%\ifJAPANESE
% \def\chuui{\@ifnextchar[{\@chuui}{\@chuui[注意:]}}
%\else
% \def\chuui{\@ifnextchar[{\@chuui}{\@chuui[Attention: ]}}
%\fi
% \def\@chuui[#1]{\par\vskip.5\baselineskip
%   \noindent{\em #1}\par\bgroup\gtfamily\sffamily}
% \def\endchuui{\egroup\vskip.5\baselineskip}
% \makeatother
%
% \iffalse
%<*driver|pldoc>
\def\eTeX{$\varepsilon$-\TeX}
\def\pTeX{p\kern-.15em\TeX}
\def\epTeX{$\varepsilon$-\pTeX}
\def\pLaTeX{p\kern-.05em\LaTeX}
\def\pLaTeXe{p\kern-.05em\LaTeXe}
\def\upTeX{u\pTeX}
\def\eupTeX{$\varepsilon$-\upTeX}
\def\upLaTeX{u\pLaTeX}
\def\upLaTeXe{u\pLaTeXe}
%</driver|pldoc>
% \fi
%
% \StopEventually{}
%
% \iffalse
%\ifJAPANESE
% \changes{v1.0c-u00}{2011/05/07}{\pLaTeX{}用から\upLaTeX{}用に修正。
%     (based on platex.dtx 1997/01/29 v1.0c)}
% \changes{v1.0e-u00}{2016/04/06}{\pLaTeX{}の変更に追随。
%     (based on platex.dtx 2016/02/16 v1.0e)}
% \changes{v1.0h-u00}{2016/05/08}{ドキュメントから\file{uplpatch.ltx}を除外
%     (based on platex.dtx 2016/05/08 v1.0h)}
% \changes{v1.0k-u00}{2016/05/21}{\pLaTeX{}の変更に追随。
%     (based on platex.dtx 2016/05/21 v1.0k)}
% \changes{v1.0k-u01}{2016/06/06}{\upLaTeX{}用にドキュメントを全体的に改訂}
% \changes{v1.0l-u01}{2016/06/19}{パッチレベルを\file{uplvers.dtx}から取得
%     (based on platex.dtx 2016/06/19 v1.0l)}
% \changes{v1.0m-u01}{2016/08/26}{\file{uplatex.cfg}の読み込みを
%    \file{uplcore.ltx}から\file{uplatex.ltx}へ移動
%     (based on platex.dtx 2016/08/26 v1.0m)}
% \changes{v1.0n-u01}{2016/09/14}{\pLaTeX{}の変更に追随。
%     (based on platex.dtx 2016/09/14 v1.0n)}
% \changes{v1.0p-u01}{2017/11/11}{\pLaTeX{}の変更に追随。
%     (based on platex.dtx 2017/11/11 v1.0p)}
% \changes{v1.0q-u01}{2017/11/29}{英語版ドキュメントを追加
%     (based on platex.dtx 2017/11/29 v1.0q)}
% \changes{v1.0r-u01}{2017/12/02}{\upLaTeX{}と\upTeX{}の参考文献も追加}
% \changes{v1.0s-u01}{2017/12/05}{デフォルト設定ファイルの読み込みを
%    \file{uplcore.ltx}から\file{uplatex.ltx}へ移動
%     (based on platex.dtx 2017/12/05 v1.0s)}
% \changes{v1.0s-u02}{2017/12/10}{\file{uplcore.ltx}の前に
%    \file{plcore.ltx}を読み込むようにした(最近の\pLaTeX{}が前提)}
% \changes{v1.0u-u02}{2018/02/18}{\pLaTeX{}の変更に追随。
%     (based on platex.dtx 2018/02/18 v1.0u)}
% \changes{v1.0v-u02}{2018/04/06}{最新のsource2eへの追随
%     (based on platex.dtx 2018/04/06 v1.0v)}
% \changes{v1.0w-u02}{2018/04/08}{安全のためフォーマット作成時の
%    バナー表示をやめた
%     (based on platex.dtx 2018/04/08 v1.0w)}
% \changes{v1.0x-u02}{2018/09/03}{ドキュメントを更新
%     (based on platex.dtx 2018/09/03 v1.0x)}
% \changes{v1.0y-u02}{2018/09/22}{最終更新日を\file{upldoc.pdf}に表示
%     (based on platex.dtx 2018/09/22 v1.0y)}
% \changes{v1.0y-u03}{2019/05/22}{ドキュメントを更新}
% \changes{v1.1b-u03}{2020/09/28}{defs読込後にフック追加}
% \changes{v1.1c-u03}{2021/02/25}{\file{latex.ltx}の読込チェック}
%\else
% \changes{v1.0c-u00}{2011/05/07}{Created \upLaTeX\ version based on \pLaTeX\ one
%     (based on platex.dtx 1997/01/29 v1.0c)}
% \changes{v1.0e-u00}{2016/04/06}{Sync with \pLaTeX.
%     (based on platex.dtx 2016/02/16 v1.0e)}
% \changes{v1.0h-u00}{2016/05/08}{Exclude \file{uplpatch.ltx} from the document
%     (based on platex.dtx 2016/05/08 v1.0h)}
% \changes{v1.0k-u00}{2016/05/21}{Sync with \pLaTeX.
%     (based on platex.dtx 2016/05/21 v1.0k)}
% \changes{v1.0k-u01}{2016/06/06}{Update documents for \upLaTeX.}
% \changes{v1.0l-u01}{2016/06/19}{Get the patch level from \file{uplvers.dtx}
%     (based on platex.dtx 2016/06/19 v1.0l)}
% \changes{v1.0m-u01}{2016/08/26}{Moved loading \file{uplatex.cfg}
%    from \file{uplcore.ltx} to \file{uplatex.ltx}
%     (based on platex.dtx 2016/08/26 v1.0m)}
% \changes{v1.0n-u01}{2016/09/14}{Sync with \pLaTeX.
%     (based on platex.dtx 2016/09/14 v1.0n)}
% \changes{v1.0p-u01}{2017/11/11}{Sync with \pLaTeX.
%     (based on platex.dtx 2017/11/11 v1.0p)}
% \changes{v1.0q-u01}{2017/11/29}{New English documentation added!
%     (based on platex.dtx 2017/11/29 v1.0q)}
% \changes{v1.0r-u01}{2017/12/02}{\upLaTeX\ and \upTeX\ references added}
% \changes{v1.0s-u01}{2017/12/05}{Moved loading default settings
%    from \file{uplcore.ltx} to \file{uplatex.ltx}
%     (based on platex.dtx 2017/12/05 v1.0s)}
% \changes{v1.0s-u02}{2017/12/10}{Load \file{plcore.ltx} before
%    \file{uplcore.ltx} (recent version of \pLaTeX\ is assumed)}
% \changes{v1.0u-u02}{2018/02/18}{Sync with \pLaTeX.
%     (based on platex.dtx 2018/02/18 v1.0u)}
% \changes{v1.0v-u02}{2018/04/06}{Sync with the latest \file{source2e.tex}
%     (based on platex.dtx 2018/04/06 v1.0v)}
% \changes{v1.0w-u02}{2018/04/08}{Stop showing banner during
%    format generation for safety
%     (based on platex.dtx 2018/04/08 v1.0w)}
% \changes{v1.0x-u02}{2018/09/03}{Update document.
%     (based on platex.dtx 2018/09/03 v1.0x)}
% \changes{v1.0y-u02}{2018/09/22}{Show last update info on \file{upldoc.pdf}
%     (based on platex.dtx 2018/09/22 v1.0y)}
% \changes{v1.0y-u03}{2019/05/22}{Update document.}
% \changes{v1.1b-u03}{2020/09/28}{Add hook after loading defs}
% \changes{v1.1c-u03}{2021/02/25}{Check for \file{latex.ltx} status}
%\fi
% \fi
%
% \iffalse
%<*driver>
\NeedsTeXFormat{pLaTeX2e}
% \fi
\ProvidesFile{uplatex.dtx}[2021/02/25 v1.1c-u03 upLaTeX document file]
% \iffalse
\documentclass{jltxdoc}
\usepackage{plext}
\GetFileInfo{uplatex.dtx}
\ifJAPANESE
\title{\upLaTeXe{}について}
\author{中野 賢 \& 日本語\TeX{}開発コミュニティ \& TTK}
\date{作成日:\filedate}
\renewcommand{\refname}{参考文献}
\GlossaryPrologue{\section*{変更履歴}%
                  \markboth{変更履歴}{変更履歴}%
                  \addcontentsline{toc}{section}{変更履歴}}
\else
\title{About \upLaTeXe{}}
\author{Ken Nakano \& Japanese \TeX\ Development Community \& TTK}
\date{Date: \filedate}
\renewcommand{\refname}{References}
\GlossaryPrologue{\section*{Change History}%
                  \markboth{Change History}{Change History}%
                  \addcontentsline{toc}{section}{Change History}}
\fi
\makeatletter
\ifJAPANESE
\def\levelchar{>・}
\fi
\def\changes@#1#2#3{%
  \let\protect\@unexpandable@protect
  \edef\@tempa{\noexpand\glossary{#2\space#1\levelchar
    \ifx\saved@macroname\@empty
%     \space\actualchar\generalname: %% comment out (uplatex.dtx only)
    \else
      \expandafter\@gobble
      \saved@macroname\actualchar
      \string\verb\quotechar*%
      \verbatimchar\saved@macroname
      \verbatimchar:
    \fi
    #3}}%
  \@tempa\endgroup\@esphack}
\makeatother
\RecordChanges
\begin{document}
   \MakeShortVerb{\+}
   \maketitle
   \DocInput{\filename}
   \StopEventually{\end{document}}
   \clearpage
   % Make TeX shut up.
   \hbadness=10000
   \newcount\hbadness
   \hfuzz=\maxdimen
   \PrintChanges
   \let\PrintChanges\relax
\end{document}
%</driver>
% \fi
%
%
%\ifJAPANESE
% \changes{v1.0c-u00}{2011/05/07}{\pLaTeX{}用から\upLaTeX{}用に修正。
%     (based on platex.dtx 1997/01/29 v1.0c)}
% \changes{v1.0k-u01}{2016/06/06}{\upLaTeX{}用にドキュメントを全体的に改訂}
% \changes{v1.0q-u01}{2017/11/29}{英語版ドキュメントを追加
%     (based on platex.dtx 2017/11/29 v1.0q)}
% \changes{v1.0x-u02}{2018/09/03}{ドキュメントを更新
%     (based on platex.dtx 2018/09/03 v1.0x)}
% \changes{v1.0y-u03}{2019/05/22}{ドキュメントを更新}
%\else
% \changes{v1.0c-u00}{2011/05/07}{Created \upLaTeX\ version based on \pLaTeX\ one
%     (based on platex.dtx 1997/01/29 v1.0c)}
% \changes{v1.0k-u01}{2016/06/06}{Update documents for \upLaTeX.}
% \changes{v1.0q-u01}{2017/11/29}{New English documentation added!
%     (based on platex.dtx 2017/11/29 v1.0q)}
% \changes{v1.0x-u02}{2018/09/03}{Update document.
%     (based on platex.dtx 2018/09/03 v1.0x)}
% \changes{v1.0y-u03}{2019/05/22}{Update document.}
%\fi
%\ifJAPANESE
% \upLaTeX{}は、内部コードをUnicode化した\pLaTeX{}の拡張版です。
% このバージョンは、「コミュニティ版\pLaTeXe{}」をベースにしています。
%\else
% \upLaTeX\ is a Unicode version of Japanese \pLaTeXe.
% This version is based on `\pLaTeXe\ Community Edition.'
%\fi
%
%\ifJAPANESE
% \pTeX{}は、高品質の日本語組版ソフトウェアとしてデファクト
% スタンダードの地位にあるといえます。しかし、\pTeX{}には
% \begin{itemize}
% \item 直接使える文字集合が原則的にJIS X 0208(JIS第1,2水準)の範囲に限定
%   されていること、
% \item 8bitの非英語欧文との親和性が高いとは言えないこと、
% \item \pTeX{}の利用が日本語に限られ、中国語・韓国語との混植への利用が
%   進んでいないこと
% \end{itemize}
% といった弱点がありました。
%
% これらの弱点を克服するため、\pTeX{}の内部コードをUnicode化した拡張版
% が\upTeX{}です。また、\upTeX{}上で用いるUnicode版\pLaTeX{}が\upLaTeX{}で
% す\footnote{\texttt{http://www.t-lab.opal.ne.jp/tex/uptex.html}}。
% 現在の\upLaTeX{}は、日本語\TeX{}開発コミュニティが配布しているコミュニティ
% 版\pLaTeX{}\footnote{\texttt{https://github.com/texjporg/platex}}を
% ベースにしており、\eupTeX{}というエンジン(\upTeX{}の\epTeX{}拡張版)で
% 動作します。
%
% 開発中の版は\pLaTeX{}と同様に、GitHubの
% リポジトリ\footnote{\texttt{https://github.com/texjporg/uplatex}}で
% 管理しています。\upLaTeX{}はアスキーとは無関係ですので、
% バグレポートはアスキー宛てではなく、日本語\TeX{}開発コミュニティに報告
% してください。\TeX\ ForumやGitHubのIssueシステムが利用できます。
%\else
% \pTeX\ is the most popular \TeX\ engine in Japan and is widely
% used for a high-quality typesetting, even for commercial printing.
% However, \pTeX\ has some limitations:
% \begin{itemize}
% \item The character set available is limited to JIS X 0208,
%   namely JIS level-1 and level-2
% \item Difficulty in handling 8-bit Latin, due to conflict with
%   legacy multibyte Japanese encodings
% \item Difficulty in typesetting CJK (Chinese, Japanese and Korean)
%  multilingual documents
% \end{itemize}
%
% To overcome these weak points,
% a Unicode extension of \pTeX, \upTeX, has been
% developed.\footnote{\texttt{http://www.t-lab.opal.ne.jp/tex/uptex.html}}
% The Unicode \pLaTeX\ format run on \upTeX\ is called \upLaTeX.
% Current \upLaTeX\ is maintained by Japanese \TeX\ Development
% Community,\footnote{\texttt{https://texjp.org}}
% in sync with \pLaTeX\ community
% edition.\footnote{\texttt{https://github.com/texjporg/platex}}
% It runs on \eupTeX, an engine with both \upTeX\ and \epTeX\ features.
%
% The development version is available from
% GitHub repository\footnote{\texttt{https://github.com/texjporg/uplatex}}.
% Any bug reports and requests should be sent to
% Japanese \TeX\ Development Community, using GitHub Issue system.
%\fi
%
%
% \clearpage
%
%\ifJAPANESE
% \section{この文書について}\label{platex:intro}
% この文書は\upLaTeXe{}の概要を示していますが、使い方のガイドでは
% ありません。ほとんどの機能は元となっている\pLaTeXe{}や\LaTeXe{}と
% 同等ですので、それぞれの付属文書などを参照してください。
%
% \upTeX{}については公式ウェブサイトあるいは\cite{tb108tanaka}(英語)を
% 参照してください。
%\else
% \section{Introduction to this document}\label{platex:intro}
% This document briefly describes \upLaTeXe, but is not a manual of \upLaTeXe.
% The basic functions of \upLaTeXe\ are almost the same with those of
% \pLaTeXe\ and \LaTeXe, so please refer to the documentation of those formats.
%
% For \upTeX, please refer to the official website or
% \cite{tb108tanaka} (in English).
%\fi
%
%\ifJAPANESE
% この文書の構成は次のようになっています。
%
% \begin{quote}
% \begin{description}
% \item[第\ref{platex:intro}節]
%  この節です。この文書についての概要を述べています。
%
% \item[第\ref{platex:plcore}節]
%  \upLaTeXe{}で拡張した機能についての概要です。
%  付属のクラスファイルやパッケージファイルについても簡単に
%  説明しています。
%
% \item[第\ref{platex:compatibility}節]
%  現在のバージョンの\upLaTeX{}と旧バージョン、あるいは元となっている
% \pLaTeX{}/\LaTeX{}との互換性について述べています。
%
% \item[付録\ref{app:dst}]
%  この文書ソース(uplatex.dtx)の
%  \dst{}のためのオプションについて述べています。
%
% \item[付録\ref{app:pldoc}]
%  \upLaTeXe{}のdtxファイルをまとめて、一つのソースコード説明書に
%  するための文書ファイルの説明をしています。
%
% \item[付録\ref{app:omake}]
%  付録\ref{app:pldoc}で説明した文書ファイルを処理するshスクリプト(手順)
%  などについて説明しています。
% \end{description}
% \end{quote}
%\else
% This document consists of following parts:
%
% \begin{quote}
% \begin{description}
% \item[Section \ref{platex:intro}]
%  This section; describes this document itself.
%
% \item[Section \ref{platex:plcore}]
%  Brief explanation of extensions in \upLaTeXe.
%  Also describes the standard classes and packages.
%
% \item[Section \ref{platex:compatibility}]
%  The compatibility note for users of the old version of
%  \upLaTeXe\ or those of the original \pLaTeXe/\LaTeXe.
%
% \item[Appendix \ref{app:dst}]
%  Describes \dst\ Options for this document.
%
% \item[Appendix \ref{app:pldoc}]
%  Description of `upldoc.tex' (counterpart for `source2e.tex' in \LaTeXe).
%
% \item[Appendix \ref{app:omake}]
%  Description of a shell script to process `upldoc.tex', etc.
% \end{description}
% \end{quote}
%\fi
%
%
%\ifJAPANESE
% \section{\upLaTeXe{}の機能について}\label{platex:plcore}
% \upLaTeXe{}が提供するファイルは、次の3種類に分類することができます。
% この構成は\pLaTeXe{}と同様です。
%
% \begin{itemize}
% \item フォーマットファイル
% \item クラスファイル
% \item パッケージファイル
% \end{itemize}
%\else
% \section{About Functions of \pLaTeXe}\label{platex:plcore}
% The structure of \upLaTeXe\ is similar to that of \pLaTeXe;
% it consists of 3 types of files: a format (uplatex.ltx),
% classes and packages.
%\fi
%
%\ifJAPANESE
% \subsection{フォーマットファイル}
% \upLaTeX{}のフォーマットファイルを作成するには、
% ソースファイル``uplatex.ltx''を\eupTeX{}のINIモードで処理します
% \footnote{2016年以前は\upTeX{}と\eupTeX{}のどちらでもフォーマットを作成する
% ことができましたが、2017年に\LaTeX{}が\eTeX{}必須となったことに伴い、
% \upLaTeX{}も\eupTeX{}が必須となりました。}。
% ただし、\TeX\ LiveやW32\TeX{}ではこの処理を簡単にする|fmtutil-sys|あるいは
% |fmtutil|というプログラムが用意されています。
% 以下を実行すれば、フォーマットファイル\file{uplatex.fmt}が作成されます。
%\else
% \subsection{About the Format}
% To make a format for \upLaTeX,
% process ``uplatex.ltx'' with INI mode of \eupTeX.\footnote{Formerly
% both \upTeX\ and \eupTeX\ can make the format file for \upLaTeX, however,
% it's not true anymore because \LaTeX\ requires \eTeX\ since 2017.}
% A handy command `fmtutil-sys' (or `fmtutil') for this purpose
% is available in \TeX\ Live. The following command generates \file{uplatex.fmt}.
%\fi
%\begin{verbatim}
%   fmtutil-sys --byfmt uplatex
%\end{verbatim}
%
%\ifJAPANESE
% 次のリストが、\file{uplatex.ltx}の内容です。
% ただし、このバージョンでは、\LaTeX{}から\upLaTeX{}への拡張を
% \file{plcore.ltx}(\pLaTeX{}によって提供される)および
% \file{uplcore.ltx}をロードすることで行ない、
% \file{latex.ltx}には直接、手を加えないようにしています。
% したがって\file{uplatex.ltx}はとても短いものとなっています。
% \file{latex.ltx}には\LaTeX{}のコマンドが、
% \file{uplcore.ltx}には\upLaTeX{}で拡張したコマンドが定義されています。
%\else
% The content of \file{uplatex.ltx} is shown below.
% In the current version of \upLaTeX,
% first we simply load \file{latex.ltx} and
% modify/extend some definitions by loading
% \file{plcore.ltx} (available from \pLaTeX) and \file{uplcore.ltx}.
%\fi
%    \begin{macrocode}
%<*plcore>
%    \end{macrocode}
%
%\ifJAPANESE
% \file{latex.ltx}の末尾で使われている|\dump|をいったん無効化します。
%\else
% Temporarily disable |\dump| at the end of \file{latex.ltx}.
%\fi
%    \begin{macrocode}
\let\orgdump\dump
\let\dump\relax
%    \end{macrocode}
%
%\ifJAPANESE
% \file{latex.ltx}を読み込みます。
% \TeX\ Liveの標準的インストールでは、この中でBabel由来の
% ハイフネーション・パターン\file{hyphen.cfg}が読み込まれるはずです。
%\else
% Load \file{latex.ltx} here.
% Within the standard installation of \TeX\ Live, \file{hyphen.cfg}
% provided by ``Babel'' package will be used.
%\fi
%    \begin{macrocode}
\input latex.ltx
%    \end{macrocode}
%
%\ifJAPANESE
% この時点で|\typeout|が未定義なら、\LaTeX{}カーネルの読み込みに
% 失敗していますので、強制終了します(\LaTeXe\ 2017/01/01以降を
% 非\eTeX{}拡張でフォーマット作成しようとした場合など)。
% \changes{v1.1c-u03}{2021/02/25}{\file{latex.ltx}の読込チェック}
%\else
% If |\typeout| is still undefined, the input of \LaTeX~kernel
% should have failed; abort now.
% \changes{v1.1c-u03}{2021/02/25}{Check for \file{latex.ltx} status}
%\fi
%    \begin{macrocode}
\ifx\typeout\undefined
  \errhelp{Please reinstall LaTeX, or check e-TeX availability.}%
  \errmessage{Failed to load `latex.ltx' properly}%
  \expandafter\end
\fi
%    \end{macrocode}
%
%\ifJAPANESE
% \file{plcore.ltx}と\file{uplcore.ltx}を読み込みます。
% \changes{v1.0s-u02}{2017/12/10}{\file{uplcore.ltx}の前に
%    \file{plcore.ltx}を読み込むようにした(最近の\pLaTeX{}が前提)}
%\else
% Load \file{plcore.ltx} and \file{uplcore.ltx}.
% \changes{v1.0s-u02}{2017/12/10}{Load \file{plcore.ltx} before
%    \file{uplcore.ltx} (recent version of \pLaTeX\ is assumed)}
%\fi
%    \begin{macrocode}
\typeout{**************************^^J%
         *^^J%
         * making upLaTeX format^^J%
         *^^J%
         **************************}
\makeatletter
\input plcore.ltx
\input uplcore.ltx
%    \end{macrocode}
%
%\ifJAPANESE
% フォント関連のデフォルト設定ファイルである、
% \file{upldefs.ltx}を読み込みます。
% \TeX{}の入力ファイル検索パスに設定されている
% ディレクトリに\file{upldefs.cfg}ファイルがある場合は、
% そのファイルを使います。
% 読み込み後にコードが実行されるかもしれません。
% \changes{v1.0s-u01}{2017/12/05}{デフォルト設定ファイルの読み込みを
%    \file{uplcore.ltx}から\file{uplatex.ltx}へ移動
%     (based on platex.dtx 2017/12/05 v1.0s)}
% \changes{v1.1b-u03}{2020/09/28}{defs読込後にフック追加}
%\else
% Load font-related default settings, \file{upldefs.ltx}.
% If a file \file{upldefs.cfg} is found, then that file will be
% used instead.
% Some code may be executed after loading.
% \changes{v1.0s-u01}{2017/12/05}{Moved loading default settings
%    from \file{uplcore.ltx} to \file{uplatex.ltx}
%     (based on platex.dtx 2017/12/05 v1.0s)}
% \changes{v1.1b-u03}{2020/09/28}{Add hook after loading defs}
%\fi
%    \begin{macrocode}
\InputIfFileExists{upldefs.cfg}
           {\typeout{*************************************^^J%
                     * Local config file upldefs.cfg used^^J%
                     *************************************}}%
           {\input{upldefs.ltx}}
\ifx\code@after@pldefs\@undefined\else \code@after@pldefs \fi
%    \end{macrocode}
%
%\ifJAPANESE
% 以前のバージョンでは、フォーマット作成時に\upLaTeX{}のバージョンが
% わかるように、端末に表示していましたが、|\everyjob| にバナー表示
% 以外のコードが含まれる可能性を考慮し、安全のためやめました。
% \changes{v1.0w-u02}{2018/04/08}{安全のためフォーマット作成時の
%    バナー表示をやめた
%     (based on platex.dtx 2018/04/08 v1.0w)}
%\else
% In the previous version, we displayed \upLaTeX\ version
% on the terminal, so that it can be easily recognized
% during format creation; however |\everyjob| can contain
% any code other than showing a banner, so now disabled.
% \changes{v1.0w-u02}{2018/04/08}{Stop showing banner during
%    format generation for safety
%     (based on platex.dtx 2018/04/08 v1.0w)}
%\fi
%    \begin{macrocode}
%\the\everyjob
%    \end{macrocode}
%
%\ifJAPANESE
% \upLaTeXe{}の起動時に\file{uplatex.cfg}がある場合、それを読み込む
% ようにします(\pLaTeXe{}が\file{platex.cfg}を読み込むのと同様)。
% バージョン2016/07/01ではコードを\file{uplcore.ltx}に入れていました
% が、\file{uplatex.ltx}へ移動しました。
% \changes{v1.0m-u01}{2016/08/26}{\file{uplatex.cfg}の読み込みを
%    \file{uplcore.ltx}から\file{uplatex.ltx}へ移動
%     (based on platex.dtx 2016/08/26 v1.0m)}
%\else
% Load \file{uplatex.cfg} if it exists at runtime of \upLaTeXe.
% (Counterpart of \file{platex.cfg} in \pLaTeXe.)
% \changes{v1.0m-u01}{2016/08/26}{Moved loading \file{uplatex.cfg}
%    from \file{uplcore.ltx} to \file{uplatex.ltx}
%     (based on platex.dtx 2016/08/26 v1.0m)}
%\fi
%    \begin{macrocode}
\everyjob\expandafter{%
  \the\everyjob
  \IfFileExists{uplatex.cfg}{%
    \typeout{*************************^^J%
             * Loading uplatex.cfg.^^J%
             *************************}%
    % \iffalse meta-comment
%% File: uplatex.dtx
%
%    pLaTeX base file:
%       Copyright 1995,1996 ASCII Corporation.
%    and modified for upLaTeX
%
%  Copyright (c) 2010 ASCII MEDIA WORKS
%  Copyright (c) 2016 Takuji Tanaka
%  Copyright (c) 2016-2021 Japanese TeX Development Community
%
%  This file is part of the upLaTeX2e system (community edition).
%  --------------------------------------------------------------
%
% \fi
%
% \iffalse
%<*driver|pldoc>
\ifx\JAPANESEtrue\undefined
  \expandafter\newif\csname ifJAPANESE\endcsname
  \JAPANESEtrue
\fi
%</driver|pldoc>
% \fi
%
% \setcounter{StandardModuleDepth}{1}
% \makeatletter
%\ifJAPANESE
% \def\chuui{\@ifnextchar[{\@chuui}{\@chuui[注意:]}}
%\else
% \def\chuui{\@ifnextchar[{\@chuui}{\@chuui[Attention: ]}}
%\fi
% \def\@chuui[#1]{\par\vskip.5\baselineskip
%   \noindent{\em #1}\par\bgroup\gtfamily\sffamily}
% \def\endchuui{\egroup\vskip.5\baselineskip}
% \makeatother
%
% \iffalse
%<*driver|pldoc>
\def\eTeX{$\varepsilon$-\TeX}
\def\pTeX{p\kern-.15em\TeX}
\def\epTeX{$\varepsilon$-\pTeX}
\def\pLaTeX{p\kern-.05em\LaTeX}
\def\pLaTeXe{p\kern-.05em\LaTeXe}
\def\upTeX{u\pTeX}
\def\eupTeX{$\varepsilon$-\upTeX}
\def\upLaTeX{u\pLaTeX}
\def\upLaTeXe{u\pLaTeXe}
%</driver|pldoc>
% \fi
%
% \StopEventually{}
%
% \iffalse
%\ifJAPANESE
% \changes{v1.0c-u00}{2011/05/07}{\pLaTeX{}用から\upLaTeX{}用に修正。
%     (based on platex.dtx 1997/01/29 v1.0c)}
% \changes{v1.0e-u00}{2016/04/06}{\pLaTeX{}の変更に追随。
%     (based on platex.dtx 2016/02/16 v1.0e)}
% \changes{v1.0h-u00}{2016/05/08}{ドキュメントから\file{uplpatch.ltx}を除外
%     (based on platex.dtx 2016/05/08 v1.0h)}
% \changes{v1.0k-u00}{2016/05/21}{\pLaTeX{}の変更に追随。
%     (based on platex.dtx 2016/05/21 v1.0k)}
% \changes{v1.0k-u01}{2016/06/06}{\upLaTeX{}用にドキュメントを全体的に改訂}
% \changes{v1.0l-u01}{2016/06/19}{パッチレベルを\file{uplvers.dtx}から取得
%     (based on platex.dtx 2016/06/19 v1.0l)}
% \changes{v1.0m-u01}{2016/08/26}{\file{uplatex.cfg}の読み込みを
%    \file{uplcore.ltx}から\file{uplatex.ltx}へ移動
%     (based on platex.dtx 2016/08/26 v1.0m)}
% \changes{v1.0n-u01}{2016/09/14}{\pLaTeX{}の変更に追随。
%     (based on platex.dtx 2016/09/14 v1.0n)}
% \changes{v1.0p-u01}{2017/11/11}{\pLaTeX{}の変更に追随。
%     (based on platex.dtx 2017/11/11 v1.0p)}
% \changes{v1.0q-u01}{2017/11/29}{英語版ドキュメントを追加
%     (based on platex.dtx 2017/11/29 v1.0q)}
% \changes{v1.0r-u01}{2017/12/02}{\upLaTeX{}と\upTeX{}の参考文献も追加}
% \changes{v1.0s-u01}{2017/12/05}{デフォルト設定ファイルの読み込みを
%    \file{uplcore.ltx}から\file{uplatex.ltx}へ移動
%     (based on platex.dtx 2017/12/05 v1.0s)}
% \changes{v1.0s-u02}{2017/12/10}{\file{uplcore.ltx}の前に
%    \file{plcore.ltx}を読み込むようにした(最近の\pLaTeX{}が前提)}
% \changes{v1.0u-u02}{2018/02/18}{\pLaTeX{}の変更に追随。
%     (based on platex.dtx 2018/02/18 v1.0u)}
% \changes{v1.0v-u02}{2018/04/06}{最新のsource2eへの追随
%     (based on platex.dtx 2018/04/06 v1.0v)}
% \changes{v1.0w-u02}{2018/04/08}{安全のためフォーマット作成時の
%    バナー表示をやめた
%     (based on platex.dtx 2018/04/08 v1.0w)}
% \changes{v1.0x-u02}{2018/09/03}{ドキュメントを更新
%     (based on platex.dtx 2018/09/03 v1.0x)}
% \changes{v1.0y-u02}{2018/09/22}{最終更新日を\file{upldoc.pdf}に表示
%     (based on platex.dtx 2018/09/22 v1.0y)}
% \changes{v1.0y-u03}{2019/05/22}{ドキュメントを更新}
% \changes{v1.1b-u03}{2020/09/28}{defs読込後にフック追加}
% \changes{v1.1c-u03}{2021/02/25}{\file{latex.ltx}の読込チェック}
%\else
% \changes{v1.0c-u00}{2011/05/07}{Created \upLaTeX\ version based on \pLaTeX\ one
%     (based on platex.dtx 1997/01/29 v1.0c)}
% \changes{v1.0e-u00}{2016/04/06}{Sync with \pLaTeX.
%     (based on platex.dtx 2016/02/16 v1.0e)}
% \changes{v1.0h-u00}{2016/05/08}{Exclude \file{uplpatch.ltx} from the document
%     (based on platex.dtx 2016/05/08 v1.0h)}
% \changes{v1.0k-u00}{2016/05/21}{Sync with \pLaTeX.
%     (based on platex.dtx 2016/05/21 v1.0k)}
% \changes{v1.0k-u01}{2016/06/06}{Update documents for \upLaTeX.}
% \changes{v1.0l-u01}{2016/06/19}{Get the patch level from \file{uplvers.dtx}
%     (based on platex.dtx 2016/06/19 v1.0l)}
% \changes{v1.0m-u01}{2016/08/26}{Moved loading \file{uplatex.cfg}
%    from \file{uplcore.ltx} to \file{uplatex.ltx}
%     (based on platex.dtx 2016/08/26 v1.0m)}
% \changes{v1.0n-u01}{2016/09/14}{Sync with \pLaTeX.
%     (based on platex.dtx 2016/09/14 v1.0n)}
% \changes{v1.0p-u01}{2017/11/11}{Sync with \pLaTeX.
%     (based on platex.dtx 2017/11/11 v1.0p)}
% \changes{v1.0q-u01}{2017/11/29}{New English documentation added!
%     (based on platex.dtx 2017/11/29 v1.0q)}
% \changes{v1.0r-u01}{2017/12/02}{\upLaTeX\ and \upTeX\ references added}
% \changes{v1.0s-u01}{2017/12/05}{Moved loading default settings
%    from \file{uplcore.ltx} to \file{uplatex.ltx}
%     (based on platex.dtx 2017/12/05 v1.0s)}
% \changes{v1.0s-u02}{2017/12/10}{Load \file{plcore.ltx} before
%    \file{uplcore.ltx} (recent version of \pLaTeX\ is assumed)}
% \changes{v1.0u-u02}{2018/02/18}{Sync with \pLaTeX.
%     (based on platex.dtx 2018/02/18 v1.0u)}
% \changes{v1.0v-u02}{2018/04/06}{Sync with the latest \file{source2e.tex}
%     (based on platex.dtx 2018/04/06 v1.0v)}
% \changes{v1.0w-u02}{2018/04/08}{Stop showing banner during
%    format generation for safety
%     (based on platex.dtx 2018/04/08 v1.0w)}
% \changes{v1.0x-u02}{2018/09/03}{Update document.
%     (based on platex.dtx 2018/09/03 v1.0x)}
% \changes{v1.0y-u02}{2018/09/22}{Show last update info on \file{upldoc.pdf}
%     (based on platex.dtx 2018/09/22 v1.0y)}
% \changes{v1.0y-u03}{2019/05/22}{Update document.}
% \changes{v1.1b-u03}{2020/09/28}{Add hook after loading defs}
% \changes{v1.1c-u03}{2021/02/25}{Check for \file{latex.ltx} status}
%\fi
% \fi
%
% \iffalse
%<*driver>
\NeedsTeXFormat{pLaTeX2e}
% \fi
\ProvidesFile{uplatex.dtx}[2021/02/25 v1.1c-u03 upLaTeX document file]
% \iffalse
\documentclass{jltxdoc}
\usepackage{plext}
\GetFileInfo{uplatex.dtx}
\ifJAPANESE
\title{\upLaTeXe{}について}
\author{中野 賢 \& 日本語\TeX{}開発コミュニティ \& TTK}
\date{作成日:\filedate}
\renewcommand{\refname}{参考文献}
\GlossaryPrologue{\section*{変更履歴}%
                  \markboth{変更履歴}{変更履歴}%
                  \addcontentsline{toc}{section}{変更履歴}}
\else
\title{About \upLaTeXe{}}
\author{Ken Nakano \& Japanese \TeX\ Development Community \& TTK}
\date{Date: \filedate}
\renewcommand{\refname}{References}
\GlossaryPrologue{\section*{Change History}%
                  \markboth{Change History}{Change History}%
                  \addcontentsline{toc}{section}{Change History}}
\fi
\makeatletter
\ifJAPANESE
\def\levelchar{>・}
\fi
\def\changes@#1#2#3{%
  \let\protect\@unexpandable@protect
  \edef\@tempa{\noexpand\glossary{#2\space#1\levelchar
    \ifx\saved@macroname\@empty
%     \space\actualchar\generalname: %% comment out (uplatex.dtx only)
    \else
      \expandafter\@gobble
      \saved@macroname\actualchar
      \string\verb\quotechar*%
      \verbatimchar\saved@macroname
      \verbatimchar:
    \fi
    #3}}%
  \@tempa\endgroup\@esphack}
\makeatother
\RecordChanges
\begin{document}
   \MakeShortVerb{\+}
   \maketitle
   \DocInput{\filename}
   \StopEventually{\end{document}}
   \clearpage
   % Make TeX shut up.
   \hbadness=10000
   \newcount\hbadness
   \hfuzz=\maxdimen
   \PrintChanges
   \let\PrintChanges\relax
\end{document}
%</driver>
% \fi
%
%
%\ifJAPANESE
% \changes{v1.0c-u00}{2011/05/07}{\pLaTeX{}用から\upLaTeX{}用に修正。
%     (based on platex.dtx 1997/01/29 v1.0c)}
% \changes{v1.0k-u01}{2016/06/06}{\upLaTeX{}用にドキュメントを全体的に改訂}
% \changes{v1.0q-u01}{2017/11/29}{英語版ドキュメントを追加
%     (based on platex.dtx 2017/11/29 v1.0q)}
% \changes{v1.0x-u02}{2018/09/03}{ドキュメントを更新
%     (based on platex.dtx 2018/09/03 v1.0x)}
% \changes{v1.0y-u03}{2019/05/22}{ドキュメントを更新}
%\else
% \changes{v1.0c-u00}{2011/05/07}{Created \upLaTeX\ version based on \pLaTeX\ one
%     (based on platex.dtx 1997/01/29 v1.0c)}
% \changes{v1.0k-u01}{2016/06/06}{Update documents for \upLaTeX.}
% \changes{v1.0q-u01}{2017/11/29}{New English documentation added!
%     (based on platex.dtx 2017/11/29 v1.0q)}
% \changes{v1.0x-u02}{2018/09/03}{Update document.
%     (based on platex.dtx 2018/09/03 v1.0x)}
% \changes{v1.0y-u03}{2019/05/22}{Update document.}
%\fi
%\ifJAPANESE
% \upLaTeX{}は、内部コードをUnicode化した\pLaTeX{}の拡張版です。
% このバージョンは、「コミュニティ版\pLaTeXe{}」をベースにしています。
%\else
% \upLaTeX\ is a Unicode version of Japanese \pLaTeXe.
% This version is based on `\pLaTeXe\ Community Edition.'
%\fi
%
%\ifJAPANESE
% \pTeX{}は、高品質の日本語組版ソフトウェアとしてデファクト
% スタンダードの地位にあるといえます。しかし、\pTeX{}には
% \begin{itemize}
% \item 直接使える文字集合が原則的にJIS X 0208(JIS第1,2水準)の範囲に限定
%   されていること、
% \item 8bitの非英語欧文との親和性が高いとは言えないこと、
% \item \pTeX{}の利用が日本語に限られ、中国語・韓国語との混植への利用が
%   進んでいないこと
% \end{itemize}
% といった弱点がありました。
%
% これらの弱点を克服するため、\pTeX{}の内部コードをUnicode化した拡張版
% が\upTeX{}です。また、\upTeX{}上で用いるUnicode版\pLaTeX{}が\upLaTeX{}で
% す\footnote{\texttt{http://www.t-lab.opal.ne.jp/tex/uptex.html}}。
% 現在の\upLaTeX{}は、日本語\TeX{}開発コミュニティが配布しているコミュニティ
% 版\pLaTeX{}\footnote{\texttt{https://github.com/texjporg/platex}}を
% ベースにしており、\eupTeX{}というエンジン(\upTeX{}の\epTeX{}拡張版)で
% 動作します。
%
% 開発中の版は\pLaTeX{}と同様に、GitHubの
% リポジトリ\footnote{\texttt{https://github.com/texjporg/uplatex}}で
% 管理しています。\upLaTeX{}はアスキーとは無関係ですので、
% バグレポートはアスキー宛てではなく、日本語\TeX{}開発コミュニティに報告
% してください。\TeX\ ForumやGitHubのIssueシステムが利用できます。
%\else
% \pTeX\ is the most popular \TeX\ engine in Japan and is widely
% used for a high-quality typesetting, even for commercial printing.
% However, \pTeX\ has some limitations:
% \begin{itemize}
% \item The character set available is limited to JIS X 0208,
%   namely JIS level-1 and level-2
% \item Difficulty in handling 8-bit Latin, due to conflict with
%   legacy multibyte Japanese encodings
% \item Difficulty in typesetting CJK (Chinese, Japanese and Korean)
%  multilingual documents
% \end{itemize}
%
% To overcome these weak points,
% a Unicode extension of \pTeX, \upTeX, has been
% developed.\footnote{\texttt{http://www.t-lab.opal.ne.jp/tex/uptex.html}}
% The Unicode \pLaTeX\ format run on \upTeX\ is called \upLaTeX.
% Current \upLaTeX\ is maintained by Japanese \TeX\ Development
% Community,\footnote{\texttt{https://texjp.org}}
% in sync with \pLaTeX\ community
% edition.\footnote{\texttt{https://github.com/texjporg/platex}}
% It runs on \eupTeX, an engine with both \upTeX\ and \epTeX\ features.
%
% The development version is available from
% GitHub repository\footnote{\texttt{https://github.com/texjporg/uplatex}}.
% Any bug reports and requests should be sent to
% Japanese \TeX\ Development Community, using GitHub Issue system.
%\fi
%
%
% \clearpage
%
%\ifJAPANESE
% \section{この文書について}\label{platex:intro}
% この文書は\upLaTeXe{}の概要を示していますが、使い方のガイドでは
% ありません。ほとんどの機能は元となっている\pLaTeXe{}や\LaTeXe{}と
% 同等ですので、それぞれの付属文書などを参照してください。
%
% \upTeX{}については公式ウェブサイトあるいは\cite{tb108tanaka}(英語)を
% 参照してください。
%\else
% \section{Introduction to this document}\label{platex:intro}
% This document briefly describes \upLaTeXe, but is not a manual of \upLaTeXe.
% The basic functions of \upLaTeXe\ are almost the same with those of
% \pLaTeXe\ and \LaTeXe, so please refer to the documentation of those formats.
%
% For \upTeX, please refer to the official website or
% \cite{tb108tanaka} (in English).
%\fi
%
%\ifJAPANESE
% この文書の構成は次のようになっています。
%
% \begin{quote}
% \begin{description}
% \item[第\ref{platex:intro}節]
%  この節です。この文書についての概要を述べています。
%
% \item[第\ref{platex:plcore}節]
%  \upLaTeXe{}で拡張した機能についての概要です。
%  付属のクラスファイルやパッケージファイルについても簡単に
%  説明しています。
%
% \item[第\ref{platex:compatibility}節]
%  現在のバージョンの\upLaTeX{}と旧バージョン、あるいは元となっている
% \pLaTeX{}/\LaTeX{}との互換性について述べています。
%
% \item[付録\ref{app:dst}]
%  この文書ソース(uplatex.dtx)の
%  \dst{}のためのオプションについて述べています。
%
% \item[付録\ref{app:pldoc}]
%  \upLaTeXe{}のdtxファイルをまとめて、一つのソースコード説明書に
%  するための文書ファイルの説明をしています。
%
% \item[付録\ref{app:omake}]
%  付録\ref{app:pldoc}で説明した文書ファイルを処理するshスクリプト(手順)
%  などについて説明しています。
% \end{description}
% \end{quote}
%\else
% This document consists of following parts:
%
% \begin{quote}
% \begin{description}
% \item[Section \ref{platex:intro}]
%  This section; describes this document itself.
%
% \item[Section \ref{platex:plcore}]
%  Brief explanation of extensions in \upLaTeXe.
%  Also describes the standard classes and packages.
%
% \item[Section \ref{platex:compatibility}]
%  The compatibility note for users of the old version of
%  \upLaTeXe\ or those of the original \pLaTeXe/\LaTeXe.
%
% \item[Appendix \ref{app:dst}]
%  Describes \dst\ Options for this document.
%
% \item[Appendix \ref{app:pldoc}]
%  Description of `upldoc.tex' (counterpart for `source2e.tex' in \LaTeXe).
%
% \item[Appendix \ref{app:omake}]
%  Description of a shell script to process `upldoc.tex', etc.
% \end{description}
% \end{quote}
%\fi
%
%
%\ifJAPANESE
% \section{\upLaTeXe{}の機能について}\label{platex:plcore}
% \upLaTeXe{}が提供するファイルは、次の3種類に分類することができます。
% この構成は\pLaTeXe{}と同様です。
%
% \begin{itemize}
% \item フォーマットファイル
% \item クラスファイル
% \item パッケージファイル
% \end{itemize}
%\else
% \section{About Functions of \pLaTeXe}\label{platex:plcore}
% The structure of \upLaTeXe\ is similar to that of \pLaTeXe;
% it consists of 3 types of files: a format (uplatex.ltx),
% classes and packages.
%\fi
%
%\ifJAPANESE
% \subsection{フォーマットファイル}
% \upLaTeX{}のフォーマットファイルを作成するには、
% ソースファイル``uplatex.ltx''を\eupTeX{}のINIモードで処理します
% \footnote{2016年以前は\upTeX{}と\eupTeX{}のどちらでもフォーマットを作成する
% ことができましたが、2017年に\LaTeX{}が\eTeX{}必須となったことに伴い、
% \upLaTeX{}も\eupTeX{}が必須となりました。}。
% ただし、\TeX\ LiveやW32\TeX{}ではこの処理を簡単にする|fmtutil-sys|あるいは
% |fmtutil|というプログラムが用意されています。
% 以下を実行すれば、フォーマットファイル\file{uplatex.fmt}が作成されます。
%\else
% \subsection{About the Format}
% To make a format for \upLaTeX,
% process ``uplatex.ltx'' with INI mode of \eupTeX.\footnote{Formerly
% both \upTeX\ and \eupTeX\ can make the format file for \upLaTeX, however,
% it's not true anymore because \LaTeX\ requires \eTeX\ since 2017.}
% A handy command `fmtutil-sys' (or `fmtutil') for this purpose
% is available in \TeX\ Live. The following command generates \file{uplatex.fmt}.
%\fi
%\begin{verbatim}
%   fmtutil-sys --byfmt uplatex
%\end{verbatim}
%
%\ifJAPANESE
% 次のリストが、\file{uplatex.ltx}の内容です。
% ただし、このバージョンでは、\LaTeX{}から\upLaTeX{}への拡張を
% \file{plcore.ltx}(\pLaTeX{}によって提供される)および
% \file{uplcore.ltx}をロードすることで行ない、
% \file{latex.ltx}には直接、手を加えないようにしています。
% したがって\file{uplatex.ltx}はとても短いものとなっています。
% \file{latex.ltx}には\LaTeX{}のコマンドが、
% \file{uplcore.ltx}には\upLaTeX{}で拡張したコマンドが定義されています。
%\else
% The content of \file{uplatex.ltx} is shown below.
% In the current version of \upLaTeX,
% first we simply load \file{latex.ltx} and
% modify/extend some definitions by loading
% \file{plcore.ltx} (available from \pLaTeX) and \file{uplcore.ltx}.
%\fi
%    \begin{macrocode}
%<*plcore>
%    \end{macrocode}
%
%\ifJAPANESE
% \file{latex.ltx}の末尾で使われている|\dump|をいったん無効化します。
%\else
% Temporarily disable |\dump| at the end of \file{latex.ltx}.
%\fi
%    \begin{macrocode}
\let\orgdump\dump
\let\dump\relax
%    \end{macrocode}
%
%\ifJAPANESE
% \file{latex.ltx}を読み込みます。
% \TeX\ Liveの標準的インストールでは、この中でBabel由来の
% ハイフネーション・パターン\file{hyphen.cfg}が読み込まれるはずです。
%\else
% Load \file{latex.ltx} here.
% Within the standard installation of \TeX\ Live, \file{hyphen.cfg}
% provided by ``Babel'' package will be used.
%\fi
%    \begin{macrocode}
\input latex.ltx
%    \end{macrocode}
%
%\ifJAPANESE
% この時点で|\typeout|が未定義なら、\LaTeX{}カーネルの読み込みに
% 失敗していますので、強制終了します(\LaTeXe\ 2017/01/01以降を
% 非\eTeX{}拡張でフォーマット作成しようとした場合など)。
% \changes{v1.1c-u03}{2021/02/25}{\file{latex.ltx}の読込チェック}
%\else
% If |\typeout| is still undefined, the input of \LaTeX~kernel
% should have failed; abort now.
% \changes{v1.1c-u03}{2021/02/25}{Check for \file{latex.ltx} status}
%\fi
%    \begin{macrocode}
\ifx\typeout\undefined
  \errhelp{Please reinstall LaTeX, or check e-TeX availability.}%
  \errmessage{Failed to load `latex.ltx' properly}%
  \expandafter\end
\fi
%    \end{macrocode}
%
%\ifJAPANESE
% \file{plcore.ltx}と\file{uplcore.ltx}を読み込みます。
% \changes{v1.0s-u02}{2017/12/10}{\file{uplcore.ltx}の前に
%    \file{plcore.ltx}を読み込むようにした(最近の\pLaTeX{}が前提)}
%\else
% Load \file{plcore.ltx} and \file{uplcore.ltx}.
% \changes{v1.0s-u02}{2017/12/10}{Load \file{plcore.ltx} before
%    \file{uplcore.ltx} (recent version of \pLaTeX\ is assumed)}
%\fi
%    \begin{macrocode}
\typeout{**************************^^J%
         *^^J%
         * making upLaTeX format^^J%
         *^^J%
         **************************}
\makeatletter
\input plcore.ltx
\input uplcore.ltx
%    \end{macrocode}
%
%\ifJAPANESE
% フォント関連のデフォルト設定ファイルである、
% \file{upldefs.ltx}を読み込みます。
% \TeX{}の入力ファイル検索パスに設定されている
% ディレクトリに\file{upldefs.cfg}ファイルがある場合は、
% そのファイルを使います。
% 読み込み後にコードが実行されるかもしれません。
% \changes{v1.0s-u01}{2017/12/05}{デフォルト設定ファイルの読み込みを
%    \file{uplcore.ltx}から\file{uplatex.ltx}へ移動
%     (based on platex.dtx 2017/12/05 v1.0s)}
% \changes{v1.1b-u03}{2020/09/28}{defs読込後にフック追加}
%\else
% Load font-related default settings, \file{upldefs.ltx}.
% If a file \file{upldefs.cfg} is found, then that file will be
% used instead.
% Some code may be executed after loading.
% \changes{v1.0s-u01}{2017/12/05}{Moved loading default settings
%    from \file{uplcore.ltx} to \file{uplatex.ltx}
%     (based on platex.dtx 2017/12/05 v1.0s)}
% \changes{v1.1b-u03}{2020/09/28}{Add hook after loading defs}
%\fi
%    \begin{macrocode}
\InputIfFileExists{upldefs.cfg}
           {\typeout{*************************************^^J%
                     * Local config file upldefs.cfg used^^J%
                     *************************************}}%
           {\input{upldefs.ltx}}
\ifx\code@after@pldefs\@undefined\else \code@after@pldefs \fi
%    \end{macrocode}
%
%\ifJAPANESE
% 以前のバージョンでは、フォーマット作成時に\upLaTeX{}のバージョンが
% わかるように、端末に表示していましたが、|\everyjob| にバナー表示
% 以外のコードが含まれる可能性を考慮し、安全のためやめました。
% \changes{v1.0w-u02}{2018/04/08}{安全のためフォーマット作成時の
%    バナー表示をやめた
%     (based on platex.dtx 2018/04/08 v1.0w)}
%\else
% In the previous version, we displayed \upLaTeX\ version
% on the terminal, so that it can be easily recognized
% during format creation; however |\everyjob| can contain
% any code other than showing a banner, so now disabled.
% \changes{v1.0w-u02}{2018/04/08}{Stop showing banner during
%    format generation for safety
%     (based on platex.dtx 2018/04/08 v1.0w)}
%\fi
%    \begin{macrocode}
%\the\everyjob
%    \end{macrocode}
%
%\ifJAPANESE
% \upLaTeXe{}の起動時に\file{uplatex.cfg}がある場合、それを読み込む
% ようにします(\pLaTeXe{}が\file{platex.cfg}を読み込むのと同様)。
% バージョン2016/07/01ではコードを\file{uplcore.ltx}に入れていました
% が、\file{uplatex.ltx}へ移動しました。
% \changes{v1.0m-u01}{2016/08/26}{\file{uplatex.cfg}の読み込みを
%    \file{uplcore.ltx}から\file{uplatex.ltx}へ移動
%     (based on platex.dtx 2016/08/26 v1.0m)}
%\else
% Load \file{uplatex.cfg} if it exists at runtime of \upLaTeXe.
% (Counterpart of \file{platex.cfg} in \pLaTeXe.)
% \changes{v1.0m-u01}{2016/08/26}{Moved loading \file{uplatex.cfg}
%    from \file{uplcore.ltx} to \file{uplatex.ltx}
%     (based on platex.dtx 2016/08/26 v1.0m)}
%\fi
%    \begin{macrocode}
\everyjob\expandafter{%
  \the\everyjob
  \IfFileExists{uplatex.cfg}{%
    \typeout{*************************^^J%
             * Loading uplatex.cfg.^^J%
             *************************}%
    % \iffalse meta-comment
%% File: uplatex.dtx
%
%    pLaTeX base file:
%       Copyright 1995,1996 ASCII Corporation.
%    and modified for upLaTeX
%
%  Copyright (c) 2010 ASCII MEDIA WORKS
%  Copyright (c) 2016 Takuji Tanaka
%  Copyright (c) 2016-2021 Japanese TeX Development Community
%
%  This file is part of the upLaTeX2e system (community edition).
%  --------------------------------------------------------------
%
% \fi
%
% \iffalse
%<*driver|pldoc>
\ifx\JAPANESEtrue\undefined
  \expandafter\newif\csname ifJAPANESE\endcsname
  \JAPANESEtrue
\fi
%</driver|pldoc>
% \fi
%
% \setcounter{StandardModuleDepth}{1}
% \makeatletter
%\ifJAPANESE
% \def\chuui{\@ifnextchar[{\@chuui}{\@chuui[注意:]}}
%\else
% \def\chuui{\@ifnextchar[{\@chuui}{\@chuui[Attention: ]}}
%\fi
% \def\@chuui[#1]{\par\vskip.5\baselineskip
%   \noindent{\em #1}\par\bgroup\gtfamily\sffamily}
% \def\endchuui{\egroup\vskip.5\baselineskip}
% \makeatother
%
% \iffalse
%<*driver|pldoc>
\def\eTeX{$\varepsilon$-\TeX}
\def\pTeX{p\kern-.15em\TeX}
\def\epTeX{$\varepsilon$-\pTeX}
\def\pLaTeX{p\kern-.05em\LaTeX}
\def\pLaTeXe{p\kern-.05em\LaTeXe}
\def\upTeX{u\pTeX}
\def\eupTeX{$\varepsilon$-\upTeX}
\def\upLaTeX{u\pLaTeX}
\def\upLaTeXe{u\pLaTeXe}
%</driver|pldoc>
% \fi
%
% \StopEventually{}
%
% \iffalse
%\ifJAPANESE
% \changes{v1.0c-u00}{2011/05/07}{\pLaTeX{}用から\upLaTeX{}用に修正。
%     (based on platex.dtx 1997/01/29 v1.0c)}
% \changes{v1.0e-u00}{2016/04/06}{\pLaTeX{}の変更に追随。
%     (based on platex.dtx 2016/02/16 v1.0e)}
% \changes{v1.0h-u00}{2016/05/08}{ドキュメントから\file{uplpatch.ltx}を除外
%     (based on platex.dtx 2016/05/08 v1.0h)}
% \changes{v1.0k-u00}{2016/05/21}{\pLaTeX{}の変更に追随。
%     (based on platex.dtx 2016/05/21 v1.0k)}
% \changes{v1.0k-u01}{2016/06/06}{\upLaTeX{}用にドキュメントを全体的に改訂}
% \changes{v1.0l-u01}{2016/06/19}{パッチレベルを\file{uplvers.dtx}から取得
%     (based on platex.dtx 2016/06/19 v1.0l)}
% \changes{v1.0m-u01}{2016/08/26}{\file{uplatex.cfg}の読み込みを
%    \file{uplcore.ltx}から\file{uplatex.ltx}へ移動
%     (based on platex.dtx 2016/08/26 v1.0m)}
% \changes{v1.0n-u01}{2016/09/14}{\pLaTeX{}の変更に追随。
%     (based on platex.dtx 2016/09/14 v1.0n)}
% \changes{v1.0p-u01}{2017/11/11}{\pLaTeX{}の変更に追随。
%     (based on platex.dtx 2017/11/11 v1.0p)}
% \changes{v1.0q-u01}{2017/11/29}{英語版ドキュメントを追加
%     (based on platex.dtx 2017/11/29 v1.0q)}
% \changes{v1.0r-u01}{2017/12/02}{\upLaTeX{}と\upTeX{}の参考文献も追加}
% \changes{v1.0s-u01}{2017/12/05}{デフォルト設定ファイルの読み込みを
%    \file{uplcore.ltx}から\file{uplatex.ltx}へ移動
%     (based on platex.dtx 2017/12/05 v1.0s)}
% \changes{v1.0s-u02}{2017/12/10}{\file{uplcore.ltx}の前に
%    \file{plcore.ltx}を読み込むようにした(最近の\pLaTeX{}が前提)}
% \changes{v1.0u-u02}{2018/02/18}{\pLaTeX{}の変更に追随。
%     (based on platex.dtx 2018/02/18 v1.0u)}
% \changes{v1.0v-u02}{2018/04/06}{最新のsource2eへの追随
%     (based on platex.dtx 2018/04/06 v1.0v)}
% \changes{v1.0w-u02}{2018/04/08}{安全のためフォーマット作成時の
%    バナー表示をやめた
%     (based on platex.dtx 2018/04/08 v1.0w)}
% \changes{v1.0x-u02}{2018/09/03}{ドキュメントを更新
%     (based on platex.dtx 2018/09/03 v1.0x)}
% \changes{v1.0y-u02}{2018/09/22}{最終更新日を\file{upldoc.pdf}に表示
%     (based on platex.dtx 2018/09/22 v1.0y)}
% \changes{v1.0y-u03}{2019/05/22}{ドキュメントを更新}
% \changes{v1.1b-u03}{2020/09/28}{defs読込後にフック追加}
% \changes{v1.1c-u03}{2021/02/25}{\file{latex.ltx}の読込チェック}
%\else
% \changes{v1.0c-u00}{2011/05/07}{Created \upLaTeX\ version based on \pLaTeX\ one
%     (based on platex.dtx 1997/01/29 v1.0c)}
% \changes{v1.0e-u00}{2016/04/06}{Sync with \pLaTeX.
%     (based on platex.dtx 2016/02/16 v1.0e)}
% \changes{v1.0h-u00}{2016/05/08}{Exclude \file{uplpatch.ltx} from the document
%     (based on platex.dtx 2016/05/08 v1.0h)}
% \changes{v1.0k-u00}{2016/05/21}{Sync with \pLaTeX.
%     (based on platex.dtx 2016/05/21 v1.0k)}
% \changes{v1.0k-u01}{2016/06/06}{Update documents for \upLaTeX.}
% \changes{v1.0l-u01}{2016/06/19}{Get the patch level from \file{uplvers.dtx}
%     (based on platex.dtx 2016/06/19 v1.0l)}
% \changes{v1.0m-u01}{2016/08/26}{Moved loading \file{uplatex.cfg}
%    from \file{uplcore.ltx} to \file{uplatex.ltx}
%     (based on platex.dtx 2016/08/26 v1.0m)}
% \changes{v1.0n-u01}{2016/09/14}{Sync with \pLaTeX.
%     (based on platex.dtx 2016/09/14 v1.0n)}
% \changes{v1.0p-u01}{2017/11/11}{Sync with \pLaTeX.
%     (based on platex.dtx 2017/11/11 v1.0p)}
% \changes{v1.0q-u01}{2017/11/29}{New English documentation added!
%     (based on platex.dtx 2017/11/29 v1.0q)}
% \changes{v1.0r-u01}{2017/12/02}{\upLaTeX\ and \upTeX\ references added}
% \changes{v1.0s-u01}{2017/12/05}{Moved loading default settings
%    from \file{uplcore.ltx} to \file{uplatex.ltx}
%     (based on platex.dtx 2017/12/05 v1.0s)}
% \changes{v1.0s-u02}{2017/12/10}{Load \file{plcore.ltx} before
%    \file{uplcore.ltx} (recent version of \pLaTeX\ is assumed)}
% \changes{v1.0u-u02}{2018/02/18}{Sync with \pLaTeX.
%     (based on platex.dtx 2018/02/18 v1.0u)}
% \changes{v1.0v-u02}{2018/04/06}{Sync with the latest \file{source2e.tex}
%     (based on platex.dtx 2018/04/06 v1.0v)}
% \changes{v1.0w-u02}{2018/04/08}{Stop showing banner during
%    format generation for safety
%     (based on platex.dtx 2018/04/08 v1.0w)}
% \changes{v1.0x-u02}{2018/09/03}{Update document.
%     (based on platex.dtx 2018/09/03 v1.0x)}
% \changes{v1.0y-u02}{2018/09/22}{Show last update info on \file{upldoc.pdf}
%     (based on platex.dtx 2018/09/22 v1.0y)}
% \changes{v1.0y-u03}{2019/05/22}{Update document.}
% \changes{v1.1b-u03}{2020/09/28}{Add hook after loading defs}
% \changes{v1.1c-u03}{2021/02/25}{Check for \file{latex.ltx} status}
%\fi
% \fi
%
% \iffalse
%<*driver>
\NeedsTeXFormat{pLaTeX2e}
% \fi
\ProvidesFile{uplatex.dtx}[2021/02/25 v1.1c-u03 upLaTeX document file]
% \iffalse
\documentclass{jltxdoc}
\usepackage{plext}
\GetFileInfo{uplatex.dtx}
\ifJAPANESE
\title{\upLaTeXe{}について}
\author{中野 賢 \& 日本語\TeX{}開発コミュニティ \& TTK}
\date{作成日:\filedate}
\renewcommand{\refname}{参考文献}
\GlossaryPrologue{\section*{変更履歴}%
                  \markboth{変更履歴}{変更履歴}%
                  \addcontentsline{toc}{section}{変更履歴}}
\else
\title{About \upLaTeXe{}}
\author{Ken Nakano \& Japanese \TeX\ Development Community \& TTK}
\date{Date: \filedate}
\renewcommand{\refname}{References}
\GlossaryPrologue{\section*{Change History}%
                  \markboth{Change History}{Change History}%
                  \addcontentsline{toc}{section}{Change History}}
\fi
\makeatletter
\ifJAPANESE
\def\levelchar{>・}
\fi
\def\changes@#1#2#3{%
  \let\protect\@unexpandable@protect
  \edef\@tempa{\noexpand\glossary{#2\space#1\levelchar
    \ifx\saved@macroname\@empty
%     \space\actualchar\generalname: %% comment out (uplatex.dtx only)
    \else
      \expandafter\@gobble
      \saved@macroname\actualchar
      \string\verb\quotechar*%
      \verbatimchar\saved@macroname
      \verbatimchar:
    \fi
    #3}}%
  \@tempa\endgroup\@esphack}
\makeatother
\RecordChanges
\begin{document}
   \MakeShortVerb{\+}
   \maketitle
   \DocInput{\filename}
   \StopEventually{\end{document}}
   \clearpage
   % Make TeX shut up.
   \hbadness=10000
   \newcount\hbadness
   \hfuzz=\maxdimen
   \PrintChanges
   \let\PrintChanges\relax
\end{document}
%</driver>
% \fi
%
%
%\ifJAPANESE
% \changes{v1.0c-u00}{2011/05/07}{\pLaTeX{}用から\upLaTeX{}用に修正。
%     (based on platex.dtx 1997/01/29 v1.0c)}
% \changes{v1.0k-u01}{2016/06/06}{\upLaTeX{}用にドキュメントを全体的に改訂}
% \changes{v1.0q-u01}{2017/11/29}{英語版ドキュメントを追加
%     (based on platex.dtx 2017/11/29 v1.0q)}
% \changes{v1.0x-u02}{2018/09/03}{ドキュメントを更新
%     (based on platex.dtx 2018/09/03 v1.0x)}
% \changes{v1.0y-u03}{2019/05/22}{ドキュメントを更新}
%\else
% \changes{v1.0c-u00}{2011/05/07}{Created \upLaTeX\ version based on \pLaTeX\ one
%     (based on platex.dtx 1997/01/29 v1.0c)}
% \changes{v1.0k-u01}{2016/06/06}{Update documents for \upLaTeX.}
% \changes{v1.0q-u01}{2017/11/29}{New English documentation added!
%     (based on platex.dtx 2017/11/29 v1.0q)}
% \changes{v1.0x-u02}{2018/09/03}{Update document.
%     (based on platex.dtx 2018/09/03 v1.0x)}
% \changes{v1.0y-u03}{2019/05/22}{Update document.}
%\fi
%\ifJAPANESE
% \upLaTeX{}は、内部コードをUnicode化した\pLaTeX{}の拡張版です。
% このバージョンは、「コミュニティ版\pLaTeXe{}」をベースにしています。
%\else
% \upLaTeX\ is a Unicode version of Japanese \pLaTeXe.
% This version is based on `\pLaTeXe\ Community Edition.'
%\fi
%
%\ifJAPANESE
% \pTeX{}は、高品質の日本語組版ソフトウェアとしてデファクト
% スタンダードの地位にあるといえます。しかし、\pTeX{}には
% \begin{itemize}
% \item 直接使える文字集合が原則的にJIS X 0208(JIS第1,2水準)の範囲に限定
%   されていること、
% \item 8bitの非英語欧文との親和性が高いとは言えないこと、
% \item \pTeX{}の利用が日本語に限られ、中国語・韓国語との混植への利用が
%   進んでいないこと
% \end{itemize}
% といった弱点がありました。
%
% これらの弱点を克服するため、\pTeX{}の内部コードをUnicode化した拡張版
% が\upTeX{}です。また、\upTeX{}上で用いるUnicode版\pLaTeX{}が\upLaTeX{}で
% す\footnote{\texttt{http://www.t-lab.opal.ne.jp/tex/uptex.html}}。
% 現在の\upLaTeX{}は、日本語\TeX{}開発コミュニティが配布しているコミュニティ
% 版\pLaTeX{}\footnote{\texttt{https://github.com/texjporg/platex}}を
% ベースにしており、\eupTeX{}というエンジン(\upTeX{}の\epTeX{}拡張版)で
% 動作します。
%
% 開発中の版は\pLaTeX{}と同様に、GitHubの
% リポジトリ\footnote{\texttt{https://github.com/texjporg/uplatex}}で
% 管理しています。\upLaTeX{}はアスキーとは無関係ですので、
% バグレポートはアスキー宛てではなく、日本語\TeX{}開発コミュニティに報告
% してください。\TeX\ ForumやGitHubのIssueシステムが利用できます。
%\else
% \pTeX\ is the most popular \TeX\ engine in Japan and is widely
% used for a high-quality typesetting, even for commercial printing.
% However, \pTeX\ has some limitations:
% \begin{itemize}
% \item The character set available is limited to JIS X 0208,
%   namely JIS level-1 and level-2
% \item Difficulty in handling 8-bit Latin, due to conflict with
%   legacy multibyte Japanese encodings
% \item Difficulty in typesetting CJK (Chinese, Japanese and Korean)
%  multilingual documents
% \end{itemize}
%
% To overcome these weak points,
% a Unicode extension of \pTeX, \upTeX, has been
% developed.\footnote{\texttt{http://www.t-lab.opal.ne.jp/tex/uptex.html}}
% The Unicode \pLaTeX\ format run on \upTeX\ is called \upLaTeX.
% Current \upLaTeX\ is maintained by Japanese \TeX\ Development
% Community,\footnote{\texttt{https://texjp.org}}
% in sync with \pLaTeX\ community
% edition.\footnote{\texttt{https://github.com/texjporg/platex}}
% It runs on \eupTeX, an engine with both \upTeX\ and \epTeX\ features.
%
% The development version is available from
% GitHub repository\footnote{\texttt{https://github.com/texjporg/uplatex}}.
% Any bug reports and requests should be sent to
% Japanese \TeX\ Development Community, using GitHub Issue system.
%\fi
%
%
% \clearpage
%
%\ifJAPANESE
% \section{この文書について}\label{platex:intro}
% この文書は\upLaTeXe{}の概要を示していますが、使い方のガイドでは
% ありません。ほとんどの機能は元となっている\pLaTeXe{}や\LaTeXe{}と
% 同等ですので、それぞれの付属文書などを参照してください。
%
% \upTeX{}については公式ウェブサイトあるいは\cite{tb108tanaka}(英語)を
% 参照してください。
%\else
% \section{Introduction to this document}\label{platex:intro}
% This document briefly describes \upLaTeXe, but is not a manual of \upLaTeXe.
% The basic functions of \upLaTeXe\ are almost the same with those of
% \pLaTeXe\ and \LaTeXe, so please refer to the documentation of those formats.
%
% For \upTeX, please refer to the official website or
% \cite{tb108tanaka} (in English).
%\fi
%
%\ifJAPANESE
% この文書の構成は次のようになっています。
%
% \begin{quote}
% \begin{description}
% \item[第\ref{platex:intro}節]
%  この節です。この文書についての概要を述べています。
%
% \item[第\ref{platex:plcore}節]
%  \upLaTeXe{}で拡張した機能についての概要です。
%  付属のクラスファイルやパッケージファイルについても簡単に
%  説明しています。
%
% \item[第\ref{platex:compatibility}節]
%  現在のバージョンの\upLaTeX{}と旧バージョン、あるいは元となっている
% \pLaTeX{}/\LaTeX{}との互換性について述べています。
%
% \item[付録\ref{app:dst}]
%  この文書ソース(uplatex.dtx)の
%  \dst{}のためのオプションについて述べています。
%
% \item[付録\ref{app:pldoc}]
%  \upLaTeXe{}のdtxファイルをまとめて、一つのソースコード説明書に
%  するための文書ファイルの説明をしています。
%
% \item[付録\ref{app:omake}]
%  付録\ref{app:pldoc}で説明した文書ファイルを処理するshスクリプト(手順)
%  などについて説明しています。
% \end{description}
% \end{quote}
%\else
% This document consists of following parts:
%
% \begin{quote}
% \begin{description}
% \item[Section \ref{platex:intro}]
%  This section; describes this document itself.
%
% \item[Section \ref{platex:plcore}]
%  Brief explanation of extensions in \upLaTeXe.
%  Also describes the standard classes and packages.
%
% \item[Section \ref{platex:compatibility}]
%  The compatibility note for users of the old version of
%  \upLaTeXe\ or those of the original \pLaTeXe/\LaTeXe.
%
% \item[Appendix \ref{app:dst}]
%  Describes \dst\ Options for this document.
%
% \item[Appendix \ref{app:pldoc}]
%  Description of `upldoc.tex' (counterpart for `source2e.tex' in \LaTeXe).
%
% \item[Appendix \ref{app:omake}]
%  Description of a shell script to process `upldoc.tex', etc.
% \end{description}
% \end{quote}
%\fi
%
%
%\ifJAPANESE
% \section{\upLaTeXe{}の機能について}\label{platex:plcore}
% \upLaTeXe{}が提供するファイルは、次の3種類に分類することができます。
% この構成は\pLaTeXe{}と同様です。
%
% \begin{itemize}
% \item フォーマットファイル
% \item クラスファイル
% \item パッケージファイル
% \end{itemize}
%\else
% \section{About Functions of \pLaTeXe}\label{platex:plcore}
% The structure of \upLaTeXe\ is similar to that of \pLaTeXe;
% it consists of 3 types of files: a format (uplatex.ltx),
% classes and packages.
%\fi
%
%\ifJAPANESE
% \subsection{フォーマットファイル}
% \upLaTeX{}のフォーマットファイルを作成するには、
% ソースファイル``uplatex.ltx''を\eupTeX{}のINIモードで処理します
% \footnote{2016年以前は\upTeX{}と\eupTeX{}のどちらでもフォーマットを作成する
% ことができましたが、2017年に\LaTeX{}が\eTeX{}必須となったことに伴い、
% \upLaTeX{}も\eupTeX{}が必須となりました。}。
% ただし、\TeX\ LiveやW32\TeX{}ではこの処理を簡単にする|fmtutil-sys|あるいは
% |fmtutil|というプログラムが用意されています。
% 以下を実行すれば、フォーマットファイル\file{uplatex.fmt}が作成されます。
%\else
% \subsection{About the Format}
% To make a format for \upLaTeX,
% process ``uplatex.ltx'' with INI mode of \eupTeX.\footnote{Formerly
% both \upTeX\ and \eupTeX\ can make the format file for \upLaTeX, however,
% it's not true anymore because \LaTeX\ requires \eTeX\ since 2017.}
% A handy command `fmtutil-sys' (or `fmtutil') for this purpose
% is available in \TeX\ Live. The following command generates \file{uplatex.fmt}.
%\fi
%\begin{verbatim}
%   fmtutil-sys --byfmt uplatex
%\end{verbatim}
%
%\ifJAPANESE
% 次のリストが、\file{uplatex.ltx}の内容です。
% ただし、このバージョンでは、\LaTeX{}から\upLaTeX{}への拡張を
% \file{plcore.ltx}(\pLaTeX{}によって提供される)および
% \file{uplcore.ltx}をロードすることで行ない、
% \file{latex.ltx}には直接、手を加えないようにしています。
% したがって\file{uplatex.ltx}はとても短いものとなっています。
% \file{latex.ltx}には\LaTeX{}のコマンドが、
% \file{uplcore.ltx}には\upLaTeX{}で拡張したコマンドが定義されています。
%\else
% The content of \file{uplatex.ltx} is shown below.
% In the current version of \upLaTeX,
% first we simply load \file{latex.ltx} and
% modify/extend some definitions by loading
% \file{plcore.ltx} (available from \pLaTeX) and \file{uplcore.ltx}.
%\fi
%    \begin{macrocode}
%<*plcore>
%    \end{macrocode}
%
%\ifJAPANESE
% \file{latex.ltx}の末尾で使われている|\dump|をいったん無効化します。
%\else
% Temporarily disable |\dump| at the end of \file{latex.ltx}.
%\fi
%    \begin{macrocode}
\let\orgdump\dump
\let\dump\relax
%    \end{macrocode}
%
%\ifJAPANESE
% \file{latex.ltx}を読み込みます。
% \TeX\ Liveの標準的インストールでは、この中でBabel由来の
% ハイフネーション・パターン\file{hyphen.cfg}が読み込まれるはずです。
%\else
% Load \file{latex.ltx} here.
% Within the standard installation of \TeX\ Live, \file{hyphen.cfg}
% provided by ``Babel'' package will be used.
%\fi
%    \begin{macrocode}
\input latex.ltx
%    \end{macrocode}
%
%\ifJAPANESE
% この時点で|\typeout|が未定義なら、\LaTeX{}カーネルの読み込みに
% 失敗していますので、強制終了します(\LaTeXe\ 2017/01/01以降を
% 非\eTeX{}拡張でフォーマット作成しようとした場合など)。
% \changes{v1.1c-u03}{2021/02/25}{\file{latex.ltx}の読込チェック}
%\else
% If |\typeout| is still undefined, the input of \LaTeX~kernel
% should have failed; abort now.
% \changes{v1.1c-u03}{2021/02/25}{Check for \file{latex.ltx} status}
%\fi
%    \begin{macrocode}
\ifx\typeout\undefined
  \errhelp{Please reinstall LaTeX, or check e-TeX availability.}%
  \errmessage{Failed to load `latex.ltx' properly}%
  \expandafter\end
\fi
%    \end{macrocode}
%
%\ifJAPANESE
% \file{plcore.ltx}と\file{uplcore.ltx}を読み込みます。
% \changes{v1.0s-u02}{2017/12/10}{\file{uplcore.ltx}の前に
%    \file{plcore.ltx}を読み込むようにした(最近の\pLaTeX{}が前提)}
%\else
% Load \file{plcore.ltx} and \file{uplcore.ltx}.
% \changes{v1.0s-u02}{2017/12/10}{Load \file{plcore.ltx} before
%    \file{uplcore.ltx} (recent version of \pLaTeX\ is assumed)}
%\fi
%    \begin{macrocode}
\typeout{**************************^^J%
         *^^J%
         * making upLaTeX format^^J%
         *^^J%
         **************************}
\makeatletter
\input plcore.ltx
\input uplcore.ltx
%    \end{macrocode}
%
%\ifJAPANESE
% フォント関連のデフォルト設定ファイルである、
% \file{upldefs.ltx}を読み込みます。
% \TeX{}の入力ファイル検索パスに設定されている
% ディレクトリに\file{upldefs.cfg}ファイルがある場合は、
% そのファイルを使います。
% 読み込み後にコードが実行されるかもしれません。
% \changes{v1.0s-u01}{2017/12/05}{デフォルト設定ファイルの読み込みを
%    \file{uplcore.ltx}から\file{uplatex.ltx}へ移動
%     (based on platex.dtx 2017/12/05 v1.0s)}
% \changes{v1.1b-u03}{2020/09/28}{defs読込後にフック追加}
%\else
% Load font-related default settings, \file{upldefs.ltx}.
% If a file \file{upldefs.cfg} is found, then that file will be
% used instead.
% Some code may be executed after loading.
% \changes{v1.0s-u01}{2017/12/05}{Moved loading default settings
%    from \file{uplcore.ltx} to \file{uplatex.ltx}
%     (based on platex.dtx 2017/12/05 v1.0s)}
% \changes{v1.1b-u03}{2020/09/28}{Add hook after loading defs}
%\fi
%    \begin{macrocode}
\InputIfFileExists{upldefs.cfg}
           {\typeout{*************************************^^J%
                     * Local config file upldefs.cfg used^^J%
                     *************************************}}%
           {\input{upldefs.ltx}}
\ifx\code@after@pldefs\@undefined\else \code@after@pldefs \fi
%    \end{macrocode}
%
%\ifJAPANESE
% 以前のバージョンでは、フォーマット作成時に\upLaTeX{}のバージョンが
% わかるように、端末に表示していましたが、|\everyjob| にバナー表示
% 以外のコードが含まれる可能性を考慮し、安全のためやめました。
% \changes{v1.0w-u02}{2018/04/08}{安全のためフォーマット作成時の
%    バナー表示をやめた
%     (based on platex.dtx 2018/04/08 v1.0w)}
%\else
% In the previous version, we displayed \upLaTeX\ version
% on the terminal, so that it can be easily recognized
% during format creation; however |\everyjob| can contain
% any code other than showing a banner, so now disabled.
% \changes{v1.0w-u02}{2018/04/08}{Stop showing banner during
%    format generation for safety
%     (based on platex.dtx 2018/04/08 v1.0w)}
%\fi
%    \begin{macrocode}
%\the\everyjob
%    \end{macrocode}
%
%\ifJAPANESE
% \upLaTeXe{}の起動時に\file{uplatex.cfg}がある場合、それを読み込む
% ようにします(\pLaTeXe{}が\file{platex.cfg}を読み込むのと同様)。
% バージョン2016/07/01ではコードを\file{uplcore.ltx}に入れていました
% が、\file{uplatex.ltx}へ移動しました。
% \changes{v1.0m-u01}{2016/08/26}{\file{uplatex.cfg}の読み込みを
%    \file{uplcore.ltx}から\file{uplatex.ltx}へ移動
%     (based on platex.dtx 2016/08/26 v1.0m)}
%\else
% Load \file{uplatex.cfg} if it exists at runtime of \upLaTeXe.
% (Counterpart of \file{platex.cfg} in \pLaTeXe.)
% \changes{v1.0m-u01}{2016/08/26}{Moved loading \file{uplatex.cfg}
%    from \file{uplcore.ltx} to \file{uplatex.ltx}
%     (based on platex.dtx 2016/08/26 v1.0m)}
%\fi
%    \begin{macrocode}
\everyjob\expandafter{%
  \the\everyjob
  \IfFileExists{uplatex.cfg}{%
    \typeout{*************************^^J%
             * Loading uplatex.cfg.^^J%
             *************************}%
    % \iffalse meta-comment
%% File: uplatex.dtx
%
%    pLaTeX base file:
%       Copyright 1995,1996 ASCII Corporation.
%    and modified for upLaTeX
%
%  Copyright (c) 2010 ASCII MEDIA WORKS
%  Copyright (c) 2016 Takuji Tanaka
%  Copyright (c) 2016-2021 Japanese TeX Development Community
%
%  This file is part of the upLaTeX2e system (community edition).
%  --------------------------------------------------------------
%
% \fi
%
% \iffalse
%<*driver|pldoc>
\ifx\JAPANESEtrue\undefined
  \expandafter\newif\csname ifJAPANESE\endcsname
  \JAPANESEtrue
\fi
%</driver|pldoc>
% \fi
%
% \setcounter{StandardModuleDepth}{1}
% \makeatletter
%\ifJAPANESE
% \def\chuui{\@ifnextchar[{\@chuui}{\@chuui[注意:]}}
%\else
% \def\chuui{\@ifnextchar[{\@chuui}{\@chuui[Attention: ]}}
%\fi
% \def\@chuui[#1]{\par\vskip.5\baselineskip
%   \noindent{\em #1}\par\bgroup\gtfamily\sffamily}
% \def\endchuui{\egroup\vskip.5\baselineskip}
% \makeatother
%
% \iffalse
%<*driver|pldoc>
\def\eTeX{$\varepsilon$-\TeX}
\def\pTeX{p\kern-.15em\TeX}
\def\epTeX{$\varepsilon$-\pTeX}
\def\pLaTeX{p\kern-.05em\LaTeX}
\def\pLaTeXe{p\kern-.05em\LaTeXe}
\def\upTeX{u\pTeX}
\def\eupTeX{$\varepsilon$-\upTeX}
\def\upLaTeX{u\pLaTeX}
\def\upLaTeXe{u\pLaTeXe}
%</driver|pldoc>
% \fi
%
% \StopEventually{}
%
% \iffalse
%\ifJAPANESE
% \changes{v1.0c-u00}{2011/05/07}{\pLaTeX{}用から\upLaTeX{}用に修正。
%     (based on platex.dtx 1997/01/29 v1.0c)}
% \changes{v1.0e-u00}{2016/04/06}{\pLaTeX{}の変更に追随。
%     (based on platex.dtx 2016/02/16 v1.0e)}
% \changes{v1.0h-u00}{2016/05/08}{ドキュメントから\file{uplpatch.ltx}を除外
%     (based on platex.dtx 2016/05/08 v1.0h)}
% \changes{v1.0k-u00}{2016/05/21}{\pLaTeX{}の変更に追随。
%     (based on platex.dtx 2016/05/21 v1.0k)}
% \changes{v1.0k-u01}{2016/06/06}{\upLaTeX{}用にドキュメントを全体的に改訂}
% \changes{v1.0l-u01}{2016/06/19}{パッチレベルを\file{uplvers.dtx}から取得
%     (based on platex.dtx 2016/06/19 v1.0l)}
% \changes{v1.0m-u01}{2016/08/26}{\file{uplatex.cfg}の読み込みを
%    \file{uplcore.ltx}から\file{uplatex.ltx}へ移動
%     (based on platex.dtx 2016/08/26 v1.0m)}
% \changes{v1.0n-u01}{2016/09/14}{\pLaTeX{}の変更に追随。
%     (based on platex.dtx 2016/09/14 v1.0n)}
% \changes{v1.0p-u01}{2017/11/11}{\pLaTeX{}の変更に追随。
%     (based on platex.dtx 2017/11/11 v1.0p)}
% \changes{v1.0q-u01}{2017/11/29}{英語版ドキュメントを追加
%     (based on platex.dtx 2017/11/29 v1.0q)}
% \changes{v1.0r-u01}{2017/12/02}{\upLaTeX{}と\upTeX{}の参考文献も追加}
% \changes{v1.0s-u01}{2017/12/05}{デフォルト設定ファイルの読み込みを
%    \file{uplcore.ltx}から\file{uplatex.ltx}へ移動
%     (based on platex.dtx 2017/12/05 v1.0s)}
% \changes{v1.0s-u02}{2017/12/10}{\file{uplcore.ltx}の前に
%    \file{plcore.ltx}を読み込むようにした(最近の\pLaTeX{}が前提)}
% \changes{v1.0u-u02}{2018/02/18}{\pLaTeX{}の変更に追随。
%     (based on platex.dtx 2018/02/18 v1.0u)}
% \changes{v1.0v-u02}{2018/04/06}{最新のsource2eへの追随
%     (based on platex.dtx 2018/04/06 v1.0v)}
% \changes{v1.0w-u02}{2018/04/08}{安全のためフォーマット作成時の
%    バナー表示をやめた
%     (based on platex.dtx 2018/04/08 v1.0w)}
% \changes{v1.0x-u02}{2018/09/03}{ドキュメントを更新
%     (based on platex.dtx 2018/09/03 v1.0x)}
% \changes{v1.0y-u02}{2018/09/22}{最終更新日を\file{upldoc.pdf}に表示
%     (based on platex.dtx 2018/09/22 v1.0y)}
% \changes{v1.0y-u03}{2019/05/22}{ドキュメントを更新}
% \changes{v1.1b-u03}{2020/09/28}{defs読込後にフック追加}
% \changes{v1.1c-u03}{2021/02/25}{\file{latex.ltx}の読込チェック}
%\else
% \changes{v1.0c-u00}{2011/05/07}{Created \upLaTeX\ version based on \pLaTeX\ one
%     (based on platex.dtx 1997/01/29 v1.0c)}
% \changes{v1.0e-u00}{2016/04/06}{Sync with \pLaTeX.
%     (based on platex.dtx 2016/02/16 v1.0e)}
% \changes{v1.0h-u00}{2016/05/08}{Exclude \file{uplpatch.ltx} from the document
%     (based on platex.dtx 2016/05/08 v1.0h)}
% \changes{v1.0k-u00}{2016/05/21}{Sync with \pLaTeX.
%     (based on platex.dtx 2016/05/21 v1.0k)}
% \changes{v1.0k-u01}{2016/06/06}{Update documents for \upLaTeX.}
% \changes{v1.0l-u01}{2016/06/19}{Get the patch level from \file{uplvers.dtx}
%     (based on platex.dtx 2016/06/19 v1.0l)}
% \changes{v1.0m-u01}{2016/08/26}{Moved loading \file{uplatex.cfg}
%    from \file{uplcore.ltx} to \file{uplatex.ltx}
%     (based on platex.dtx 2016/08/26 v1.0m)}
% \changes{v1.0n-u01}{2016/09/14}{Sync with \pLaTeX.
%     (based on platex.dtx 2016/09/14 v1.0n)}
% \changes{v1.0p-u01}{2017/11/11}{Sync with \pLaTeX.
%     (based on platex.dtx 2017/11/11 v1.0p)}
% \changes{v1.0q-u01}{2017/11/29}{New English documentation added!
%     (based on platex.dtx 2017/11/29 v1.0q)}
% \changes{v1.0r-u01}{2017/12/02}{\upLaTeX\ and \upTeX\ references added}
% \changes{v1.0s-u01}{2017/12/05}{Moved loading default settings
%    from \file{uplcore.ltx} to \file{uplatex.ltx}
%     (based on platex.dtx 2017/12/05 v1.0s)}
% \changes{v1.0s-u02}{2017/12/10}{Load \file{plcore.ltx} before
%    \file{uplcore.ltx} (recent version of \pLaTeX\ is assumed)}
% \changes{v1.0u-u02}{2018/02/18}{Sync with \pLaTeX.
%     (based on platex.dtx 2018/02/18 v1.0u)}
% \changes{v1.0v-u02}{2018/04/06}{Sync with the latest \file{source2e.tex}
%     (based on platex.dtx 2018/04/06 v1.0v)}
% \changes{v1.0w-u02}{2018/04/08}{Stop showing banner during
%    format generation for safety
%     (based on platex.dtx 2018/04/08 v1.0w)}
% \changes{v1.0x-u02}{2018/09/03}{Update document.
%     (based on platex.dtx 2018/09/03 v1.0x)}
% \changes{v1.0y-u02}{2018/09/22}{Show last update info on \file{upldoc.pdf}
%     (based on platex.dtx 2018/09/22 v1.0y)}
% \changes{v1.0y-u03}{2019/05/22}{Update document.}
% \changes{v1.1b-u03}{2020/09/28}{Add hook after loading defs}
% \changes{v1.1c-u03}{2021/02/25}{Check for \file{latex.ltx} status}
%\fi
% \fi
%
% \iffalse
%<*driver>
\NeedsTeXFormat{pLaTeX2e}
% \fi
\ProvidesFile{uplatex.dtx}[2021/02/25 v1.1c-u03 upLaTeX document file]
% \iffalse
\documentclass{jltxdoc}
\usepackage{plext}
\GetFileInfo{uplatex.dtx}
\ifJAPANESE
\title{\upLaTeXe{}について}
\author{中野 賢 \& 日本語\TeX{}開発コミュニティ \& TTK}
\date{作成日:\filedate}
\renewcommand{\refname}{参考文献}
\GlossaryPrologue{\section*{変更履歴}%
                  \markboth{変更履歴}{変更履歴}%
                  \addcontentsline{toc}{section}{変更履歴}}
\else
\title{About \upLaTeXe{}}
\author{Ken Nakano \& Japanese \TeX\ Development Community \& TTK}
\date{Date: \filedate}
\renewcommand{\refname}{References}
\GlossaryPrologue{\section*{Change History}%
                  \markboth{Change History}{Change History}%
                  \addcontentsline{toc}{section}{Change History}}
\fi
\makeatletter
\ifJAPANESE
\def\levelchar{>・}
\fi
\def\changes@#1#2#3{%
  \let\protect\@unexpandable@protect
  \edef\@tempa{\noexpand\glossary{#2\space#1\levelchar
    \ifx\saved@macroname\@empty
%     \space\actualchar\generalname: %% comment out (uplatex.dtx only)
    \else
      \expandafter\@gobble
      \saved@macroname\actualchar
      \string\verb\quotechar*%
      \verbatimchar\saved@macroname
      \verbatimchar:
    \fi
    #3}}%
  \@tempa\endgroup\@esphack}
\makeatother
\RecordChanges
\begin{document}
   \MakeShortVerb{\+}
   \maketitle
   \DocInput{\filename}
   \StopEventually{\end{document}}
   \clearpage
   % Make TeX shut up.
   \hbadness=10000
   \newcount\hbadness
   \hfuzz=\maxdimen
   \PrintChanges
   \let\PrintChanges\relax
\end{document}
%</driver>
% \fi
%
%
%\ifJAPANESE
% \changes{v1.0c-u00}{2011/05/07}{\pLaTeX{}用から\upLaTeX{}用に修正。
%     (based on platex.dtx 1997/01/29 v1.0c)}
% \changes{v1.0k-u01}{2016/06/06}{\upLaTeX{}用にドキュメントを全体的に改訂}
% \changes{v1.0q-u01}{2017/11/29}{英語版ドキュメントを追加
%     (based on platex.dtx 2017/11/29 v1.0q)}
% \changes{v1.0x-u02}{2018/09/03}{ドキュメントを更新
%     (based on platex.dtx 2018/09/03 v1.0x)}
% \changes{v1.0y-u03}{2019/05/22}{ドキュメントを更新}
%\else
% \changes{v1.0c-u00}{2011/05/07}{Created \upLaTeX\ version based on \pLaTeX\ one
%     (based on platex.dtx 1997/01/29 v1.0c)}
% \changes{v1.0k-u01}{2016/06/06}{Update documents for \upLaTeX.}
% \changes{v1.0q-u01}{2017/11/29}{New English documentation added!
%     (based on platex.dtx 2017/11/29 v1.0q)}
% \changes{v1.0x-u02}{2018/09/03}{Update document.
%     (based on platex.dtx 2018/09/03 v1.0x)}
% \changes{v1.0y-u03}{2019/05/22}{Update document.}
%\fi
%\ifJAPANESE
% \upLaTeX{}は、内部コードをUnicode化した\pLaTeX{}の拡張版です。
% このバージョンは、「コミュニティ版\pLaTeXe{}」をベースにしています。
%\else
% \upLaTeX\ is a Unicode version of Japanese \pLaTeXe.
% This version is based on `\pLaTeXe\ Community Edition.'
%\fi
%
%\ifJAPANESE
% \pTeX{}は、高品質の日本語組版ソフトウェアとしてデファクト
% スタンダードの地位にあるといえます。しかし、\pTeX{}には
% \begin{itemize}
% \item 直接使える文字集合が原則的にJIS X 0208(JIS第1,2水準)の範囲に限定
%   されていること、
% \item 8bitの非英語欧文との親和性が高いとは言えないこと、
% \item \pTeX{}の利用が日本語に限られ、中国語・韓国語との混植への利用が
%   進んでいないこと
% \end{itemize}
% といった弱点がありました。
%
% これらの弱点を克服するため、\pTeX{}の内部コードをUnicode化した拡張版
% が\upTeX{}です。また、\upTeX{}上で用いるUnicode版\pLaTeX{}が\upLaTeX{}で
% す\footnote{\texttt{http://www.t-lab.opal.ne.jp/tex/uptex.html}}。
% 現在の\upLaTeX{}は、日本語\TeX{}開発コミュニティが配布しているコミュニティ
% 版\pLaTeX{}\footnote{\texttt{https://github.com/texjporg/platex}}を
% ベースにしており、\eupTeX{}というエンジン(\upTeX{}の\epTeX{}拡張版)で
% 動作します。
%
% 開発中の版は\pLaTeX{}と同様に、GitHubの
% リポジトリ\footnote{\texttt{https://github.com/texjporg/uplatex}}で
% 管理しています。\upLaTeX{}はアスキーとは無関係ですので、
% バグレポートはアスキー宛てではなく、日本語\TeX{}開発コミュニティに報告
% してください。\TeX\ ForumやGitHubのIssueシステムが利用できます。
%\else
% \pTeX\ is the most popular \TeX\ engine in Japan and is widely
% used for a high-quality typesetting, even for commercial printing.
% However, \pTeX\ has some limitations:
% \begin{itemize}
% \item The character set available is limited to JIS X 0208,
%   namely JIS level-1 and level-2
% \item Difficulty in handling 8-bit Latin, due to conflict with
%   legacy multibyte Japanese encodings
% \item Difficulty in typesetting CJK (Chinese, Japanese and Korean)
%  multilingual documents
% \end{itemize}
%
% To overcome these weak points,
% a Unicode extension of \pTeX, \upTeX, has been
% developed.\footnote{\texttt{http://www.t-lab.opal.ne.jp/tex/uptex.html}}
% The Unicode \pLaTeX\ format run on \upTeX\ is called \upLaTeX.
% Current \upLaTeX\ is maintained by Japanese \TeX\ Development
% Community,\footnote{\texttt{https://texjp.org}}
% in sync with \pLaTeX\ community
% edition.\footnote{\texttt{https://github.com/texjporg/platex}}
% It runs on \eupTeX, an engine with both \upTeX\ and \epTeX\ features.
%
% The development version is available from
% GitHub repository\footnote{\texttt{https://github.com/texjporg/uplatex}}.
% Any bug reports and requests should be sent to
% Japanese \TeX\ Development Community, using GitHub Issue system.
%\fi
%
%
% \clearpage
%
%\ifJAPANESE
% \section{この文書について}\label{platex:intro}
% この文書は\upLaTeXe{}の概要を示していますが、使い方のガイドでは
% ありません。ほとんどの機能は元となっている\pLaTeXe{}や\LaTeXe{}と
% 同等ですので、それぞれの付属文書などを参照してください。
%
% \upTeX{}については公式ウェブサイトあるいは\cite{tb108tanaka}(英語)を
% 参照してください。
%\else
% \section{Introduction to this document}\label{platex:intro}
% This document briefly describes \upLaTeXe, but is not a manual of \upLaTeXe.
% The basic functions of \upLaTeXe\ are almost the same with those of
% \pLaTeXe\ and \LaTeXe, so please refer to the documentation of those formats.
%
% For \upTeX, please refer to the official website or
% \cite{tb108tanaka} (in English).
%\fi
%
%\ifJAPANESE
% この文書の構成は次のようになっています。
%
% \begin{quote}
% \begin{description}
% \item[第\ref{platex:intro}節]
%  この節です。この文書についての概要を述べています。
%
% \item[第\ref{platex:plcore}節]
%  \upLaTeXe{}で拡張した機能についての概要です。
%  付属のクラスファイルやパッケージファイルについても簡単に
%  説明しています。
%
% \item[第\ref{platex:compatibility}節]
%  現在のバージョンの\upLaTeX{}と旧バージョン、あるいは元となっている
% \pLaTeX{}/\LaTeX{}との互換性について述べています。
%
% \item[付録\ref{app:dst}]
%  この文書ソース(uplatex.dtx)の
%  \dst{}のためのオプションについて述べています。
%
% \item[付録\ref{app:pldoc}]
%  \upLaTeXe{}のdtxファイルをまとめて、一つのソースコード説明書に
%  するための文書ファイルの説明をしています。
%
% \item[付録\ref{app:omake}]
%  付録\ref{app:pldoc}で説明した文書ファイルを処理するshスクリプト(手順)
%  などについて説明しています。
% \end{description}
% \end{quote}
%\else
% This document consists of following parts:
%
% \begin{quote}
% \begin{description}
% \item[Section \ref{platex:intro}]
%  This section; describes this document itself.
%
% \item[Section \ref{platex:plcore}]
%  Brief explanation of extensions in \upLaTeXe.
%  Also describes the standard classes and packages.
%
% \item[Section \ref{platex:compatibility}]
%  The compatibility note for users of the old version of
%  \upLaTeXe\ or those of the original \pLaTeXe/\LaTeXe.
%
% \item[Appendix \ref{app:dst}]
%  Describes \dst\ Options for this document.
%
% \item[Appendix \ref{app:pldoc}]
%  Description of `upldoc.tex' (counterpart for `source2e.tex' in \LaTeXe).
%
% \item[Appendix \ref{app:omake}]
%  Description of a shell script to process `upldoc.tex', etc.
% \end{description}
% \end{quote}
%\fi
%
%
%\ifJAPANESE
% \section{\upLaTeXe{}の機能について}\label{platex:plcore}
% \upLaTeXe{}が提供するファイルは、次の3種類に分類することができます。
% この構成は\pLaTeXe{}と同様です。
%
% \begin{itemize}
% \item フォーマットファイル
% \item クラスファイル
% \item パッケージファイル
% \end{itemize}
%\else
% \section{About Functions of \pLaTeXe}\label{platex:plcore}
% The structure of \upLaTeXe\ is similar to that of \pLaTeXe;
% it consists of 3 types of files: a format (uplatex.ltx),
% classes and packages.
%\fi
%
%\ifJAPANESE
% \subsection{フォーマットファイル}
% \upLaTeX{}のフォーマットファイルを作成するには、
% ソースファイル``uplatex.ltx''を\eupTeX{}のINIモードで処理します
% \footnote{2016年以前は\upTeX{}と\eupTeX{}のどちらでもフォーマットを作成する
% ことができましたが、2017年に\LaTeX{}が\eTeX{}必須となったことに伴い、
% \upLaTeX{}も\eupTeX{}が必須となりました。}。
% ただし、\TeX\ LiveやW32\TeX{}ではこの処理を簡単にする|fmtutil-sys|あるいは
% |fmtutil|というプログラムが用意されています。
% 以下を実行すれば、フォーマットファイル\file{uplatex.fmt}が作成されます。
%\else
% \subsection{About the Format}
% To make a format for \upLaTeX,
% process ``uplatex.ltx'' with INI mode of \eupTeX.\footnote{Formerly
% both \upTeX\ and \eupTeX\ can make the format file for \upLaTeX, however,
% it's not true anymore because \LaTeX\ requires \eTeX\ since 2017.}
% A handy command `fmtutil-sys' (or `fmtutil') for this purpose
% is available in \TeX\ Live. The following command generates \file{uplatex.fmt}.
%\fi
%\begin{verbatim}
%   fmtutil-sys --byfmt uplatex
%\end{verbatim}
%
%\ifJAPANESE
% 次のリストが、\file{uplatex.ltx}の内容です。
% ただし、このバージョンでは、\LaTeX{}から\upLaTeX{}への拡張を
% \file{plcore.ltx}(\pLaTeX{}によって提供される)および
% \file{uplcore.ltx}をロードすることで行ない、
% \file{latex.ltx}には直接、手を加えないようにしています。
% したがって\file{uplatex.ltx}はとても短いものとなっています。
% \file{latex.ltx}には\LaTeX{}のコマンドが、
% \file{uplcore.ltx}には\upLaTeX{}で拡張したコマンドが定義されています。
%\else
% The content of \file{uplatex.ltx} is shown below.
% In the current version of \upLaTeX,
% first we simply load \file{latex.ltx} and
% modify/extend some definitions by loading
% \file{plcore.ltx} (available from \pLaTeX) and \file{uplcore.ltx}.
%\fi
%    \begin{macrocode}
%<*plcore>
%    \end{macrocode}
%
%\ifJAPANESE
% \file{latex.ltx}の末尾で使われている|\dump|をいったん無効化します。
%\else
% Temporarily disable |\dump| at the end of \file{latex.ltx}.
%\fi
%    \begin{macrocode}
\let\orgdump\dump
\let\dump\relax
%    \end{macrocode}
%
%\ifJAPANESE
% \file{latex.ltx}を読み込みます。
% \TeX\ Liveの標準的インストールでは、この中でBabel由来の
% ハイフネーション・パターン\file{hyphen.cfg}が読み込まれるはずです。
%\else
% Load \file{latex.ltx} here.
% Within the standard installation of \TeX\ Live, \file{hyphen.cfg}
% provided by ``Babel'' package will be used.
%\fi
%    \begin{macrocode}
\input latex.ltx
%    \end{macrocode}
%
%\ifJAPANESE
% この時点で|\typeout|が未定義なら、\LaTeX{}カーネルの読み込みに
% 失敗していますので、強制終了します(\LaTeXe\ 2017/01/01以降を
% 非\eTeX{}拡張でフォーマット作成しようとした場合など)。
% \changes{v1.1c-u03}{2021/02/25}{\file{latex.ltx}の読込チェック}
%\else
% If |\typeout| is still undefined, the input of \LaTeX~kernel
% should have failed; abort now.
% \changes{v1.1c-u03}{2021/02/25}{Check for \file{latex.ltx} status}
%\fi
%    \begin{macrocode}
\ifx\typeout\undefined
  \errhelp{Please reinstall LaTeX, or check e-TeX availability.}%
  \errmessage{Failed to load `latex.ltx' properly}%
  \expandafter\end
\fi
%    \end{macrocode}
%
%\ifJAPANESE
% \file{plcore.ltx}と\file{uplcore.ltx}を読み込みます。
% \changes{v1.0s-u02}{2017/12/10}{\file{uplcore.ltx}の前に
%    \file{plcore.ltx}を読み込むようにした(最近の\pLaTeX{}が前提)}
%\else
% Load \file{plcore.ltx} and \file{uplcore.ltx}.
% \changes{v1.0s-u02}{2017/12/10}{Load \file{plcore.ltx} before
%    \file{uplcore.ltx} (recent version of \pLaTeX\ is assumed)}
%\fi
%    \begin{macrocode}
\typeout{**************************^^J%
         *^^J%
         * making upLaTeX format^^J%
         *^^J%
         **************************}
\makeatletter
\input plcore.ltx
\input uplcore.ltx
%    \end{macrocode}
%
%\ifJAPANESE
% フォント関連のデフォルト設定ファイルである、
% \file{upldefs.ltx}を読み込みます。
% \TeX{}の入力ファイル検索パスに設定されている
% ディレクトリに\file{upldefs.cfg}ファイルがある場合は、
% そのファイルを使います。
% 読み込み後にコードが実行されるかもしれません。
% \changes{v1.0s-u01}{2017/12/05}{デフォルト設定ファイルの読み込みを
%    \file{uplcore.ltx}から\file{uplatex.ltx}へ移動
%     (based on platex.dtx 2017/12/05 v1.0s)}
% \changes{v1.1b-u03}{2020/09/28}{defs読込後にフック追加}
%\else
% Load font-related default settings, \file{upldefs.ltx}.
% If a file \file{upldefs.cfg} is found, then that file will be
% used instead.
% Some code may be executed after loading.
% \changes{v1.0s-u01}{2017/12/05}{Moved loading default settings
%    from \file{uplcore.ltx} to \file{uplatex.ltx}
%     (based on platex.dtx 2017/12/05 v1.0s)}
% \changes{v1.1b-u03}{2020/09/28}{Add hook after loading defs}
%\fi
%    \begin{macrocode}
\InputIfFileExists{upldefs.cfg}
           {\typeout{*************************************^^J%
                     * Local config file upldefs.cfg used^^J%
                     *************************************}}%
           {\input{upldefs.ltx}}
\ifx\code@after@pldefs\@undefined\else \code@after@pldefs \fi
%    \end{macrocode}
%
%\ifJAPANESE
% 以前のバージョンでは、フォーマット作成時に\upLaTeX{}のバージョンが
% わかるように、端末に表示していましたが、|\everyjob| にバナー表示
% 以外のコードが含まれる可能性を考慮し、安全のためやめました。
% \changes{v1.0w-u02}{2018/04/08}{安全のためフォーマット作成時の
%    バナー表示をやめた
%     (based on platex.dtx 2018/04/08 v1.0w)}
%\else
% In the previous version, we displayed \upLaTeX\ version
% on the terminal, so that it can be easily recognized
% during format creation; however |\everyjob| can contain
% any code other than showing a banner, so now disabled.
% \changes{v1.0w-u02}{2018/04/08}{Stop showing banner during
%    format generation for safety
%     (based on platex.dtx 2018/04/08 v1.0w)}
%\fi
%    \begin{macrocode}
%\the\everyjob
%    \end{macrocode}
%
%\ifJAPANESE
% \upLaTeXe{}の起動時に\file{uplatex.cfg}がある場合、それを読み込む
% ようにします(\pLaTeXe{}が\file{platex.cfg}を読み込むのと同様)。
% バージョン2016/07/01ではコードを\file{uplcore.ltx}に入れていました
% が、\file{uplatex.ltx}へ移動しました。
% \changes{v1.0m-u01}{2016/08/26}{\file{uplatex.cfg}の読み込みを
%    \file{uplcore.ltx}から\file{uplatex.ltx}へ移動
%     (based on platex.dtx 2016/08/26 v1.0m)}
%\else
% Load \file{uplatex.cfg} if it exists at runtime of \upLaTeXe.
% (Counterpart of \file{platex.cfg} in \pLaTeXe.)
% \changes{v1.0m-u01}{2016/08/26}{Moved loading \file{uplatex.cfg}
%    from \file{uplcore.ltx} to \file{uplatex.ltx}
%     (based on platex.dtx 2016/08/26 v1.0m)}
%\fi
%    \begin{macrocode}
\everyjob\expandafter{%
  \the\everyjob
  \IfFileExists{uplatex.cfg}{%
    \typeout{*************************^^J%
             * Loading uplatex.cfg.^^J%
             *************************}%
    \input{uplatex.cfg}}{}%
}
%    \end{macrocode}
%
%\ifJAPANESE
% フォーマットファイルにダンプします。
%\else
% Dump to the format file.
%\fi
%    \begin{macrocode}
\let\dump\orgdump
\let\orgdump\@undefined
\makeatother
\dump
%\endinput
%    \end{macrocode}
%
%    \begin{macrocode}
%</plcore>
%    \end{macrocode}
%
%\ifJAPANESE
% 実際に\upLaTeXe{}への拡張を行なっている\file{uplcore.ltx}は、
% \dst{}プログラムによって、次のファイルの断片が連結されたものです。
%
% \begin{itemize}
% \item \file{uplvers.dtx}は、\upLaTeXe{}のフォーマットバージョンを
%   定義しています。
% \end{itemize}
%
% また、プリロードフォントや組版パラメータなどのデフォルト設定は、
% \file{uplatex.ltx}の中で\file{upldefs.ltx}をロードすることにより行います
% \footnote{旧版では\file{uplcore.ltx}の中でロードしていましたが、
% 2018年以降の新しいコミュニティ版\upLaTeX{}では
% \file{uplatex.ltx}から読み込むことにしました。}。
% このファイル\file{upldefs.ltx}も\file{uplfonts.dtx}から生成されます。
% \begin{chuui}
% このファイルに記述されている設定を変更すれば
% \upLaTeXe{}をカスタマイズすることができますが、
% その場合は\file{upldefs.ltx}を直接修正するのではなく、いったん
% \file{upldefs.cfg}という名前でコピーして、そのファイルを編集してください。
% フォーマット作成時に\file{upldefs.cfg}が存在した場合は、そちらが
% \file{upldefs.ltx}の代わりに読み込まれます。
% \end{chuui}
%\else
% The file \file{uplcore.ltx}, which provides modifications/extensions
% to make \upLaTeXe, is a concatenation of stripped files below
% using \dst\ program.
%
% \begin{itemize}
% \item \file{uplvers.dtx} defines the format version of \upLaTeXe.
% \item \file{uplfonts.dtx} extends \NFSS2 for Japanese font selection.
% \item \file{plcore.dtx} (the same content as \pLaTeXe); defines other
%   modifications to \LaTeXe.
% \end{itemize}
%
% Moreover, default settings of pre-loaded fonts and typesetting parameters
% are done by loading \file{upldefs.ltx} inside
% \file{uplatex.ltx}.\footnote{Older \upLaTeX\ loaded \file{upldefs.ltx}
% inside \file{uplcore.ltx}; however, \upLaTeX\ community edition newer than
% 2018 loads \file{upldefs.ltx} inside \file{uplatex.ltx}.}
% This file \file{upldefs.ltx} is also stripped from \file{uplfonts.dtx}.
% \begin{chuui}
% You can customize \upLaTeXe\ by tuning these settings.
% If you need to do that, copy/rename it as \file{upldefs.cfg} and edit it,
% instead of overwriting \file{upldefs.ltx} itself.
% If a file named \file{upldefs.cfg} is found at a format creation
% time, it will be read as a substitute of \file{upldefs.ltx}.
% \end{chuui}
%\fi
%
%\ifJAPANESE
% ここまで見てきたように、\upLaTeX{}の各ファイルはそれぞれ\pLaTeX{}での
% 対応するファイル名の頭に``u''を付けた名前になっています。
%\else
% As shown above, the files in \upLaTeX\ is named after \pLaTeX\ ones,
% prefixed with ``u.''
%\fi
%
%
%\ifJAPANESE
% \subsubsection{バージョン}
% \upLaTeXe{}のバージョンやフォーマットファイル名は、
% \file{uplvers.dtx}で定義しています。これは、\pLaTeXe{}のバージョンや
% フォーマットファイル名が\file{plvers.dtx}で定義されているのと同じです。
%\else
% \subsubsection{Version}
% The version (like ``\pfmtversion'') and the format name
% (``\pfmtname'') of \upLaTeXe\ are defined in \file{uplvers.dtx}.
% This is similar to \pLaTeXe, which defines those in \file{plvers.dtx}.
%\fi
%
%
%\ifJAPANESE
% \subsubsection{\NFSS2コマンド}
% \upLaTeXe{}は\pLaTeXe{}と共通の\file{plcore.ltx}を使用していますので、
% \NFSS2の和文フォント選択への拡張が有効になっています。
%\else
% \subsubsection{\NFSS2 Commands}
% \upLaTeXe\ shares \file{plcore.dtx} with \pLaTeXe, so
% the extensions of \NFSS2 for selecting Japanese fonts are available.
%\fi
%
%
%\ifJAPANESE
% \subsubsection{出力ルーチンとフロート}
% \upLaTeXe{}は\pLaTeXe{}と共通の\file{plcore.ltx}を使用していますので、
% 出力ルーチンや脚注マクロなどは\pLaTeXe{}と同じように動作します。
%\else
% \subsubsection{Output Routine and Floats}
% \upLaTeXe\ shares \file{plcore.dtx} with \pLaTeXe, so
% the output routine and footnote macros will behave similar to \pLaTeXe.
%\fi
%
%
%\ifJAPANESE
% \subsection{クラスファイルとパッケージファイル}
%
% \upLaTeXe{}が提供をするクラスファイルやパッケージファイルは、
% \pLaTeXe{}に含まれるファイルを基にしています。
%
% \upLaTeXe{}に付属のクラスファイルは、次のとおりです。
%
% \begin{itemize}
% \item ujarticle.cls, ujbook.cls, ujreport.cls\par
%   横組用の標準クラスファイル。
%   \file{ujclasses.dtx}から作成される。
%   それぞれjarticle.cls, jbook.cls, jreport.clsの\upLaTeX{}版。
%
% \item utarticle.cls, utbook.cls, utreport.cls\par
%   縦組用の標準クラスファイル。
%   \file{ujclasses.dtx}から作成される。
%   それぞれtarticle.cls, tbook.cls, treport.clsの\upLaTeX{}版。
% \end{itemize}
%
% なおjltxdoc.clsの\upLaTeX{}版はありませんが、これは\pLaTeX{}のものが
% \upLaTeX{}でもそのまま使えます。
%\else
% \subsection{Classes and Packages}
%
% Classes and packages bundled with \upLaTeXe\ are based on
% those in original \pLaTeXe, and modified some parameters.
%
% \upLaTeXe\ classes:
%
% \begin{itemize}
% \item ujarticle.cls, ujbook.cls, ujreport.cls\par
%   Standard \emph{yoko-kumi} (horizontal writing) classes;
%   stripped from \file{ujclasses.dtx}.
%   \upLaTeX\ edition of jarticle.cls, jbook.cls and jreport.cls.
%
% \item utarticle.cls, utbook.cls, utreport.cls\par
%   Standard \emph{tate-kumi} (vertical writing) classes;
%   stripped from \file{ujclasses.dtx}.
%   \upLaTeX\ edition of tarticle.cls, tbook.cls and treport.cls.
% \end{itemize}
%
% We don't provide \upLaTeX\ edition of jltxdoc.cls, but the one
% from \pLaTeX\ can be used also on \upLaTeX\ without problem.
%\fi
%
%\ifJAPANESE
% また、\upLaTeXe{}に付属のパッケージファイルは、次のとおりです。
%
% \begin{itemize}
% \item uptrace.sty\par
%   ptrace.styの\upLaTeX{}版。
%   \LaTeX{}でフォント選択コマンドのトレースに使う\file{tracefnt.sty}が
%   再定義してしまう\NFSS2コマンドを、\upLaTeXe{}用に再々定義するための
%   パッケージ。
%   \file{uplfonts.dtx}から作成される。
% \end{itemize}
%
% 他の\pLaTeX{}のパッケージは、\upLaTeX{}でもそのまま動作します。
%\else
% \upLaTeXe\ packages:
%
% \begin{itemize}
% \item uptrace.sty\par
%   \upLaTeXe\ version of \file{tracefnt.sty};
%   the package \file{tracefnt.sty} overwrites \upLaTeXe-style \NFSS2
%   commands, so \file{uptrace.sty} provides redefinitions to recover
%   \upLaTeXe\ extensions. Stripped from \file{uplfonts.dtx}.
% \end{itemize}
%
% Other \pLaTeX\ packages work also on \upLaTeX.
%\fi
%
%
%\ifJAPANESE
% \section{他のフォーマット・旧バージョンとの互換性}
% \label{platex:compatibility}
% ここでは、この\upLaTeXe{}のバージョンと以前のバージョン、あるいは
% \pLaTeXe{}/\LaTeXe{}との互換性について説明をしています。
%
% \subsection{\pLaTeXe{}および\LaTeXe{}との互換性}
% \upLaTeXe{}は、\pLaTeXe{}の上位互換という形を取っていますので、
% クラスファイルやいくつかのコマンドを置き換えるだけで、
% たいていの\pLaTeXe{}文書を簡単に\upLaTeXe{}文書に変更することができます。
% ただし、\upLaTeXe{}のデフォルトの日本語フォントメトリックは\pLaTeXe{}の
% それと異なりますので、レイアウトが変化することがあります。
% また、\LaTeXe{}のいくつかの命令の定義も変更していますので、
% \LaTeXe{}で処理できるファイルを\upLaTeXe{}で処理した場合に
% 完全に同じ結果になるとは限りません。
%
% また、\upLaTeXe{}は新しいマクロパッケージですので、2.09互換モードを
% サポートしていません。\LaTeXe{}の仕様に従ってドキュメントを作成して
% ください。
%
% \pLaTeXe{}向けあるいは\LaTeXe{}向けに作られた多くのクラスファイルや
% パッケージファイルはそのまま使えると思います。
% ただし、例えばクラスファイルが\pLaTeX{}標準の
% 漢字エンコーディング(JY1, JT1)を前提としている場合は、
% \upLaTeX{}で採用した漢字エンコーディング(JY2, JT2)と合致せずに
% エラーが発生してしまいます。この場合は、そのクラスファイルが
% \upLaTeX{}に対応していないことになります。このような場合は、
% \pLaTeX{}を使い続けるか、その作者に連絡して\upLaTeX{}に対応して
% もらうなどの対応をとってください。
%\else
% \section{Compatibility with Other Formats and Older Versions}
% \label{platex:compatibility}
% Here we provide some information about the compatibility between
% current \upLaTeXe\ and older versions or original \pLaTeXe/\LaTeXe.
%
% \subsection{Compatibility with \pLaTeXe/\LaTeXe}
% \upLaTeXe\ is in most part upward compatible with \pLaTeXe,
% so you can move from \pLaTeXe\ to \upLaTeXe\ by simply replacing
% the document class and some macros. However, the default Japanese
% font metrics in \upLaTeXe\ is different from those in \pLaTeXe;
% therefore, you should not expect identical output from both
% \pLaTeXe\ and \upLaTeXe.
%
% Note that \upLaTeX\ is a new format, so we do \emph{not} provide support
% for 2.09 compatibility mode. Follow the standard \LaTeXe\ convention!
%
% We hope that most classes and packages meant for \LaTeXe/\pLaTeXe\ works
% also for \upLaTeXe\ without any modification. However for example,
% if a class or a package uses Kanji encoding `JY1' or `JT1' (default on
% \pLaTeXe), an error complaining the mismatch of Kanji encoding might
% happen on \upLaTeX, in which the default is `JY2' and `JT2.'
% In this case, we have to say that the class or package does not support
% \upLaTeXe; you should use \pLaTeX, or report to the author of the
% package or class.
%\fi
%
%\ifJAPANESE
% \subsection{latexreleaseパッケージへの対応}
% \LaTeX\ \texttt{<2015/01/01>}で導入されたlatexreleaseパッケージを
% もとに、新しい\pLaTeX{}ではplatexreleaseパッケージが用意されました。
% 本来は\upLaTeX{}でも同様のパッケージを用意するのがよいのですが、
% 現在は\pLaTeX{}から\upLaTeX{}への変更点が含まれていませんので、
% 幸いplatexreleaseパッケージをそのまま用いることができます。
% このため、\upLaTeX{}で独自のパッケージを用意することはしていません。
% platexreleaseパッケージを用いると、過去の\upLaTeX{}をエミュレート
% したり、フォーマットを作り直すことなく新しい\upLaTeX{}を試したりする
% ことができます。詳細はplatexreleaseのドキュメントを参照してください。
%\else
% \subsection{Support for Package `latexrelease'}
% \pLaTeX\ provides `platexrelease' package, which is based on
% `latexrelease' package (introduced in \LaTeX\ \texttt{<2015/01/01>}).
% It could be better if we also provide a similar package on \upLaTeX,
% but currently we don't need it; \upLaTeX\ does not have any recent
% \upLaTeX-specific changes. So, you can safely use `platexrelease'
% package for emulating the specified format date.
%\fi
%
%
%
% \appendix
%
%\ifJAPANESE
% \section{\dst{}プログラムのためのオプション}\label{app:dst}
% この文書のソース(\file{uplatex.dtx})を\dst{}プログラムで
% 処理することによって、
% いくつかの異なるファイルを生成することができます。
% \dst{}プログラムの詳細は、\file{docstrip.dtx}を参照してください。
%
% この文書の\dst{}プログラムのためのオプションは、次のとおりです。
%
% \DeleteShortVerb{\|}
% \begin{center}
% \begin{tabular}{l|p{.8\linewidth}}
% \emph{オプション} & \emph{意味}\\\hline
% plcore & フォーマットファイルを作るためのファイルを生成\\
% pldoc  & \upLaTeXe{}のソースファイルをまとめて組版するための
%          文書ファイル(upldoc.tex)を生成\\[2mm]
% shprog & 上記のファイルを作成するためのshスクリプトを生成\\
% Xins   & 上記のshスクリプトやperlスクリプトを取り出すための
%          \dst{}バッチファイル(Xins.ins)を生成\\
% \end{tabular}
% \end{center}
% \MakeShortVerb{\|}
%\else
% \section{\dst\ Options}\label{app:dst}
% By processing \file{uplatex.dtx} with \dst\ program,
% different files can be generated.
% Here are the \dst\ options for this document:
%
% \DeleteShortVerb{\|}
% \begin{center}
% \begin{tabular}{l|p{.8\linewidth}}
% \emph{Option} & \emph{Function}\\\hline
% plcore & Generates a fragment of format sources\\
% pldoc  & Generates `upldoc.tex' for typesetting
%          \upLaTeXe\ sources\\[2mm]
% shprog & Generates a shell script to process `upldoc.tex'\\
% Xins   & Generates a \dst\ batch file `Xins.ins' for
%          generating the above shell/perl scripts\\
% \end{tabular}
% \end{center}
% \MakeShortVerb{\|}
%\fi
%
%
%\ifJAPANESE
% \section{文書ファイル}\label{app:pldoc}
% ここでは、このパッケージに含まれているdtxファイルをまとめて組版し、
% ソースコード説明書を得るための文書ファイル\file{upldoc.tex}について
% 説明をしています。個別に処理した場合と異なり、
% 変更履歴や索引も付きます。
%
% デフォルトではソースコードの説明が日本語で書かれます。
% もし英語の説明書を読みたい場合は、\par\medskip
% \begin{minipage}{.5\textwidth}\ttfamily
% | |\cs{newif}\cs{ifJAPANESE}
% \end{minipage}\par\medskip\noindent
% という内容の\file{uplatex.cfg}を予め用意してから\file{upldoc.tex}を
% 処理してください(2016年7月1日以降の\upLaTeXe{}が必要)。
%
% コードは\pLaTeXe{}のものと(ファイル名を除き)ほぼ同一なので、
% ここでは違っている部分だけ説明します。
%\else
% \section{Documentation of \upLaTeXe\ sources}\label{app:pldoc}
% The contents of `upldoc.tex' for typesetting \upLaTeXe\ sources
% is described here. Compared to individual processings,
% batch processing using `upldoc.tex' prints also changes and an index.
%
% By default, the description of \upLaTeXe\ sources is written in
% Japanese. If you need English version, first save\par\medskip
% \begin{minipage}{.5\textwidth}\ttfamily
% | |\cs{newif}\cs{ifJAPANESE}
% \end{minipage}\par\medskip\noindent
% as \file{uplatex.cfg}, and process \file{upldoc.tex}
% (\upLaTeXe\ newer than July 2016 is required).
%
% Here we explain only difference between \file{pldoc.tex} (\pLaTeXe)
% and \file{upldoc.tex} (\upLaTeXe).
%\fi
%
%    \begin{macrocode}
%<*pldoc>
\begin{filecontents}{upldoc.dic}
西暦    せいれき
和暦    われき
\end{filecontents}
%    \end{macrocode}
%\ifJAPANESE
% \pLaTeXe{}のドキュメントでは、\file{plext.dtx}の中で組み立てるサンプル
% のために\file{plext}パッケージが必要ですが、\upLaTeXe{}のファイル
% にはそのようなサンプルが含まれないので除外しています。
%\else
% The document of \pLaTeXe\ requires \file{plext} package,
% since \file{plext.dtx} contains several examples of partial
% vertical writing. However, we don't have such examples in
% \upLaTeXe\ files, so no need for it.
%\fi
%    \begin{macrocode}
\documentclass{jltxdoc}
%\usepackage{plext} %% comment out for upLaTeX
\listfiles

\DoNotIndex{\def,\long,\edef,\xdef,\gdef,\let,\global}
\DoNotIndex{\if,\ifnum,\ifdim,\ifcat,\ifmmode,\ifvmode,\ifhmode,%
            \iftrue,\iffalse,\ifvoid,\ifx,\ifeof,\ifcase,\else,\or,\fi}
\DoNotIndex{\box,\copy,\setbox,\unvbox,\unhbox,\hbox,%
            \vbox,\vtop,\vcenter}
\DoNotIndex{\@empty,\immediate,\write}
\DoNotIndex{\egroup,\bgroup,\expandafter,\begingroup,\endgroup}
\DoNotIndex{\divide,\advance,\multiply,\count,\dimen}
\DoNotIndex{\relax,\space,\string}
\DoNotIndex{\csname,\endcsname,\@spaces,\openin,\openout,%
            \closein,\closeout}
\DoNotIndex{\catcode,\endinput}
\DoNotIndex{\jobname,\message,\read,\the,\m@ne,\noexpand}
\DoNotIndex{\hsize,\vsize,\hskip,\vskip,\kern,\hfil,\hfill,\hss,\vss,\unskip}
\DoNotIndex{\m@ne,\z@,\z@skip,\@ne,\tw@,\p@,\@minus,\@plus}
\DoNotIndex{\dp,\wd,\ht,\setlength,\addtolength}
\DoNotIndex{\newcommand, \renewcommand}

\ifJAPANESE
\IndexPrologue{\part*{索 引}%
                 \markboth{索 引}{索 引}%
                 \addcontentsline{toc}{part}{索 引}%
イタリック体の数字は、その項目が説明されているページを示しています。
下線の引かれた数字は、定義されているページを示しています。
その他の数字は、その項目が使われているページを示しています。}
\else
\IndexPrologue{\part*{Index}%
                 \markboth{Index}{Index}%
                 \addcontentsline{toc}{part}{Index}%
The italic numbers denote the pages where the corresponding entry
is described, numbers underlined point to the definition,
all others indicate the places where it is used.}
\fi
%
\ifJAPANESE
\GlossaryPrologue{\part*{変更履歴}%
                 \markboth{変更履歴}{変更履歴}%
                 \addcontentsline{toc}{part}{変更履歴}}
\else
\GlossaryPrologue{\part*{Change History}%
                 \markboth{Change History}{Change History}%
                 \addcontentsline{toc}{part}{Change History}}
\fi

\makeatletter
\def\changes@#1#2#3{%
  \let\protect\@unexpandable@protect
  \edef\@tempa{\noexpand\glossary{#2\space
               \currentfile\space#1\levelchar
               \ifx\saved@macroname\@empty
                  \space\actualchar\generalname
               \else
                  \expandafter\@gobble
                  \saved@macroname\actualchar
                  \string\verb\quotechar*%
                  \verbatimchar\saved@macroname
                  \verbatimchar
               \fi
               :\levelchar #3}}%
  \@tempa\endgroup\@esphack}
\renewcommand*\MacroFont{\fontencoding\encodingdefault
                   \fontfamily\ttdefault
                   \fontseries\mddefault
                   \fontshape\updefault
                   \small
                   \hfuzz 6pt\relax}
\renewcommand*\l@subsection{\@dottedtocline{2}{1.5em}{2.8em}}
\renewcommand*\l@subsubsection{\@dottedtocline{3}{3.8em}{3.4em}}
\makeatother
\RecordChanges
\CodelineIndex
\EnableCrossrefs
\setcounter{IndexColumns}{2}
\settowidth\MacroIndent{\ttfamily\scriptsize 000\ }
%    \end{macrocode}
%\ifJAPANESE
% この文書のタイトル・著者・日付を設定します。
% \changes{v1.0h-u00}{2016/05/08}{ドキュメントから\file{uplpatch.ltx}を除外
%     (based on platex.dtx 2016/05/08 v1.0h)}
% \changes{v1.0l-u01}{2016/06/19}{パッチレベルを\file{uplvers.dtx}から取得
%     (based on platex.dtx 2016/06/19 v1.0l)}
% \changes{v1.0y-u02}{2018/09/22}{最終更新日を\file{upldoc.pdf}に表示
%     (based on platex.dtx 2018/09/22 v1.0y)}
%\else
% Set the title, authors and the date for this document.
% \changes{v1.0h-u00}{2016/05/08}{Exclude \file{uplpatch.ltx} from the document
%     (based on platex.dtx 2016/05/08 v1.0h)}
% \changes{v1.0l-u01}{2016/06/19}{Get the patch level from \file{uplvers.dtx}
%     (based on platex.dtx 2016/06/19 v1.0l)}
% \changes{v1.0y-u02}{2018/09/22}{Show last update info on \file{upldoc.pdf}
%     (based on platex.dtx 2018/09/22 v1.0y)}
%\fi
%    \begin{macrocode}
 \title{The \upLaTeXe\ Sources}
 \author{Ken Nakano \& Japanese \TeX\ Development Community \& TTK}

% Get the (temporary) date and up-patch level from uplvers.dtx
\makeatletter
\let\patchdate=\@empty
\begingroup
   \def\ProvidesFile#1[#2 #3]#4\def\uppatch@level#5{%
      \date{#2}\xdef\patchdate{#5}\endinput}
   \input{uplvers.dtx}
\endgroup

% Add the patch version if available.
\def\Xpatch{}
\ifx\patchdate\Xpatch\else
  \edef\@date{\@date\space version \patchdate}
\fi

% Obtain the last update info, as upLaTeX does not change format date
% -> if successful, reconstruct the date completely
\def\lastupd@te{0000/00/00}
\begingroup
   \def\ProvidesFile#1[#2 #3]{%
      \def\@tempd@te{#2}\endinput
      \@ifl@t@r{\@tempd@te}{\lastupd@te}{%
         \global\let\lastupd@te\@tempd@te
      }{}}
   \let\ProvidesClass\ProvidesFile
   \let\ProvidesPackage\ProvidesFile
   \input{uplvers.dtx}
   \input{uplfonts.dtx}
   \input{ukinsoku.dtx}
   \input{ujclasses.dtx}
\endgroup
\@ifl@t@r{\lastupd@te}{0000/00/00}{%
  \date{Version \patchdate\break (last updated: \lastupd@te)}%
}{}
\makeatother
%    \end{macrocode}
%\ifJAPANESE
% ここからが本文ページとなります。
%\else
% Here starts the document body.
%\fi
%    \begin{macrocode}
\begin{document}
\pagenumbering{roman}
\maketitle
\renewcommand\maketitle{}
\tableofcontents
\clearpage
\pagenumbering{arabic}

\DocInclude{uplvers}   % upLaTeX version

\DocInclude{uplfonts}  % NFSS2 commands

\DocInclude{ukinsoku}  % kinsoku parameter

\DocInclude{ujclasses} % Standard class

\StopEventually{\end{document}}

\clearpage
\pagestyle{headings}
% Make TeX shut up.
\hbadness=10000
\newcount\hbadness
\hfuzz=\maxdimen
%
\PrintChanges
\clearpage
%
\begingroup
  \def\endash{--}
  \catcode`\-\active
  \def-{\futurelet\temp\indexdash}
  \def\indexdash{\ifx\temp-\endash\fi}

  \PrintIndex
\endgroup
\let\PrintChanges\relax
\let\PrintIndex\relax
\end{document}
%</pldoc>
%    \end{macrocode}
%
%
%
%\ifJAPANESE
% \section{おまけプログラム}\label{app:omake}
%
% \subsection{シェルスクリプト\file{mkpldoc.sh}}\label{app:shprog}
% \upLaTeXe{}のマクロ定義ファイルをまとめて組版し、変更履歴と索引も
% 付けるときに便利なシェルスクリプトです。
% このシェルスクリプトの使用方法は次のとおりです。
%\begin{verbatim}
%    sh mkpldoc.sh
%\end{verbatim}
%
% コードは\pLaTeXe{}のものと(ファイル名を除き)ほぼ同一なので、
% ここでは違っている部分だけ説明します。
%\else
% \section{Additional Utility Programs}\label{app:omake}
%
% \subsection{Shell Script \file{mkpldoc.sh}}\label{app:shprog}
% A shell script to process `pldoc.tex' and produce a fully indexed
% source code description. Run |sh mkpldoc.sh| to use it.
%
% The script is almost identical to that in \pLaTeXe, so
% here we describe only the difference.
%\fi
%
%    \begin{macrocode}
%<*shprog>
%<ja>rm -f upldoc.toc upldoc.idx upldoc.glo
%<en>rm -f upldoc-en.toc upldoc-en.idx upldoc-en.glo
echo "" > ltxdoc.cfg
%<ja>uplatex upldoc.tex
%<en>uplatex -jobname=upldoc-en upldoc.tex
%    \end{macrocode}
%\ifJAPANESE
% 変更履歴や索引の生成にはmendexを用いますが、
% \upLaTeX{}の場合はUTF-8モードで実行する必要がありますので、
% |-U|というオプションを付けます\footnote{uplatexコマンドも
% 実際にはUTF-8モードで実行する必要がありますが、デフォルトの内部漢字
% コードがUTF-8に設定されているはずですので、\texttt{-kanji=utf8}を
% 付けなくても処理できると思います。}。
% makeindexコマンドには、このオプションがありません。
%\else
% To make the Change log and Glossary (Change History) for
% \upLaTeX\ using `mendex,' we need to run it in UTF-8 mode.
% So, option |-U| is important.\footnote{The command `uplatex'
% should be also in UTF-8 mode, but it defaults to UTF-8 mode;
% therefore, we don't need to add \texttt{-kanji=utf8} explicitly.}
%\fi
%    \begin{macrocode}
%<ja>mendex -U -s gind.ist -d upldoc.dic -o upldoc.ind upldoc.idx
%<en>mendex -U -s gind.ist -d upldoc.dic -o upldoc-en.ind upldoc-en.idx
%<ja>mendex -U -f -s gglo.ist -o upldoc.gls upldoc.glo
%<en>mendex -U -f -s gglo.ist -o upldoc-en.gls upldoc-en.glo
echo "\includeonly{}" > ltxdoc.cfg
%<ja>uplatex upldoc.tex
%<en>uplatex -jobname=upldoc-en upldoc.tex
echo "" > ltxdoc.cfg
%<ja>uplatex upldoc.tex
%<en>uplatex -jobname=upldoc-en upldoc.tex
# EOT
%</shprog>
%    \end{macrocode}
%
%
%\ifJAPANESE
% \subsection{perlスクリプト\file{dstcheck.pl}}\label{app:plprog}
% \pLaTeXe{}のものがそのまま使えるので、\upLaTeXe{}では省略します。
%\else
% \subsection{Perl Script \file{dstcheck.pl}}\label{app:plprog}
% The one from \pLaTeXe\ can be use without any change, so
% omitted here in \upLaTeXe.
%\fi
%
%
%\ifJAPANESE
% \subsection{\dst{}バッチファイル}
% 付録\ref{app:shprog}で説明をしたスクリプトを、このファイルから
% 取り出すための\dst{}バッチファイルです。コードは\pLaTeXe{}の
% ものと(ファイル名を除き)ほぼ同一なので、説明は割愛します。
%\else
% \subsection{\dst{} Batch file}
% Here we introduce a \dst\ batch file `Xins.ins,' which generates the
% script described in Appendix \ref{app:shprog}.
% The code is almost identical to that in \pLaTeXe.
%\fi
%
%    \begin{macrocode}
%<*Xins>
\input docstrip
\keepsilent
%    \end{macrocode}
%
%    \begin{macrocode}
{\catcode`#=12 \gdef\MetaPrefix{## }}
%    \end{macrocode}
%
%    \begin{macrocode}
\declarepreamble\thispre
\endpreamble
\usepreamble\thispre
%    \end{macrocode}
%
%    \begin{macrocode}
\declarepostamble\thispost
\endpostamble
\usepostamble\thispost
%    \end{macrocode}
%
%    \begin{macrocode}
\generate{
   \file{mkpldoc.sh}{\from{uplatex.dtx}{shprog,ja}}
   \file{mkpldoc-en.sh}{\from{uplatex.dtx}{shprog,en}}
}
\endbatchfile
%</Xins>
%    \end{macrocode}
%
% \newpage
% \begin{thebibliography}{9}
% \bibitem{tb108tanaka}
% Takuji Tanaka,
% \newblock Up\TeX\ --- Unicode version of \pTeX\ with CJK extensions.
% \newblock TUGboat issue 34:3, 2013.\\
%   (\texttt{http://tug.org/TUGboat/tb34-3/tb108tanaka.pdf})
% \end{thebibliography}
%
% \iffalse
% ここで、このあとに組版されるかもしれない文書のために、
% 節見出しの番号を算用数字に戻します。
% \fi
%
% \renewcommand{\thesection}{\arabic{section}}
%
% \Finale
%
\endinput
}{}%
}
%    \end{macrocode}
%
%\ifJAPANESE
% フォーマットファイルにダンプします。
%\else
% Dump to the format file.
%\fi
%    \begin{macrocode}
\let\dump\orgdump
\let\orgdump\@undefined
\makeatother
\dump
%\endinput
%    \end{macrocode}
%
%    \begin{macrocode}
%</plcore>
%    \end{macrocode}
%
%\ifJAPANESE
% 実際に\upLaTeXe{}への拡張を行なっている\file{uplcore.ltx}は、
% \dst{}プログラムによって、次のファイルの断片が連結されたものです。
%
% \begin{itemize}
% \item \file{uplvers.dtx}は、\upLaTeXe{}のフォーマットバージョンを
%   定義しています。
% \end{itemize}
%
% また、プリロードフォントや組版パラメータなどのデフォルト設定は、
% \file{uplatex.ltx}の中で\file{upldefs.ltx}をロードすることにより行います
% \footnote{旧版では\file{uplcore.ltx}の中でロードしていましたが、
% 2018年以降の新しいコミュニティ版\upLaTeX{}では
% \file{uplatex.ltx}から読み込むことにしました。}。
% このファイル\file{upldefs.ltx}も\file{uplfonts.dtx}から生成されます。
% \begin{chuui}
% このファイルに記述されている設定を変更すれば
% \upLaTeXe{}をカスタマイズすることができますが、
% その場合は\file{upldefs.ltx}を直接修正するのではなく、いったん
% \file{upldefs.cfg}という名前でコピーして、そのファイルを編集してください。
% フォーマット作成時に\file{upldefs.cfg}が存在した場合は、そちらが
% \file{upldefs.ltx}の代わりに読み込まれます。
% \end{chuui}
%\else
% The file \file{uplcore.ltx}, which provides modifications/extensions
% to make \upLaTeXe, is a concatenation of stripped files below
% using \dst\ program.
%
% \begin{itemize}
% \item \file{uplvers.dtx} defines the format version of \upLaTeXe.
% \item \file{uplfonts.dtx} extends \NFSS2 for Japanese font selection.
% \item \file{plcore.dtx} (the same content as \pLaTeXe); defines other
%   modifications to \LaTeXe.
% \end{itemize}
%
% Moreover, default settings of pre-loaded fonts and typesetting parameters
% are done by loading \file{upldefs.ltx} inside
% \file{uplatex.ltx}.\footnote{Older \upLaTeX\ loaded \file{upldefs.ltx}
% inside \file{uplcore.ltx}; however, \upLaTeX\ community edition newer than
% 2018 loads \file{upldefs.ltx} inside \file{uplatex.ltx}.}
% This file \file{upldefs.ltx} is also stripped from \file{uplfonts.dtx}.
% \begin{chuui}
% You can customize \upLaTeXe\ by tuning these settings.
% If you need to do that, copy/rename it as \file{upldefs.cfg} and edit it,
% instead of overwriting \file{upldefs.ltx} itself.
% If a file named \file{upldefs.cfg} is found at a format creation
% time, it will be read as a substitute of \file{upldefs.ltx}.
% \end{chuui}
%\fi
%
%\ifJAPANESE
% ここまで見てきたように、\upLaTeX{}の各ファイルはそれぞれ\pLaTeX{}での
% 対応するファイル名の頭に``u''を付けた名前になっています。
%\else
% As shown above, the files in \upLaTeX\ is named after \pLaTeX\ ones,
% prefixed with ``u.''
%\fi
%
%
%\ifJAPANESE
% \subsubsection{バージョン}
% \upLaTeXe{}のバージョンやフォーマットファイル名は、
% \file{uplvers.dtx}で定義しています。これは、\pLaTeXe{}のバージョンや
% フォーマットファイル名が\file{plvers.dtx}で定義されているのと同じです。
%\else
% \subsubsection{Version}
% The version (like ``\pfmtversion'') and the format name
% (``\pfmtname'') of \upLaTeXe\ are defined in \file{uplvers.dtx}.
% This is similar to \pLaTeXe, which defines those in \file{plvers.dtx}.
%\fi
%
%
%\ifJAPANESE
% \subsubsection{\NFSS2コマンド}
% \upLaTeXe{}は\pLaTeXe{}と共通の\file{plcore.ltx}を使用していますので、
% \NFSS2の和文フォント選択への拡張が有効になっています。
%\else
% \subsubsection{\NFSS2 Commands}
% \upLaTeXe\ shares \file{plcore.dtx} with \pLaTeXe, so
% the extensions of \NFSS2 for selecting Japanese fonts are available.
%\fi
%
%
%\ifJAPANESE
% \subsubsection{出力ルーチンとフロート}
% \upLaTeXe{}は\pLaTeXe{}と共通の\file{plcore.ltx}を使用していますので、
% 出力ルーチンや脚注マクロなどは\pLaTeXe{}と同じように動作します。
%\else
% \subsubsection{Output Routine and Floats}
% \upLaTeXe\ shares \file{plcore.dtx} with \pLaTeXe, so
% the output routine and footnote macros will behave similar to \pLaTeXe.
%\fi
%
%
%\ifJAPANESE
% \subsection{クラスファイルとパッケージファイル}
%
% \upLaTeXe{}が提供をするクラスファイルやパッケージファイルは、
% \pLaTeXe{}に含まれるファイルを基にしています。
%
% \upLaTeXe{}に付属のクラスファイルは、次のとおりです。
%
% \begin{itemize}
% \item ujarticle.cls, ujbook.cls, ujreport.cls\par
%   横組用の標準クラスファイル。
%   \file{ujclasses.dtx}から作成される。
%   それぞれjarticle.cls, jbook.cls, jreport.clsの\upLaTeX{}版。
%
% \item utarticle.cls, utbook.cls, utreport.cls\par
%   縦組用の標準クラスファイル。
%   \file{ujclasses.dtx}から作成される。
%   それぞれtarticle.cls, tbook.cls, treport.clsの\upLaTeX{}版。
% \end{itemize}
%
% なおjltxdoc.clsの\upLaTeX{}版はありませんが、これは\pLaTeX{}のものが
% \upLaTeX{}でもそのまま使えます。
%\else
% \subsection{Classes and Packages}
%
% Classes and packages bundled with \upLaTeXe\ are based on
% those in original \pLaTeXe, and modified some parameters.
%
% \upLaTeXe\ classes:
%
% \begin{itemize}
% \item ujarticle.cls, ujbook.cls, ujreport.cls\par
%   Standard \emph{yoko-kumi} (horizontal writing) classes;
%   stripped from \file{ujclasses.dtx}.
%   \upLaTeX\ edition of jarticle.cls, jbook.cls and jreport.cls.
%
% \item utarticle.cls, utbook.cls, utreport.cls\par
%   Standard \emph{tate-kumi} (vertical writing) classes;
%   stripped from \file{ujclasses.dtx}.
%   \upLaTeX\ edition of tarticle.cls, tbook.cls and treport.cls.
% \end{itemize}
%
% We don't provide \upLaTeX\ edition of jltxdoc.cls, but the one
% from \pLaTeX\ can be used also on \upLaTeX\ without problem.
%\fi
%
%\ifJAPANESE
% また、\upLaTeXe{}に付属のパッケージファイルは、次のとおりです。
%
% \begin{itemize}
% \item uptrace.sty\par
%   ptrace.styの\upLaTeX{}版。
%   \LaTeX{}でフォント選択コマンドのトレースに使う\file{tracefnt.sty}が
%   再定義してしまう\NFSS2コマンドを、\upLaTeXe{}用に再々定義するための
%   パッケージ。
%   \file{uplfonts.dtx}から作成される。
% \end{itemize}
%
% 他の\pLaTeX{}のパッケージは、\upLaTeX{}でもそのまま動作します。
%\else
% \upLaTeXe\ packages:
%
% \begin{itemize}
% \item uptrace.sty\par
%   \upLaTeXe\ version of \file{tracefnt.sty};
%   the package \file{tracefnt.sty} overwrites \upLaTeXe-style \NFSS2
%   commands, so \file{uptrace.sty} provides redefinitions to recover
%   \upLaTeXe\ extensions. Stripped from \file{uplfonts.dtx}.
% \end{itemize}
%
% Other \pLaTeX\ packages work also on \upLaTeX.
%\fi
%
%
%\ifJAPANESE
% \section{他のフォーマット・旧バージョンとの互換性}
% \label{platex:compatibility}
% ここでは、この\upLaTeXe{}のバージョンと以前のバージョン、あるいは
% \pLaTeXe{}/\LaTeXe{}との互換性について説明をしています。
%
% \subsection{\pLaTeXe{}および\LaTeXe{}との互換性}
% \upLaTeXe{}は、\pLaTeXe{}の上位互換という形を取っていますので、
% クラスファイルやいくつかのコマンドを置き換えるだけで、
% たいていの\pLaTeXe{}文書を簡単に\upLaTeXe{}文書に変更することができます。
% ただし、\upLaTeXe{}のデフォルトの日本語フォントメトリックは\pLaTeXe{}の
% それと異なりますので、レイアウトが変化することがあります。
% また、\LaTeXe{}のいくつかの命令の定義も変更していますので、
% \LaTeXe{}で処理できるファイルを\upLaTeXe{}で処理した場合に
% 完全に同じ結果になるとは限りません。
%
% また、\upLaTeXe{}は新しいマクロパッケージですので、2.09互換モードを
% サポートしていません。\LaTeXe{}の仕様に従ってドキュメントを作成して
% ください。
%
% \pLaTeXe{}向けあるいは\LaTeXe{}向けに作られた多くのクラスファイルや
% パッケージファイルはそのまま使えると思います。
% ただし、例えばクラスファイルが\pLaTeX{}標準の
% 漢字エンコーディング(JY1, JT1)を前提としている場合は、
% \upLaTeX{}で採用した漢字エンコーディング(JY2, JT2)と合致せずに
% エラーが発生してしまいます。この場合は、そのクラスファイルが
% \upLaTeX{}に対応していないことになります。このような場合は、
% \pLaTeX{}を使い続けるか、その作者に連絡して\upLaTeX{}に対応して
% もらうなどの対応をとってください。
%\else
% \section{Compatibility with Other Formats and Older Versions}
% \label{platex:compatibility}
% Here we provide some information about the compatibility between
% current \upLaTeXe\ and older versions or original \pLaTeXe/\LaTeXe.
%
% \subsection{Compatibility with \pLaTeXe/\LaTeXe}
% \upLaTeXe\ is in most part upward compatible with \pLaTeXe,
% so you can move from \pLaTeXe\ to \upLaTeXe\ by simply replacing
% the document class and some macros. However, the default Japanese
% font metrics in \upLaTeXe\ is different from those in \pLaTeXe;
% therefore, you should not expect identical output from both
% \pLaTeXe\ and \upLaTeXe.
%
% Note that \upLaTeX\ is a new format, so we do \emph{not} provide support
% for 2.09 compatibility mode. Follow the standard \LaTeXe\ convention!
%
% We hope that most classes and packages meant for \LaTeXe/\pLaTeXe\ works
% also for \upLaTeXe\ without any modification. However for example,
% if a class or a package uses Kanji encoding `JY1' or `JT1' (default on
% \pLaTeXe), an error complaining the mismatch of Kanji encoding might
% happen on \upLaTeX, in which the default is `JY2' and `JT2.'
% In this case, we have to say that the class or package does not support
% \upLaTeXe; you should use \pLaTeX, or report to the author of the
% package or class.
%\fi
%
%\ifJAPANESE
% \subsection{latexreleaseパッケージへの対応}
% \LaTeX\ \texttt{<2015/01/01>}で導入されたlatexreleaseパッケージを
% もとに、新しい\pLaTeX{}ではplatexreleaseパッケージが用意されました。
% 本来は\upLaTeX{}でも同様のパッケージを用意するのがよいのですが、
% 現在は\pLaTeX{}から\upLaTeX{}への変更点が含まれていませんので、
% 幸いplatexreleaseパッケージをそのまま用いることができます。
% このため、\upLaTeX{}で独自のパッケージを用意することはしていません。
% platexreleaseパッケージを用いると、過去の\upLaTeX{}をエミュレート
% したり、フォーマットを作り直すことなく新しい\upLaTeX{}を試したりする
% ことができます。詳細はplatexreleaseのドキュメントを参照してください。
%\else
% \subsection{Support for Package `latexrelease'}
% \pLaTeX\ provides `platexrelease' package, which is based on
% `latexrelease' package (introduced in \LaTeX\ \texttt{<2015/01/01>}).
% It could be better if we also provide a similar package on \upLaTeX,
% but currently we don't need it; \upLaTeX\ does not have any recent
% \upLaTeX-specific changes. So, you can safely use `platexrelease'
% package for emulating the specified format date.
%\fi
%
%
%
% \appendix
%
%\ifJAPANESE
% \section{\dst{}プログラムのためのオプション}\label{app:dst}
% この文書のソース(\file{uplatex.dtx})を\dst{}プログラムで
% 処理することによって、
% いくつかの異なるファイルを生成することができます。
% \dst{}プログラムの詳細は、\file{docstrip.dtx}を参照してください。
%
% この文書の\dst{}プログラムのためのオプションは、次のとおりです。
%
% \DeleteShortVerb{\|}
% \begin{center}
% \begin{tabular}{l|p{.8\linewidth}}
% \emph{オプション} & \emph{意味}\\\hline
% plcore & フォーマットファイルを作るためのファイルを生成\\
% pldoc  & \upLaTeXe{}のソースファイルをまとめて組版するための
%          文書ファイル(upldoc.tex)を生成\\[2mm]
% shprog & 上記のファイルを作成するためのshスクリプトを生成\\
% Xins   & 上記のshスクリプトやperlスクリプトを取り出すための
%          \dst{}バッチファイル(Xins.ins)を生成\\
% \end{tabular}
% \end{center}
% \MakeShortVerb{\|}
%\else
% \section{\dst\ Options}\label{app:dst}
% By processing \file{uplatex.dtx} with \dst\ program,
% different files can be generated.
% Here are the \dst\ options for this document:
%
% \DeleteShortVerb{\|}
% \begin{center}
% \begin{tabular}{l|p{.8\linewidth}}
% \emph{Option} & \emph{Function}\\\hline
% plcore & Generates a fragment of format sources\\
% pldoc  & Generates `upldoc.tex' for typesetting
%          \upLaTeXe\ sources\\[2mm]
% shprog & Generates a shell script to process `upldoc.tex'\\
% Xins   & Generates a \dst\ batch file `Xins.ins' for
%          generating the above shell/perl scripts\\
% \end{tabular}
% \end{center}
% \MakeShortVerb{\|}
%\fi
%
%
%\ifJAPANESE
% \section{文書ファイル}\label{app:pldoc}
% ここでは、このパッケージに含まれているdtxファイルをまとめて組版し、
% ソースコード説明書を得るための文書ファイル\file{upldoc.tex}について
% 説明をしています。個別に処理した場合と異なり、
% 変更履歴や索引も付きます。
%
% デフォルトではソースコードの説明が日本語で書かれます。
% もし英語の説明書を読みたい場合は、\par\medskip
% \begin{minipage}{.5\textwidth}\ttfamily
% | |\cs{newif}\cs{ifJAPANESE}
% \end{minipage}\par\medskip\noindent
% という内容の\file{uplatex.cfg}を予め用意してから\file{upldoc.tex}を
% 処理してください(2016年7月1日以降の\upLaTeXe{}が必要)。
%
% コードは\pLaTeXe{}のものと(ファイル名を除き)ほぼ同一なので、
% ここでは違っている部分だけ説明します。
%\else
% \section{Documentation of \upLaTeXe\ sources}\label{app:pldoc}
% The contents of `upldoc.tex' for typesetting \upLaTeXe\ sources
% is described here. Compared to individual processings,
% batch processing using `upldoc.tex' prints also changes and an index.
%
% By default, the description of \upLaTeXe\ sources is written in
% Japanese. If you need English version, first save\par\medskip
% \begin{minipage}{.5\textwidth}\ttfamily
% | |\cs{newif}\cs{ifJAPANESE}
% \end{minipage}\par\medskip\noindent
% as \file{uplatex.cfg}, and process \file{upldoc.tex}
% (\upLaTeXe\ newer than July 2016 is required).
%
% Here we explain only difference between \file{pldoc.tex} (\pLaTeXe)
% and \file{upldoc.tex} (\upLaTeXe).
%\fi
%
%    \begin{macrocode}
%<*pldoc>
\begin{filecontents}{upldoc.dic}
西暦    せいれき
和暦    われき
\end{filecontents}
%    \end{macrocode}
%\ifJAPANESE
% \pLaTeXe{}のドキュメントでは、\file{plext.dtx}の中で組み立てるサンプル
% のために\file{plext}パッケージが必要ですが、\upLaTeXe{}のファイル
% にはそのようなサンプルが含まれないので除外しています。
%\else
% The document of \pLaTeXe\ requires \file{plext} package,
% since \file{plext.dtx} contains several examples of partial
% vertical writing. However, we don't have such examples in
% \upLaTeXe\ files, so no need for it.
%\fi
%    \begin{macrocode}
\documentclass{jltxdoc}
%\usepackage{plext} %% comment out for upLaTeX
\listfiles

\DoNotIndex{\def,\long,\edef,\xdef,\gdef,\let,\global}
\DoNotIndex{\if,\ifnum,\ifdim,\ifcat,\ifmmode,\ifvmode,\ifhmode,%
            \iftrue,\iffalse,\ifvoid,\ifx,\ifeof,\ifcase,\else,\or,\fi}
\DoNotIndex{\box,\copy,\setbox,\unvbox,\unhbox,\hbox,%
            \vbox,\vtop,\vcenter}
\DoNotIndex{\@empty,\immediate,\write}
\DoNotIndex{\egroup,\bgroup,\expandafter,\begingroup,\endgroup}
\DoNotIndex{\divide,\advance,\multiply,\count,\dimen}
\DoNotIndex{\relax,\space,\string}
\DoNotIndex{\csname,\endcsname,\@spaces,\openin,\openout,%
            \closein,\closeout}
\DoNotIndex{\catcode,\endinput}
\DoNotIndex{\jobname,\message,\read,\the,\m@ne,\noexpand}
\DoNotIndex{\hsize,\vsize,\hskip,\vskip,\kern,\hfil,\hfill,\hss,\vss,\unskip}
\DoNotIndex{\m@ne,\z@,\z@skip,\@ne,\tw@,\p@,\@minus,\@plus}
\DoNotIndex{\dp,\wd,\ht,\setlength,\addtolength}
\DoNotIndex{\newcommand, \renewcommand}

\ifJAPANESE
\IndexPrologue{\part*{索 引}%
                 \markboth{索 引}{索 引}%
                 \addcontentsline{toc}{part}{索 引}%
イタリック体の数字は、その項目が説明されているページを示しています。
下線の引かれた数字は、定義されているページを示しています。
その他の数字は、その項目が使われているページを示しています。}
\else
\IndexPrologue{\part*{Index}%
                 \markboth{Index}{Index}%
                 \addcontentsline{toc}{part}{Index}%
The italic numbers denote the pages where the corresponding entry
is described, numbers underlined point to the definition,
all others indicate the places where it is used.}
\fi
%
\ifJAPANESE
\GlossaryPrologue{\part*{変更履歴}%
                 \markboth{変更履歴}{変更履歴}%
                 \addcontentsline{toc}{part}{変更履歴}}
\else
\GlossaryPrologue{\part*{Change History}%
                 \markboth{Change History}{Change History}%
                 \addcontentsline{toc}{part}{Change History}}
\fi

\makeatletter
\def\changes@#1#2#3{%
  \let\protect\@unexpandable@protect
  \edef\@tempa{\noexpand\glossary{#2\space
               \currentfile\space#1\levelchar
               \ifx\saved@macroname\@empty
                  \space\actualchar\generalname
               \else
                  \expandafter\@gobble
                  \saved@macroname\actualchar
                  \string\verb\quotechar*%
                  \verbatimchar\saved@macroname
                  \verbatimchar
               \fi
               :\levelchar #3}}%
  \@tempa\endgroup\@esphack}
\renewcommand*\MacroFont{\fontencoding\encodingdefault
                   \fontfamily\ttdefault
                   \fontseries\mddefault
                   \fontshape\updefault
                   \small
                   \hfuzz 6pt\relax}
\renewcommand*\l@subsection{\@dottedtocline{2}{1.5em}{2.8em}}
\renewcommand*\l@subsubsection{\@dottedtocline{3}{3.8em}{3.4em}}
\makeatother
\RecordChanges
\CodelineIndex
\EnableCrossrefs
\setcounter{IndexColumns}{2}
\settowidth\MacroIndent{\ttfamily\scriptsize 000\ }
%    \end{macrocode}
%\ifJAPANESE
% この文書のタイトル・著者・日付を設定します。
% \changes{v1.0h-u00}{2016/05/08}{ドキュメントから\file{uplpatch.ltx}を除外
%     (based on platex.dtx 2016/05/08 v1.0h)}
% \changes{v1.0l-u01}{2016/06/19}{パッチレベルを\file{uplvers.dtx}から取得
%     (based on platex.dtx 2016/06/19 v1.0l)}
% \changes{v1.0y-u02}{2018/09/22}{最終更新日を\file{upldoc.pdf}に表示
%     (based on platex.dtx 2018/09/22 v1.0y)}
%\else
% Set the title, authors and the date for this document.
% \changes{v1.0h-u00}{2016/05/08}{Exclude \file{uplpatch.ltx} from the document
%     (based on platex.dtx 2016/05/08 v1.0h)}
% \changes{v1.0l-u01}{2016/06/19}{Get the patch level from \file{uplvers.dtx}
%     (based on platex.dtx 2016/06/19 v1.0l)}
% \changes{v1.0y-u02}{2018/09/22}{Show last update info on \file{upldoc.pdf}
%     (based on platex.dtx 2018/09/22 v1.0y)}
%\fi
%    \begin{macrocode}
 \title{The \upLaTeXe\ Sources}
 \author{Ken Nakano \& Japanese \TeX\ Development Community \& TTK}

% Get the (temporary) date and up-patch level from uplvers.dtx
\makeatletter
\let\patchdate=\@empty
\begingroup
   \def\ProvidesFile#1[#2 #3]#4\def\uppatch@level#5{%
      \date{#2}\xdef\patchdate{#5}\endinput}
   % \iffalse meta-comment
%% File: uplvers.dtx
%
%    pLaTeX version setting file:
%       Copyright 1995-2006 ASCII Corporation.
%    and modified for upLaTeX
%
%  Copyright (c) 2010 ASCII MEDIA WORKS
%  Copyright (c) 2016 Takuji Tanaka
%  Copyright (c) 2016-2017 Japanese TeX Development Community
%
%  This file is part of the upLaTeX2e system (community edition).
%  --------------------------------------------------------------
%
% \fi
%
%
% \setcounter{StandardModuleDepth}{1}
% \StopEventually{}
%
% \iffalse
% \changes{v1.0}{1995/05/16}{p\LaTeXe\ 用に\file{ltvers.dtx}を修正}
% \changes{v1.0a}{1995/08/30}{\LaTeX\ \texttt{!<1995/06/01!>}版用に修正}
% \changes{v1.0b}{1996/01/31}{\LaTeX\ \texttt{!<1995/12/01!>}版用に修正}
% \changes{v1.0c}{1997/01/11}{\LaTeX\ \texttt{!<1996/06/01!>}版用に修正}
% \changes{v1.0d}{1997/01/23}{\LaTeX\ \texttt{!<1996/12/01!>}版用に修正}
% \changes{v1.0e}{1997/07/02}{\LaTeX\ \texttt{!<1997/06/01!>}版用に修正}
% \changes{v1.0f}{1998/02/17}{\LaTeX\ \texttt{!<1997/12/01!>}版用に修正}
% \changes{v1.0g}{1998/09/01}{\LaTeX\ \texttt{!<1998/06/01!>}版用に修正}
% \changes{v1.0h}{1999/04/05}{\LaTeX\ \texttt{!<1998/12/01!>}版用に修正}
% \changes{v1.0i}{1999/08/09}{\LaTeX\ \texttt{!<1999/06/01!>}版用に修正}
% \changes{v1.0j}{2000/02/29}{\LaTeX\ \texttt{!<1999/12/01!>}版用に修正}
% \changes{v1.0k}{2000/11/03}{\LaTeX\ \texttt{!<2000/06/01!>}版用に修正}
% \changes{v1.0l}{2001/09/04}{\LaTeX\ \texttt{!<2001/06/01!>}版用に修正}
% \changes{v1.0m}{2004/08/10}{\LaTeX\ \texttt{!<2003/12/01!>}版対応確認}
% \changes{v1.0n}{2005/01/04}{plfonts.dtxバグ修正}
% \changes{v1.0o}{2006/01/04}{plfonts.dtxバグ修正}
% \changes{v1.0p}{2006/06/27}{plfonts.dtx \LaTeX\ \texttt{!<2005/12/01!>}対応}
% \changes{v1.0q}{2006/11/10}{plfonts.dtxバグ修正}
% \changes{v1.0q-u00}{2011/05/07}{p\LaTeX{}用からup\LaTeX{}用に修正。}
% \changes{v1.0r}{2016/01/26}{plcore.dtx p\TeX\ (r28720)対応}
% \changes{v1.0s}{2016/02/01}{\LaTeX\ \texttt{!<2015/01/01!>}のlatexreleaseに
%    対応するためのコードを導入}
% \changes{v1.0t}{2016/02/03}{\cs{plIncludeInRelease}と
%    \cs{plEndIncludeInRelease}を新設。}
% \changes{v1.0u}{2016/04/17}{\LaTeX\ \texttt{!<2016/03/31!>}版対応確認}
% \changes{v1.0u-u00}{2016/04/17}{p\LaTeX{}の変更に追随。}
% \changes{v1.0v}{2016/05/07}{パッチファイルをロードするのをやめた。}
% \changes{v1.0v}{2016/05/07}{起動時の文字列を最新の\LaTeX{}に合わせた。}
% \changes{v1.0w}{2016/05/12}{起動時の文字列に入れる\LaTeX{}のバージョンを
%    元の\LaTeX{}のバナーから引き継ぐように改良}
% \changes{v1.0w-u00}{2016/05/12}{起動時の文字列に入れるBabelのバージョンを
%    元の\LaTeX{}のバナーから取得するコードを\file{uplatex.ini}から取り入れた}
% \changes{v1.0w-u01}{2016/05/21}{サポート外の\LaTeX~2.09互換モードが
%    使われた場合に明確なエラーを出すようにした。}
% \changes{v1.0x}{2016/06/19}{パッチレベルを\file{plvers.dtx}で設定}
% \changes{v1.0x-u01}{2016/06/19}{p\LaTeX{}の変更に追随。}
% \changes{v1.0y-u01}{2016/06/29}{\file{uplatex.cfg}の読み込みを追加}
% \changes{v1.0z-u01}{2016/08/26}{\file{uplatex.cfg}の読み込みを
%    \file{uplcore.ltx}から\file{uplatex.ltx}へ移動}
% \changes{v1.1}{2016/09/14}{起動時のバナーを取得するコードを改良}
% \changes{v1.1-u01}{2016/09/14}{p\LaTeX{}の変更に追随。}
% \changes{v1.1a}{2017/02/20}{\LaTeX\ \texttt{!<2017/01/01!>}版対応確認}
% \changes{v1.1a-u01}{2017/03/05}{p\LaTeX{}の変更に追随。}
% \changes{v1.1b}{2017/03/19}{\cs{l@nohyphenation}の定義を保証
%    (sync with ltfinal 2017/03/09 v2.0t)}
% \changes{v1.1b}{2017/03/19}{\cs{document@default@language}の定義を保証
%    (sync with ltfinal 2017/03/09 v2.0t)}
% \changes{v1.1b-u01}{2017/03/19}{p\LaTeX{}の変更に追随。}
% \changes{v1.1c}{2017/04/23}{\LaTeX\ \texttt{!<2017/04/15!>}版対応確認}
% \changes{v1.1c-u01}{2017/05/04}{p\LaTeX{}の変更に追随。}
% \changes{v1.1d}{2017/09/24}{パッチレベルが負の数の場合をpre-release扱いへ}
% \changes{v1.1d-u01}{2017/09/24}{p\LaTeX{}の変更に追随。}
% \fi
%
% \iffalse
%<*driver>
% \fi
\ProvidesFile{uplvers.dtx}[2017/09/24 v1.1d-u01 upLaTeX Kernel (Version Info)]
% \iffalse
\documentclass{jltxdoc}
\GetFileInfo{uplvers.dtx}
\author{Ken Nakano \& Hideaki Togashi \& TTK}
\title{\filename}
\date{作成日:\filedate}
\begin{document}
  \maketitle
  \DocInput{\filename}
\end{document}
%</driver>
% \fi
%
% \section{バージョンの設定}
% まず、このディストリビューションでのup\LaTeXe{}の日付とバージョン番号
% を定義します。また、up\LaTeXe{}が起動されたときに表示される文字列の
% 設定もします。
%
% \changes{v1.0}{1995/05/16}{p\LaTeXe\ 用に\file{ltvers.dtx}を修正}
% \changes{v1.0a}{1995/08/30}{\LaTeX\ \texttt{!<1995/06/01!>}版用に修正}
% \changes{v1.0b}{1996/01/31}{\LaTeX\ \texttt{!<1995/12/01!>}版用に修正}
% \changes{v1.0c}{1997/01/11}{\LaTeX\ \texttt{!<1996/06/01!>}版用に修正}
% \changes{v1.0d}{1997/01/23}{\LaTeX\ \texttt{!<1996/12/01!>}版用に修正}
% \changes{v1.0e}{1997/07/02}{\LaTeX\ \texttt{!<1997/06/01!>}版用に修正}
% \changes{v1.0f}{1998/02/17}{\LaTeX\ \texttt{!<1997/12/01!>}版用に修正}
% \changes{v1.0g}{1998/09/01}{\LaTeX\ \texttt{!<1998/06/01!>}版用に修正}
% \changes{v1.0h}{1999/04/05}{\LaTeX\ \texttt{!<1998/12/01!>}版用に修正}
% \changes{v1.0i}{1999/08/09}{\LaTeX\ \texttt{!<1999/06/01!>}版用に修正}
% \changes{v1.0j}{2000/02/29}{\LaTeX\ \texttt{!<1999/12/01!>}版用に修正}
% \changes{v1.0k}{2000/11/03}{\LaTeX\ \texttt{!<2000/06/01!>}版用に修正}
% \changes{v1.0l}{2001/09/04}{\LaTeX\ \texttt{!<2001/06/01!>}版用に修正}
% \changes{v1.0m}{2004/08/10}{\LaTeX\ \texttt{!<2003/12/01!>}版対応確認}
% \changes{v1.0s}{2016/02/01}{\LaTeX\ \texttt{!<2015/01/01!>}版用に修正}
% \changes{v1.0u}{2016/04/17}{\LaTeX\ \texttt{!<2016/03/31!>}版対応確認}
% \changes{v1.1a}{2017/02/20}{\LaTeX\ \texttt{!<2017/01/01!>}版対応確認}
% \changes{v1.1c}{2017/04/23}{\LaTeX\ \texttt{!<2017/04/15!>}版対応確認}
%
% このバージョンのup\LaTeXe{}は、次のバージョンの\LaTeX{}\footnote{%
% \LaTeX\ authors: Johannes Braams, David Carlisle, Alan Jeffrey,
%   Leslie Lamport, Frank Mittelbach, Chris Rowley, Rainer Sch\"opf}を
% もとにしています。
%    \begin{macrocode}
%<*2ekernel>
%\def\fmtname{LaTeX2e}
%\edef\fmtversion
%</2ekernel>
%<latexrelease>\edef\latexreleaseversion
%<platexrelease>\edef\p@known@latexreleaseversion
%<*2ekernel|latexrelease|platexrelease>
   {2017/04/15}
%</2ekernel|latexrelease|platexrelease>
%    \end{macrocode}
%
% \begin{macro}{\pfmtname}
% \begin{macro}{\pfmtversion}
% \begin{macro}{\ppatch@level}
% up\LaTeXe{}のフォーマットファイル名とバージョンです。
% \changes{v1.0x}{2016/06/19}{パッチレベルを\file{plvers.dtx}で設定}
%    \begin{macrocode}
%<*plcore>
\def\pfmtname{pLaTeX2e}
\def\pfmtversion
%</plcore>
%<platexrelease>\edef\platexreleaseversion
%<*plcore|platexrelease>
   {2017/10/28u01}
%</plcore|platexrelease>
%<*plcore>
\def\ppatch@level{1}
%</plcore>
%    \end{macrocode}
% \end{macro}
% \end{macro}
% \end{macro}
%
% \subsection{\LaTeX~2.09互換モードの抑制}
%
% \begin{macro}{\documentstyle}
% p\LaTeX{}は、|\documentclass|の代わりに|\documentstyle|が使われると
% \LaTeX~2.09互換モードに入ります。しかし、up\LaTeX{}は新しいマクロ
% パッケージですので、\LaTeX~2.09互換モードをサポートしません。
% このため、\file{plcore.dtx}の定義を上書きして明確なエラーを出します。
% \changes{v1.0w-u01}{2016/05/21}{サポート外の\LaTeX~2.09互換モードが
%    使われた場合に明確なエラーを出すようにした。}
%    \begin{macrocode}
%<*plfinal>
\def\documentstyle{%
  \@latex@error{upLaTeX does NOT support LaTeX 2.09 compatibility
    mode.\MessageBreak Use \noexpand\documentclass instead}{%
    \noexpand\documentstyle is an old convention of LaTeX 2.09,
    which has been\MessageBreak obsolete since 1995. upLaTeX is
    first released in 2007, so we do\MessageBreak not provide any
    emulation of the LaTeX 2.09 author environment.\MessageBreak
    New documents should use Standard LaTeX conventions, and
    start\MessageBreak with the \noexpand\documentclass command.}%
  \documentclass}
%    \end{macrocode}
% \end{macro}
%
% \subsection{パッチファイルのロード}
%
% 次の部分は、up\LaTeXe{}のパッチファイルをロードするためのコードです。
% バグを修正するためのパッチを配布するかもしれません。
%
% パッチファイルをロードするコードはコメントアウトしました。
% \changes{v1.0v}{2016/05/07}{パッチファイルをロードするのをやめた。}
%    \begin{macrocode}
%\IfFileExists{uplpatch.ltx}
%  {\typeout{************************************^^J%
%            * Appliying patch file uplpatch.ltx *^^J%
%            ************************************}
%  \def\pfmtversion@topatch{unknown}
%  \input{uplpatch.ltx}
%  \ifx\pfmtversion\pfmtversion@topatch
%    \ifx\ppatch@level\@undefined
%      \typeout{^^J^^J^^J%
%   !!!!!!!!!!!!!!!!!!!!!!!!!!!!!!!!!!!!!!!!!!!!!!!!!!!!!!!^^J%
%   !! Patch file `uplpatch.ltx' (for version <\pfmtversion@topatch>)^^J%
%   !! is not suitable for version <\pfmtversion> of upLaTeX.^^J^^J%
%   !! Please check if iniptex found an old patch file:^^J%
%   !! --- if so, rename it or delete it, and redo the^^J%
%   !!     iniptex run.^^J%
%   !!!!!!!!!!!!!!!!!!!!!!!!!!!!!!!!!!!!!!!!!!!!!!!!!!!!!!!^^J}%
%      \batchmode \@@end
%    \fi
%  \else
%      \typeout{^^J^^J^^J%
%   !!!!!!!!!!!!!!!!!!!!!!!!!!!!!!!!!!!!!!!!!!!!!!!!!!!!!!!^^J%
%   !! Patch file `uplpatch.ltx' (for version <\pfmtversion@topatch>)^^J%
%   !! is not suitable for version <\pfmtversion> of upLaTeX.^^J%
%   !!^^J%
%   !! Please check if iniptex found an old patch file:^^J%
%   !! --- if so, rename it or delete it, and redo the^^J%
%   !!     iniptex run.^^J%
%   !!!!!!!!!!!!!!!!!!!!!!!!!!!!!!!!!!!!!!!!!!!!!!!!!!!!!!!^^J}%
%      \batchmode \@@end
%  \fi
%  \let\pfmtversion@topatch\relax
%  }{}
%    \end{macrocode}
%
% \subsection{起動時に表示するバナー}
%
% \begin{macro}{\everyjob}
% 起動時に表示される文字列です。
% \LaTeX{}にパッチがあてられている場合は、それも表示します。
%
%\iffalse
% この実装については\file{uplatex.dtx}のコメントを参照。(2016/09/14)
%\fi
%
% \changes{v1.0v}{2016/05/07}{起動時の文字列を最新の\LaTeX{}に合わせた。}
% \changes{v1.0w}{2016/05/12}{起動時の文字列に入れる\LaTeX{}のバージョンを
%    元の\LaTeX{}のバナーから引き継ぐように改良}
% \changes{v1.1}{2016/09/14}{起動時のバナーを取得するコードを改良}
% \changes{v1.1d}{2017/09/24}{パッチレベルが負の数の場合をpre-release扱いへ}
%    \begin{macrocode}
\ifx\patch@level\@undefined % fallback if undefined in LaTeX
  \def\patch@level{0}\fi
\ifx\ppatch@level\@undefined % fallback if undefined in upLaTeX
  \def\ppatch@level{0}\fi
\begingroup
  \def\parse@@BANNER\typeout#1\typeout#2#3\relax{#1}
  \edef\platexTMP{%
    \ifnum\ppatch@level=0
      \everyjob{\noexpand\typeout{%
        \pfmtname\space<\pfmtversion>\space
          (based on \expandafter\parse@@BANNER\platexBANNER)}}%
    \else\ifnum\ppatch@level>0
      \everyjob{\noexpand\typeout{%
        \pfmtname\space<\pfmtversion>+\ppatch@level\space
          (based on \expandafter\parse@@BANNER\platexBANNER)}}%
    \else
      \everyjob{\noexpand\typeout{%
        \pfmtname\space<\pfmtversion>-pre\ppatch@level\space
          (based on \expandafter\parse@@BANNER\platexBANNER)}}%
    \fi\fi
  }
\expandafter
\endgroup \platexTMP
%    \end{macrocode}
%
% p\LaTeX{}やup\LaTeX{}は、独自のハイフネーション・パターンを定義していません。
% \TeX\ Liveの標準的インストールでは、代わりに\LaTeX{}が読み込んでいる
% Babelパッケージのものが適用されるはずですから、起動時の文字列にも
% \file{hyphen.cfg}のバージョンを反映します(Babelパッケージの
% \file{hyphen.cfg}でない場合は、何も表示されず空行になるはずです)。
%
%\iffalse
% この実装については\file{uplatex.dtx}のコメントを参照。(2016/09/14)
%\fi
%
% \changes{v1.0w-u00}{2016/05/12}{起動時の文字列に入れるBabelのバージョンを
%    元の\LaTeX{}のバナーから取得するコードを\file{uplatex.ini}から取り入れた}
%    \begin{macrocode}
\begingroup
  \def\parse@@BANNER\typeout#1\typeout#2#3\relax{#2}
  \edef\platexTMP{%
    \the\everyjob\noexpand\typeout{\expandafter\parse@@BANNER\platexBANNER}%
  }
  \everyjob=\expandafter{\platexTMP}%
  \edef\platexTMP{%
    \noexpand\let\noexpand\platexBANNER=\noexpand\@undefined
    \noexpand\everyjob={\the\everyjob}%
  }
  \expandafter
\endgroup \platexTMP
%</plfinal>
%    \end{macrocode}
% \end{macro}
%
% ^^A 起動時に\file{uplatex.cfg}がある場合、それを読み込むようにする
% ^^A コードは、\file{uplcore.ltx}から\file{uplatex.ltx}へ移動しました。
% \changes{v1.0y-u01}{2016/06/29}{\file{uplatex.cfg}の読み込みを追加}
% \changes{v1.0z-u01}{2016/08/26}{\file{uplatex.cfg}の読み込みを
%    \file{uplcore.ltx}から\file{uplatex.ltx}へ移動}
%
% \subsection{ハイフネーション関連}
%
% \begin{macro}{\l@nohyphenation}
% \LaTeXe\ 2017-04-15で、|\verb|の途中でハイフネーションが起きないように
% する修正が入りました。この修正には|\l@nohyphenation|が定義済みでなければ
% なりませんが、通常はBabelの定義ファイルによって提供されています。
% \LaTeXe{}は特殊な状況も想定してltfinalで対策しているようですので、
% p\LaTeXe{}も念のためplfinalで対策します(参考:latex2e svn r1405)。
% \changes{v1.1b}{2017/03/19}{\cs{l@nohyphenation}の定義を保証
%    (sync with ltfinal 2017/03/09 v2.0t)}
%    \begin{macrocode}
%<*plfinal>
\ifx\l@nohyphenation \@undefined
  \newlanguage\l@nohyphenation
\fi
%    \end{macrocode}
% \end{macro}
%
% \begin{macro}{\document@default@language}
% \LaTeXe\ 2017-04-15で導入されたパラメータです。更新タイミングのずれの
% 可能性を考慮し、p\LaTeXe{}でも準備しておきます。verbatim環境の途中で
% 改ページが起きた場合にヘッダでハイフネーションが抑制されないように、
% |\@outputpage|で|\language|をリセットするときに使われます
% (参考:latex2e svn r1407)。
% \changes{v1.1b}{2017/03/19}{\cs{document@default@language}の定義を保証
%    (sync with ltfinal 2017/03/09 v2.0t)}
%    \begin{macrocode}
\ifx\document@default@language \@undefined
  \let\document@default@language\m@ne
\fi
%</plfinal>
%    \end{macrocode}
% \end{macro}
%
% \subsection{latexreleaseパッケージへの対応}
%
% 最後に、latexreleaseパッケージへの対応です。
% \begin{macro}{\plIncludeInRelease}
% \changes{v1.0t}{2016/02/03}{\cs{plIncludeInRelease}と
%    \cs{plEndIncludeInRelease}を新設。}
%    \begin{macrocode}
%<*plcore|platexrelease>
\def\plIncludeInRelease#1{\kernel@ifnextchar[%
  {\@plIncludeInRelease{#1}}
  {\@plIncludeInRelease{#1}[#1]}}
%    \end{macrocode}
%
%    \begin{macrocode}
\def\@plIncludeInRelease#1[#2]{\@plIncludeInRele@se{#2}}
%    \end{macrocode}
%
%    \begin{macrocode}
\def\@plIncludeInRele@se#1#2#3{%
  \toks@{[#1] #3}%
  \expandafter\ifx\csname\string#2+\@currname+IIR\endcsname\relax
    \ifnum\expandafter\@parse@version#1//00\@nil
          >\expandafter\@parse@version\pfmtversion//00\@nil
      \GenericInfo{}{Skipping: \the\toks@}%
     \expandafter\expandafter\expandafter\@gobble@plIncludeInRelease
    \else
      \GenericInfo{}{Applying: \the\toks@}%
      \expandafter\let\csname\string#2+\@currname+IIR\endcsname\@empty
    \fi
  \else
    \GenericInfo{}{Already applied: \the\toks@}%
    \expandafter\@gobble@plIncludeInRelease
  \fi
}
%    \end{macrocode}
%
%    \begin{macrocode}
\long\def\@gobble@plIncludeInRelease#1\plEndIncludeInRelease{}
\let\plEndIncludeInRelease\relax
%</plcore|platexrelease>
%    \end{macrocode}
% \end{macro}
%
% \LaTeXe{}が提供するlatexreleaseパッケージが読み込まれていて、
% かつp\LaTeXe{}が提供するplatexreleaseパッケージが読み込まれていない
% 場合は、警告を出します。
% \changes{v1.0s}{2016/02/01}{latexrelease利用時に警告を出すようにした}
%    \begin{macrocode}
%<*plfinal>
\AtBeginDocument{%
  \@ifpackageloaded{latexrelease}{%
    \@ifpackageloaded{platexrelease}{}{%
      \@latex@warning@no@line{%
        Package latexrelease is loaded.\MessageBreak
        Some patches in pLaTeX2e core may be overwritten.\MessageBreak
        Consider using platexrelease.\MessageBreak
        See platex.pdf for detail}%
    }%
  }{}%
}
%</plfinal>
%    \end{macrocode}
%
% \Finale
%
\endinput

\endgroup

% Add the patch version if available.
\def\Xpatch{}
\ifx\patchdate\Xpatch\else
  \edef\@date{\@date\space version \patchdate}
\fi

% Obtain the last update info, as upLaTeX does not change format date
% -> if successful, reconstruct the date completely
\def\lastupd@te{0000/00/00}
\begingroup
   \def\ProvidesFile#1[#2 #3]{%
      \def\@tempd@te{#2}\endinput
      \@ifl@t@r{\@tempd@te}{\lastupd@te}{%
         \global\let\lastupd@te\@tempd@te
      }{}}
   \let\ProvidesClass\ProvidesFile
   \let\ProvidesPackage\ProvidesFile
   % \iffalse meta-comment
%% File: uplvers.dtx
%
%    pLaTeX version setting file:
%       Copyright 1995-2006 ASCII Corporation.
%    and modified for upLaTeX
%
%  Copyright (c) 2010 ASCII MEDIA WORKS
%  Copyright (c) 2016 Takuji Tanaka
%  Copyright (c) 2016-2017 Japanese TeX Development Community
%
%  This file is part of the upLaTeX2e system (community edition).
%  --------------------------------------------------------------
%
% \fi
%
%
% \setcounter{StandardModuleDepth}{1}
% \StopEventually{}
%
% \iffalse
% \changes{v1.0}{1995/05/16}{p\LaTeXe\ 用に\file{ltvers.dtx}を修正}
% \changes{v1.0a}{1995/08/30}{\LaTeX\ \texttt{!<1995/06/01!>}版用に修正}
% \changes{v1.0b}{1996/01/31}{\LaTeX\ \texttt{!<1995/12/01!>}版用に修正}
% \changes{v1.0c}{1997/01/11}{\LaTeX\ \texttt{!<1996/06/01!>}版用に修正}
% \changes{v1.0d}{1997/01/23}{\LaTeX\ \texttt{!<1996/12/01!>}版用に修正}
% \changes{v1.0e}{1997/07/02}{\LaTeX\ \texttt{!<1997/06/01!>}版用に修正}
% \changes{v1.0f}{1998/02/17}{\LaTeX\ \texttt{!<1997/12/01!>}版用に修正}
% \changes{v1.0g}{1998/09/01}{\LaTeX\ \texttt{!<1998/06/01!>}版用に修正}
% \changes{v1.0h}{1999/04/05}{\LaTeX\ \texttt{!<1998/12/01!>}版用に修正}
% \changes{v1.0i}{1999/08/09}{\LaTeX\ \texttt{!<1999/06/01!>}版用に修正}
% \changes{v1.0j}{2000/02/29}{\LaTeX\ \texttt{!<1999/12/01!>}版用に修正}
% \changes{v1.0k}{2000/11/03}{\LaTeX\ \texttt{!<2000/06/01!>}版用に修正}
% \changes{v1.0l}{2001/09/04}{\LaTeX\ \texttt{!<2001/06/01!>}版用に修正}
% \changes{v1.0m}{2004/08/10}{\LaTeX\ \texttt{!<2003/12/01!>}版対応確認}
% \changes{v1.0n}{2005/01/04}{plfonts.dtxバグ修正}
% \changes{v1.0o}{2006/01/04}{plfonts.dtxバグ修正}
% \changes{v1.0p}{2006/06/27}{plfonts.dtx \LaTeX\ \texttt{!<2005/12/01!>}対応}
% \changes{v1.0q}{2006/11/10}{plfonts.dtxバグ修正}
% \changes{v1.0q-u00}{2011/05/07}{p\LaTeX{}用からup\LaTeX{}用に修正。}
% \changes{v1.0r}{2016/01/26}{plcore.dtx p\TeX\ (r28720)対応}
% \changes{v1.0s}{2016/02/01}{\LaTeX\ \texttt{!<2015/01/01!>}のlatexreleaseに
%    対応するためのコードを導入}
% \changes{v1.0t}{2016/02/03}{\cs{plIncludeInRelease}と
%    \cs{plEndIncludeInRelease}を新設。}
% \changes{v1.0u}{2016/04/17}{\LaTeX\ \texttt{!<2016/03/31!>}版対応確認}
% \changes{v1.0u-u00}{2016/04/17}{p\LaTeX{}の変更に追随。}
% \changes{v1.0v}{2016/05/07}{パッチファイルをロードするのをやめた。}
% \changes{v1.0v}{2016/05/07}{起動時の文字列を最新の\LaTeX{}に合わせた。}
% \changes{v1.0w}{2016/05/12}{起動時の文字列に入れる\LaTeX{}のバージョンを
%    元の\LaTeX{}のバナーから引き継ぐように改良}
% \changes{v1.0w-u00}{2016/05/12}{起動時の文字列に入れるBabelのバージョンを
%    元の\LaTeX{}のバナーから取得するコードを\file{uplatex.ini}から取り入れた}
% \changes{v1.0w-u01}{2016/05/21}{サポート外の\LaTeX~2.09互換モードが
%    使われた場合に明確なエラーを出すようにした。}
% \changes{v1.0x}{2016/06/19}{パッチレベルを\file{plvers.dtx}で設定}
% \changes{v1.0x-u01}{2016/06/19}{p\LaTeX{}の変更に追随。}
% \changes{v1.0y-u01}{2016/06/29}{\file{uplatex.cfg}の読み込みを追加}
% \changes{v1.0z-u01}{2016/08/26}{\file{uplatex.cfg}の読み込みを
%    \file{uplcore.ltx}から\file{uplatex.ltx}へ移動}
% \changes{v1.1}{2016/09/14}{起動時のバナーを取得するコードを改良}
% \changes{v1.1-u01}{2016/09/14}{p\LaTeX{}の変更に追随。}
% \changes{v1.1a}{2017/02/20}{\LaTeX\ \texttt{!<2017/01/01!>}版対応確認}
% \changes{v1.1a-u01}{2017/03/05}{p\LaTeX{}の変更に追随。}
% \changes{v1.1b}{2017/03/19}{\cs{l@nohyphenation}の定義を保証
%    (sync with ltfinal 2017/03/09 v2.0t)}
% \changes{v1.1b}{2017/03/19}{\cs{document@default@language}の定義を保証
%    (sync with ltfinal 2017/03/09 v2.0t)}
% \changes{v1.1b-u01}{2017/03/19}{p\LaTeX{}の変更に追随。}
% \changes{v1.1c}{2017/04/23}{\LaTeX\ \texttt{!<2017/04/15!>}版対応確認}
% \changes{v1.1c-u01}{2017/05/04}{p\LaTeX{}の変更に追随。}
% \changes{v1.1d}{2017/09/24}{パッチレベルが負の数の場合をpre-release扱いへ}
% \changes{v1.1d-u01}{2017/09/24}{p\LaTeX{}の変更に追随。}
% \fi
%
% \iffalse
%<*driver>
% \fi
\ProvidesFile{uplvers.dtx}[2017/09/24 v1.1d-u01 upLaTeX Kernel (Version Info)]
% \iffalse
\documentclass{jltxdoc}
\GetFileInfo{uplvers.dtx}
\author{Ken Nakano \& Hideaki Togashi \& TTK}
\title{\filename}
\date{作成日:\filedate}
\begin{document}
  \maketitle
  \DocInput{\filename}
\end{document}
%</driver>
% \fi
%
% \section{バージョンの設定}
% まず、このディストリビューションでのup\LaTeXe{}の日付とバージョン番号
% を定義します。また、up\LaTeXe{}が起動されたときに表示される文字列の
% 設定もします。
%
% \changes{v1.0}{1995/05/16}{p\LaTeXe\ 用に\file{ltvers.dtx}を修正}
% \changes{v1.0a}{1995/08/30}{\LaTeX\ \texttt{!<1995/06/01!>}版用に修正}
% \changes{v1.0b}{1996/01/31}{\LaTeX\ \texttt{!<1995/12/01!>}版用に修正}
% \changes{v1.0c}{1997/01/11}{\LaTeX\ \texttt{!<1996/06/01!>}版用に修正}
% \changes{v1.0d}{1997/01/23}{\LaTeX\ \texttt{!<1996/12/01!>}版用に修正}
% \changes{v1.0e}{1997/07/02}{\LaTeX\ \texttt{!<1997/06/01!>}版用に修正}
% \changes{v1.0f}{1998/02/17}{\LaTeX\ \texttt{!<1997/12/01!>}版用に修正}
% \changes{v1.0g}{1998/09/01}{\LaTeX\ \texttt{!<1998/06/01!>}版用に修正}
% \changes{v1.0h}{1999/04/05}{\LaTeX\ \texttt{!<1998/12/01!>}版用に修正}
% \changes{v1.0i}{1999/08/09}{\LaTeX\ \texttt{!<1999/06/01!>}版用に修正}
% \changes{v1.0j}{2000/02/29}{\LaTeX\ \texttt{!<1999/12/01!>}版用に修正}
% \changes{v1.0k}{2000/11/03}{\LaTeX\ \texttt{!<2000/06/01!>}版用に修正}
% \changes{v1.0l}{2001/09/04}{\LaTeX\ \texttt{!<2001/06/01!>}版用に修正}
% \changes{v1.0m}{2004/08/10}{\LaTeX\ \texttt{!<2003/12/01!>}版対応確認}
% \changes{v1.0s}{2016/02/01}{\LaTeX\ \texttt{!<2015/01/01!>}版用に修正}
% \changes{v1.0u}{2016/04/17}{\LaTeX\ \texttt{!<2016/03/31!>}版対応確認}
% \changes{v1.1a}{2017/02/20}{\LaTeX\ \texttt{!<2017/01/01!>}版対応確認}
% \changes{v1.1c}{2017/04/23}{\LaTeX\ \texttt{!<2017/04/15!>}版対応確認}
%
% このバージョンのup\LaTeXe{}は、次のバージョンの\LaTeX{}\footnote{%
% \LaTeX\ authors: Johannes Braams, David Carlisle, Alan Jeffrey,
%   Leslie Lamport, Frank Mittelbach, Chris Rowley, Rainer Sch\"opf}を
% もとにしています。
%    \begin{macrocode}
%<*2ekernel>
%\def\fmtname{LaTeX2e}
%\edef\fmtversion
%</2ekernel>
%<latexrelease>\edef\latexreleaseversion
%<platexrelease>\edef\p@known@latexreleaseversion
%<*2ekernel|latexrelease|platexrelease>
   {2017/04/15}
%</2ekernel|latexrelease|platexrelease>
%    \end{macrocode}
%
% \begin{macro}{\pfmtname}
% \begin{macro}{\pfmtversion}
% \begin{macro}{\ppatch@level}
% up\LaTeXe{}のフォーマットファイル名とバージョンです。
% \changes{v1.0x}{2016/06/19}{パッチレベルを\file{plvers.dtx}で設定}
%    \begin{macrocode}
%<*plcore>
\def\pfmtname{pLaTeX2e}
\def\pfmtversion
%</plcore>
%<platexrelease>\edef\platexreleaseversion
%<*plcore|platexrelease>
   {2017/10/28u01}
%</plcore|platexrelease>
%<*plcore>
\def\ppatch@level{1}
%</plcore>
%    \end{macrocode}
% \end{macro}
% \end{macro}
% \end{macro}
%
% \subsection{\LaTeX~2.09互換モードの抑制}
%
% \begin{macro}{\documentstyle}
% p\LaTeX{}は、|\documentclass|の代わりに|\documentstyle|が使われると
% \LaTeX~2.09互換モードに入ります。しかし、up\LaTeX{}は新しいマクロ
% パッケージですので、\LaTeX~2.09互換モードをサポートしません。
% このため、\file{plcore.dtx}の定義を上書きして明確なエラーを出します。
% \changes{v1.0w-u01}{2016/05/21}{サポート外の\LaTeX~2.09互換モードが
%    使われた場合に明確なエラーを出すようにした。}
%    \begin{macrocode}
%<*plfinal>
\def\documentstyle{%
  \@latex@error{upLaTeX does NOT support LaTeX 2.09 compatibility
    mode.\MessageBreak Use \noexpand\documentclass instead}{%
    \noexpand\documentstyle is an old convention of LaTeX 2.09,
    which has been\MessageBreak obsolete since 1995. upLaTeX is
    first released in 2007, so we do\MessageBreak not provide any
    emulation of the LaTeX 2.09 author environment.\MessageBreak
    New documents should use Standard LaTeX conventions, and
    start\MessageBreak with the \noexpand\documentclass command.}%
  \documentclass}
%    \end{macrocode}
% \end{macro}
%
% \subsection{パッチファイルのロード}
%
% 次の部分は、up\LaTeXe{}のパッチファイルをロードするためのコードです。
% バグを修正するためのパッチを配布するかもしれません。
%
% パッチファイルをロードするコードはコメントアウトしました。
% \changes{v1.0v}{2016/05/07}{パッチファイルをロードするのをやめた。}
%    \begin{macrocode}
%\IfFileExists{uplpatch.ltx}
%  {\typeout{************************************^^J%
%            * Appliying patch file uplpatch.ltx *^^J%
%            ************************************}
%  \def\pfmtversion@topatch{unknown}
%  \input{uplpatch.ltx}
%  \ifx\pfmtversion\pfmtversion@topatch
%    \ifx\ppatch@level\@undefined
%      \typeout{^^J^^J^^J%
%   !!!!!!!!!!!!!!!!!!!!!!!!!!!!!!!!!!!!!!!!!!!!!!!!!!!!!!!^^J%
%   !! Patch file `uplpatch.ltx' (for version <\pfmtversion@topatch>)^^J%
%   !! is not suitable for version <\pfmtversion> of upLaTeX.^^J^^J%
%   !! Please check if iniptex found an old patch file:^^J%
%   !! --- if so, rename it or delete it, and redo the^^J%
%   !!     iniptex run.^^J%
%   !!!!!!!!!!!!!!!!!!!!!!!!!!!!!!!!!!!!!!!!!!!!!!!!!!!!!!!^^J}%
%      \batchmode \@@end
%    \fi
%  \else
%      \typeout{^^J^^J^^J%
%   !!!!!!!!!!!!!!!!!!!!!!!!!!!!!!!!!!!!!!!!!!!!!!!!!!!!!!!^^J%
%   !! Patch file `uplpatch.ltx' (for version <\pfmtversion@topatch>)^^J%
%   !! is not suitable for version <\pfmtversion> of upLaTeX.^^J%
%   !!^^J%
%   !! Please check if iniptex found an old patch file:^^J%
%   !! --- if so, rename it or delete it, and redo the^^J%
%   !!     iniptex run.^^J%
%   !!!!!!!!!!!!!!!!!!!!!!!!!!!!!!!!!!!!!!!!!!!!!!!!!!!!!!!^^J}%
%      \batchmode \@@end
%  \fi
%  \let\pfmtversion@topatch\relax
%  }{}
%    \end{macrocode}
%
% \subsection{起動時に表示するバナー}
%
% \begin{macro}{\everyjob}
% 起動時に表示される文字列です。
% \LaTeX{}にパッチがあてられている場合は、それも表示します。
%
%\iffalse
% この実装については\file{uplatex.dtx}のコメントを参照。(2016/09/14)
%\fi
%
% \changes{v1.0v}{2016/05/07}{起動時の文字列を最新の\LaTeX{}に合わせた。}
% \changes{v1.0w}{2016/05/12}{起動時の文字列に入れる\LaTeX{}のバージョンを
%    元の\LaTeX{}のバナーから引き継ぐように改良}
% \changes{v1.1}{2016/09/14}{起動時のバナーを取得するコードを改良}
% \changes{v1.1d}{2017/09/24}{パッチレベルが負の数の場合をpre-release扱いへ}
%    \begin{macrocode}
\ifx\patch@level\@undefined % fallback if undefined in LaTeX
  \def\patch@level{0}\fi
\ifx\ppatch@level\@undefined % fallback if undefined in upLaTeX
  \def\ppatch@level{0}\fi
\begingroup
  \def\parse@@BANNER\typeout#1\typeout#2#3\relax{#1}
  \edef\platexTMP{%
    \ifnum\ppatch@level=0
      \everyjob{\noexpand\typeout{%
        \pfmtname\space<\pfmtversion>\space
          (based on \expandafter\parse@@BANNER\platexBANNER)}}%
    \else\ifnum\ppatch@level>0
      \everyjob{\noexpand\typeout{%
        \pfmtname\space<\pfmtversion>+\ppatch@level\space
          (based on \expandafter\parse@@BANNER\platexBANNER)}}%
    \else
      \everyjob{\noexpand\typeout{%
        \pfmtname\space<\pfmtversion>-pre\ppatch@level\space
          (based on \expandafter\parse@@BANNER\platexBANNER)}}%
    \fi\fi
  }
\expandafter
\endgroup \platexTMP
%    \end{macrocode}
%
% p\LaTeX{}やup\LaTeX{}は、独自のハイフネーション・パターンを定義していません。
% \TeX\ Liveの標準的インストールでは、代わりに\LaTeX{}が読み込んでいる
% Babelパッケージのものが適用されるはずですから、起動時の文字列にも
% \file{hyphen.cfg}のバージョンを反映します(Babelパッケージの
% \file{hyphen.cfg}でない場合は、何も表示されず空行になるはずです)。
%
%\iffalse
% この実装については\file{uplatex.dtx}のコメントを参照。(2016/09/14)
%\fi
%
% \changes{v1.0w-u00}{2016/05/12}{起動時の文字列に入れるBabelのバージョンを
%    元の\LaTeX{}のバナーから取得するコードを\file{uplatex.ini}から取り入れた}
%    \begin{macrocode}
\begingroup
  \def\parse@@BANNER\typeout#1\typeout#2#3\relax{#2}
  \edef\platexTMP{%
    \the\everyjob\noexpand\typeout{\expandafter\parse@@BANNER\platexBANNER}%
  }
  \everyjob=\expandafter{\platexTMP}%
  \edef\platexTMP{%
    \noexpand\let\noexpand\platexBANNER=\noexpand\@undefined
    \noexpand\everyjob={\the\everyjob}%
  }
  \expandafter
\endgroup \platexTMP
%</plfinal>
%    \end{macrocode}
% \end{macro}
%
% ^^A 起動時に\file{uplatex.cfg}がある場合、それを読み込むようにする
% ^^A コードは、\file{uplcore.ltx}から\file{uplatex.ltx}へ移動しました。
% \changes{v1.0y-u01}{2016/06/29}{\file{uplatex.cfg}の読み込みを追加}
% \changes{v1.0z-u01}{2016/08/26}{\file{uplatex.cfg}の読み込みを
%    \file{uplcore.ltx}から\file{uplatex.ltx}へ移動}
%
% \subsection{ハイフネーション関連}
%
% \begin{macro}{\l@nohyphenation}
% \LaTeXe\ 2017-04-15で、|\verb|の途中でハイフネーションが起きないように
% する修正が入りました。この修正には|\l@nohyphenation|が定義済みでなければ
% なりませんが、通常はBabelの定義ファイルによって提供されています。
% \LaTeXe{}は特殊な状況も想定してltfinalで対策しているようですので、
% p\LaTeXe{}も念のためplfinalで対策します(参考:latex2e svn r1405)。
% \changes{v1.1b}{2017/03/19}{\cs{l@nohyphenation}の定義を保証
%    (sync with ltfinal 2017/03/09 v2.0t)}
%    \begin{macrocode}
%<*plfinal>
\ifx\l@nohyphenation \@undefined
  \newlanguage\l@nohyphenation
\fi
%    \end{macrocode}
% \end{macro}
%
% \begin{macro}{\document@default@language}
% \LaTeXe\ 2017-04-15で導入されたパラメータです。更新タイミングのずれの
% 可能性を考慮し、p\LaTeXe{}でも準備しておきます。verbatim環境の途中で
% 改ページが起きた場合にヘッダでハイフネーションが抑制されないように、
% |\@outputpage|で|\language|をリセットするときに使われます
% (参考:latex2e svn r1407)。
% \changes{v1.1b}{2017/03/19}{\cs{document@default@language}の定義を保証
%    (sync with ltfinal 2017/03/09 v2.0t)}
%    \begin{macrocode}
\ifx\document@default@language \@undefined
  \let\document@default@language\m@ne
\fi
%</plfinal>
%    \end{macrocode}
% \end{macro}
%
% \subsection{latexreleaseパッケージへの対応}
%
% 最後に、latexreleaseパッケージへの対応です。
% \begin{macro}{\plIncludeInRelease}
% \changes{v1.0t}{2016/02/03}{\cs{plIncludeInRelease}と
%    \cs{plEndIncludeInRelease}を新設。}
%    \begin{macrocode}
%<*plcore|platexrelease>
\def\plIncludeInRelease#1{\kernel@ifnextchar[%
  {\@plIncludeInRelease{#1}}
  {\@plIncludeInRelease{#1}[#1]}}
%    \end{macrocode}
%
%    \begin{macrocode}
\def\@plIncludeInRelease#1[#2]{\@plIncludeInRele@se{#2}}
%    \end{macrocode}
%
%    \begin{macrocode}
\def\@plIncludeInRele@se#1#2#3{%
  \toks@{[#1] #3}%
  \expandafter\ifx\csname\string#2+\@currname+IIR\endcsname\relax
    \ifnum\expandafter\@parse@version#1//00\@nil
          >\expandafter\@parse@version\pfmtversion//00\@nil
      \GenericInfo{}{Skipping: \the\toks@}%
     \expandafter\expandafter\expandafter\@gobble@plIncludeInRelease
    \else
      \GenericInfo{}{Applying: \the\toks@}%
      \expandafter\let\csname\string#2+\@currname+IIR\endcsname\@empty
    \fi
  \else
    \GenericInfo{}{Already applied: \the\toks@}%
    \expandafter\@gobble@plIncludeInRelease
  \fi
}
%    \end{macrocode}
%
%    \begin{macrocode}
\long\def\@gobble@plIncludeInRelease#1\plEndIncludeInRelease{}
\let\plEndIncludeInRelease\relax
%</plcore|platexrelease>
%    \end{macrocode}
% \end{macro}
%
% \LaTeXe{}が提供するlatexreleaseパッケージが読み込まれていて、
% かつp\LaTeXe{}が提供するplatexreleaseパッケージが読み込まれていない
% 場合は、警告を出します。
% \changes{v1.0s}{2016/02/01}{latexrelease利用時に警告を出すようにした}
%    \begin{macrocode}
%<*plfinal>
\AtBeginDocument{%
  \@ifpackageloaded{latexrelease}{%
    \@ifpackageloaded{platexrelease}{}{%
      \@latex@warning@no@line{%
        Package latexrelease is loaded.\MessageBreak
        Some patches in pLaTeX2e core may be overwritten.\MessageBreak
        Consider using platexrelease.\MessageBreak
        See platex.pdf for detail}%
    }%
  }{}%
}
%</plfinal>
%    \end{macrocode}
%
% \Finale
%
\endinput

   % \iffalse meta-comment
%% File: uplfonts.dtx
%
%    pLaTeX fonts files:
%       Copyright 1994-2006 ASCII Corporation.
%    and modified for upLaTeX
%
%  Copyright (c) 2010 ASCII MEDIA WORKS
%  Copyright (c) 2016 Takuji Tanaka
%  Copyright (c) 2016-2020 Japanese TeX Development Community
%
%  This file is part of the upLaTeX2e system (community edition).
%  --------------------------------------------------------------
%
% \fi
%
% \iffalse
%<*driver>
\ifx\JAPANESEtrue\undefined
  \expandafter\newif\csname ifJAPANESE\endcsname
  \JAPANESEtrue
\fi
\def\eTeX{$\varepsilon$-\TeX}
\def\pTeX{p\kern-.15em\TeX}
\def\epTeX{$\varepsilon$-\pTeX}
\def\pLaTeX{p\kern-.05em\LaTeX}
\def\pLaTeXe{p\kern-.05em\LaTeXe}
\def\upTeX{u\pTeX}
\def\eupTeX{$\varepsilon$-\upTeX}
\def\upLaTeX{u\pLaTeX}
\def\upLaTeXe{u\pLaTeXe}
%</driver>
% \fi
%
% \setcounter{StandardModuleDepth}{1}
% \StopEventually{}
%
% \iffalse
% \changes{v1.5-u00}{2011/05/07}{p\LaTeX{}用からup\LaTeX{}用に修正。
%     (based on plfonts.dtx 2006/11/10 v1.5)}
% \changes{v1.6a-u00}{2016/04/06}{p\LaTeX{}の変更に追随。
%     (based on plfonts.dtx 2016/04/01 v1.6a)}
% \changes{v1.6b-u00}{2016/04/30}{uptrace.styの冒頭でtracefnt.styを
%    \cs{RequirePackageWithOptions}するようにした
%     (based on plfonts.dtx 2016/04/30 v1.6b)}
% \changes{v1.6c-u00}{2016/06/06}{p\LaTeX{}の変更に追随。
%     (based on plfonts.dtx 2016/06/06 v1.6c)}
% \changes{v1.6d-u00}{2016/06/19}{p\LaTeX{}の変更に追随。
%     (based on plfonts.dtx 2016/06/19 v1.6d)}
% \changes{v1.6e-u00}{2016/06/29}{p\LaTeX{}の変更に追随。
%     (based on plfonts.dtx 2016/06/26 v1.6e)}
% \changes{v1.6f-u00}{2017/03/05}{uptrace.styのplatexrelease対応
%     (based on plfonts.dtx 2017/02/20 v1.6f)}
% \changes{v1.6g-u00}{2017/03/08}{p\LaTeX{}の変更に追随。
%     (based on plfonts.dtx 2017/03/07 v1.6g)}
% \changes{v1.6h-u00}{2017/08/05}{p\LaTeX{}の変更に追随。
%     (based on plfonts.dtx 2017/08/05 v1.6h)}
% \changes{v1.6i-u00}{2017/09/24}{p\LaTeX{}の変更に追随。
%     (based on plfonts.dtx 2017/09/24 v1.6i)}
% \changes{v1.6j-u00}{2017/11/06}{p\LaTeX{}の変更に追随。
%     (based on plfonts.dtx 2017/11/06 v1.6j)}
% \changes{v1.6k-u00}{2017/12/05}{デフォルト設定ファイルの読み込みを
%    \file{uplcore.ltx}から\file{uplatex.ltx}へ移動
%     (based on plfonts.dtx 2017/12/05 v1.6k)}
% \changes{v1.6k-u01}{2017/12/10}{uptraceパッケージは
%    ptraceパッケージを読み込むだけとした}
% \changes{v1.6k-u02}{2017/12/10}{p\LaTeX{}との統合のため、
%    up\LaTeX{}用の最小限の変更だけを定義するようにした}
% \changes{v1.6l-u02}{2018/02/04}{p\LaTeX{}の変更に追随。
%     (based on plfonts.dtx 2018/02/04 v1.6l)}
% \changes{v1.6q-u02}{2018/07/03}{p\LaTeX{}の変更に追随。
%     (based on plfonts.dtx 2018/07/03 v1.6q)}
% \changes{v1.6t-u02}{2019/09/22}{p\LaTeX{}の変更に追随。
%     (based on plfonts.dtx 2019/09/16 v1.6t)}
% \changes{v1.6v-u02}{2020/02/01}{p\LaTeX{}の変更に追随。
%     (based on plfonts.dtx 2020/02/01 v1.6v)}
% \fi
%
% \iffalse
%<*driver>
\NeedsTeXFormat{pLaTeX2e}
% \fi
\ProvidesFile{uplfonts.dtx}[2020/02/01 v1.6v-u02 upLaTeX New Font Selection Scheme]
% \iffalse
\documentclass{jltxdoc}
\GetFileInfo{uplfonts.dtx}
\title{up\LaTeXe{}のフォントコマンド\space\fileversion}
\author{Ken Nakano \& Hideaki Togashi \& TTK}
\date{作成日:\filedate}
\begin{document}
   \maketitle
   \tableofcontents
   \DocInput{\filename}
\end{document}
%</driver>
% \fi
%
% \section{概要}\label{plfonts:intro}
% ここでは、和文書体を\NFSS2のインターフェイスで選択するための
% コマンドやマクロについて説明をしています。
% また、フォント定義ファイルや初期設定ファイルなどの説明もしています。
% 新しいフォント選択コマンドの使い方については、\file{fntguide.tex}や
% \file{usrguide.tex}を参照してください。
% \changes{v1.5-u00}{2011/05/07}{p\LaTeX{}用からup\LaTeX{}用に修正。
%     (based on plfonts.dtx 2006/11/10 v1.5)}
% \changes{v1.6k-u02}{2017/12/10}{p\LaTeX{}との統合のため、
%    up\LaTeX{}用の最小限の変更だけを定義するようにした}
%
% \begin{description}
% \item[第\ref{plfonts:intro}節] この節です。このファイルの概要と
%    \dst{}プログラムのためのオプションを示しています。
% \item[第\ref{plfonts:codes}節] 実際のコードの部分です。
% \item[第\ref{plfonts:pldefs}節] プリロードフォントやエラーフォントなどの
%  初期設定について説明をしています。
% \item[第\ref{plfonts:fontdef}節] フォント定義ファイルについて
%    説明をしています。
% \end{description}
%
%
% \subsection{\dst{}プログラムのためのオプション}
% \dst{}プログラムのためのオプションを次に示します。
%
% \DeleteShortVerb{\|}
% \begin{center}
% \begin{tabular}{l|p{0.7\linewidth}}
% \emph{オプション} & \emph{意味}\\\hline
% plcore & \file{uplcore.ltx}の断片を生成するオプションでしたが、削除。\\
% trace  & \file{uptrace.sty}を生成します。\\
% JY2mc  & 横組用、明朝体のフォント定義ファイルを生成します。\\
% JY2gt  & 横組用、ゴシック体のフォント定義ファイルを生成します。\\
% JT2mc  & 縦組用、明朝体のフォント定義ファイルを生成します。\\
% JT2gt  & 縦組用、ゴシック体のフォント定義ファイルを生成します。\\
% pldefs & \file{upldefs.ltx}を生成します。次の4つのオプションを付加する
%          ことで、プリロードするフォントを選択することができます。
%          デフォルトは10ptです。\\
% xpt    & 10pt プリロード\\
% xipt   & 11pt プリロード\\
% xiipt  & 12pt プリロード\\
% ori    & \file{plfonts.tex}に似たプリロード\\
% \end{tabular}
% \end{center}
% \MakeShortVerb{\|}
%
%
%
% \section{コード}\label{plfonts:codes}
% \NFSS2の拡張は、p\LaTeX{}において\file{plfonts.dtx}から生成される
% \file{plcore.ltx}をそのまま利用するので、up\LaTeX{}では定義しません。
% 後方互換性のため、\file{uptrace.sty}を提供しますが、
% これも単に\file{ptrace.sty}を読み込むだけとします。
%
% \changes{v1.6b-u00}{2016/04/30}{uptrace.styの冒頭でtracefnt.styを
%    \cs{RequirePackageWithOptions}するようにした}
% \changes{v1.6k-u01}{2017/12/10}{uptraceパッケージは
%    ptraceパッケージを読み込むだけとした}
%    \begin{macrocode}
%<*trace>
\NeedsTeXFormat{pLaTeX2e}
\ProvidesPackage{uptrace}
     [2019/09/22 v1.6t-u02 Standard upLaTeX package (font tracing)]
\RequirePackageWithOptions{ptrace}
%</trace>
%    \end{macrocode}
%
% デフォルト設定ファイル\file{upldefs.ltx}は、もともと\file{uplcore.ltx}の途中で
% 読み込んでいましたが、2018年以降の新しいコミュニティ版\upLaTeX{}では
% \file{uplatex.ltx}から読み込むことにしました。
% 実際の中身については、第\ref{plfonts:pldefs}節を参照してください。
% \changes{v1.6k-u00}{2017/12/05}{デフォルト設定ファイルの読み込みを
%    \file{uplcore.ltx}から\file{uplatex.ltx}へ移動
%     (based on plfonts.dtx 2017/12/05 v1.6k)}
%
%
% \section{デフォルト設定ファイル}\label{plfonts:pldefs}
% ここでは、フォーマットファイルに読み込まれるデフォルト値を設定しています。
% この節での内容は\file{upldefs.ltx}に出力されます。
% このファイルの内容を\file{uplcore.ltx}に含めてもよいのですが、
% デフォルトの設定を参照しやすいように、別ファイルにしてあります。
%
% プリロードサイズは、\dst{}プログラムのオプションで変更することができます。
% これ以外の設定を変更したい場合は、\file{upldefs.ltx}を
% 直接、修正するのではなく、このファイルを\file{upldefs.cfg}という名前で
% コピーをして、そのファイルに対して修正を加えるようにしてください。
%    \begin{macrocode}
%<*pldefs>
\ProvidesFile{upldefs.ltx}
      [2020/02/01 v1.6v-u02 upLaTeX Kernel (Default settings)]
%</pldefs>
%    \end{macrocode}
%
% \subsection{テキストフォント}
% テキストフォントのための属性やエラー書体などの宣言です。
% p\LaTeX{}のデフォルトの横組エンコードはJY1、縦組エンコードはJT1ですが、
% up\LaTeX{}では横組エンコードはJY2、縦組エンコードはJT2とします。
%
% \changes{v1.6s}{2019/08/13}{Explicitly set some defaults
%    after \cs{DeclareErrorKanjiFont} change
%    (sync with ltfssini.dtx 2019/07/09 v3.1c)}
% \noindent
% 縦横エンコード共通:
%    \begin{macrocode}
%<*pldefs>
\DeclareKanjiEncodingDefaults{}{}
\DeclareErrorKanjiFont{JY2}{mc}{m}{n}{10}
\kanjifamily{mc}
\kanjiseries{m}
\kanjishape{n}
\fontsize{10}{10}
%    \end{macrocode}
% 横組エンコード:
%    \begin{macrocode}
\DeclareYokoKanjiEncoding{JY2}{}{}
\DeclareKanjiSubstitution{JY2}{mc}{m}{n}
%    \end{macrocode}
% 縦組エンコード:
%    \begin{macrocode}
\DeclareTateKanjiEncoding{JT2}{}{}
\DeclareKanjiSubstitution{JT2}{mc}{m}{n}
%    \end{macrocode}
% 縦横のエンコーディングのセット化:
% \changes{v1.6j}{2017/11/06}{縦横のエンコーディングのセット化を
%    plcoreからpldefsへ移動}
%    \begin{macrocode}
\KanjiEncodingPair{JY2}{JT2}
%    \end{macrocode}
% フォント属性のデフォルト値:
% \LaTeXe~2019-10-01までは|\shapedefault|は|\updefault|でしたが、
% \LaTeXe~2020-02-02で|\updefault|が``n''から``up''へと修正されたことに
% 伴い、|\shapedefault|は明示的に``n''に設定されました。
% \changes{v1.6v}{2020/02/01}{Set \cs{kanjishapedefault} explicitly to ``n''
%    (sync with fontdef.dtx 2019/12/17 v3.0e)}
%    \begin{macrocode}
\newcommand\mcdefault{mc}
\newcommand\gtdefault{gt}
\newcommand\kanjiencodingdefault{JY2}
\newcommand\kanjifamilydefault{\mcdefault}
\newcommand\kanjiseriesdefault{\mddefault}
\newcommand\kanjishapedefault{n}% formerly \updefault
%    \end{macrocode}
% 和文エンコードの指定:
%    \begin{macrocode}
\kanjiencoding{JY2}
%    \end{macrocode}
% フォント定義:
% これらの具体的な内容は第\ref{plfonts:fontdef}節を参照してください。
% \changes{v1.3}{1997/01/24}{Rename font definition filename.}
%    \begin{macrocode}
\input{jy2mc.fd}
\input{jy2gt.fd}
\input{jt2mc.fd}
\input{jt2gt.fd}
%    \end{macrocode}
% フォントを有効にします。
%    \begin{macrocode}
\fontencoding{JT2}\selectfont
\fontencoding{JY2}\selectfont
%    \end{macrocode}
%
% \changes{v1.3b}{1997/01/30}{数式用フォントの宣言をクラスファイルに移動した}
%
%
% \subsection{プリロードフォント}
% あらかじめフォーマットファイルにロードされるフォントの宣言です。
% \dst{}プログラムのオプションでロードされるフォントのサイズを
% 変更することができます。\file{uplfmt.ins}では|xpt|を指定しています。
%    \begin{macrocode}
%<*xpt>
\DeclarePreloadSizes{JY2}{mc}{m}{n}{5,7,10,12}
\DeclarePreloadSizes{JY2}{gt}{m}{n}{5,7,10,12}
\DeclarePreloadSizes{JT2}{mc}{m}{n}{5,7,10,12}
\DeclarePreloadSizes{JT2}{gt}{m}{n}{5,7,10,12}
%</xpt>
%<*xipt>
\DeclarePreloadSizes{JY2}{mc}{m}{n}{5,7,10.95,12}
\DeclarePreloadSizes{JY2}{gt}{m}{n}{5,7,10.95,12}
\DeclarePreloadSizes{JT2}{mc}{m}{n}{5,7,10.95,12}
\DeclarePreloadSizes{JT2}{gt}{m}{n}{5,7,10.95,12}
%</xipt>
%<*xiipt>
\DeclarePreloadSizes{JY2}{mc}{m}{n}{7,9,12,14.4}
\DeclarePreloadSizes{JY2}{gt}{m}{n}{7,9,12,14.4}
\DeclarePreloadSizes{JT2}{mc}{m}{n}{7,9,12,14.4}
\DeclarePreloadSizes{JT2}{gt}{m}{n}{7,9,12,14.4}
%</xiipt>
%<*ori>
\DeclarePreloadSizes{JY2}{mc}{m}{n}
        {5,6,7,8,9,10,10.95,12,14.4,17.28,20.74,24.88}
\DeclarePreloadSizes{JY2}{gt}{m}{n}
        {5,6,7,8,9,10,10.95,12,14.4,17.28,20.74,24.88}
\DeclarePreloadSizes{JT2}{mc}{m}{n}
        {5,6,7,8,9,10,10.95,12,14.4,17.28,20.74,24.88}
\DeclarePreloadSizes{JT2}{gt}{m}{n}
        {5,6,7,8,9,10,10.95,12,14.4,17.28,20.74,24.88}
%</ori>
%    \end{macrocode}
%
%
% \subsection{組版パラメータ}
% 禁則パラメータや文字間へ挿入するスペースの設定などです。
% 実際の各文字への禁則パラメータおよびスペースの挿入の許可設定などは、
% \file{ukinsoku.tex}で行なっています。
% 具体的な設定については、\file{ukinsoku.dtx}を参照してください。
%    \begin{macrocode}
\InputIfFileExists{ukinsoku.tex}%
  {\message{Loading kinsoku patterns for japanese.}}
  {\errhelp{The configuration for kinsoku is incorrectly installed.^^J%
            If you don't understand this error message you need
            to seek^^Jexpert advice.}%
   \errmessage{OOPS! I can't find any kinsoku patterns for japanese^^J%
               \space Think of getting some or the
               uplatex2e setup will never succeed}\@@end}
%    \end{macrocode}
%
% 組版パラメータの設定をします。
% |\kanjiskip|は、漢字と漢字の間に挿入されるグルーです。
% |\noautospacing|で、挿入を中止することができます。
% デフォルトは|\autospacing|です。
%    \begin{macrocode}
\kanjiskip=0pt plus .4pt minus .5pt
\autospacing
%    \end{macrocode}
% |\xkanjiskip|は、和欧文間に自動的に挿入されるグルーです。
% |\noautoxspacing|で、挿入を中止することができます。
% デフォルトは|\autoxspacing|です。
% \changes{v1.1c}{1995/09/12}{\cs{xkanjiskip}のデフォルト値}
%    \begin{macrocode}
\xkanjiskip=.25zw plus1pt minus1pt
\autoxspacing
%    \end{macrocode}
% |\jcharwidowpenalty|は、パラグラフに対する禁則です。
% パラグラフの最後の行が1文字だけにならないように調整するために使われます。
%    \begin{macrocode}
\jcharwidowpenalty=500
%    \end{macrocode}
%
% ここまでが、\file{pldefs.ltx}の内容です。
%    \begin{macrocode}
%</pldefs>
%    \end{macrocode}
%
%
%
% \section{フォント定義ファイル}\label{plfonts:fontdef}
% \changes{v1.3}{1997/01/24}{Rename provided font definition filename.}
% ここでは、フォント定義ファイルの設定をしています。フォント定義ファイルは、
% \LaTeX{}のフォント属性を\TeX{}フォントに置き換えるためのファイルです。
% 記述方法についての詳細は、|fntguide.tex|を参照してください。
%
% 欧文書体の設定については、
% \file{cmfonts.fdd}や\file{slides.fdd}などを参照してください。
% \file{skfonts.fdd}には、写研代用書体を使うためのパッケージと
% フォント定義が記述されています。
%    \begin{macrocode}
%<JY2mc>\ProvidesFile{jy2mc.fd}
%<JY2gt>\ProvidesFile{jy2gt.fd}
%<JT2mc>\ProvidesFile{jt2mc.fd}
%<JT2gt>\ProvidesFile{jt2gt.fd}
%<JY2mc,JY2gt,JT2mc,JT2gt>       [2018/07/03 v1.6q-u02 KANJI font defines]
%    \end{macrocode}
% 横組用、縦組用ともに、
% 明朝体のシリーズ|bx|がゴシック体となるように宣言しています。
% \changes{v1.2}{1995/11/24}{it, sl, scの宣言を外した}
% \changes{v1.3b}{1997/01/29}{フォント定義ファイルのサイズ指定の調整}
% \changes{v1.3b}{1997/03/11}{すべてのサイズをロード可能にした}
% また、シリーズ|b|は同じ書体の|bx|と等価になるように宣言します。
% \changes{v1.6q}{2018/07/03}{シリーズbがbxと等価になるように宣言}
%
% p\LaTeX{}では従属書体にOT1エンコーディングを指定していましたが、
% up\LaTeX{}ではT1エンコーディングを用いるように変更しました。
% また、要求サイズ(指定されたフォントサイズ)が10ptのとき、
% 全角幅の実寸が9.62216ptとなるようにしますので、
% 和文スケール値($1\,\mathrm{zw} \div \textmc{要求サイズ}$)は
% $9.62216\,\mathrm{pt}/10\,\mathrm{pt}=0.962216$です。
% upjis系のメトリックは全角幅が10ptでデザインされているので、
% これを0.962216倍で読込みます。
% \changes{v1.6l}{2018/02/04}{和文スケール値を明文化}
%    \begin{macrocode}
%<*JY2mc>
\DeclareKanjiFamily{JY2}{mc}{}
\DeclareRelationFont{JY2}{mc}{m}{}{T1}{cmr}{m}{}
\DeclareRelationFont{JY2}{mc}{bx}{}{T1}{cmr}{bx}{}
\DeclareFontShape{JY2}{mc}{m}{n}{<->s*[0.962216]upjisr-h}{}
\DeclareFontShape{JY2}{mc}{bx}{n}{<->ssub*gt/m/n}{}
\DeclareFontShape{JY2}{mc}{b}{n}{<->ssub*mc/bx/n}{}
%</JY2mc>
%<*JT2mc>
\DeclareKanjiFamily{JT2}{mc}{}
\DeclareRelationFont{JT2}{mc}{m}{}{T1}{cmr}{m}{}
\DeclareRelationFont{JT2}{mc}{bx}{}{T1}{cmr}{bx}{}
\DeclareFontShape{JT2}{mc}{m}{n}{<->s*[0.962216]upjisr-v}{}
\DeclareFontShape{JT2}{mc}{bx}{n}{<->ssub*gt/m/n}{}
\DeclareFontShape{JT2}{mc}{b}{n}{<->ssub*mc/bx/n}{}
%</JT2mc>
%<*JY2gt>
\DeclareKanjiFamily{JY2}{gt}{}
\DeclareRelationFont{JY2}{gt}{m}{}{T1}{cmr}{bx}{}
\DeclareFontShape{JY2}{gt}{m}{n}{<->s*[0.962216]upjisg-h}{}
\DeclareFontShape{JY2}{gt}{bx}{n}{<->ssub*gt/m/n}{}
\DeclareFontShape{JY2}{gt}{b}{n}{<->ssub*gt/bx/n}{}
%</JY2gt>
%<*JT2gt>
\DeclareKanjiFamily{JT2}{gt}{}
\DeclareRelationFont{JT2}{gt}{m}{}{T1}{cmr}{bx}{}
\DeclareFontShape{JT2}{gt}{m}{n}{<->s*[0.962216]upjisg-v}{}
\DeclareFontShape{JT2}{gt}{bx}{n}{<->ssub*gt/m/n}{}
\DeclareFontShape{JT2}{gt}{b}{n}{<->ssub*gt/bx/n}{}
%</JT2gt>
%    \end{macrocode}
%
%
% \Finale
%
\endinput

   % \iffalse meta-comment
%% File: ukinsoku.dtx
%
%    pLaTeX kinsoku file:
%       Copyright 1995 ASCII Corporation.
%    and modified for upLaTeX
%
%  Copyright (c) 2010 ASCII MEDIA WORKS
%  Copyright (c) 2016 Takuji Tanaka
%  Copyright (c) 2016-2021 Japanese TeX Development Community
%
%  This file is part of the upLaTeX2e system (community edition).
%  --------------------------------------------------------------
%
% \fi
%
%
% \setcounter{StandardModuleDepth}{1}
% \StopEventually{}
%
% \iffalse
% \changes{v1.0-u00}{2011/05/07}{p\LaTeX{}用からup\LaTeX{}用に修正。}
% \changes{v1.0-u01}{2017/08/02}{U+00B7 (MIDDLE DOT; JIS X 0213)の
%    前禁則ペナルティをU+30FBと同じ値に設定、注意点を明文化}
% \changes{v1.0b}{2017/08/05}{%、&、\%、\&の禁則ペナルティが
%      誤っていたのを修正(post $\rightarrow$ pre)}
% \changes{v1.0b-u01}{2017/08/05}{p\LaTeX{}の変更に追随}
% \changes{v1.0b-u02}{2018/01/27}{up\TeX{}の将来の変更に備え、
%      Latin-1 Supplementのうち属性がLatinのもの
%      (Latin-1 letters)をコードポイントで指定}
% \changes{v1.0b-u03}{2018/04/08}{\LaTeX\ 2018-04-01対策}
% \changes{v1.0b-u04}{2019/01/29}{内部Unicode化されていることを確認}
% \changes{v1.0b-u05}{2019/05/19}{up\TeX~v1.24の\cs{kcatcode}の既定値のバグ回避}
% \changes{v1.0b-u06}{2019/09/22}{バグ回避コードがかえって有害なため除去}
% \changes{v1.0c}{2020/09/28}{!の\cs{inhibitxspcode}を設定}
% \changes{v1.0c-u06}{2020/09/28}{p\LaTeX{}の変更に追随}
% \changes{v1.0d}{2021/03/04}{:の\cs{inhibitxspcode}と:の\cs{xspcode}を設定}
% \changes{v1.0d-u06}{2021/03/04}{p\LaTeX{}の変更に追随}
% \fi
%
% \iffalse
%<*driver>
\NeedsTeXFormat{pLaTeX2e}
% \fi
\ProvidesFile{ukinsoku.dtx}[2021/03/04 v1.0d-u06 upLaTeX Kernel]
% \iffalse
\documentclass{jltxdoc}
\GetFileInfo{ukinsoku.dtx}
\title{禁則パラメータ\space\fileversion}
\author{Ken Nakano \& TTK}
\date{作成日:\filedate}
\begin{document}
   \maketitle
   \DocInput{\filename}
\end{document}
%</driver>
% \fi
%
% このファイルは、禁則と文字間スペースの設定について説明をしています。
% 日本語\TeX{}の機能についての詳細は、『日本語\TeX テクニカルブックI』を
% 参照してください。
%
% なお、このファイルのコード部分は、
% p\TeX{}やp\LaTeX{}で配布されている\file{kinsoku.tex}に、
% JIS X 0213の定義文字などの設定を追加したものです。
% このファイルは内部コードUnicode (|uptex|)なup\TeX{}エンジンで読まれる
% 必要があります。
% \changes{v1.0-u00}{2011/05/07}{p\LaTeX{}用からup\LaTeX{}用に修正。}
% \changes{v1.0b-u04}{2019/01/29}{内部コードがUnicodeであることを確認}
%
%    \begin{macrocode}
%<*plcore>
\ifnum\ucs"3000="3000 \else
    \errhelp{Please try to run (e)uptex with option
             `-kanji-internal=uptex'.}%
    \errmessage{This file should be read with
                internal Kanji encoding Unicode}\@@end
\fi
%    \end{macrocode}
%
% \changes{v1.0b-u05}{2019/05/19}{up\TeX~v1.24の\cs{kcatcode}の既定値のバグ回避}
% \changes{v1.0b-u06}{2019/09/22}{バグ回避コードがかえって有害なため除去}
%
% \section{禁則}
%
% ある文字を行頭禁則の対象にするには、|\prebreakpenalty|に正の値を指定します。
% ある文字を行末禁則の対象にするには、|\postbreakpenalty|に正の値を指定します。
% 数値が大きいほど、行頭、あるいは行末で改行されにくくなります。
%
% \subsection{半角文字に対する禁則}
% ここでは、半角文字に対する禁則の設定を行なっています。
% \changes{v1.0b}{2017/08/05}{%、&、\%、\&の禁則ペナルティが
%      誤っていたのを修正(post $\rightarrow$ pre)}
%
%    \begin{macrocode}
%%
%% 行頭、行末禁則パラメータ
%%
%% 1byte characters
\prebreakpenalty`!=10000
\prebreakpenalty`"=10000
\postbreakpenalty`\#=500
\postbreakpenalty`\$=500
\prebreakpenalty`\%=500
\prebreakpenalty`\&=500
\postbreakpenalty`\`=10000
\prebreakpenalty`'=10000
\prebreakpenalty`)=10000
\postbreakpenalty`(=10000
\prebreakpenalty`*=500
\prebreakpenalty`+=500
\prebreakpenalty`-=10000
\prebreakpenalty`.=10000
\prebreakpenalty`,=10000
\prebreakpenalty`/=500
\prebreakpenalty`;=10000
\prebreakpenalty`?=10000
\prebreakpenalty`:=10000
\prebreakpenalty`]=10000
\postbreakpenalty`[=10000
%    \end{macrocode}
%
% \subsection{全角文字に対する禁則}
% ここでは、全角文字に対する禁則の設定を行なっています。
%
% up\TeX{}/up\LaTeX{}の場合、JIS X 0213(日本)・KS C 5601(韓国)・
% GB2312(中国)・Big5(台湾)などの文字集合に含まれる、
% いわゆる全角文字の一部が、8-bit Latinと同じコードポイントを
% 共有します。すなわち、同じコードポイントが、CJKトークンとしても
% non-CJKトークンとしても有効に扱われることがあります。
% 以下に例を示します\footnote{ここで表示しているnon-CJKトークンと
% して扱われた結果は、up\LaTeX{}のデフォルト従属欧文エンコーディング
% であるT1の場合のものです。}。
% {\font\lmr=rm-lmr10\lmr
% \begin{itemize}
% \item \texttt{0xA1}: \kchar"A1 (CJK) vs. \char"A1\ (non-CJK)
% \item \texttt{0xAB}: \kchar"AB (CJK) vs. \char"AB\ (non-CJK)
% \item \texttt{0xB7}: \kchar"B7 (CJK) vs. \char"B7\ (non-CJK)
% \item \texttt{0xB9}: \kchar"B9 (CJK) vs. \char"B9\ (non-CJK)
% \item …
% \end{itemize}}
% \file{ukinsoku.tex}ではCJKトークンを優先した禁則設定を行っています。
% この設定により、同じコードポイントをnon-CJKトークンとして扱う場合に
% 予期せずLatin-1の文字が禁則対象になってしまいます。
% 問題が起きた場合は禁則の設定を調整してください。
% \changes{v1.0-u01}{2017/08/02}{U+00B7 (MIDDLE DOT; JIS X 0213)の
%    前禁則ペナルティをU+30FBと同じ値に設定、注意点を明文化}
% \changes{v1.0b-u02}{2018/01/27}{up\TeX{}の将来の変更に備え、
%      Latin-1 Supplementのうち属性がLatinのもの
%      (Latin-1 letters)をコードポイントで指定}
%
% なお、以下で複数回登場する |"AA| と |"BA| はそれぞれªとºですが、
% \LaTeXe\ 2018-04-01でUTF-8入力になった影響で、これらの文字は
% |macrocode| 環境内のコードに(たとえ |%| に続くコメントであっても)
% 書けなくなってしまったようです。これらの文字で
% docstrip処理中にエラー
%\begin{verbatim}
%   ! Argument of \@font@info has an extra }.
%\end{verbatim}
% が出ないように、コメントからも削除しました。
% \changes{v1.0b-u03}{2018/04/08}{\LaTeX\ 2018-04-01対策}
%
% \changes{v1.0d}{2021/03/04}{:の\cs{xspcode}を設定}
%    \begin{macrocode}
%%全角文字
\prebreakpenalty`、=10000
\prebreakpenalty`。=10000
\prebreakpenalty`,=10000
\prebreakpenalty`.=10000
\prebreakpenalty`・=10000
\prebreakpenalty`:=10000
\prebreakpenalty`;=10000
\prebreakpenalty`?=10000
\prebreakpenalty`!=10000
\prebreakpenalty`゛=10000%\jis"212B
\prebreakpenalty`゜=10000%\jis"212C
\prebreakpenalty`´=10000%\jis"212D
\postbreakpenalty``=10000%\jis"212E
\prebreakpenalty`々=10000%\jis"2139
\prebreakpenalty`…=250%\jis"2144
\prebreakpenalty`‥=250%\jis"2145
\postbreakpenalty`‘=10000%\jis"2146
\prebreakpenalty`’=10000%\jis"2147
\postbreakpenalty`“=10000%\jis"2148
\prebreakpenalty`”=10000%\jis"2149
\prebreakpenalty`)=10000
\postbreakpenalty`(=10000
\prebreakpenalty`}=10000
\postbreakpenalty`{=10000
\prebreakpenalty`]=10000
\postbreakpenalty`[=10000
%%\postbreakpenalty`‘=10000
%%\prebreakpenalty`’=10000
\postbreakpenalty`〔=10000%\jis"214C
\prebreakpenalty`〕=10000%\jis"214D
\postbreakpenalty`〈=10000%\jis"2152
\prebreakpenalty`〉=10000%\jis"2153
\postbreakpenalty`《=10000%\jis"2154
\prebreakpenalty`》=10000%\jis"2155
\postbreakpenalty`「=10000%\jis"2156
\prebreakpenalty`」=10000%\jis"2157
\postbreakpenalty`『=10000%\jis"2158
\prebreakpenalty`』=10000%\jis"2159
\postbreakpenalty`【=10000%\jis"215A
\prebreakpenalty`】=10000%\jis"215B
\prebreakpenalty`ー=10000
\prebreakpenalty`+=200
\prebreakpenalty`−=200% U+2212 MINUS SIGN
\prebreakpenalty`-=200% U+FF0D FULLWIDTH HYPHEN-MINUS
\prebreakpenalty`==200
\postbreakpenalty`#=200
\postbreakpenalty`$=200
\prebreakpenalty`%=200
\prebreakpenalty`&=200
\prebreakpenalty`ぁ=150
\prebreakpenalty`ぃ=150
\prebreakpenalty`ぅ=150
\prebreakpenalty`ぇ=150
\prebreakpenalty`ぉ=150
\prebreakpenalty`っ=150
\prebreakpenalty`ゃ=150
\prebreakpenalty`ゅ=150
\prebreakpenalty`ょ=150
\prebreakpenalty`ゎ=150%\jis"246E
\prebreakpenalty`ァ=150
\prebreakpenalty`ィ=150
\prebreakpenalty`ゥ=150
\prebreakpenalty`ェ=150
\prebreakpenalty`ォ=150
\prebreakpenalty`ッ=150
\prebreakpenalty`ャ=150
\prebreakpenalty`ュ=150
\prebreakpenalty`ョ=150
\prebreakpenalty`ヮ=150%\jis"256E
\prebreakpenalty`ヵ=150%\jis"2575
\prebreakpenalty`ヶ=150%\jis"2576
%% kinsoku  JIS X 0208 additional
\prebreakpenalty`ヽ=10000
\prebreakpenalty`ヾ=10000
\prebreakpenalty`ゝ=10000
\prebreakpenalty`ゞ=10000
%%
%% kinsoku  JIS X 0213
%%
\prebreakpenalty`〳=10000
\prebreakpenalty`〴=10000
\prebreakpenalty`〵=10000
\prebreakpenalty`〻=10000
\postbreakpenalty`⦅=10000
\prebreakpenalty`⦆=10000
\postbreakpenalty`⦅=10000
\prebreakpenalty`⦆=10000
\postbreakpenalty`〘=10000
\prebreakpenalty`〙=10000
\postbreakpenalty`〖=10000
\prebreakpenalty`〗=10000
\postbreakpenalty`«=10000
\prebreakpenalty`»=10000
\postbreakpenalty`〝=10000
\prebreakpenalty`〟=10000
\prebreakpenalty`‼=10000
\prebreakpenalty`⁇=10000
\prebreakpenalty`⁈=10000
\prebreakpenalty`⁉=10000
\postbreakpenalty`¡=10000
\postbreakpenalty`¿=10000
\prebreakpenalty`ː=10000
\prebreakpenalty`·=10000
\prebreakpenalty"AA=10000
\prebreakpenalty"BA=10000
\prebreakpenalty`¹=10000
\prebreakpenalty`²=10000
\prebreakpenalty`³=10000
\postbreakpenalty`€=10000
\prebreakpenalty`ゕ=150
\prebreakpenalty`ゖ=150
\prebreakpenalty`ㇰ=150
\prebreakpenalty`ㇱ=150
\prebreakpenalty`ㇲ=150
\prebreakpenalty`ㇳ=150
\prebreakpenalty`ㇴ=150
\prebreakpenalty`ㇵ=150
\prebreakpenalty`ㇶ=150
\prebreakpenalty`ㇷ=150
\prebreakpenalty`ㇸ=150
\prebreakpenalty`ㇹ=150
%%\prebreakpenalty`ㇷ゚=150
\prebreakpenalty`ㇺ=150
\prebreakpenalty`ㇻ=150
\prebreakpenalty`ㇼ=150
\prebreakpenalty`ㇽ=150
\prebreakpenalty`ㇾ=150
\prebreakpenalty`ㇿ=150
%%
%% kinsoku  JIS X 0212
%%
%%\postbreakpenalty`¡=10000
%%\postbreakpenalty`¿=10000
%%\prebreakpenalty"BA=10000
%%\prebreakpenalty"AA=10000
\prebreakpenalty`™=10000
%%
%% kinsoku  半角片仮名
%%
\prebreakpenalty`。=10000
\prebreakpenalty`、=10000
\prebreakpenalty`゙=10000
\prebreakpenalty`゚=10000
\prebreakpenalty`」=10000
\postbreakpenalty`「=10000
%    \end{macrocode}
%
% \section{文字間のスペース}
%
% ある英字の前後と、その文字に隣合う漢字に挿入されるスペースを制御するには、
% |\xspcode|を用います。
%
% ある漢字の前後と、その文字に隣合う英字に挿入されるスペースを制御するには、
% |\inhibitxspcode|を用います。
%
% \subsection{ある英字と前後の漢字の間の制御}
% ここでは、英字に対する設定を行なっています。
%
% 指定する数値とその意味は次のとおりです。
%
% \begin{center}
% \begin{tabular}{ll}
% 0 & 前後の漢字の間での処理を禁止する。\\
% 1 & 直前の漢字との間にのみ、スペースの挿入を許可する。\\
% 2 & 直後の漢字との間にのみ、スペースの挿入を許可する。\\
% 3 & 前後の漢字との間でのスペースの挿入を許可する。\\
% \end{tabular}
% \end{center}
%
%    \begin{macrocode}
%%
%% xspcode
\xspcode`(=1
\xspcode`)=2
\xspcode`[=1
\xspcode`]=2
\xspcode``=1
\xspcode`'=2
\xspcode`:=2
\xspcode`;=2
\xspcode`,=2
\xspcode`.=2
%%  for 8bit Latin
\xspcode"80=3
\xspcode"81=3
\xspcode"82=3
\xspcode"83=3
\xspcode"84=3
\xspcode"85=3
\xspcode"86=3
\xspcode"87=3
\xspcode"88=3
\xspcode"89=3
\xspcode"8A=3
\xspcode"8B=3
\xspcode"8C=3
\xspcode"8D=3
\xspcode"8E=3
\xspcode"8F=3
\xspcode"90=3
\xspcode"91=3
\xspcode"92=3
\xspcode"93=3
\xspcode"94=3
\xspcode"95=3
\xspcode"96=3
\xspcode"97=3
\xspcode"98=3
\xspcode"99=3
\xspcode"9A=3
\xspcode"9B=3
\xspcode"9C=3
\xspcode"9D=3
\xspcode"9E=3
\xspcode"9F=3
\xspcode"A0=3
\xspcode"A1=3
\xspcode"A2=3
\xspcode"A3=3
\xspcode"A4=3
\xspcode"A5=3
\xspcode"A6=3
\xspcode"A7=3
\xspcode"A8=3
\xspcode"A9=3
\xspcode"AA=3
\xspcode"AB=3
\xspcode"AC=3
\xspcode"AD=3
\xspcode"AE=3
\xspcode"AF=3
\xspcode"B0=3
\xspcode"B1=3
\xspcode"B2=3
\xspcode"B3=3
\xspcode"B4=3
\xspcode"B5=3
\xspcode"B6=3
\xspcode"B7=3
\xspcode"B8=3
\xspcode"B9=3
\xspcode"BA=3
\xspcode"BB=3
\xspcode"BC=3
\xspcode"BD=3
\xspcode"BE=3
\xspcode"BF=3
\xspcode"C0=3
\xspcode"C1=3
\xspcode"C2=3
\xspcode"C3=3
\xspcode"C4=3
\xspcode"C5=3
\xspcode"C6=3
\xspcode"C7=3
\xspcode"C8=3
\xspcode"C9=3
\xspcode"CA=3
\xspcode"CB=3
\xspcode"CC=3
\xspcode"CD=3
\xspcode"CE=3
\xspcode"CF=3
\xspcode"D0=3
\xspcode"D1=3
\xspcode"D2=3
\xspcode"D3=3
\xspcode"D4=3
\xspcode"D5=3
\xspcode"D6=3
\xspcode"D7=3
\xspcode"D8=3
\xspcode"D9=3
\xspcode"DA=3
\xspcode"DB=3
\xspcode"DC=3
\xspcode"DD=3
\xspcode"DE=3
\xspcode"DF=3
\xspcode"E0=3
\xspcode"E1=3
\xspcode"E2=3
\xspcode"E3=3
\xspcode"E4=3
\xspcode"E5=3
\xspcode"E6=3
\xspcode"E7=3
\xspcode"E8=3
\xspcode"E9=3
\xspcode"EA=3
\xspcode"EB=3
\xspcode"EC=3
\xspcode"ED=3
\xspcode"EE=3
\xspcode"EF=3
\xspcode"F0=3
\xspcode"F1=3
\xspcode"F2=3
\xspcode"F3=3
\xspcode"F4=3
\xspcode"F5=3
\xspcode"F6=3
\xspcode"F7=3
\xspcode"F8=3
\xspcode"F9=3
\xspcode"FA=3
\xspcode"FB=3
\xspcode"FC=3
\xspcode"FD=3
\xspcode"FE=3
\xspcode"FF=3
%    \end{macrocode}
%
% \subsection{ある漢字と前後の英字の間の制御}
% ここでは、漢字に対する設定を行なっています。
%
% 指定する数値とその意味は次のとおりです。
%
% \begin{center}
% \begin{tabular}{ll}
% 0 & 前後の英字との間にスペースを挿入することを禁止する。\\
% 1 & 直前の英字との間にスペースを挿入することを禁止する。\\
% 2 & 直後の英字との間にスペースを挿入することを禁止する。\\
% 3 & 前後の英字との間でのスペースの挿入を許可する。\\
% \end{tabular}
% \end{center}
%
% \changes{v1.0c}{2020/09/28}{!の\cs{inhibitxspcode}を設定}
% \changes{v1.0d}{2021/03/04}{:の\cs{inhibitxspcode}を設定}
%    \begin{macrocode}
%%
%% inhibitxspcode
\inhibitxspcode`、=1
\inhibitxspcode`。=1
\inhibitxspcode`,=1
\inhibitxspcode`.=1
\inhibitxspcode`:=1
\inhibitxspcode`;=1
\inhibitxspcode`?=1
\inhibitxspcode`!=1
\inhibitxspcode`)=1
\inhibitxspcode`(=2
\inhibitxspcode`]=1
\inhibitxspcode`[=2
\inhibitxspcode`}=1
\inhibitxspcode`{=2
\inhibitxspcode`‘=2
\inhibitxspcode`’=1
\inhibitxspcode`“=2
\inhibitxspcode`”=1
\inhibitxspcode`〔=2
\inhibitxspcode`〕=1
\inhibitxspcode`〈=2
\inhibitxspcode`〉=1
\inhibitxspcode`《=2
\inhibitxspcode`》=1
\inhibitxspcode`「=2
\inhibitxspcode`」=1
\inhibitxspcode`『=2
\inhibitxspcode`』=1
\inhibitxspcode`【=2
\inhibitxspcode`】=1
\inhibitxspcode`—=0% U+2014 EM DASH
\inhibitxspcode`―=0% U+2015 HORIZONTAL BAR
\inhibitxspcode`〜=0% U+301C WAVE DASH
\inhibitxspcode`~=0% U+FF5E FULLWIDTH TILDE
\inhibitxspcode`…=0
\inhibitxspcode`¥=0% U+00A5 YEN SIGN
\inhibitxspcode`¥=0% U+FFE5 FULLWIDTH YEN SIGN
\inhibitxspcode`°=1
\inhibitxspcode`′=1
\inhibitxspcode`″=1
%%
%% inhibitxspcode  JIS X 0213
%%
\inhibitxspcode`⦅=2
\inhibitxspcode`⦆=1
\inhibitxspcode`⦅=2
\inhibitxspcode`⦆=1
\inhibitxspcode`〘=2
\inhibitxspcode`〙=1
\inhibitxspcode`〖=2
\inhibitxspcode`〗=1
\inhibitxspcode`«=2
\inhibitxspcode`»=1
\inhibitxspcode`〝=2
\inhibitxspcode`〟=1
\inhibitxspcode`‼=1
\inhibitxspcode`⁇=1
\inhibitxspcode`⁈=1
\inhibitxspcode`⁉=1
\inhibitxspcode`¡=2
\inhibitxspcode`¿=2
\inhibitxspcode"AA=1
\inhibitxspcode"BA=1
\inhibitxspcode`¹=1
\inhibitxspcode`²=1
\inhibitxspcode`³=1
\inhibitxspcode`€=2
%%
%% inhibitxspcode  JIS X 0212
%%
%%\inhibitxspcode`¡=2
%%\inhibitxspcode`¿=2
%%\inhibitxspcode"BA=1
%%\inhibitxspcode"AA=1
\inhibitxspcode`™=1
%%
%% inhibitxspcode  半角片仮名
%%
\inhibitxspcode`。=1
\inhibitxspcode`、=1
\inhibitxspcode`「=2
\inhibitxspcode`」=1
%    \end{macrocode}
%
%    \begin{macrocode}
%</plcore>
%    \end{macrocode}
%
% \Finale
%
\endinput

   \input{ujclasses.dtx}
\endgroup
\@ifl@t@r{\lastupd@te}{0000/00/00}{%
  \date{Version \patchdate\break (last updated: \lastupd@te)}%
}{}
\makeatother
%    \end{macrocode}
%\ifJAPANESE
% ここからが本文ページとなります。
%\else
% Here starts the document body.
%\fi
%    \begin{macrocode}
\begin{document}
\pagenumbering{roman}
\maketitle
\renewcommand\maketitle{}
\tableofcontents
\clearpage
\pagenumbering{arabic}

\DocInclude{uplvers}   % upLaTeX version

\DocInclude{uplfonts}  % NFSS2 commands

\DocInclude{ukinsoku}  % kinsoku parameter

\DocInclude{ujclasses} % Standard class

\StopEventually{\end{document}}

\clearpage
\pagestyle{headings}
% Make TeX shut up.
\hbadness=10000
\newcount\hbadness
\hfuzz=\maxdimen
%
\PrintChanges
\clearpage
%
\begingroup
  \def\endash{--}
  \catcode`\-\active
  \def-{\futurelet\temp\indexdash}
  \def\indexdash{\ifx\temp-\endash\fi}

  \PrintIndex
\endgroup
\let\PrintChanges\relax
\let\PrintIndex\relax
\end{document}
%</pldoc>
%    \end{macrocode}
%
%
%
%\ifJAPANESE
% \section{おまけプログラム}\label{app:omake}
%
% \subsection{シェルスクリプト\file{mkpldoc.sh}}\label{app:shprog}
% \upLaTeXe{}のマクロ定義ファイルをまとめて組版し、変更履歴と索引も
% 付けるときに便利なシェルスクリプトです。
% このシェルスクリプトの使用方法は次のとおりです。
%\begin{verbatim}
%    sh mkpldoc.sh
%\end{verbatim}
%
% コードは\pLaTeXe{}のものと(ファイル名を除き)ほぼ同一なので、
% ここでは違っている部分だけ説明します。
%\else
% \section{Additional Utility Programs}\label{app:omake}
%
% \subsection{Shell Script \file{mkpldoc.sh}}\label{app:shprog}
% A shell script to process `pldoc.tex' and produce a fully indexed
% source code description. Run |sh mkpldoc.sh| to use it.
%
% The script is almost identical to that in \pLaTeXe, so
% here we describe only the difference.
%\fi
%
%    \begin{macrocode}
%<*shprog>
%<ja>rm -f upldoc.toc upldoc.idx upldoc.glo
%<en>rm -f upldoc-en.toc upldoc-en.idx upldoc-en.glo
echo "" > ltxdoc.cfg
%<ja>uplatex upldoc.tex
%<en>uplatex -jobname=upldoc-en upldoc.tex
%    \end{macrocode}
%\ifJAPANESE
% 変更履歴や索引の生成にはmendexを用いますが、
% \upLaTeX{}の場合はUTF-8モードで実行する必要がありますので、
% |-U|というオプションを付けます\footnote{uplatexコマンドも
% 実際にはUTF-8モードで実行する必要がありますが、デフォルトの内部漢字
% コードがUTF-8に設定されているはずですので、\texttt{-kanji=utf8}を
% 付けなくても処理できると思います。}。
% makeindexコマンドには、このオプションがありません。
%\else
% To make the Change log and Glossary (Change History) for
% \upLaTeX\ using `mendex,' we need to run it in UTF-8 mode.
% So, option |-U| is important.\footnote{The command `uplatex'
% should be also in UTF-8 mode, but it defaults to UTF-8 mode;
% therefore, we don't need to add \texttt{-kanji=utf8} explicitly.}
%\fi
%    \begin{macrocode}
%<ja>mendex -U -s gind.ist -d upldoc.dic -o upldoc.ind upldoc.idx
%<en>mendex -U -s gind.ist -d upldoc.dic -o upldoc-en.ind upldoc-en.idx
%<ja>mendex -U -f -s gglo.ist -o upldoc.gls upldoc.glo
%<en>mendex -U -f -s gglo.ist -o upldoc-en.gls upldoc-en.glo
echo "\includeonly{}" > ltxdoc.cfg
%<ja>uplatex upldoc.tex
%<en>uplatex -jobname=upldoc-en upldoc.tex
echo "" > ltxdoc.cfg
%<ja>uplatex upldoc.tex
%<en>uplatex -jobname=upldoc-en upldoc.tex
# EOT
%</shprog>
%    \end{macrocode}
%
%
%\ifJAPANESE
% \subsection{perlスクリプト\file{dstcheck.pl}}\label{app:plprog}
% \pLaTeXe{}のものがそのまま使えるので、\upLaTeXe{}では省略します。
%\else
% \subsection{Perl Script \file{dstcheck.pl}}\label{app:plprog}
% The one from \pLaTeXe\ can be use without any change, so
% omitted here in \upLaTeXe.
%\fi
%
%
%\ifJAPANESE
% \subsection{\dst{}バッチファイル}
% 付録\ref{app:shprog}で説明をしたスクリプトを、このファイルから
% 取り出すための\dst{}バッチファイルです。コードは\pLaTeXe{}の
% ものと(ファイル名を除き)ほぼ同一なので、説明は割愛します。
%\else
% \subsection{\dst{} Batch file}
% Here we introduce a \dst\ batch file `Xins.ins,' which generates the
% script described in Appendix \ref{app:shprog}.
% The code is almost identical to that in \pLaTeXe.
%\fi
%
%    \begin{macrocode}
%<*Xins>
\input docstrip
\keepsilent
%    \end{macrocode}
%
%    \begin{macrocode}
{\catcode`#=12 \gdef\MetaPrefix{## }}
%    \end{macrocode}
%
%    \begin{macrocode}
\declarepreamble\thispre
\endpreamble
\usepreamble\thispre
%    \end{macrocode}
%
%    \begin{macrocode}
\declarepostamble\thispost
\endpostamble
\usepostamble\thispost
%    \end{macrocode}
%
%    \begin{macrocode}
\generate{
   \file{mkpldoc.sh}{\from{uplatex.dtx}{shprog,ja}}
   \file{mkpldoc-en.sh}{\from{uplatex.dtx}{shprog,en}}
}
\endbatchfile
%</Xins>
%    \end{macrocode}
%
% \newpage
% \begin{thebibliography}{9}
% \bibitem{tb108tanaka}
% Takuji Tanaka,
% \newblock Up\TeX\ --- Unicode version of \pTeX\ with CJK extensions.
% \newblock TUGboat issue 34:3, 2013.\\
%   (\texttt{http://tug.org/TUGboat/tb34-3/tb108tanaka.pdf})
% \end{thebibliography}
%
% \iffalse
% ここで、このあとに組版されるかもしれない文書のために、
% 節見出しの番号を算用数字に戻します。
% \fi
%
% \renewcommand{\thesection}{\arabic{section}}
%
% \Finale
%
\endinput
}{}%
}
%    \end{macrocode}
%
%\ifJAPANESE
% フォーマットファイルにダンプします。
%\else
% Dump to the format file.
%\fi
%    \begin{macrocode}
\let\dump\orgdump
\let\orgdump\@undefined
\makeatother
\dump
%\endinput
%    \end{macrocode}
%
%    \begin{macrocode}
%</plcore>
%    \end{macrocode}
%
%\ifJAPANESE
% 実際に\upLaTeXe{}への拡張を行なっている\file{uplcore.ltx}は、
% \dst{}プログラムによって、次のファイルの断片が連結されたものです。
%
% \begin{itemize}
% \item \file{uplvers.dtx}は、\upLaTeXe{}のフォーマットバージョンを
%   定義しています。
% \end{itemize}
%
% また、プリロードフォントや組版パラメータなどのデフォルト設定は、
% \file{uplatex.ltx}の中で\file{upldefs.ltx}をロードすることにより行います
% \footnote{旧版では\file{uplcore.ltx}の中でロードしていましたが、
% 2018年以降の新しいコミュニティ版\upLaTeX{}では
% \file{uplatex.ltx}から読み込むことにしました。}。
% このファイル\file{upldefs.ltx}も\file{uplfonts.dtx}から生成されます。
% \begin{chuui}
% このファイルに記述されている設定を変更すれば
% \upLaTeXe{}をカスタマイズすることができますが、
% その場合は\file{upldefs.ltx}を直接修正するのではなく、いったん
% \file{upldefs.cfg}という名前でコピーして、そのファイルを編集してください。
% フォーマット作成時に\file{upldefs.cfg}が存在した場合は、そちらが
% \file{upldefs.ltx}の代わりに読み込まれます。
% \end{chuui}
%\else
% The file \file{uplcore.ltx}, which provides modifications/extensions
% to make \upLaTeXe, is a concatenation of stripped files below
% using \dst\ program.
%
% \begin{itemize}
% \item \file{uplvers.dtx} defines the format version of \upLaTeXe.
% \item \file{uplfonts.dtx} extends \NFSS2 for Japanese font selection.
% \item \file{plcore.dtx} (the same content as \pLaTeXe); defines other
%   modifications to \LaTeXe.
% \end{itemize}
%
% Moreover, default settings of pre-loaded fonts and typesetting parameters
% are done by loading \file{upldefs.ltx} inside
% \file{uplatex.ltx}.\footnote{Older \upLaTeX\ loaded \file{upldefs.ltx}
% inside \file{uplcore.ltx}; however, \upLaTeX\ community edition newer than
% 2018 loads \file{upldefs.ltx} inside \file{uplatex.ltx}.}
% This file \file{upldefs.ltx} is also stripped from \file{uplfonts.dtx}.
% \begin{chuui}
% You can customize \upLaTeXe\ by tuning these settings.
% If you need to do that, copy/rename it as \file{upldefs.cfg} and edit it,
% instead of overwriting \file{upldefs.ltx} itself.
% If a file named \file{upldefs.cfg} is found at a format creation
% time, it will be read as a substitute of \file{upldefs.ltx}.
% \end{chuui}
%\fi
%
%\ifJAPANESE
% ここまで見てきたように、\upLaTeX{}の各ファイルはそれぞれ\pLaTeX{}での
% 対応するファイル名の頭に``u''を付けた名前になっています。
%\else
% As shown above, the files in \upLaTeX\ is named after \pLaTeX\ ones,
% prefixed with ``u.''
%\fi
%
%
%\ifJAPANESE
% \subsubsection{バージョン}
% \upLaTeXe{}のバージョンやフォーマットファイル名は、
% \file{uplvers.dtx}で定義しています。これは、\pLaTeXe{}のバージョンや
% フォーマットファイル名が\file{plvers.dtx}で定義されているのと同じです。
%\else
% \subsubsection{Version}
% The version (like ``\pfmtversion'') and the format name
% (``\pfmtname'') of \upLaTeXe\ are defined in \file{uplvers.dtx}.
% This is similar to \pLaTeXe, which defines those in \file{plvers.dtx}.
%\fi
%
%
%\ifJAPANESE
% \subsubsection{\NFSS2コマンド}
% \upLaTeXe{}は\pLaTeXe{}と共通の\file{plcore.ltx}を使用していますので、
% \NFSS2の和文フォント選択への拡張が有効になっています。
%\else
% \subsubsection{\NFSS2 Commands}
% \upLaTeXe\ shares \file{plcore.dtx} with \pLaTeXe, so
% the extensions of \NFSS2 for selecting Japanese fonts are available.
%\fi
%
%
%\ifJAPANESE
% \subsubsection{出力ルーチンとフロート}
% \upLaTeXe{}は\pLaTeXe{}と共通の\file{plcore.ltx}を使用していますので、
% 出力ルーチンや脚注マクロなどは\pLaTeXe{}と同じように動作します。
%\else
% \subsubsection{Output Routine and Floats}
% \upLaTeXe\ shares \file{plcore.dtx} with \pLaTeXe, so
% the output routine and footnote macros will behave similar to \pLaTeXe.
%\fi
%
%
%\ifJAPANESE
% \subsection{クラスファイルとパッケージファイル}
%
% \upLaTeXe{}が提供をするクラスファイルやパッケージファイルは、
% \pLaTeXe{}に含まれるファイルを基にしています。
%
% \upLaTeXe{}に付属のクラスファイルは、次のとおりです。
%
% \begin{itemize}
% \item ujarticle.cls, ujbook.cls, ujreport.cls\par
%   横組用の標準クラスファイル。
%   \file{ujclasses.dtx}から作成される。
%   それぞれjarticle.cls, jbook.cls, jreport.clsの\upLaTeX{}版。
%
% \item utarticle.cls, utbook.cls, utreport.cls\par
%   縦組用の標準クラスファイル。
%   \file{ujclasses.dtx}から作成される。
%   それぞれtarticle.cls, tbook.cls, treport.clsの\upLaTeX{}版。
% \end{itemize}
%
% なおjltxdoc.clsの\upLaTeX{}版はありませんが、これは\pLaTeX{}のものが
% \upLaTeX{}でもそのまま使えます。
%\else
% \subsection{Classes and Packages}
%
% Classes and packages bundled with \upLaTeXe\ are based on
% those in original \pLaTeXe, and modified some parameters.
%
% \upLaTeXe\ classes:
%
% \begin{itemize}
% \item ujarticle.cls, ujbook.cls, ujreport.cls\par
%   Standard \emph{yoko-kumi} (horizontal writing) classes;
%   stripped from \file{ujclasses.dtx}.
%   \upLaTeX\ edition of jarticle.cls, jbook.cls and jreport.cls.
%
% \item utarticle.cls, utbook.cls, utreport.cls\par
%   Standard \emph{tate-kumi} (vertical writing) classes;
%   stripped from \file{ujclasses.dtx}.
%   \upLaTeX\ edition of tarticle.cls, tbook.cls and treport.cls.
% \end{itemize}
%
% We don't provide \upLaTeX\ edition of jltxdoc.cls, but the one
% from \pLaTeX\ can be used also on \upLaTeX\ without problem.
%\fi
%
%\ifJAPANESE
% また、\upLaTeXe{}に付属のパッケージファイルは、次のとおりです。
%
% \begin{itemize}
% \item uptrace.sty\par
%   ptrace.styの\upLaTeX{}版。
%   \LaTeX{}でフォント選択コマンドのトレースに使う\file{tracefnt.sty}が
%   再定義してしまう\NFSS2コマンドを、\upLaTeXe{}用に再々定義するための
%   パッケージ。
%   \file{uplfonts.dtx}から作成される。
% \end{itemize}
%
% 他の\pLaTeX{}のパッケージは、\upLaTeX{}でもそのまま動作します。
%\else
% \upLaTeXe\ packages:
%
% \begin{itemize}
% \item uptrace.sty\par
%   \upLaTeXe\ version of \file{tracefnt.sty};
%   the package \file{tracefnt.sty} overwrites \upLaTeXe-style \NFSS2
%   commands, so \file{uptrace.sty} provides redefinitions to recover
%   \upLaTeXe\ extensions. Stripped from \file{uplfonts.dtx}.
% \end{itemize}
%
% Other \pLaTeX\ packages work also on \upLaTeX.
%\fi
%
%
%\ifJAPANESE
% \section{他のフォーマット・旧バージョンとの互換性}
% \label{platex:compatibility}
% ここでは、この\upLaTeXe{}のバージョンと以前のバージョン、あるいは
% \pLaTeXe{}/\LaTeXe{}との互換性について説明をしています。
%
% \subsection{\pLaTeXe{}および\LaTeXe{}との互換性}
% \upLaTeXe{}は、\pLaTeXe{}の上位互換という形を取っていますので、
% クラスファイルやいくつかのコマンドを置き換えるだけで、
% たいていの\pLaTeXe{}文書を簡単に\upLaTeXe{}文書に変更することができます。
% ただし、\upLaTeXe{}のデフォルトの日本語フォントメトリックは\pLaTeXe{}の
% それと異なりますので、レイアウトが変化することがあります。
% また、\LaTeXe{}のいくつかの命令の定義も変更していますので、
% \LaTeXe{}で処理できるファイルを\upLaTeXe{}で処理した場合に
% 完全に同じ結果になるとは限りません。
%
% また、\upLaTeXe{}は新しいマクロパッケージですので、2.09互換モードを
% サポートしていません。\LaTeXe{}の仕様に従ってドキュメントを作成して
% ください。
%
% \pLaTeXe{}向けあるいは\LaTeXe{}向けに作られた多くのクラスファイルや
% パッケージファイルはそのまま使えると思います。
% ただし、例えばクラスファイルが\pLaTeX{}標準の
% 漢字エンコーディング(JY1, JT1)を前提としている場合は、
% \upLaTeX{}で採用した漢字エンコーディング(JY2, JT2)と合致せずに
% エラーが発生してしまいます。この場合は、そのクラスファイルが
% \upLaTeX{}に対応していないことになります。このような場合は、
% \pLaTeX{}を使い続けるか、その作者に連絡して\upLaTeX{}に対応して
% もらうなどの対応をとってください。
%\else
% \section{Compatibility with Other Formats and Older Versions}
% \label{platex:compatibility}
% Here we provide some information about the compatibility between
% current \upLaTeXe\ and older versions or original \pLaTeXe/\LaTeXe.
%
% \subsection{Compatibility with \pLaTeXe/\LaTeXe}
% \upLaTeXe\ is in most part upward compatible with \pLaTeXe,
% so you can move from \pLaTeXe\ to \upLaTeXe\ by simply replacing
% the document class and some macros. However, the default Japanese
% font metrics in \upLaTeXe\ is different from those in \pLaTeXe;
% therefore, you should not expect identical output from both
% \pLaTeXe\ and \upLaTeXe.
%
% Note that \upLaTeX\ is a new format, so we do \emph{not} provide support
% for 2.09 compatibility mode. Follow the standard \LaTeXe\ convention!
%
% We hope that most classes and packages meant for \LaTeXe/\pLaTeXe\ works
% also for \upLaTeXe\ without any modification. However for example,
% if a class or a package uses Kanji encoding `JY1' or `JT1' (default on
% \pLaTeXe), an error complaining the mismatch of Kanji encoding might
% happen on \upLaTeX, in which the default is `JY2' and `JT2.'
% In this case, we have to say that the class or package does not support
% \upLaTeXe; you should use \pLaTeX, or report to the author of the
% package or class.
%\fi
%
%\ifJAPANESE
% \subsection{latexreleaseパッケージへの対応}
% \LaTeX\ \texttt{<2015/01/01>}で導入されたlatexreleaseパッケージを
% もとに、新しい\pLaTeX{}ではplatexreleaseパッケージが用意されました。
% 本来は\upLaTeX{}でも同様のパッケージを用意するのがよいのですが、
% 現在は\pLaTeX{}から\upLaTeX{}への変更点が含まれていませんので、
% 幸いplatexreleaseパッケージをそのまま用いることができます。
% このため、\upLaTeX{}で独自のパッケージを用意することはしていません。
% platexreleaseパッケージを用いると、過去の\upLaTeX{}をエミュレート
% したり、フォーマットを作り直すことなく新しい\upLaTeX{}を試したりする
% ことができます。詳細はplatexreleaseのドキュメントを参照してください。
%\else
% \subsection{Support for Package `latexrelease'}
% \pLaTeX\ provides `platexrelease' package, which is based on
% `latexrelease' package (introduced in \LaTeX\ \texttt{<2015/01/01>}).
% It could be better if we also provide a similar package on \upLaTeX,
% but currently we don't need it; \upLaTeX\ does not have any recent
% \upLaTeX-specific changes. So, you can safely use `platexrelease'
% package for emulating the specified format date.
%\fi
%
%
%
% \appendix
%
%\ifJAPANESE
% \section{\dst{}プログラムのためのオプション}\label{app:dst}
% この文書のソース(\file{uplatex.dtx})を\dst{}プログラムで
% 処理することによって、
% いくつかの異なるファイルを生成することができます。
% \dst{}プログラムの詳細は、\file{docstrip.dtx}を参照してください。
%
% この文書の\dst{}プログラムのためのオプションは、次のとおりです。
%
% \DeleteShortVerb{\|}
% \begin{center}
% \begin{tabular}{l|p{.8\linewidth}}
% \emph{オプション} & \emph{意味}\\\hline
% plcore & フォーマットファイルを作るためのファイルを生成\\
% pldoc  & \upLaTeXe{}のソースファイルをまとめて組版するための
%          文書ファイル(upldoc.tex)を生成\\[2mm]
% shprog & 上記のファイルを作成するためのshスクリプトを生成\\
% Xins   & 上記のshスクリプトやperlスクリプトを取り出すための
%          \dst{}バッチファイル(Xins.ins)を生成\\
% \end{tabular}
% \end{center}
% \MakeShortVerb{\|}
%\else
% \section{\dst\ Options}\label{app:dst}
% By processing \file{uplatex.dtx} with \dst\ program,
% different files can be generated.
% Here are the \dst\ options for this document:
%
% \DeleteShortVerb{\|}
% \begin{center}
% \begin{tabular}{l|p{.8\linewidth}}
% \emph{Option} & \emph{Function}\\\hline
% plcore & Generates a fragment of format sources\\
% pldoc  & Generates `upldoc.tex' for typesetting
%          \upLaTeXe\ sources\\[2mm]
% shprog & Generates a shell script to process `upldoc.tex'\\
% Xins   & Generates a \dst\ batch file `Xins.ins' for
%          generating the above shell/perl scripts\\
% \end{tabular}
% \end{center}
% \MakeShortVerb{\|}
%\fi
%
%
%\ifJAPANESE
% \section{文書ファイル}\label{app:pldoc}
% ここでは、このパッケージに含まれているdtxファイルをまとめて組版し、
% ソースコード説明書を得るための文書ファイル\file{upldoc.tex}について
% 説明をしています。個別に処理した場合と異なり、
% 変更履歴や索引も付きます。
%
% デフォルトではソースコードの説明が日本語で書かれます。
% もし英語の説明書を読みたい場合は、\par\medskip
% \begin{minipage}{.5\textwidth}\ttfamily
% | |\cs{newif}\cs{ifJAPANESE}
% \end{minipage}\par\medskip\noindent
% という内容の\file{uplatex.cfg}を予め用意してから\file{upldoc.tex}を
% 処理してください(2016年7月1日以降の\upLaTeXe{}が必要)。
%
% コードは\pLaTeXe{}のものと(ファイル名を除き)ほぼ同一なので、
% ここでは違っている部分だけ説明します。
%\else
% \section{Documentation of \upLaTeXe\ sources}\label{app:pldoc}
% The contents of `upldoc.tex' for typesetting \upLaTeXe\ sources
% is described here. Compared to individual processings,
% batch processing using `upldoc.tex' prints also changes and an index.
%
% By default, the description of \upLaTeXe\ sources is written in
% Japanese. If you need English version, first save\par\medskip
% \begin{minipage}{.5\textwidth}\ttfamily
% | |\cs{newif}\cs{ifJAPANESE}
% \end{minipage}\par\medskip\noindent
% as \file{uplatex.cfg}, and process \file{upldoc.tex}
% (\upLaTeXe\ newer than July 2016 is required).
%
% Here we explain only difference between \file{pldoc.tex} (\pLaTeXe)
% and \file{upldoc.tex} (\upLaTeXe).
%\fi
%
%    \begin{macrocode}
%<*pldoc>
\begin{filecontents}{upldoc.dic}
西暦    せいれき
和暦    われき
\end{filecontents}
%    \end{macrocode}
%\ifJAPANESE
% \pLaTeXe{}のドキュメントでは、\file{plext.dtx}の中で組み立てるサンプル
% のために\file{plext}パッケージが必要ですが、\upLaTeXe{}のファイル
% にはそのようなサンプルが含まれないので除外しています。
%\else
% The document of \pLaTeXe\ requires \file{plext} package,
% since \file{plext.dtx} contains several examples of partial
% vertical writing. However, we don't have such examples in
% \upLaTeXe\ files, so no need for it.
%\fi
%    \begin{macrocode}
\documentclass{jltxdoc}
%\usepackage{plext} %% comment out for upLaTeX
\listfiles

\DoNotIndex{\def,\long,\edef,\xdef,\gdef,\let,\global}
\DoNotIndex{\if,\ifnum,\ifdim,\ifcat,\ifmmode,\ifvmode,\ifhmode,%
            \iftrue,\iffalse,\ifvoid,\ifx,\ifeof,\ifcase,\else,\or,\fi}
\DoNotIndex{\box,\copy,\setbox,\unvbox,\unhbox,\hbox,%
            \vbox,\vtop,\vcenter}
\DoNotIndex{\@empty,\immediate,\write}
\DoNotIndex{\egroup,\bgroup,\expandafter,\begingroup,\endgroup}
\DoNotIndex{\divide,\advance,\multiply,\count,\dimen}
\DoNotIndex{\relax,\space,\string}
\DoNotIndex{\csname,\endcsname,\@spaces,\openin,\openout,%
            \closein,\closeout}
\DoNotIndex{\catcode,\endinput}
\DoNotIndex{\jobname,\message,\read,\the,\m@ne,\noexpand}
\DoNotIndex{\hsize,\vsize,\hskip,\vskip,\kern,\hfil,\hfill,\hss,\vss,\unskip}
\DoNotIndex{\m@ne,\z@,\z@skip,\@ne,\tw@,\p@,\@minus,\@plus}
\DoNotIndex{\dp,\wd,\ht,\setlength,\addtolength}
\DoNotIndex{\newcommand, \renewcommand}

\ifJAPANESE
\IndexPrologue{\part*{索 引}%
                 \markboth{索 引}{索 引}%
                 \addcontentsline{toc}{part}{索 引}%
イタリック体の数字は、その項目が説明されているページを示しています。
下線の引かれた数字は、定義されているページを示しています。
その他の数字は、その項目が使われているページを示しています。}
\else
\IndexPrologue{\part*{Index}%
                 \markboth{Index}{Index}%
                 \addcontentsline{toc}{part}{Index}%
The italic numbers denote the pages where the corresponding entry
is described, numbers underlined point to the definition,
all others indicate the places where it is used.}
\fi
%
\ifJAPANESE
\GlossaryPrologue{\part*{変更履歴}%
                 \markboth{変更履歴}{変更履歴}%
                 \addcontentsline{toc}{part}{変更履歴}}
\else
\GlossaryPrologue{\part*{Change History}%
                 \markboth{Change History}{Change History}%
                 \addcontentsline{toc}{part}{Change History}}
\fi

\makeatletter
\def\changes@#1#2#3{%
  \let\protect\@unexpandable@protect
  \edef\@tempa{\noexpand\glossary{#2\space
               \currentfile\space#1\levelchar
               \ifx\saved@macroname\@empty
                  \space\actualchar\generalname
               \else
                  \expandafter\@gobble
                  \saved@macroname\actualchar
                  \string\verb\quotechar*%
                  \verbatimchar\saved@macroname
                  \verbatimchar
               \fi
               :\levelchar #3}}%
  \@tempa\endgroup\@esphack}
\renewcommand*\MacroFont{\fontencoding\encodingdefault
                   \fontfamily\ttdefault
                   \fontseries\mddefault
                   \fontshape\updefault
                   \small
                   \hfuzz 6pt\relax}
\renewcommand*\l@subsection{\@dottedtocline{2}{1.5em}{2.8em}}
\renewcommand*\l@subsubsection{\@dottedtocline{3}{3.8em}{3.4em}}
\makeatother
\RecordChanges
\CodelineIndex
\EnableCrossrefs
\setcounter{IndexColumns}{2}
\settowidth\MacroIndent{\ttfamily\scriptsize 000\ }
%    \end{macrocode}
%\ifJAPANESE
% この文書のタイトル・著者・日付を設定します。
% \changes{v1.0h-u00}{2016/05/08}{ドキュメントから\file{uplpatch.ltx}を除外
%     (based on platex.dtx 2016/05/08 v1.0h)}
% \changes{v1.0l-u01}{2016/06/19}{パッチレベルを\file{uplvers.dtx}から取得
%     (based on platex.dtx 2016/06/19 v1.0l)}
% \changes{v1.0y-u02}{2018/09/22}{最終更新日を\file{upldoc.pdf}に表示
%     (based on platex.dtx 2018/09/22 v1.0y)}
%\else
% Set the title, authors and the date for this document.
% \changes{v1.0h-u00}{2016/05/08}{Exclude \file{uplpatch.ltx} from the document
%     (based on platex.dtx 2016/05/08 v1.0h)}
% \changes{v1.0l-u01}{2016/06/19}{Get the patch level from \file{uplvers.dtx}
%     (based on platex.dtx 2016/06/19 v1.0l)}
% \changes{v1.0y-u02}{2018/09/22}{Show last update info on \file{upldoc.pdf}
%     (based on platex.dtx 2018/09/22 v1.0y)}
%\fi
%    \begin{macrocode}
 \title{The \upLaTeXe\ Sources}
 \author{Ken Nakano \& Japanese \TeX\ Development Community \& TTK}

% Get the (temporary) date and up-patch level from uplvers.dtx
\makeatletter
\let\patchdate=\@empty
\begingroup
   \def\ProvidesFile#1[#2 #3]#4\def\uppatch@level#5{%
      \date{#2}\xdef\patchdate{#5}\endinput}
   % \iffalse meta-comment
%% File: uplvers.dtx
%
%    pLaTeX version setting file:
%       Copyright 1995-2006 ASCII Corporation.
%    and modified for upLaTeX
%
%  Copyright (c) 2010 ASCII MEDIA WORKS
%  Copyright (c) 2016 Takuji Tanaka
%  Copyright (c) 2016-2017 Japanese TeX Development Community
%
%  This file is part of the upLaTeX2e system (community edition).
%  --------------------------------------------------------------
%
% \fi
%
%
% \setcounter{StandardModuleDepth}{1}
% \StopEventually{}
%
% \iffalse
% \changes{v1.0}{1995/05/16}{p\LaTeXe\ 用に\file{ltvers.dtx}を修正}
% \changes{v1.0a}{1995/08/30}{\LaTeX\ \texttt{!<1995/06/01!>}版用に修正}
% \changes{v1.0b}{1996/01/31}{\LaTeX\ \texttt{!<1995/12/01!>}版用に修正}
% \changes{v1.0c}{1997/01/11}{\LaTeX\ \texttt{!<1996/06/01!>}版用に修正}
% \changes{v1.0d}{1997/01/23}{\LaTeX\ \texttt{!<1996/12/01!>}版用に修正}
% \changes{v1.0e}{1997/07/02}{\LaTeX\ \texttt{!<1997/06/01!>}版用に修正}
% \changes{v1.0f}{1998/02/17}{\LaTeX\ \texttt{!<1997/12/01!>}版用に修正}
% \changes{v1.0g}{1998/09/01}{\LaTeX\ \texttt{!<1998/06/01!>}版用に修正}
% \changes{v1.0h}{1999/04/05}{\LaTeX\ \texttt{!<1998/12/01!>}版用に修正}
% \changes{v1.0i}{1999/08/09}{\LaTeX\ \texttt{!<1999/06/01!>}版用に修正}
% \changes{v1.0j}{2000/02/29}{\LaTeX\ \texttt{!<1999/12/01!>}版用に修正}
% \changes{v1.0k}{2000/11/03}{\LaTeX\ \texttt{!<2000/06/01!>}版用に修正}
% \changes{v1.0l}{2001/09/04}{\LaTeX\ \texttt{!<2001/06/01!>}版用に修正}
% \changes{v1.0m}{2004/08/10}{\LaTeX\ \texttt{!<2003/12/01!>}版対応確認}
% \changes{v1.0n}{2005/01/04}{plfonts.dtxバグ修正}
% \changes{v1.0o}{2006/01/04}{plfonts.dtxバグ修正}
% \changes{v1.0p}{2006/06/27}{plfonts.dtx \LaTeX\ \texttt{!<2005/12/01!>}対応}
% \changes{v1.0q}{2006/11/10}{plfonts.dtxバグ修正}
% \changes{v1.0q-u00}{2011/05/07}{p\LaTeX{}用からup\LaTeX{}用に修正。}
% \changes{v1.0r}{2016/01/26}{plcore.dtx p\TeX\ (r28720)対応}
% \changes{v1.0s}{2016/02/01}{\LaTeX\ \texttt{!<2015/01/01!>}のlatexreleaseに
%    対応するためのコードを導入}
% \changes{v1.0t}{2016/02/03}{\cs{plIncludeInRelease}と
%    \cs{plEndIncludeInRelease}を新設。}
% \changes{v1.0u}{2016/04/17}{\LaTeX\ \texttt{!<2016/03/31!>}版対応確認}
% \changes{v1.0u-u00}{2016/04/17}{p\LaTeX{}の変更に追随。}
% \changes{v1.0v}{2016/05/07}{パッチファイルをロードするのをやめた。}
% \changes{v1.0v}{2016/05/07}{起動時の文字列を最新の\LaTeX{}に合わせた。}
% \changes{v1.0w}{2016/05/12}{起動時の文字列に入れる\LaTeX{}のバージョンを
%    元の\LaTeX{}のバナーから引き継ぐように改良}
% \changes{v1.0w-u00}{2016/05/12}{起動時の文字列に入れるBabelのバージョンを
%    元の\LaTeX{}のバナーから取得するコードを\file{uplatex.ini}から取り入れた}
% \changes{v1.0w-u01}{2016/05/21}{サポート外の\LaTeX~2.09互換モードが
%    使われた場合に明確なエラーを出すようにした。}
% \changes{v1.0x}{2016/06/19}{パッチレベルを\file{plvers.dtx}で設定}
% \changes{v1.0x-u01}{2016/06/19}{p\LaTeX{}の変更に追随。}
% \changes{v1.0y-u01}{2016/06/29}{\file{uplatex.cfg}の読み込みを追加}
% \changes{v1.0z-u01}{2016/08/26}{\file{uplatex.cfg}の読み込みを
%    \file{uplcore.ltx}から\file{uplatex.ltx}へ移動}
% \changes{v1.1}{2016/09/14}{起動時のバナーを取得するコードを改良}
% \changes{v1.1-u01}{2016/09/14}{p\LaTeX{}の変更に追随。}
% \changes{v1.1a}{2017/02/20}{\LaTeX\ \texttt{!<2017/01/01!>}版対応確認}
% \changes{v1.1a-u01}{2017/03/05}{p\LaTeX{}の変更に追随。}
% \changes{v1.1b}{2017/03/19}{\cs{l@nohyphenation}の定義を保証
%    (sync with ltfinal 2017/03/09 v2.0t)}
% \changes{v1.1b}{2017/03/19}{\cs{document@default@language}の定義を保証
%    (sync with ltfinal 2017/03/09 v2.0t)}
% \changes{v1.1b-u01}{2017/03/19}{p\LaTeX{}の変更に追随。}
% \changes{v1.1c}{2017/04/23}{\LaTeX\ \texttt{!<2017/04/15!>}版対応確認}
% \changes{v1.1c-u01}{2017/05/04}{p\LaTeX{}の変更に追随。}
% \changes{v1.1d}{2017/09/24}{パッチレベルが負の数の場合をpre-release扱いへ}
% \changes{v1.1d-u01}{2017/09/24}{p\LaTeX{}の変更に追随。}
% \fi
%
% \iffalse
%<*driver>
% \fi
\ProvidesFile{uplvers.dtx}[2017/09/24 v1.1d-u01 upLaTeX Kernel (Version Info)]
% \iffalse
\documentclass{jltxdoc}
\GetFileInfo{uplvers.dtx}
\author{Ken Nakano \& Hideaki Togashi \& TTK}
\title{\filename}
\date{作成日:\filedate}
\begin{document}
  \maketitle
  \DocInput{\filename}
\end{document}
%</driver>
% \fi
%
% \section{バージョンの設定}
% まず、このディストリビューションでのup\LaTeXe{}の日付とバージョン番号
% を定義します。また、up\LaTeXe{}が起動されたときに表示される文字列の
% 設定もします。
%
% \changes{v1.0}{1995/05/16}{p\LaTeXe\ 用に\file{ltvers.dtx}を修正}
% \changes{v1.0a}{1995/08/30}{\LaTeX\ \texttt{!<1995/06/01!>}版用に修正}
% \changes{v1.0b}{1996/01/31}{\LaTeX\ \texttt{!<1995/12/01!>}版用に修正}
% \changes{v1.0c}{1997/01/11}{\LaTeX\ \texttt{!<1996/06/01!>}版用に修正}
% \changes{v1.0d}{1997/01/23}{\LaTeX\ \texttt{!<1996/12/01!>}版用に修正}
% \changes{v1.0e}{1997/07/02}{\LaTeX\ \texttt{!<1997/06/01!>}版用に修正}
% \changes{v1.0f}{1998/02/17}{\LaTeX\ \texttt{!<1997/12/01!>}版用に修正}
% \changes{v1.0g}{1998/09/01}{\LaTeX\ \texttt{!<1998/06/01!>}版用に修正}
% \changes{v1.0h}{1999/04/05}{\LaTeX\ \texttt{!<1998/12/01!>}版用に修正}
% \changes{v1.0i}{1999/08/09}{\LaTeX\ \texttt{!<1999/06/01!>}版用に修正}
% \changes{v1.0j}{2000/02/29}{\LaTeX\ \texttt{!<1999/12/01!>}版用に修正}
% \changes{v1.0k}{2000/11/03}{\LaTeX\ \texttt{!<2000/06/01!>}版用に修正}
% \changes{v1.0l}{2001/09/04}{\LaTeX\ \texttt{!<2001/06/01!>}版用に修正}
% \changes{v1.0m}{2004/08/10}{\LaTeX\ \texttt{!<2003/12/01!>}版対応確認}
% \changes{v1.0s}{2016/02/01}{\LaTeX\ \texttt{!<2015/01/01!>}版用に修正}
% \changes{v1.0u}{2016/04/17}{\LaTeX\ \texttt{!<2016/03/31!>}版対応確認}
% \changes{v1.1a}{2017/02/20}{\LaTeX\ \texttt{!<2017/01/01!>}版対応確認}
% \changes{v1.1c}{2017/04/23}{\LaTeX\ \texttt{!<2017/04/15!>}版対応確認}
%
% このバージョンのup\LaTeXe{}は、次のバージョンの\LaTeX{}\footnote{%
% \LaTeX\ authors: Johannes Braams, David Carlisle, Alan Jeffrey,
%   Leslie Lamport, Frank Mittelbach, Chris Rowley, Rainer Sch\"opf}を
% もとにしています。
%    \begin{macrocode}
%<*2ekernel>
%\def\fmtname{LaTeX2e}
%\edef\fmtversion
%</2ekernel>
%<latexrelease>\edef\latexreleaseversion
%<platexrelease>\edef\p@known@latexreleaseversion
%<*2ekernel|latexrelease|platexrelease>
   {2017/04/15}
%</2ekernel|latexrelease|platexrelease>
%    \end{macrocode}
%
% \begin{macro}{\pfmtname}
% \begin{macro}{\pfmtversion}
% \begin{macro}{\ppatch@level}
% up\LaTeXe{}のフォーマットファイル名とバージョンです。
% \changes{v1.0x}{2016/06/19}{パッチレベルを\file{plvers.dtx}で設定}
%    \begin{macrocode}
%<*plcore>
\def\pfmtname{pLaTeX2e}
\def\pfmtversion
%</plcore>
%<platexrelease>\edef\platexreleaseversion
%<*plcore|platexrelease>
   {2017/10/28u01}
%</plcore|platexrelease>
%<*plcore>
\def\ppatch@level{1}
%</plcore>
%    \end{macrocode}
% \end{macro}
% \end{macro}
% \end{macro}
%
% \subsection{\LaTeX~2.09互換モードの抑制}
%
% \begin{macro}{\documentstyle}
% p\LaTeX{}は、|\documentclass|の代わりに|\documentstyle|が使われると
% \LaTeX~2.09互換モードに入ります。しかし、up\LaTeX{}は新しいマクロ
% パッケージですので、\LaTeX~2.09互換モードをサポートしません。
% このため、\file{plcore.dtx}の定義を上書きして明確なエラーを出します。
% \changes{v1.0w-u01}{2016/05/21}{サポート外の\LaTeX~2.09互換モードが
%    使われた場合に明確なエラーを出すようにした。}
%    \begin{macrocode}
%<*plfinal>
\def\documentstyle{%
  \@latex@error{upLaTeX does NOT support LaTeX 2.09 compatibility
    mode.\MessageBreak Use \noexpand\documentclass instead}{%
    \noexpand\documentstyle is an old convention of LaTeX 2.09,
    which has been\MessageBreak obsolete since 1995. upLaTeX is
    first released in 2007, so we do\MessageBreak not provide any
    emulation of the LaTeX 2.09 author environment.\MessageBreak
    New documents should use Standard LaTeX conventions, and
    start\MessageBreak with the \noexpand\documentclass command.}%
  \documentclass}
%    \end{macrocode}
% \end{macro}
%
% \subsection{パッチファイルのロード}
%
% 次の部分は、up\LaTeXe{}のパッチファイルをロードするためのコードです。
% バグを修正するためのパッチを配布するかもしれません。
%
% パッチファイルをロードするコードはコメントアウトしました。
% \changes{v1.0v}{2016/05/07}{パッチファイルをロードするのをやめた。}
%    \begin{macrocode}
%\IfFileExists{uplpatch.ltx}
%  {\typeout{************************************^^J%
%            * Appliying patch file uplpatch.ltx *^^J%
%            ************************************}
%  \def\pfmtversion@topatch{unknown}
%  \input{uplpatch.ltx}
%  \ifx\pfmtversion\pfmtversion@topatch
%    \ifx\ppatch@level\@undefined
%      \typeout{^^J^^J^^J%
%   !!!!!!!!!!!!!!!!!!!!!!!!!!!!!!!!!!!!!!!!!!!!!!!!!!!!!!!^^J%
%   !! Patch file `uplpatch.ltx' (for version <\pfmtversion@topatch>)^^J%
%   !! is not suitable for version <\pfmtversion> of upLaTeX.^^J^^J%
%   !! Please check if iniptex found an old patch file:^^J%
%   !! --- if so, rename it or delete it, and redo the^^J%
%   !!     iniptex run.^^J%
%   !!!!!!!!!!!!!!!!!!!!!!!!!!!!!!!!!!!!!!!!!!!!!!!!!!!!!!!^^J}%
%      \batchmode \@@end
%    \fi
%  \else
%      \typeout{^^J^^J^^J%
%   !!!!!!!!!!!!!!!!!!!!!!!!!!!!!!!!!!!!!!!!!!!!!!!!!!!!!!!^^J%
%   !! Patch file `uplpatch.ltx' (for version <\pfmtversion@topatch>)^^J%
%   !! is not suitable for version <\pfmtversion> of upLaTeX.^^J%
%   !!^^J%
%   !! Please check if iniptex found an old patch file:^^J%
%   !! --- if so, rename it or delete it, and redo the^^J%
%   !!     iniptex run.^^J%
%   !!!!!!!!!!!!!!!!!!!!!!!!!!!!!!!!!!!!!!!!!!!!!!!!!!!!!!!^^J}%
%      \batchmode \@@end
%  \fi
%  \let\pfmtversion@topatch\relax
%  }{}
%    \end{macrocode}
%
% \subsection{起動時に表示するバナー}
%
% \begin{macro}{\everyjob}
% 起動時に表示される文字列です。
% \LaTeX{}にパッチがあてられている場合は、それも表示します。
%
%\iffalse
% この実装については\file{uplatex.dtx}のコメントを参照。(2016/09/14)
%\fi
%
% \changes{v1.0v}{2016/05/07}{起動時の文字列を最新の\LaTeX{}に合わせた。}
% \changes{v1.0w}{2016/05/12}{起動時の文字列に入れる\LaTeX{}のバージョンを
%    元の\LaTeX{}のバナーから引き継ぐように改良}
% \changes{v1.1}{2016/09/14}{起動時のバナーを取得するコードを改良}
% \changes{v1.1d}{2017/09/24}{パッチレベルが負の数の場合をpre-release扱いへ}
%    \begin{macrocode}
\ifx\patch@level\@undefined % fallback if undefined in LaTeX
  \def\patch@level{0}\fi
\ifx\ppatch@level\@undefined % fallback if undefined in upLaTeX
  \def\ppatch@level{0}\fi
\begingroup
  \def\parse@@BANNER\typeout#1\typeout#2#3\relax{#1}
  \edef\platexTMP{%
    \ifnum\ppatch@level=0
      \everyjob{\noexpand\typeout{%
        \pfmtname\space<\pfmtversion>\space
          (based on \expandafter\parse@@BANNER\platexBANNER)}}%
    \else\ifnum\ppatch@level>0
      \everyjob{\noexpand\typeout{%
        \pfmtname\space<\pfmtversion>+\ppatch@level\space
          (based on \expandafter\parse@@BANNER\platexBANNER)}}%
    \else
      \everyjob{\noexpand\typeout{%
        \pfmtname\space<\pfmtversion>-pre\ppatch@level\space
          (based on \expandafter\parse@@BANNER\platexBANNER)}}%
    \fi\fi
  }
\expandafter
\endgroup \platexTMP
%    \end{macrocode}
%
% p\LaTeX{}やup\LaTeX{}は、独自のハイフネーション・パターンを定義していません。
% \TeX\ Liveの標準的インストールでは、代わりに\LaTeX{}が読み込んでいる
% Babelパッケージのものが適用されるはずですから、起動時の文字列にも
% \file{hyphen.cfg}のバージョンを反映します(Babelパッケージの
% \file{hyphen.cfg}でない場合は、何も表示されず空行になるはずです)。
%
%\iffalse
% この実装については\file{uplatex.dtx}のコメントを参照。(2016/09/14)
%\fi
%
% \changes{v1.0w-u00}{2016/05/12}{起動時の文字列に入れるBabelのバージョンを
%    元の\LaTeX{}のバナーから取得するコードを\file{uplatex.ini}から取り入れた}
%    \begin{macrocode}
\begingroup
  \def\parse@@BANNER\typeout#1\typeout#2#3\relax{#2}
  \edef\platexTMP{%
    \the\everyjob\noexpand\typeout{\expandafter\parse@@BANNER\platexBANNER}%
  }
  \everyjob=\expandafter{\platexTMP}%
  \edef\platexTMP{%
    \noexpand\let\noexpand\platexBANNER=\noexpand\@undefined
    \noexpand\everyjob={\the\everyjob}%
  }
  \expandafter
\endgroup \platexTMP
%</plfinal>
%    \end{macrocode}
% \end{macro}
%
% ^^A 起動時に\file{uplatex.cfg}がある場合、それを読み込むようにする
% ^^A コードは、\file{uplcore.ltx}から\file{uplatex.ltx}へ移動しました。
% \changes{v1.0y-u01}{2016/06/29}{\file{uplatex.cfg}の読み込みを追加}
% \changes{v1.0z-u01}{2016/08/26}{\file{uplatex.cfg}の読み込みを
%    \file{uplcore.ltx}から\file{uplatex.ltx}へ移動}
%
% \subsection{ハイフネーション関連}
%
% \begin{macro}{\l@nohyphenation}
% \LaTeXe\ 2017-04-15で、|\verb|の途中でハイフネーションが起きないように
% する修正が入りました。この修正には|\l@nohyphenation|が定義済みでなければ
% なりませんが、通常はBabelの定義ファイルによって提供されています。
% \LaTeXe{}は特殊な状況も想定してltfinalで対策しているようですので、
% p\LaTeXe{}も念のためplfinalで対策します(参考:latex2e svn r1405)。
% \changes{v1.1b}{2017/03/19}{\cs{l@nohyphenation}の定義を保証
%    (sync with ltfinal 2017/03/09 v2.0t)}
%    \begin{macrocode}
%<*plfinal>
\ifx\l@nohyphenation \@undefined
  \newlanguage\l@nohyphenation
\fi
%    \end{macrocode}
% \end{macro}
%
% \begin{macro}{\document@default@language}
% \LaTeXe\ 2017-04-15で導入されたパラメータです。更新タイミングのずれの
% 可能性を考慮し、p\LaTeXe{}でも準備しておきます。verbatim環境の途中で
% 改ページが起きた場合にヘッダでハイフネーションが抑制されないように、
% |\@outputpage|で|\language|をリセットするときに使われます
% (参考:latex2e svn r1407)。
% \changes{v1.1b}{2017/03/19}{\cs{document@default@language}の定義を保証
%    (sync with ltfinal 2017/03/09 v2.0t)}
%    \begin{macrocode}
\ifx\document@default@language \@undefined
  \let\document@default@language\m@ne
\fi
%</plfinal>
%    \end{macrocode}
% \end{macro}
%
% \subsection{latexreleaseパッケージへの対応}
%
% 最後に、latexreleaseパッケージへの対応です。
% \begin{macro}{\plIncludeInRelease}
% \changes{v1.0t}{2016/02/03}{\cs{plIncludeInRelease}と
%    \cs{plEndIncludeInRelease}を新設。}
%    \begin{macrocode}
%<*plcore|platexrelease>
\def\plIncludeInRelease#1{\kernel@ifnextchar[%
  {\@plIncludeInRelease{#1}}
  {\@plIncludeInRelease{#1}[#1]}}
%    \end{macrocode}
%
%    \begin{macrocode}
\def\@plIncludeInRelease#1[#2]{\@plIncludeInRele@se{#2}}
%    \end{macrocode}
%
%    \begin{macrocode}
\def\@plIncludeInRele@se#1#2#3{%
  \toks@{[#1] #3}%
  \expandafter\ifx\csname\string#2+\@currname+IIR\endcsname\relax
    \ifnum\expandafter\@parse@version#1//00\@nil
          >\expandafter\@parse@version\pfmtversion//00\@nil
      \GenericInfo{}{Skipping: \the\toks@}%
     \expandafter\expandafter\expandafter\@gobble@plIncludeInRelease
    \else
      \GenericInfo{}{Applying: \the\toks@}%
      \expandafter\let\csname\string#2+\@currname+IIR\endcsname\@empty
    \fi
  \else
    \GenericInfo{}{Already applied: \the\toks@}%
    \expandafter\@gobble@plIncludeInRelease
  \fi
}
%    \end{macrocode}
%
%    \begin{macrocode}
\long\def\@gobble@plIncludeInRelease#1\plEndIncludeInRelease{}
\let\plEndIncludeInRelease\relax
%</plcore|platexrelease>
%    \end{macrocode}
% \end{macro}
%
% \LaTeXe{}が提供するlatexreleaseパッケージが読み込まれていて、
% かつp\LaTeXe{}が提供するplatexreleaseパッケージが読み込まれていない
% 場合は、警告を出します。
% \changes{v1.0s}{2016/02/01}{latexrelease利用時に警告を出すようにした}
%    \begin{macrocode}
%<*plfinal>
\AtBeginDocument{%
  \@ifpackageloaded{latexrelease}{%
    \@ifpackageloaded{platexrelease}{}{%
      \@latex@warning@no@line{%
        Package latexrelease is loaded.\MessageBreak
        Some patches in pLaTeX2e core may be overwritten.\MessageBreak
        Consider using platexrelease.\MessageBreak
        See platex.pdf for detail}%
    }%
  }{}%
}
%</plfinal>
%    \end{macrocode}
%
% \Finale
%
\endinput

\endgroup

% Add the patch version if available.
\def\Xpatch{}
\ifx\patchdate\Xpatch\else
  \edef\@date{\@date\space version \patchdate}
\fi

% Obtain the last update info, as upLaTeX does not change format date
% -> if successful, reconstruct the date completely
\def\lastupd@te{0000/00/00}
\begingroup
   \def\ProvidesFile#1[#2 #3]{%
      \def\@tempd@te{#2}\endinput
      \@ifl@t@r{\@tempd@te}{\lastupd@te}{%
         \global\let\lastupd@te\@tempd@te
      }{}}
   \let\ProvidesClass\ProvidesFile
   \let\ProvidesPackage\ProvidesFile
   % \iffalse meta-comment
%% File: uplvers.dtx
%
%    pLaTeX version setting file:
%       Copyright 1995-2006 ASCII Corporation.
%    and modified for upLaTeX
%
%  Copyright (c) 2010 ASCII MEDIA WORKS
%  Copyright (c) 2016 Takuji Tanaka
%  Copyright (c) 2016-2017 Japanese TeX Development Community
%
%  This file is part of the upLaTeX2e system (community edition).
%  --------------------------------------------------------------
%
% \fi
%
%
% \setcounter{StandardModuleDepth}{1}
% \StopEventually{}
%
% \iffalse
% \changes{v1.0}{1995/05/16}{p\LaTeXe\ 用に\file{ltvers.dtx}を修正}
% \changes{v1.0a}{1995/08/30}{\LaTeX\ \texttt{!<1995/06/01!>}版用に修正}
% \changes{v1.0b}{1996/01/31}{\LaTeX\ \texttt{!<1995/12/01!>}版用に修正}
% \changes{v1.0c}{1997/01/11}{\LaTeX\ \texttt{!<1996/06/01!>}版用に修正}
% \changes{v1.0d}{1997/01/23}{\LaTeX\ \texttt{!<1996/12/01!>}版用に修正}
% \changes{v1.0e}{1997/07/02}{\LaTeX\ \texttt{!<1997/06/01!>}版用に修正}
% \changes{v1.0f}{1998/02/17}{\LaTeX\ \texttt{!<1997/12/01!>}版用に修正}
% \changes{v1.0g}{1998/09/01}{\LaTeX\ \texttt{!<1998/06/01!>}版用に修正}
% \changes{v1.0h}{1999/04/05}{\LaTeX\ \texttt{!<1998/12/01!>}版用に修正}
% \changes{v1.0i}{1999/08/09}{\LaTeX\ \texttt{!<1999/06/01!>}版用に修正}
% \changes{v1.0j}{2000/02/29}{\LaTeX\ \texttt{!<1999/12/01!>}版用に修正}
% \changes{v1.0k}{2000/11/03}{\LaTeX\ \texttt{!<2000/06/01!>}版用に修正}
% \changes{v1.0l}{2001/09/04}{\LaTeX\ \texttt{!<2001/06/01!>}版用に修正}
% \changes{v1.0m}{2004/08/10}{\LaTeX\ \texttt{!<2003/12/01!>}版対応確認}
% \changes{v1.0n}{2005/01/04}{plfonts.dtxバグ修正}
% \changes{v1.0o}{2006/01/04}{plfonts.dtxバグ修正}
% \changes{v1.0p}{2006/06/27}{plfonts.dtx \LaTeX\ \texttt{!<2005/12/01!>}対応}
% \changes{v1.0q}{2006/11/10}{plfonts.dtxバグ修正}
% \changes{v1.0q-u00}{2011/05/07}{p\LaTeX{}用からup\LaTeX{}用に修正。}
% \changes{v1.0r}{2016/01/26}{plcore.dtx p\TeX\ (r28720)対応}
% \changes{v1.0s}{2016/02/01}{\LaTeX\ \texttt{!<2015/01/01!>}のlatexreleaseに
%    対応するためのコードを導入}
% \changes{v1.0t}{2016/02/03}{\cs{plIncludeInRelease}と
%    \cs{plEndIncludeInRelease}を新設。}
% \changes{v1.0u}{2016/04/17}{\LaTeX\ \texttt{!<2016/03/31!>}版対応確認}
% \changes{v1.0u-u00}{2016/04/17}{p\LaTeX{}の変更に追随。}
% \changes{v1.0v}{2016/05/07}{パッチファイルをロードするのをやめた。}
% \changes{v1.0v}{2016/05/07}{起動時の文字列を最新の\LaTeX{}に合わせた。}
% \changes{v1.0w}{2016/05/12}{起動時の文字列に入れる\LaTeX{}のバージョンを
%    元の\LaTeX{}のバナーから引き継ぐように改良}
% \changes{v1.0w-u00}{2016/05/12}{起動時の文字列に入れるBabelのバージョンを
%    元の\LaTeX{}のバナーから取得するコードを\file{uplatex.ini}から取り入れた}
% \changes{v1.0w-u01}{2016/05/21}{サポート外の\LaTeX~2.09互換モードが
%    使われた場合に明確なエラーを出すようにした。}
% \changes{v1.0x}{2016/06/19}{パッチレベルを\file{plvers.dtx}で設定}
% \changes{v1.0x-u01}{2016/06/19}{p\LaTeX{}の変更に追随。}
% \changes{v1.0y-u01}{2016/06/29}{\file{uplatex.cfg}の読み込みを追加}
% \changes{v1.0z-u01}{2016/08/26}{\file{uplatex.cfg}の読み込みを
%    \file{uplcore.ltx}から\file{uplatex.ltx}へ移動}
% \changes{v1.1}{2016/09/14}{起動時のバナーを取得するコードを改良}
% \changes{v1.1-u01}{2016/09/14}{p\LaTeX{}の変更に追随。}
% \changes{v1.1a}{2017/02/20}{\LaTeX\ \texttt{!<2017/01/01!>}版対応確認}
% \changes{v1.1a-u01}{2017/03/05}{p\LaTeX{}の変更に追随。}
% \changes{v1.1b}{2017/03/19}{\cs{l@nohyphenation}の定義を保証
%    (sync with ltfinal 2017/03/09 v2.0t)}
% \changes{v1.1b}{2017/03/19}{\cs{document@default@language}の定義を保証
%    (sync with ltfinal 2017/03/09 v2.0t)}
% \changes{v1.1b-u01}{2017/03/19}{p\LaTeX{}の変更に追随。}
% \changes{v1.1c}{2017/04/23}{\LaTeX\ \texttt{!<2017/04/15!>}版対応確認}
% \changes{v1.1c-u01}{2017/05/04}{p\LaTeX{}の変更に追随。}
% \changes{v1.1d}{2017/09/24}{パッチレベルが負の数の場合をpre-release扱いへ}
% \changes{v1.1d-u01}{2017/09/24}{p\LaTeX{}の変更に追随。}
% \fi
%
% \iffalse
%<*driver>
% \fi
\ProvidesFile{uplvers.dtx}[2017/09/24 v1.1d-u01 upLaTeX Kernel (Version Info)]
% \iffalse
\documentclass{jltxdoc}
\GetFileInfo{uplvers.dtx}
\author{Ken Nakano \& Hideaki Togashi \& TTK}
\title{\filename}
\date{作成日:\filedate}
\begin{document}
  \maketitle
  \DocInput{\filename}
\end{document}
%</driver>
% \fi
%
% \section{バージョンの設定}
% まず、このディストリビューションでのup\LaTeXe{}の日付とバージョン番号
% を定義します。また、up\LaTeXe{}が起動されたときに表示される文字列の
% 設定もします。
%
% \changes{v1.0}{1995/05/16}{p\LaTeXe\ 用に\file{ltvers.dtx}を修正}
% \changes{v1.0a}{1995/08/30}{\LaTeX\ \texttt{!<1995/06/01!>}版用に修正}
% \changes{v1.0b}{1996/01/31}{\LaTeX\ \texttt{!<1995/12/01!>}版用に修正}
% \changes{v1.0c}{1997/01/11}{\LaTeX\ \texttt{!<1996/06/01!>}版用に修正}
% \changes{v1.0d}{1997/01/23}{\LaTeX\ \texttt{!<1996/12/01!>}版用に修正}
% \changes{v1.0e}{1997/07/02}{\LaTeX\ \texttt{!<1997/06/01!>}版用に修正}
% \changes{v1.0f}{1998/02/17}{\LaTeX\ \texttt{!<1997/12/01!>}版用に修正}
% \changes{v1.0g}{1998/09/01}{\LaTeX\ \texttt{!<1998/06/01!>}版用に修正}
% \changes{v1.0h}{1999/04/05}{\LaTeX\ \texttt{!<1998/12/01!>}版用に修正}
% \changes{v1.0i}{1999/08/09}{\LaTeX\ \texttt{!<1999/06/01!>}版用に修正}
% \changes{v1.0j}{2000/02/29}{\LaTeX\ \texttt{!<1999/12/01!>}版用に修正}
% \changes{v1.0k}{2000/11/03}{\LaTeX\ \texttt{!<2000/06/01!>}版用に修正}
% \changes{v1.0l}{2001/09/04}{\LaTeX\ \texttt{!<2001/06/01!>}版用に修正}
% \changes{v1.0m}{2004/08/10}{\LaTeX\ \texttt{!<2003/12/01!>}版対応確認}
% \changes{v1.0s}{2016/02/01}{\LaTeX\ \texttt{!<2015/01/01!>}版用に修正}
% \changes{v1.0u}{2016/04/17}{\LaTeX\ \texttt{!<2016/03/31!>}版対応確認}
% \changes{v1.1a}{2017/02/20}{\LaTeX\ \texttt{!<2017/01/01!>}版対応確認}
% \changes{v1.1c}{2017/04/23}{\LaTeX\ \texttt{!<2017/04/15!>}版対応確認}
%
% このバージョンのup\LaTeXe{}は、次のバージョンの\LaTeX{}\footnote{%
% \LaTeX\ authors: Johannes Braams, David Carlisle, Alan Jeffrey,
%   Leslie Lamport, Frank Mittelbach, Chris Rowley, Rainer Sch\"opf}を
% もとにしています。
%    \begin{macrocode}
%<*2ekernel>
%\def\fmtname{LaTeX2e}
%\edef\fmtversion
%</2ekernel>
%<latexrelease>\edef\latexreleaseversion
%<platexrelease>\edef\p@known@latexreleaseversion
%<*2ekernel|latexrelease|platexrelease>
   {2017/04/15}
%</2ekernel|latexrelease|platexrelease>
%    \end{macrocode}
%
% \begin{macro}{\pfmtname}
% \begin{macro}{\pfmtversion}
% \begin{macro}{\ppatch@level}
% up\LaTeXe{}のフォーマットファイル名とバージョンです。
% \changes{v1.0x}{2016/06/19}{パッチレベルを\file{plvers.dtx}で設定}
%    \begin{macrocode}
%<*plcore>
\def\pfmtname{pLaTeX2e}
\def\pfmtversion
%</plcore>
%<platexrelease>\edef\platexreleaseversion
%<*plcore|platexrelease>
   {2017/10/28u01}
%</plcore|platexrelease>
%<*plcore>
\def\ppatch@level{1}
%</plcore>
%    \end{macrocode}
% \end{macro}
% \end{macro}
% \end{macro}
%
% \subsection{\LaTeX~2.09互換モードの抑制}
%
% \begin{macro}{\documentstyle}
% p\LaTeX{}は、|\documentclass|の代わりに|\documentstyle|が使われると
% \LaTeX~2.09互換モードに入ります。しかし、up\LaTeX{}は新しいマクロ
% パッケージですので、\LaTeX~2.09互換モードをサポートしません。
% このため、\file{plcore.dtx}の定義を上書きして明確なエラーを出します。
% \changes{v1.0w-u01}{2016/05/21}{サポート外の\LaTeX~2.09互換モードが
%    使われた場合に明確なエラーを出すようにした。}
%    \begin{macrocode}
%<*plfinal>
\def\documentstyle{%
  \@latex@error{upLaTeX does NOT support LaTeX 2.09 compatibility
    mode.\MessageBreak Use \noexpand\documentclass instead}{%
    \noexpand\documentstyle is an old convention of LaTeX 2.09,
    which has been\MessageBreak obsolete since 1995. upLaTeX is
    first released in 2007, so we do\MessageBreak not provide any
    emulation of the LaTeX 2.09 author environment.\MessageBreak
    New documents should use Standard LaTeX conventions, and
    start\MessageBreak with the \noexpand\documentclass command.}%
  \documentclass}
%    \end{macrocode}
% \end{macro}
%
% \subsection{パッチファイルのロード}
%
% 次の部分は、up\LaTeXe{}のパッチファイルをロードするためのコードです。
% バグを修正するためのパッチを配布するかもしれません。
%
% パッチファイルをロードするコードはコメントアウトしました。
% \changes{v1.0v}{2016/05/07}{パッチファイルをロードするのをやめた。}
%    \begin{macrocode}
%\IfFileExists{uplpatch.ltx}
%  {\typeout{************************************^^J%
%            * Appliying patch file uplpatch.ltx *^^J%
%            ************************************}
%  \def\pfmtversion@topatch{unknown}
%  \input{uplpatch.ltx}
%  \ifx\pfmtversion\pfmtversion@topatch
%    \ifx\ppatch@level\@undefined
%      \typeout{^^J^^J^^J%
%   !!!!!!!!!!!!!!!!!!!!!!!!!!!!!!!!!!!!!!!!!!!!!!!!!!!!!!!^^J%
%   !! Patch file `uplpatch.ltx' (for version <\pfmtversion@topatch>)^^J%
%   !! is not suitable for version <\pfmtversion> of upLaTeX.^^J^^J%
%   !! Please check if iniptex found an old patch file:^^J%
%   !! --- if so, rename it or delete it, and redo the^^J%
%   !!     iniptex run.^^J%
%   !!!!!!!!!!!!!!!!!!!!!!!!!!!!!!!!!!!!!!!!!!!!!!!!!!!!!!!^^J}%
%      \batchmode \@@end
%    \fi
%  \else
%      \typeout{^^J^^J^^J%
%   !!!!!!!!!!!!!!!!!!!!!!!!!!!!!!!!!!!!!!!!!!!!!!!!!!!!!!!^^J%
%   !! Patch file `uplpatch.ltx' (for version <\pfmtversion@topatch>)^^J%
%   !! is not suitable for version <\pfmtversion> of upLaTeX.^^J%
%   !!^^J%
%   !! Please check if iniptex found an old patch file:^^J%
%   !! --- if so, rename it or delete it, and redo the^^J%
%   !!     iniptex run.^^J%
%   !!!!!!!!!!!!!!!!!!!!!!!!!!!!!!!!!!!!!!!!!!!!!!!!!!!!!!!^^J}%
%      \batchmode \@@end
%  \fi
%  \let\pfmtversion@topatch\relax
%  }{}
%    \end{macrocode}
%
% \subsection{起動時に表示するバナー}
%
% \begin{macro}{\everyjob}
% 起動時に表示される文字列です。
% \LaTeX{}にパッチがあてられている場合は、それも表示します。
%
%\iffalse
% この実装については\file{uplatex.dtx}のコメントを参照。(2016/09/14)
%\fi
%
% \changes{v1.0v}{2016/05/07}{起動時の文字列を最新の\LaTeX{}に合わせた。}
% \changes{v1.0w}{2016/05/12}{起動時の文字列に入れる\LaTeX{}のバージョンを
%    元の\LaTeX{}のバナーから引き継ぐように改良}
% \changes{v1.1}{2016/09/14}{起動時のバナーを取得するコードを改良}
% \changes{v1.1d}{2017/09/24}{パッチレベルが負の数の場合をpre-release扱いへ}
%    \begin{macrocode}
\ifx\patch@level\@undefined % fallback if undefined in LaTeX
  \def\patch@level{0}\fi
\ifx\ppatch@level\@undefined % fallback if undefined in upLaTeX
  \def\ppatch@level{0}\fi
\begingroup
  \def\parse@@BANNER\typeout#1\typeout#2#3\relax{#1}
  \edef\platexTMP{%
    \ifnum\ppatch@level=0
      \everyjob{\noexpand\typeout{%
        \pfmtname\space<\pfmtversion>\space
          (based on \expandafter\parse@@BANNER\platexBANNER)}}%
    \else\ifnum\ppatch@level>0
      \everyjob{\noexpand\typeout{%
        \pfmtname\space<\pfmtversion>+\ppatch@level\space
          (based on \expandafter\parse@@BANNER\platexBANNER)}}%
    \else
      \everyjob{\noexpand\typeout{%
        \pfmtname\space<\pfmtversion>-pre\ppatch@level\space
          (based on \expandafter\parse@@BANNER\platexBANNER)}}%
    \fi\fi
  }
\expandafter
\endgroup \platexTMP
%    \end{macrocode}
%
% p\LaTeX{}やup\LaTeX{}は、独自のハイフネーション・パターンを定義していません。
% \TeX\ Liveの標準的インストールでは、代わりに\LaTeX{}が読み込んでいる
% Babelパッケージのものが適用されるはずですから、起動時の文字列にも
% \file{hyphen.cfg}のバージョンを反映します(Babelパッケージの
% \file{hyphen.cfg}でない場合は、何も表示されず空行になるはずです)。
%
%\iffalse
% この実装については\file{uplatex.dtx}のコメントを参照。(2016/09/14)
%\fi
%
% \changes{v1.0w-u00}{2016/05/12}{起動時の文字列に入れるBabelのバージョンを
%    元の\LaTeX{}のバナーから取得するコードを\file{uplatex.ini}から取り入れた}
%    \begin{macrocode}
\begingroup
  \def\parse@@BANNER\typeout#1\typeout#2#3\relax{#2}
  \edef\platexTMP{%
    \the\everyjob\noexpand\typeout{\expandafter\parse@@BANNER\platexBANNER}%
  }
  \everyjob=\expandafter{\platexTMP}%
  \edef\platexTMP{%
    \noexpand\let\noexpand\platexBANNER=\noexpand\@undefined
    \noexpand\everyjob={\the\everyjob}%
  }
  \expandafter
\endgroup \platexTMP
%</plfinal>
%    \end{macrocode}
% \end{macro}
%
% ^^A 起動時に\file{uplatex.cfg}がある場合、それを読み込むようにする
% ^^A コードは、\file{uplcore.ltx}から\file{uplatex.ltx}へ移動しました。
% \changes{v1.0y-u01}{2016/06/29}{\file{uplatex.cfg}の読み込みを追加}
% \changes{v1.0z-u01}{2016/08/26}{\file{uplatex.cfg}の読み込みを
%    \file{uplcore.ltx}から\file{uplatex.ltx}へ移動}
%
% \subsection{ハイフネーション関連}
%
% \begin{macro}{\l@nohyphenation}
% \LaTeXe\ 2017-04-15で、|\verb|の途中でハイフネーションが起きないように
% する修正が入りました。この修正には|\l@nohyphenation|が定義済みでなければ
% なりませんが、通常はBabelの定義ファイルによって提供されています。
% \LaTeXe{}は特殊な状況も想定してltfinalで対策しているようですので、
% p\LaTeXe{}も念のためplfinalで対策します(参考:latex2e svn r1405)。
% \changes{v1.1b}{2017/03/19}{\cs{l@nohyphenation}の定義を保証
%    (sync with ltfinal 2017/03/09 v2.0t)}
%    \begin{macrocode}
%<*plfinal>
\ifx\l@nohyphenation \@undefined
  \newlanguage\l@nohyphenation
\fi
%    \end{macrocode}
% \end{macro}
%
% \begin{macro}{\document@default@language}
% \LaTeXe\ 2017-04-15で導入されたパラメータです。更新タイミングのずれの
% 可能性を考慮し、p\LaTeXe{}でも準備しておきます。verbatim環境の途中で
% 改ページが起きた場合にヘッダでハイフネーションが抑制されないように、
% |\@outputpage|で|\language|をリセットするときに使われます
% (参考:latex2e svn r1407)。
% \changes{v1.1b}{2017/03/19}{\cs{document@default@language}の定義を保証
%    (sync with ltfinal 2017/03/09 v2.0t)}
%    \begin{macrocode}
\ifx\document@default@language \@undefined
  \let\document@default@language\m@ne
\fi
%</plfinal>
%    \end{macrocode}
% \end{macro}
%
% \subsection{latexreleaseパッケージへの対応}
%
% 最後に、latexreleaseパッケージへの対応です。
% \begin{macro}{\plIncludeInRelease}
% \changes{v1.0t}{2016/02/03}{\cs{plIncludeInRelease}と
%    \cs{plEndIncludeInRelease}を新設。}
%    \begin{macrocode}
%<*plcore|platexrelease>
\def\plIncludeInRelease#1{\kernel@ifnextchar[%
  {\@plIncludeInRelease{#1}}
  {\@plIncludeInRelease{#1}[#1]}}
%    \end{macrocode}
%
%    \begin{macrocode}
\def\@plIncludeInRelease#1[#2]{\@plIncludeInRele@se{#2}}
%    \end{macrocode}
%
%    \begin{macrocode}
\def\@plIncludeInRele@se#1#2#3{%
  \toks@{[#1] #3}%
  \expandafter\ifx\csname\string#2+\@currname+IIR\endcsname\relax
    \ifnum\expandafter\@parse@version#1//00\@nil
          >\expandafter\@parse@version\pfmtversion//00\@nil
      \GenericInfo{}{Skipping: \the\toks@}%
     \expandafter\expandafter\expandafter\@gobble@plIncludeInRelease
    \else
      \GenericInfo{}{Applying: \the\toks@}%
      \expandafter\let\csname\string#2+\@currname+IIR\endcsname\@empty
    \fi
  \else
    \GenericInfo{}{Already applied: \the\toks@}%
    \expandafter\@gobble@plIncludeInRelease
  \fi
}
%    \end{macrocode}
%
%    \begin{macrocode}
\long\def\@gobble@plIncludeInRelease#1\plEndIncludeInRelease{}
\let\plEndIncludeInRelease\relax
%</plcore|platexrelease>
%    \end{macrocode}
% \end{macro}
%
% \LaTeXe{}が提供するlatexreleaseパッケージが読み込まれていて、
% かつp\LaTeXe{}が提供するplatexreleaseパッケージが読み込まれていない
% 場合は、警告を出します。
% \changes{v1.0s}{2016/02/01}{latexrelease利用時に警告を出すようにした}
%    \begin{macrocode}
%<*plfinal>
\AtBeginDocument{%
  \@ifpackageloaded{latexrelease}{%
    \@ifpackageloaded{platexrelease}{}{%
      \@latex@warning@no@line{%
        Package latexrelease is loaded.\MessageBreak
        Some patches in pLaTeX2e core may be overwritten.\MessageBreak
        Consider using platexrelease.\MessageBreak
        See platex.pdf for detail}%
    }%
  }{}%
}
%</plfinal>
%    \end{macrocode}
%
% \Finale
%
\endinput

   % \iffalse meta-comment
%% File: uplfonts.dtx
%
%    pLaTeX fonts files:
%       Copyright 1994-2006 ASCII Corporation.
%    and modified for upLaTeX
%
%  Copyright (c) 2010 ASCII MEDIA WORKS
%  Copyright (c) 2016 Takuji Tanaka
%  Copyright (c) 2016-2020 Japanese TeX Development Community
%
%  This file is part of the upLaTeX2e system (community edition).
%  --------------------------------------------------------------
%
% \fi
%
% \iffalse
%<*driver>
\ifx\JAPANESEtrue\undefined
  \expandafter\newif\csname ifJAPANESE\endcsname
  \JAPANESEtrue
\fi
\def\eTeX{$\varepsilon$-\TeX}
\def\pTeX{p\kern-.15em\TeX}
\def\epTeX{$\varepsilon$-\pTeX}
\def\pLaTeX{p\kern-.05em\LaTeX}
\def\pLaTeXe{p\kern-.05em\LaTeXe}
\def\upTeX{u\pTeX}
\def\eupTeX{$\varepsilon$-\upTeX}
\def\upLaTeX{u\pLaTeX}
\def\upLaTeXe{u\pLaTeXe}
%</driver>
% \fi
%
% \setcounter{StandardModuleDepth}{1}
% \StopEventually{}
%
% \iffalse
% \changes{v1.5-u00}{2011/05/07}{p\LaTeX{}用からup\LaTeX{}用に修正。
%     (based on plfonts.dtx 2006/11/10 v1.5)}
% \changes{v1.6a-u00}{2016/04/06}{p\LaTeX{}の変更に追随。
%     (based on plfonts.dtx 2016/04/01 v1.6a)}
% \changes{v1.6b-u00}{2016/04/30}{uptrace.styの冒頭でtracefnt.styを
%    \cs{RequirePackageWithOptions}するようにした
%     (based on plfonts.dtx 2016/04/30 v1.6b)}
% \changes{v1.6c-u00}{2016/06/06}{p\LaTeX{}の変更に追随。
%     (based on plfonts.dtx 2016/06/06 v1.6c)}
% \changes{v1.6d-u00}{2016/06/19}{p\LaTeX{}の変更に追随。
%     (based on plfonts.dtx 2016/06/19 v1.6d)}
% \changes{v1.6e-u00}{2016/06/29}{p\LaTeX{}の変更に追随。
%     (based on plfonts.dtx 2016/06/26 v1.6e)}
% \changes{v1.6f-u00}{2017/03/05}{uptrace.styのplatexrelease対応
%     (based on plfonts.dtx 2017/02/20 v1.6f)}
% \changes{v1.6g-u00}{2017/03/08}{p\LaTeX{}の変更に追随。
%     (based on plfonts.dtx 2017/03/07 v1.6g)}
% \changes{v1.6h-u00}{2017/08/05}{p\LaTeX{}の変更に追随。
%     (based on plfonts.dtx 2017/08/05 v1.6h)}
% \changes{v1.6i-u00}{2017/09/24}{p\LaTeX{}の変更に追随。
%     (based on plfonts.dtx 2017/09/24 v1.6i)}
% \changes{v1.6j-u00}{2017/11/06}{p\LaTeX{}の変更に追随。
%     (based on plfonts.dtx 2017/11/06 v1.6j)}
% \changes{v1.6k-u00}{2017/12/05}{デフォルト設定ファイルの読み込みを
%    \file{uplcore.ltx}から\file{uplatex.ltx}へ移動
%     (based on plfonts.dtx 2017/12/05 v1.6k)}
% \changes{v1.6k-u01}{2017/12/10}{uptraceパッケージは
%    ptraceパッケージを読み込むだけとした}
% \changes{v1.6k-u02}{2017/12/10}{p\LaTeX{}との統合のため、
%    up\LaTeX{}用の最小限の変更だけを定義するようにした}
% \changes{v1.6l-u02}{2018/02/04}{p\LaTeX{}の変更に追随。
%     (based on plfonts.dtx 2018/02/04 v1.6l)}
% \changes{v1.6q-u02}{2018/07/03}{p\LaTeX{}の変更に追随。
%     (based on plfonts.dtx 2018/07/03 v1.6q)}
% \changes{v1.6t-u02}{2019/09/22}{p\LaTeX{}の変更に追随。
%     (based on plfonts.dtx 2019/09/16 v1.6t)}
% \changes{v1.6v-u02}{2020/02/01}{p\LaTeX{}の変更に追随。
%     (based on plfonts.dtx 2020/02/01 v1.6v)}
% \fi
%
% \iffalse
%<*driver>
\NeedsTeXFormat{pLaTeX2e}
% \fi
\ProvidesFile{uplfonts.dtx}[2020/02/01 v1.6v-u02 upLaTeX New Font Selection Scheme]
% \iffalse
\documentclass{jltxdoc}
\GetFileInfo{uplfonts.dtx}
\title{up\LaTeXe{}のフォントコマンド\space\fileversion}
\author{Ken Nakano \& Hideaki Togashi \& TTK}
\date{作成日:\filedate}
\begin{document}
   \maketitle
   \tableofcontents
   \DocInput{\filename}
\end{document}
%</driver>
% \fi
%
% \section{概要}\label{plfonts:intro}
% ここでは、和文書体を\NFSS2のインターフェイスで選択するための
% コマンドやマクロについて説明をしています。
% また、フォント定義ファイルや初期設定ファイルなどの説明もしています。
% 新しいフォント選択コマンドの使い方については、\file{fntguide.tex}や
% \file{usrguide.tex}を参照してください。
% \changes{v1.5-u00}{2011/05/07}{p\LaTeX{}用からup\LaTeX{}用に修正。
%     (based on plfonts.dtx 2006/11/10 v1.5)}
% \changes{v1.6k-u02}{2017/12/10}{p\LaTeX{}との統合のため、
%    up\LaTeX{}用の最小限の変更だけを定義するようにした}
%
% \begin{description}
% \item[第\ref{plfonts:intro}節] この節です。このファイルの概要と
%    \dst{}プログラムのためのオプションを示しています。
% \item[第\ref{plfonts:codes}節] 実際のコードの部分です。
% \item[第\ref{plfonts:pldefs}節] プリロードフォントやエラーフォントなどの
%  初期設定について説明をしています。
% \item[第\ref{plfonts:fontdef}節] フォント定義ファイルについて
%    説明をしています。
% \end{description}
%
%
% \subsection{\dst{}プログラムのためのオプション}
% \dst{}プログラムのためのオプションを次に示します。
%
% \DeleteShortVerb{\|}
% \begin{center}
% \begin{tabular}{l|p{0.7\linewidth}}
% \emph{オプション} & \emph{意味}\\\hline
% plcore & \file{uplcore.ltx}の断片を生成するオプションでしたが、削除。\\
% trace  & \file{uptrace.sty}を生成します。\\
% JY2mc  & 横組用、明朝体のフォント定義ファイルを生成します。\\
% JY2gt  & 横組用、ゴシック体のフォント定義ファイルを生成します。\\
% JT2mc  & 縦組用、明朝体のフォント定義ファイルを生成します。\\
% JT2gt  & 縦組用、ゴシック体のフォント定義ファイルを生成します。\\
% pldefs & \file{upldefs.ltx}を生成します。次の4つのオプションを付加する
%          ことで、プリロードするフォントを選択することができます。
%          デフォルトは10ptです。\\
% xpt    & 10pt プリロード\\
% xipt   & 11pt プリロード\\
% xiipt  & 12pt プリロード\\
% ori    & \file{plfonts.tex}に似たプリロード\\
% \end{tabular}
% \end{center}
% \MakeShortVerb{\|}
%
%
%
% \section{コード}\label{plfonts:codes}
% \NFSS2の拡張は、p\LaTeX{}において\file{plfonts.dtx}から生成される
% \file{plcore.ltx}をそのまま利用するので、up\LaTeX{}では定義しません。
% 後方互換性のため、\file{uptrace.sty}を提供しますが、
% これも単に\file{ptrace.sty}を読み込むだけとします。
%
% \changes{v1.6b-u00}{2016/04/30}{uptrace.styの冒頭でtracefnt.styを
%    \cs{RequirePackageWithOptions}するようにした}
% \changes{v1.6k-u01}{2017/12/10}{uptraceパッケージは
%    ptraceパッケージを読み込むだけとした}
%    \begin{macrocode}
%<*trace>
\NeedsTeXFormat{pLaTeX2e}
\ProvidesPackage{uptrace}
     [2019/09/22 v1.6t-u02 Standard upLaTeX package (font tracing)]
\RequirePackageWithOptions{ptrace}
%</trace>
%    \end{macrocode}
%
% デフォルト設定ファイル\file{upldefs.ltx}は、もともと\file{uplcore.ltx}の途中で
% 読み込んでいましたが、2018年以降の新しいコミュニティ版\upLaTeX{}では
% \file{uplatex.ltx}から読み込むことにしました。
% 実際の中身については、第\ref{plfonts:pldefs}節を参照してください。
% \changes{v1.6k-u00}{2017/12/05}{デフォルト設定ファイルの読み込みを
%    \file{uplcore.ltx}から\file{uplatex.ltx}へ移動
%     (based on plfonts.dtx 2017/12/05 v1.6k)}
%
%
% \section{デフォルト設定ファイル}\label{plfonts:pldefs}
% ここでは、フォーマットファイルに読み込まれるデフォルト値を設定しています。
% この節での内容は\file{upldefs.ltx}に出力されます。
% このファイルの内容を\file{uplcore.ltx}に含めてもよいのですが、
% デフォルトの設定を参照しやすいように、別ファイルにしてあります。
%
% プリロードサイズは、\dst{}プログラムのオプションで変更することができます。
% これ以外の設定を変更したい場合は、\file{upldefs.ltx}を
% 直接、修正するのではなく、このファイルを\file{upldefs.cfg}という名前で
% コピーをして、そのファイルに対して修正を加えるようにしてください。
%    \begin{macrocode}
%<*pldefs>
\ProvidesFile{upldefs.ltx}
      [2020/02/01 v1.6v-u02 upLaTeX Kernel (Default settings)]
%</pldefs>
%    \end{macrocode}
%
% \subsection{テキストフォント}
% テキストフォントのための属性やエラー書体などの宣言です。
% p\LaTeX{}のデフォルトの横組エンコードはJY1、縦組エンコードはJT1ですが、
% up\LaTeX{}では横組エンコードはJY2、縦組エンコードはJT2とします。
%
% \changes{v1.6s}{2019/08/13}{Explicitly set some defaults
%    after \cs{DeclareErrorKanjiFont} change
%    (sync with ltfssini.dtx 2019/07/09 v3.1c)}
% \noindent
% 縦横エンコード共通:
%    \begin{macrocode}
%<*pldefs>
\DeclareKanjiEncodingDefaults{}{}
\DeclareErrorKanjiFont{JY2}{mc}{m}{n}{10}
\kanjifamily{mc}
\kanjiseries{m}
\kanjishape{n}
\fontsize{10}{10}
%    \end{macrocode}
% 横組エンコード:
%    \begin{macrocode}
\DeclareYokoKanjiEncoding{JY2}{}{}
\DeclareKanjiSubstitution{JY2}{mc}{m}{n}
%    \end{macrocode}
% 縦組エンコード:
%    \begin{macrocode}
\DeclareTateKanjiEncoding{JT2}{}{}
\DeclareKanjiSubstitution{JT2}{mc}{m}{n}
%    \end{macrocode}
% 縦横のエンコーディングのセット化:
% \changes{v1.6j}{2017/11/06}{縦横のエンコーディングのセット化を
%    plcoreからpldefsへ移動}
%    \begin{macrocode}
\KanjiEncodingPair{JY2}{JT2}
%    \end{macrocode}
% フォント属性のデフォルト値:
% \LaTeXe~2019-10-01までは|\shapedefault|は|\updefault|でしたが、
% \LaTeXe~2020-02-02で|\updefault|が``n''から``up''へと修正されたことに
% 伴い、|\shapedefault|は明示的に``n''に設定されました。
% \changes{v1.6v}{2020/02/01}{Set \cs{kanjishapedefault} explicitly to ``n''
%    (sync with fontdef.dtx 2019/12/17 v3.0e)}
%    \begin{macrocode}
\newcommand\mcdefault{mc}
\newcommand\gtdefault{gt}
\newcommand\kanjiencodingdefault{JY2}
\newcommand\kanjifamilydefault{\mcdefault}
\newcommand\kanjiseriesdefault{\mddefault}
\newcommand\kanjishapedefault{n}% formerly \updefault
%    \end{macrocode}
% 和文エンコードの指定:
%    \begin{macrocode}
\kanjiencoding{JY2}
%    \end{macrocode}
% フォント定義:
% これらの具体的な内容は第\ref{plfonts:fontdef}節を参照してください。
% \changes{v1.3}{1997/01/24}{Rename font definition filename.}
%    \begin{macrocode}
%%
%% This is file `jy2mc.fd',
%% generated with the docstrip utility.
%%
%% The original source files were:
%%
%% uplfonts.dtx  (with options: `JY2mc')
%% 
%% Copyright (c) 2010 ASCII MEDIA WORKS
%% Copyright (c) 2016 Takuji Tanaka
%% Copyright (c) 2016-2018 Japanese TeX Development Community
%% 
%% This file is part of the upLaTeX2e system (community edition).
%% --------------------------------------------------------------
%% 
%% File: uplfonts.dtx
\ProvidesFile{jy2mc.fd}
       [2018/07/03 v1.6q-u02 KANJI font defines]
\DeclareKanjiFamily{JY2}{mc}{}
\DeclareRelationFont{JY2}{mc}{m}{}{T1}{cmr}{m}{}
\DeclareRelationFont{JY2}{mc}{bx}{}{T1}{cmr}{bx}{}
\DeclareFontShape{JY2}{mc}{m}{n}{<->s*[0.962216]upjisr-h}{}
\DeclareFontShape{JY2}{mc}{bx}{n}{<->ssub*gt/m/n}{}
\DeclareFontShape{JY2}{mc}{b}{n}{<->ssub*mc/bx/n}{}
\endinput
%%
%% End of file `jy2mc.fd'.

%%
%% This is file `jy2gt.fd',
%% generated with the docstrip utility.
%%
%% The original source files were:
%%
%% uplfonts.dtx  (with options: `JY2gt')
%% 
%% Copyright (c) 2010 ASCII MEDIA WORKS
%% Copyright (c) 2016 Takuji Tanaka
%% Copyright (c) 2016-2018 Japanese TeX Development Community
%% 
%% This file is part of the upLaTeX2e system (community edition).
%% --------------------------------------------------------------
%% 
%% File: uplfonts.dtx
\ProvidesFile{jy2gt.fd}
       [2018/07/03 v1.6q-u02 KANJI font defines]
\DeclareKanjiFamily{JY2}{gt}{}
\DeclareRelationFont{JY2}{gt}{m}{}{T1}{cmr}{bx}{}
\DeclareFontShape{JY2}{gt}{m}{n}{<->s*[0.962216]upjisg-h}{}
\DeclareFontShape{JY2}{gt}{bx}{n}{<->ssub*gt/m/n}{}
\DeclareFontShape{JY2}{gt}{b}{n}{<->ssub*gt/bx/n}{}
\endinput
%%
%% End of file `jy2gt.fd'.

%%
%% This is file `jt2mc.fd',
%% generated with the docstrip utility.
%%
%% The original source files were:
%%
%% uplfonts.dtx  (with options: `JT2mc')
%% 
%% Copyright (c) 2010 ASCII MEDIA WORKS
%% Copyright (c) 2016 Takuji Tanaka
%% Copyright (c) 2016-2018 Japanese TeX Development Community
%% 
%% This file is part of the upLaTeX2e system (community edition).
%% --------------------------------------------------------------
%% 
%% File: uplfonts.dtx
\ProvidesFile{jt2mc.fd}
       [2018/07/03 v1.6q-u02 KANJI font defines]
\DeclareKanjiFamily{JT2}{mc}{}
\DeclareRelationFont{JT2}{mc}{m}{}{T1}{cmr}{m}{}
\DeclareRelationFont{JT2}{mc}{bx}{}{T1}{cmr}{bx}{}
\DeclareFontShape{JT2}{mc}{m}{n}{<->s*[0.962216]upjisr-v}{}
\DeclareFontShape{JT2}{mc}{bx}{n}{<->ssub*gt/m/n}{}
\DeclareFontShape{JT2}{mc}{b}{n}{<->ssub*mc/bx/n}{}
\endinput
%%
%% End of file `jt2mc.fd'.

%%
%% This is file `jt2gt.fd',
%% generated with the docstrip utility.
%%
%% The original source files were:
%%
%% uplfonts.dtx  (with options: `JT2gt')
%% 
%% Copyright (c) 2010 ASCII MEDIA WORKS
%% Copyright (c) 2016 Takuji Tanaka
%% Copyright (c) 2016 Japanese TeX Development Community
%% 
%% This file is part of the upLaTeX2e system (community edition).
%% --------------------------------------------------------------
%% 
%% File: uplfonts.dtx
\ProvidesFile{jt2gt.fd}
       [1997/01/24 v1.3 KANJI font defines]
\DeclareKanjiFamily{JT2}{gt}{}
\DeclareRelationFont{JT2}{gt}{m}{}{T1}{cmr}{bx}{}
\DeclareFontShape{JT2}{gt}{m}{n}{<->s*[0.962216]upjisg-v}{}
\DeclareFontShape{JT2}{gt}{bx}{n}{<->ssub*gt/m/n}{}
\endinput
%%
%% End of file `jt2gt.fd'.

%    \end{macrocode}
% フォントを有効にします。
%    \begin{macrocode}
\fontencoding{JT2}\selectfont
\fontencoding{JY2}\selectfont
%    \end{macrocode}
%
% \changes{v1.3b}{1997/01/30}{数式用フォントの宣言をクラスファイルに移動した}
%
%
% \subsection{プリロードフォント}
% あらかじめフォーマットファイルにロードされるフォントの宣言です。
% \dst{}プログラムのオプションでロードされるフォントのサイズを
% 変更することができます。\file{uplfmt.ins}では|xpt|を指定しています。
%    \begin{macrocode}
%<*xpt>
\DeclarePreloadSizes{JY2}{mc}{m}{n}{5,7,10,12}
\DeclarePreloadSizes{JY2}{gt}{m}{n}{5,7,10,12}
\DeclarePreloadSizes{JT2}{mc}{m}{n}{5,7,10,12}
\DeclarePreloadSizes{JT2}{gt}{m}{n}{5,7,10,12}
%</xpt>
%<*xipt>
\DeclarePreloadSizes{JY2}{mc}{m}{n}{5,7,10.95,12}
\DeclarePreloadSizes{JY2}{gt}{m}{n}{5,7,10.95,12}
\DeclarePreloadSizes{JT2}{mc}{m}{n}{5,7,10.95,12}
\DeclarePreloadSizes{JT2}{gt}{m}{n}{5,7,10.95,12}
%</xipt>
%<*xiipt>
\DeclarePreloadSizes{JY2}{mc}{m}{n}{7,9,12,14.4}
\DeclarePreloadSizes{JY2}{gt}{m}{n}{7,9,12,14.4}
\DeclarePreloadSizes{JT2}{mc}{m}{n}{7,9,12,14.4}
\DeclarePreloadSizes{JT2}{gt}{m}{n}{7,9,12,14.4}
%</xiipt>
%<*ori>
\DeclarePreloadSizes{JY2}{mc}{m}{n}
        {5,6,7,8,9,10,10.95,12,14.4,17.28,20.74,24.88}
\DeclarePreloadSizes{JY2}{gt}{m}{n}
        {5,6,7,8,9,10,10.95,12,14.4,17.28,20.74,24.88}
\DeclarePreloadSizes{JT2}{mc}{m}{n}
        {5,6,7,8,9,10,10.95,12,14.4,17.28,20.74,24.88}
\DeclarePreloadSizes{JT2}{gt}{m}{n}
        {5,6,7,8,9,10,10.95,12,14.4,17.28,20.74,24.88}
%</ori>
%    \end{macrocode}
%
%
% \subsection{組版パラメータ}
% 禁則パラメータや文字間へ挿入するスペースの設定などです。
% 実際の各文字への禁則パラメータおよびスペースの挿入の許可設定などは、
% \file{ukinsoku.tex}で行なっています。
% 具体的な設定については、\file{ukinsoku.dtx}を参照してください。
%    \begin{macrocode}
\InputIfFileExists{ukinsoku.tex}%
  {\message{Loading kinsoku patterns for japanese.}}
  {\errhelp{The configuration for kinsoku is incorrectly installed.^^J%
            If you don't understand this error message you need
            to seek^^Jexpert advice.}%
   \errmessage{OOPS! I can't find any kinsoku patterns for japanese^^J%
               \space Think of getting some or the
               uplatex2e setup will never succeed}\@@end}
%    \end{macrocode}
%
% 組版パラメータの設定をします。
% |\kanjiskip|は、漢字と漢字の間に挿入されるグルーです。
% |\noautospacing|で、挿入を中止することができます。
% デフォルトは|\autospacing|です。
%    \begin{macrocode}
\kanjiskip=0pt plus .4pt minus .5pt
\autospacing
%    \end{macrocode}
% |\xkanjiskip|は、和欧文間に自動的に挿入されるグルーです。
% |\noautoxspacing|で、挿入を中止することができます。
% デフォルトは|\autoxspacing|です。
% \changes{v1.1c}{1995/09/12}{\cs{xkanjiskip}のデフォルト値}
%    \begin{macrocode}
\xkanjiskip=.25zw plus1pt minus1pt
\autoxspacing
%    \end{macrocode}
% |\jcharwidowpenalty|は、パラグラフに対する禁則です。
% パラグラフの最後の行が1文字だけにならないように調整するために使われます。
%    \begin{macrocode}
\jcharwidowpenalty=500
%    \end{macrocode}
%
% ここまでが、\file{pldefs.ltx}の内容です。
%    \begin{macrocode}
%</pldefs>
%    \end{macrocode}
%
%
%
% \section{フォント定義ファイル}\label{plfonts:fontdef}
% \changes{v1.3}{1997/01/24}{Rename provided font definition filename.}
% ここでは、フォント定義ファイルの設定をしています。フォント定義ファイルは、
% \LaTeX{}のフォント属性を\TeX{}フォントに置き換えるためのファイルです。
% 記述方法についての詳細は、|fntguide.tex|を参照してください。
%
% 欧文書体の設定については、
% \file{cmfonts.fdd}や\file{slides.fdd}などを参照してください。
% \file{skfonts.fdd}には、写研代用書体を使うためのパッケージと
% フォント定義が記述されています。
%    \begin{macrocode}
%<JY2mc>\ProvidesFile{jy2mc.fd}
%<JY2gt>\ProvidesFile{jy2gt.fd}
%<JT2mc>\ProvidesFile{jt2mc.fd}
%<JT2gt>\ProvidesFile{jt2gt.fd}
%<JY2mc,JY2gt,JT2mc,JT2gt>       [2018/07/03 v1.6q-u02 KANJI font defines]
%    \end{macrocode}
% 横組用、縦組用ともに、
% 明朝体のシリーズ|bx|がゴシック体となるように宣言しています。
% \changes{v1.2}{1995/11/24}{it, sl, scの宣言を外した}
% \changes{v1.3b}{1997/01/29}{フォント定義ファイルのサイズ指定の調整}
% \changes{v1.3b}{1997/03/11}{すべてのサイズをロード可能にした}
% また、シリーズ|b|は同じ書体の|bx|と等価になるように宣言します。
% \changes{v1.6q}{2018/07/03}{シリーズbがbxと等価になるように宣言}
%
% p\LaTeX{}では従属書体にOT1エンコーディングを指定していましたが、
% up\LaTeX{}ではT1エンコーディングを用いるように変更しました。
% また、要求サイズ(指定されたフォントサイズ)が10ptのとき、
% 全角幅の実寸が9.62216ptとなるようにしますので、
% 和文スケール値($1\,\mathrm{zw} \div \textmc{要求サイズ}$)は
% $9.62216\,\mathrm{pt}/10\,\mathrm{pt}=0.962216$です。
% upjis系のメトリックは全角幅が10ptでデザインされているので、
% これを0.962216倍で読込みます。
% \changes{v1.6l}{2018/02/04}{和文スケール値を明文化}
%    \begin{macrocode}
%<*JY2mc>
\DeclareKanjiFamily{JY2}{mc}{}
\DeclareRelationFont{JY2}{mc}{m}{}{T1}{cmr}{m}{}
\DeclareRelationFont{JY2}{mc}{bx}{}{T1}{cmr}{bx}{}
\DeclareFontShape{JY2}{mc}{m}{n}{<->s*[0.962216]upjisr-h}{}
\DeclareFontShape{JY2}{mc}{bx}{n}{<->ssub*gt/m/n}{}
\DeclareFontShape{JY2}{mc}{b}{n}{<->ssub*mc/bx/n}{}
%</JY2mc>
%<*JT2mc>
\DeclareKanjiFamily{JT2}{mc}{}
\DeclareRelationFont{JT2}{mc}{m}{}{T1}{cmr}{m}{}
\DeclareRelationFont{JT2}{mc}{bx}{}{T1}{cmr}{bx}{}
\DeclareFontShape{JT2}{mc}{m}{n}{<->s*[0.962216]upjisr-v}{}
\DeclareFontShape{JT2}{mc}{bx}{n}{<->ssub*gt/m/n}{}
\DeclareFontShape{JT2}{mc}{b}{n}{<->ssub*mc/bx/n}{}
%</JT2mc>
%<*JY2gt>
\DeclareKanjiFamily{JY2}{gt}{}
\DeclareRelationFont{JY2}{gt}{m}{}{T1}{cmr}{bx}{}
\DeclareFontShape{JY2}{gt}{m}{n}{<->s*[0.962216]upjisg-h}{}
\DeclareFontShape{JY2}{gt}{bx}{n}{<->ssub*gt/m/n}{}
\DeclareFontShape{JY2}{gt}{b}{n}{<->ssub*gt/bx/n}{}
%</JY2gt>
%<*JT2gt>
\DeclareKanjiFamily{JT2}{gt}{}
\DeclareRelationFont{JT2}{gt}{m}{}{T1}{cmr}{bx}{}
\DeclareFontShape{JT2}{gt}{m}{n}{<->s*[0.962216]upjisg-v}{}
\DeclareFontShape{JT2}{gt}{bx}{n}{<->ssub*gt/m/n}{}
\DeclareFontShape{JT2}{gt}{b}{n}{<->ssub*gt/bx/n}{}
%</JT2gt>
%    \end{macrocode}
%
%
% \Finale
%
\endinput

   % \iffalse meta-comment
%% File: ukinsoku.dtx
%
%    pLaTeX kinsoku file:
%       Copyright 1995 ASCII Corporation.
%    and modified for upLaTeX
%
%  Copyright (c) 2010 ASCII MEDIA WORKS
%  Copyright (c) 2016 Takuji Tanaka
%  Copyright (c) 2016-2021 Japanese TeX Development Community
%
%  This file is part of the upLaTeX2e system (community edition).
%  --------------------------------------------------------------
%
% \fi
%
%
% \setcounter{StandardModuleDepth}{1}
% \StopEventually{}
%
% \iffalse
% \changes{v1.0-u00}{2011/05/07}{p\LaTeX{}用からup\LaTeX{}用に修正。}
% \changes{v1.0-u01}{2017/08/02}{U+00B7 (MIDDLE DOT; JIS X 0213)の
%    前禁則ペナルティをU+30FBと同じ値に設定、注意点を明文化}
% \changes{v1.0b}{2017/08/05}{%、&、\%、\&の禁則ペナルティが
%      誤っていたのを修正(post $\rightarrow$ pre)}
% \changes{v1.0b-u01}{2017/08/05}{p\LaTeX{}の変更に追随}
% \changes{v1.0b-u02}{2018/01/27}{up\TeX{}の将来の変更に備え、
%      Latin-1 Supplementのうち属性がLatinのもの
%      (Latin-1 letters)をコードポイントで指定}
% \changes{v1.0b-u03}{2018/04/08}{\LaTeX\ 2018-04-01対策}
% \changes{v1.0b-u04}{2019/01/29}{内部Unicode化されていることを確認}
% \changes{v1.0b-u05}{2019/05/19}{up\TeX~v1.24の\cs{kcatcode}の既定値のバグ回避}
% \changes{v1.0b-u06}{2019/09/22}{バグ回避コードがかえって有害なため除去}
% \changes{v1.0c}{2020/09/28}{!の\cs{inhibitxspcode}を設定}
% \changes{v1.0c-u06}{2020/09/28}{p\LaTeX{}の変更に追随}
% \changes{v1.0d}{2021/03/04}{:の\cs{inhibitxspcode}と:の\cs{xspcode}を設定}
% \changes{v1.0d-u06}{2021/03/04}{p\LaTeX{}の変更に追随}
% \fi
%
% \iffalse
%<*driver>
\NeedsTeXFormat{pLaTeX2e}
% \fi
\ProvidesFile{ukinsoku.dtx}[2021/03/04 v1.0d-u06 upLaTeX Kernel]
% \iffalse
\documentclass{jltxdoc}
\GetFileInfo{ukinsoku.dtx}
\title{禁則パラメータ\space\fileversion}
\author{Ken Nakano \& TTK}
\date{作成日:\filedate}
\begin{document}
   \maketitle
   \DocInput{\filename}
\end{document}
%</driver>
% \fi
%
% このファイルは、禁則と文字間スペースの設定について説明をしています。
% 日本語\TeX{}の機能についての詳細は、『日本語\TeX テクニカルブックI』を
% 参照してください。
%
% なお、このファイルのコード部分は、
% p\TeX{}やp\LaTeX{}で配布されている\file{kinsoku.tex}に、
% JIS X 0213の定義文字などの設定を追加したものです。
% このファイルは内部コードUnicode (|uptex|)なup\TeX{}エンジンで読まれる
% 必要があります。
% \changes{v1.0-u00}{2011/05/07}{p\LaTeX{}用からup\LaTeX{}用に修正。}
% \changes{v1.0b-u04}{2019/01/29}{内部コードがUnicodeであることを確認}
%
%    \begin{macrocode}
%<*plcore>
\ifnum\ucs"3000="3000 \else
    \errhelp{Please try to run (e)uptex with option
             `-kanji-internal=uptex'.}%
    \errmessage{This file should be read with
                internal Kanji encoding Unicode}\@@end
\fi
%    \end{macrocode}
%
% \changes{v1.0b-u05}{2019/05/19}{up\TeX~v1.24の\cs{kcatcode}の既定値のバグ回避}
% \changes{v1.0b-u06}{2019/09/22}{バグ回避コードがかえって有害なため除去}
%
% \section{禁則}
%
% ある文字を行頭禁則の対象にするには、|\prebreakpenalty|に正の値を指定します。
% ある文字を行末禁則の対象にするには、|\postbreakpenalty|に正の値を指定します。
% 数値が大きいほど、行頭、あるいは行末で改行されにくくなります。
%
% \subsection{半角文字に対する禁則}
% ここでは、半角文字に対する禁則の設定を行なっています。
% \changes{v1.0b}{2017/08/05}{%、&、\%、\&の禁則ペナルティが
%      誤っていたのを修正(post $\rightarrow$ pre)}
%
%    \begin{macrocode}
%%
%% 行頭、行末禁則パラメータ
%%
%% 1byte characters
\prebreakpenalty`!=10000
\prebreakpenalty`"=10000
\postbreakpenalty`\#=500
\postbreakpenalty`\$=500
\prebreakpenalty`\%=500
\prebreakpenalty`\&=500
\postbreakpenalty`\`=10000
\prebreakpenalty`'=10000
\prebreakpenalty`)=10000
\postbreakpenalty`(=10000
\prebreakpenalty`*=500
\prebreakpenalty`+=500
\prebreakpenalty`-=10000
\prebreakpenalty`.=10000
\prebreakpenalty`,=10000
\prebreakpenalty`/=500
\prebreakpenalty`;=10000
\prebreakpenalty`?=10000
\prebreakpenalty`:=10000
\prebreakpenalty`]=10000
\postbreakpenalty`[=10000
%    \end{macrocode}
%
% \subsection{全角文字に対する禁則}
% ここでは、全角文字に対する禁則の設定を行なっています。
%
% up\TeX{}/up\LaTeX{}の場合、JIS X 0213(日本)・KS C 5601(韓国)・
% GB2312(中国)・Big5(台湾)などの文字集合に含まれる、
% いわゆる全角文字の一部が、8-bit Latinと同じコードポイントを
% 共有します。すなわち、同じコードポイントが、CJKトークンとしても
% non-CJKトークンとしても有効に扱われることがあります。
% 以下に例を示します\footnote{ここで表示しているnon-CJKトークンと
% して扱われた結果は、up\LaTeX{}のデフォルト従属欧文エンコーディング
% であるT1の場合のものです。}。
% {\font\lmr=rm-lmr10\lmr
% \begin{itemize}
% \item \texttt{0xA1}: \kchar"A1 (CJK) vs. \char"A1\ (non-CJK)
% \item \texttt{0xAB}: \kchar"AB (CJK) vs. \char"AB\ (non-CJK)
% \item \texttt{0xB7}: \kchar"B7 (CJK) vs. \char"B7\ (non-CJK)
% \item \texttt{0xB9}: \kchar"B9 (CJK) vs. \char"B9\ (non-CJK)
% \item …
% \end{itemize}}
% \file{ukinsoku.tex}ではCJKトークンを優先した禁則設定を行っています。
% この設定により、同じコードポイントをnon-CJKトークンとして扱う場合に
% 予期せずLatin-1の文字が禁則対象になってしまいます。
% 問題が起きた場合は禁則の設定を調整してください。
% \changes{v1.0-u01}{2017/08/02}{U+00B7 (MIDDLE DOT; JIS X 0213)の
%    前禁則ペナルティをU+30FBと同じ値に設定、注意点を明文化}
% \changes{v1.0b-u02}{2018/01/27}{up\TeX{}の将来の変更に備え、
%      Latin-1 Supplementのうち属性がLatinのもの
%      (Latin-1 letters)をコードポイントで指定}
%
% なお、以下で複数回登場する |"AA| と |"BA| はそれぞれªとºですが、
% \LaTeXe\ 2018-04-01でUTF-8入力になった影響で、これらの文字は
% |macrocode| 環境内のコードに(たとえ |%| に続くコメントであっても)
% 書けなくなってしまったようです。これらの文字で
% docstrip処理中にエラー
%\begin{verbatim}
%   ! Argument of \@font@info has an extra }.
%\end{verbatim}
% が出ないように、コメントからも削除しました。
% \changes{v1.0b-u03}{2018/04/08}{\LaTeX\ 2018-04-01対策}
%
% \changes{v1.0d}{2021/03/04}{:の\cs{xspcode}を設定}
%    \begin{macrocode}
%%全角文字
\prebreakpenalty`、=10000
\prebreakpenalty`。=10000
\prebreakpenalty`,=10000
\prebreakpenalty`.=10000
\prebreakpenalty`・=10000
\prebreakpenalty`:=10000
\prebreakpenalty`;=10000
\prebreakpenalty`?=10000
\prebreakpenalty`!=10000
\prebreakpenalty`゛=10000%\jis"212B
\prebreakpenalty`゜=10000%\jis"212C
\prebreakpenalty`´=10000%\jis"212D
\postbreakpenalty``=10000%\jis"212E
\prebreakpenalty`々=10000%\jis"2139
\prebreakpenalty`…=250%\jis"2144
\prebreakpenalty`‥=250%\jis"2145
\postbreakpenalty`‘=10000%\jis"2146
\prebreakpenalty`’=10000%\jis"2147
\postbreakpenalty`“=10000%\jis"2148
\prebreakpenalty`”=10000%\jis"2149
\prebreakpenalty`)=10000
\postbreakpenalty`(=10000
\prebreakpenalty`}=10000
\postbreakpenalty`{=10000
\prebreakpenalty`]=10000
\postbreakpenalty`[=10000
%%\postbreakpenalty`‘=10000
%%\prebreakpenalty`’=10000
\postbreakpenalty`〔=10000%\jis"214C
\prebreakpenalty`〕=10000%\jis"214D
\postbreakpenalty`〈=10000%\jis"2152
\prebreakpenalty`〉=10000%\jis"2153
\postbreakpenalty`《=10000%\jis"2154
\prebreakpenalty`》=10000%\jis"2155
\postbreakpenalty`「=10000%\jis"2156
\prebreakpenalty`」=10000%\jis"2157
\postbreakpenalty`『=10000%\jis"2158
\prebreakpenalty`』=10000%\jis"2159
\postbreakpenalty`【=10000%\jis"215A
\prebreakpenalty`】=10000%\jis"215B
\prebreakpenalty`ー=10000
\prebreakpenalty`+=200
\prebreakpenalty`−=200% U+2212 MINUS SIGN
\prebreakpenalty`-=200% U+FF0D FULLWIDTH HYPHEN-MINUS
\prebreakpenalty`==200
\postbreakpenalty`#=200
\postbreakpenalty`$=200
\prebreakpenalty`%=200
\prebreakpenalty`&=200
\prebreakpenalty`ぁ=150
\prebreakpenalty`ぃ=150
\prebreakpenalty`ぅ=150
\prebreakpenalty`ぇ=150
\prebreakpenalty`ぉ=150
\prebreakpenalty`っ=150
\prebreakpenalty`ゃ=150
\prebreakpenalty`ゅ=150
\prebreakpenalty`ょ=150
\prebreakpenalty`ゎ=150%\jis"246E
\prebreakpenalty`ァ=150
\prebreakpenalty`ィ=150
\prebreakpenalty`ゥ=150
\prebreakpenalty`ェ=150
\prebreakpenalty`ォ=150
\prebreakpenalty`ッ=150
\prebreakpenalty`ャ=150
\prebreakpenalty`ュ=150
\prebreakpenalty`ョ=150
\prebreakpenalty`ヮ=150%\jis"256E
\prebreakpenalty`ヵ=150%\jis"2575
\prebreakpenalty`ヶ=150%\jis"2576
%% kinsoku  JIS X 0208 additional
\prebreakpenalty`ヽ=10000
\prebreakpenalty`ヾ=10000
\prebreakpenalty`ゝ=10000
\prebreakpenalty`ゞ=10000
%%
%% kinsoku  JIS X 0213
%%
\prebreakpenalty`〳=10000
\prebreakpenalty`〴=10000
\prebreakpenalty`〵=10000
\prebreakpenalty`〻=10000
\postbreakpenalty`⦅=10000
\prebreakpenalty`⦆=10000
\postbreakpenalty`⦅=10000
\prebreakpenalty`⦆=10000
\postbreakpenalty`〘=10000
\prebreakpenalty`〙=10000
\postbreakpenalty`〖=10000
\prebreakpenalty`〗=10000
\postbreakpenalty`«=10000
\prebreakpenalty`»=10000
\postbreakpenalty`〝=10000
\prebreakpenalty`〟=10000
\prebreakpenalty`‼=10000
\prebreakpenalty`⁇=10000
\prebreakpenalty`⁈=10000
\prebreakpenalty`⁉=10000
\postbreakpenalty`¡=10000
\postbreakpenalty`¿=10000
\prebreakpenalty`ː=10000
\prebreakpenalty`·=10000
\prebreakpenalty"AA=10000
\prebreakpenalty"BA=10000
\prebreakpenalty`¹=10000
\prebreakpenalty`²=10000
\prebreakpenalty`³=10000
\postbreakpenalty`€=10000
\prebreakpenalty`ゕ=150
\prebreakpenalty`ゖ=150
\prebreakpenalty`ㇰ=150
\prebreakpenalty`ㇱ=150
\prebreakpenalty`ㇲ=150
\prebreakpenalty`ㇳ=150
\prebreakpenalty`ㇴ=150
\prebreakpenalty`ㇵ=150
\prebreakpenalty`ㇶ=150
\prebreakpenalty`ㇷ=150
\prebreakpenalty`ㇸ=150
\prebreakpenalty`ㇹ=150
%%\prebreakpenalty`ㇷ゚=150
\prebreakpenalty`ㇺ=150
\prebreakpenalty`ㇻ=150
\prebreakpenalty`ㇼ=150
\prebreakpenalty`ㇽ=150
\prebreakpenalty`ㇾ=150
\prebreakpenalty`ㇿ=150
%%
%% kinsoku  JIS X 0212
%%
%%\postbreakpenalty`¡=10000
%%\postbreakpenalty`¿=10000
%%\prebreakpenalty"BA=10000
%%\prebreakpenalty"AA=10000
\prebreakpenalty`™=10000
%%
%% kinsoku  半角片仮名
%%
\prebreakpenalty`。=10000
\prebreakpenalty`、=10000
\prebreakpenalty`゙=10000
\prebreakpenalty`゚=10000
\prebreakpenalty`」=10000
\postbreakpenalty`「=10000
%    \end{macrocode}
%
% \section{文字間のスペース}
%
% ある英字の前後と、その文字に隣合う漢字に挿入されるスペースを制御するには、
% |\xspcode|を用います。
%
% ある漢字の前後と、その文字に隣合う英字に挿入されるスペースを制御するには、
% |\inhibitxspcode|を用います。
%
% \subsection{ある英字と前後の漢字の間の制御}
% ここでは、英字に対する設定を行なっています。
%
% 指定する数値とその意味は次のとおりです。
%
% \begin{center}
% \begin{tabular}{ll}
% 0 & 前後の漢字の間での処理を禁止する。\\
% 1 & 直前の漢字との間にのみ、スペースの挿入を許可する。\\
% 2 & 直後の漢字との間にのみ、スペースの挿入を許可する。\\
% 3 & 前後の漢字との間でのスペースの挿入を許可する。\\
% \end{tabular}
% \end{center}
%
%    \begin{macrocode}
%%
%% xspcode
\xspcode`(=1
\xspcode`)=2
\xspcode`[=1
\xspcode`]=2
\xspcode``=1
\xspcode`'=2
\xspcode`:=2
\xspcode`;=2
\xspcode`,=2
\xspcode`.=2
%%  for 8bit Latin
\xspcode"80=3
\xspcode"81=3
\xspcode"82=3
\xspcode"83=3
\xspcode"84=3
\xspcode"85=3
\xspcode"86=3
\xspcode"87=3
\xspcode"88=3
\xspcode"89=3
\xspcode"8A=3
\xspcode"8B=3
\xspcode"8C=3
\xspcode"8D=3
\xspcode"8E=3
\xspcode"8F=3
\xspcode"90=3
\xspcode"91=3
\xspcode"92=3
\xspcode"93=3
\xspcode"94=3
\xspcode"95=3
\xspcode"96=3
\xspcode"97=3
\xspcode"98=3
\xspcode"99=3
\xspcode"9A=3
\xspcode"9B=3
\xspcode"9C=3
\xspcode"9D=3
\xspcode"9E=3
\xspcode"9F=3
\xspcode"A0=3
\xspcode"A1=3
\xspcode"A2=3
\xspcode"A3=3
\xspcode"A4=3
\xspcode"A5=3
\xspcode"A6=3
\xspcode"A7=3
\xspcode"A8=3
\xspcode"A9=3
\xspcode"AA=3
\xspcode"AB=3
\xspcode"AC=3
\xspcode"AD=3
\xspcode"AE=3
\xspcode"AF=3
\xspcode"B0=3
\xspcode"B1=3
\xspcode"B2=3
\xspcode"B3=3
\xspcode"B4=3
\xspcode"B5=3
\xspcode"B6=3
\xspcode"B7=3
\xspcode"B8=3
\xspcode"B9=3
\xspcode"BA=3
\xspcode"BB=3
\xspcode"BC=3
\xspcode"BD=3
\xspcode"BE=3
\xspcode"BF=3
\xspcode"C0=3
\xspcode"C1=3
\xspcode"C2=3
\xspcode"C3=3
\xspcode"C4=3
\xspcode"C5=3
\xspcode"C6=3
\xspcode"C7=3
\xspcode"C8=3
\xspcode"C9=3
\xspcode"CA=3
\xspcode"CB=3
\xspcode"CC=3
\xspcode"CD=3
\xspcode"CE=3
\xspcode"CF=3
\xspcode"D0=3
\xspcode"D1=3
\xspcode"D2=3
\xspcode"D3=3
\xspcode"D4=3
\xspcode"D5=3
\xspcode"D6=3
\xspcode"D7=3
\xspcode"D8=3
\xspcode"D9=3
\xspcode"DA=3
\xspcode"DB=3
\xspcode"DC=3
\xspcode"DD=3
\xspcode"DE=3
\xspcode"DF=3
\xspcode"E0=3
\xspcode"E1=3
\xspcode"E2=3
\xspcode"E3=3
\xspcode"E4=3
\xspcode"E5=3
\xspcode"E6=3
\xspcode"E7=3
\xspcode"E8=3
\xspcode"E9=3
\xspcode"EA=3
\xspcode"EB=3
\xspcode"EC=3
\xspcode"ED=3
\xspcode"EE=3
\xspcode"EF=3
\xspcode"F0=3
\xspcode"F1=3
\xspcode"F2=3
\xspcode"F3=3
\xspcode"F4=3
\xspcode"F5=3
\xspcode"F6=3
\xspcode"F7=3
\xspcode"F8=3
\xspcode"F9=3
\xspcode"FA=3
\xspcode"FB=3
\xspcode"FC=3
\xspcode"FD=3
\xspcode"FE=3
\xspcode"FF=3
%    \end{macrocode}
%
% \subsection{ある漢字と前後の英字の間の制御}
% ここでは、漢字に対する設定を行なっています。
%
% 指定する数値とその意味は次のとおりです。
%
% \begin{center}
% \begin{tabular}{ll}
% 0 & 前後の英字との間にスペースを挿入することを禁止する。\\
% 1 & 直前の英字との間にスペースを挿入することを禁止する。\\
% 2 & 直後の英字との間にスペースを挿入することを禁止する。\\
% 3 & 前後の英字との間でのスペースの挿入を許可する。\\
% \end{tabular}
% \end{center}
%
% \changes{v1.0c}{2020/09/28}{!の\cs{inhibitxspcode}を設定}
% \changes{v1.0d}{2021/03/04}{:の\cs{inhibitxspcode}を設定}
%    \begin{macrocode}
%%
%% inhibitxspcode
\inhibitxspcode`、=1
\inhibitxspcode`。=1
\inhibitxspcode`,=1
\inhibitxspcode`.=1
\inhibitxspcode`:=1
\inhibitxspcode`;=1
\inhibitxspcode`?=1
\inhibitxspcode`!=1
\inhibitxspcode`)=1
\inhibitxspcode`(=2
\inhibitxspcode`]=1
\inhibitxspcode`[=2
\inhibitxspcode`}=1
\inhibitxspcode`{=2
\inhibitxspcode`‘=2
\inhibitxspcode`’=1
\inhibitxspcode`“=2
\inhibitxspcode`”=1
\inhibitxspcode`〔=2
\inhibitxspcode`〕=1
\inhibitxspcode`〈=2
\inhibitxspcode`〉=1
\inhibitxspcode`《=2
\inhibitxspcode`》=1
\inhibitxspcode`「=2
\inhibitxspcode`」=1
\inhibitxspcode`『=2
\inhibitxspcode`』=1
\inhibitxspcode`【=2
\inhibitxspcode`】=1
\inhibitxspcode`—=0% U+2014 EM DASH
\inhibitxspcode`―=0% U+2015 HORIZONTAL BAR
\inhibitxspcode`〜=0% U+301C WAVE DASH
\inhibitxspcode`~=0% U+FF5E FULLWIDTH TILDE
\inhibitxspcode`…=0
\inhibitxspcode`¥=0% U+00A5 YEN SIGN
\inhibitxspcode`¥=0% U+FFE5 FULLWIDTH YEN SIGN
\inhibitxspcode`°=1
\inhibitxspcode`′=1
\inhibitxspcode`″=1
%%
%% inhibitxspcode  JIS X 0213
%%
\inhibitxspcode`⦅=2
\inhibitxspcode`⦆=1
\inhibitxspcode`⦅=2
\inhibitxspcode`⦆=1
\inhibitxspcode`〘=2
\inhibitxspcode`〙=1
\inhibitxspcode`〖=2
\inhibitxspcode`〗=1
\inhibitxspcode`«=2
\inhibitxspcode`»=1
\inhibitxspcode`〝=2
\inhibitxspcode`〟=1
\inhibitxspcode`‼=1
\inhibitxspcode`⁇=1
\inhibitxspcode`⁈=1
\inhibitxspcode`⁉=1
\inhibitxspcode`¡=2
\inhibitxspcode`¿=2
\inhibitxspcode"AA=1
\inhibitxspcode"BA=1
\inhibitxspcode`¹=1
\inhibitxspcode`²=1
\inhibitxspcode`³=1
\inhibitxspcode`€=2
%%
%% inhibitxspcode  JIS X 0212
%%
%%\inhibitxspcode`¡=2
%%\inhibitxspcode`¿=2
%%\inhibitxspcode"BA=1
%%\inhibitxspcode"AA=1
\inhibitxspcode`™=1
%%
%% inhibitxspcode  半角片仮名
%%
\inhibitxspcode`。=1
\inhibitxspcode`、=1
\inhibitxspcode`「=2
\inhibitxspcode`」=1
%    \end{macrocode}
%
%    \begin{macrocode}
%</plcore>
%    \end{macrocode}
%
% \Finale
%
\endinput

   \input{ujclasses.dtx}
\endgroup
\@ifl@t@r{\lastupd@te}{0000/00/00}{%
  \date{Version \patchdate\break (last updated: \lastupd@te)}%
}{}
\makeatother
%    \end{macrocode}
%\ifJAPANESE
% ここからが本文ページとなります。
%\else
% Here starts the document body.
%\fi
%    \begin{macrocode}
\begin{document}
\pagenumbering{roman}
\maketitle
\renewcommand\maketitle{}
\tableofcontents
\clearpage
\pagenumbering{arabic}

\DocInclude{uplvers}   % upLaTeX version

\DocInclude{uplfonts}  % NFSS2 commands

\DocInclude{ukinsoku}  % kinsoku parameter

\DocInclude{ujclasses} % Standard class

\StopEventually{\end{document}}

\clearpage
\pagestyle{headings}
% Make TeX shut up.
\hbadness=10000
\newcount\hbadness
\hfuzz=\maxdimen
%
\PrintChanges
\clearpage
%
\begingroup
  \def\endash{--}
  \catcode`\-\active
  \def-{\futurelet\temp\indexdash}
  \def\indexdash{\ifx\temp-\endash\fi}

  \PrintIndex
\endgroup
\let\PrintChanges\relax
\let\PrintIndex\relax
\end{document}
%</pldoc>
%    \end{macrocode}
%
%
%
%\ifJAPANESE
% \section{おまけプログラム}\label{app:omake}
%
% \subsection{シェルスクリプト\file{mkpldoc.sh}}\label{app:shprog}
% \upLaTeXe{}のマクロ定義ファイルをまとめて組版し、変更履歴と索引も
% 付けるときに便利なシェルスクリプトです。
% このシェルスクリプトの使用方法は次のとおりです。
%\begin{verbatim}
%    sh mkpldoc.sh
%\end{verbatim}
%
% コードは\pLaTeXe{}のものと(ファイル名を除き)ほぼ同一なので、
% ここでは違っている部分だけ説明します。
%\else
% \section{Additional Utility Programs}\label{app:omake}
%
% \subsection{Shell Script \file{mkpldoc.sh}}\label{app:shprog}
% A shell script to process `pldoc.tex' and produce a fully indexed
% source code description. Run |sh mkpldoc.sh| to use it.
%
% The script is almost identical to that in \pLaTeXe, so
% here we describe only the difference.
%\fi
%
%    \begin{macrocode}
%<*shprog>
%<ja>rm -f upldoc.toc upldoc.idx upldoc.glo
%<en>rm -f upldoc-en.toc upldoc-en.idx upldoc-en.glo
echo "" > ltxdoc.cfg
%<ja>uplatex upldoc.tex
%<en>uplatex -jobname=upldoc-en upldoc.tex
%    \end{macrocode}
%\ifJAPANESE
% 変更履歴や索引の生成にはmendexを用いますが、
% \upLaTeX{}の場合はUTF-8モードで実行する必要がありますので、
% |-U|というオプションを付けます\footnote{uplatexコマンドも
% 実際にはUTF-8モードで実行する必要がありますが、デフォルトの内部漢字
% コードがUTF-8に設定されているはずですので、\texttt{-kanji=utf8}を
% 付けなくても処理できると思います。}。
% makeindexコマンドには、このオプションがありません。
%\else
% To make the Change log and Glossary (Change History) for
% \upLaTeX\ using `mendex,' we need to run it in UTF-8 mode.
% So, option |-U| is important.\footnote{The command `uplatex'
% should be also in UTF-8 mode, but it defaults to UTF-8 mode;
% therefore, we don't need to add \texttt{-kanji=utf8} explicitly.}
%\fi
%    \begin{macrocode}
%<ja>mendex -U -s gind.ist -d upldoc.dic -o upldoc.ind upldoc.idx
%<en>mendex -U -s gind.ist -d upldoc.dic -o upldoc-en.ind upldoc-en.idx
%<ja>mendex -U -f -s gglo.ist -o upldoc.gls upldoc.glo
%<en>mendex -U -f -s gglo.ist -o upldoc-en.gls upldoc-en.glo
echo "\includeonly{}" > ltxdoc.cfg
%<ja>uplatex upldoc.tex
%<en>uplatex -jobname=upldoc-en upldoc.tex
echo "" > ltxdoc.cfg
%<ja>uplatex upldoc.tex
%<en>uplatex -jobname=upldoc-en upldoc.tex
# EOT
%</shprog>
%    \end{macrocode}
%
%
%\ifJAPANESE
% \subsection{perlスクリプト\file{dstcheck.pl}}\label{app:plprog}
% \pLaTeXe{}のものがそのまま使えるので、\upLaTeXe{}では省略します。
%\else
% \subsection{Perl Script \file{dstcheck.pl}}\label{app:plprog}
% The one from \pLaTeXe\ can be use without any change, so
% omitted here in \upLaTeXe.
%\fi
%
%
%\ifJAPANESE
% \subsection{\dst{}バッチファイル}
% 付録\ref{app:shprog}で説明をしたスクリプトを、このファイルから
% 取り出すための\dst{}バッチファイルです。コードは\pLaTeXe{}の
% ものと(ファイル名を除き)ほぼ同一なので、説明は割愛します。
%\else
% \subsection{\dst{} Batch file}
% Here we introduce a \dst\ batch file `Xins.ins,' which generates the
% script described in Appendix \ref{app:shprog}.
% The code is almost identical to that in \pLaTeXe.
%\fi
%
%    \begin{macrocode}
%<*Xins>
\input docstrip
\keepsilent
%    \end{macrocode}
%
%    \begin{macrocode}
{\catcode`#=12 \gdef\MetaPrefix{## }}
%    \end{macrocode}
%
%    \begin{macrocode}
\declarepreamble\thispre
\endpreamble
\usepreamble\thispre
%    \end{macrocode}
%
%    \begin{macrocode}
\declarepostamble\thispost
\endpostamble
\usepostamble\thispost
%    \end{macrocode}
%
%    \begin{macrocode}
\generate{
   \file{mkpldoc.sh}{\from{uplatex.dtx}{shprog,ja}}
   \file{mkpldoc-en.sh}{\from{uplatex.dtx}{shprog,en}}
}
\endbatchfile
%</Xins>
%    \end{macrocode}
%
% \newpage
% \begin{thebibliography}{9}
% \bibitem{tb108tanaka}
% Takuji Tanaka,
% \newblock Up\TeX\ --- Unicode version of \pTeX\ with CJK extensions.
% \newblock TUGboat issue 34:3, 2013.\\
%   (\texttt{http://tug.org/TUGboat/tb34-3/tb108tanaka.pdf})
% \end{thebibliography}
%
% \iffalse
% ここで、このあとに組版されるかもしれない文書のために、
% 節見出しの番号を算用数字に戻します。
% \fi
%
% \renewcommand{\thesection}{\arabic{section}}
%
% \Finale
%
\endinput
}{}%
}
%    \end{macrocode}
%
%\ifJAPANESE
% フォーマットファイルにダンプします。
%\else
% Dump to the format file.
%\fi
%    \begin{macrocode}
\let\dump\orgdump
\let\orgdump\@undefined
\makeatother
\dump
%\endinput
%    \end{macrocode}
%
%    \begin{macrocode}
%</plcore>
%    \end{macrocode}
%
%\ifJAPANESE
% 実際に\upLaTeXe{}への拡張を行なっている\file{uplcore.ltx}は、
% \dst{}プログラムによって、次のファイルの断片が連結されたものです。
%
% \begin{itemize}
% \item \file{uplvers.dtx}は、\upLaTeXe{}のフォーマットバージョンを
%   定義しています。
% \end{itemize}
%
% また、プリロードフォントや組版パラメータなどのデフォルト設定は、
% \file{uplatex.ltx}の中で\file{upldefs.ltx}をロードすることにより行います
% \footnote{旧版では\file{uplcore.ltx}の中でロードしていましたが、
% 2018年以降の新しいコミュニティ版\upLaTeX{}では
% \file{uplatex.ltx}から読み込むことにしました。}。
% このファイル\file{upldefs.ltx}も\file{uplfonts.dtx}から生成されます。
% \begin{chuui}
% このファイルに記述されている設定を変更すれば
% \upLaTeXe{}をカスタマイズすることができますが、
% その場合は\file{upldefs.ltx}を直接修正するのではなく、いったん
% \file{upldefs.cfg}という名前でコピーして、そのファイルを編集してください。
% フォーマット作成時に\file{upldefs.cfg}が存在した場合は、そちらが
% \file{upldefs.ltx}の代わりに読み込まれます。
% \end{chuui}
%\else
% The file \file{uplcore.ltx}, which provides modifications/extensions
% to make \upLaTeXe, is a concatenation of stripped files below
% using \dst\ program.
%
% \begin{itemize}
% \item \file{uplvers.dtx} defines the format version of \upLaTeXe.
% \item \file{uplfonts.dtx} extends \NFSS2 for Japanese font selection.
% \item \file{plcore.dtx} (the same content as \pLaTeXe); defines other
%   modifications to \LaTeXe.
% \end{itemize}
%
% Moreover, default settings of pre-loaded fonts and typesetting parameters
% are done by loading \file{upldefs.ltx} inside
% \file{uplatex.ltx}.\footnote{Older \upLaTeX\ loaded \file{upldefs.ltx}
% inside \file{uplcore.ltx}; however, \upLaTeX\ community edition newer than
% 2018 loads \file{upldefs.ltx} inside \file{uplatex.ltx}.}
% This file \file{upldefs.ltx} is also stripped from \file{uplfonts.dtx}.
% \begin{chuui}
% You can customize \upLaTeXe\ by tuning these settings.
% If you need to do that, copy/rename it as \file{upldefs.cfg} and edit it,
% instead of overwriting \file{upldefs.ltx} itself.
% If a file named \file{upldefs.cfg} is found at a format creation
% time, it will be read as a substitute of \file{upldefs.ltx}.
% \end{chuui}
%\fi
%
%\ifJAPANESE
% ここまで見てきたように、\upLaTeX{}の各ファイルはそれぞれ\pLaTeX{}での
% 対応するファイル名の頭に``u''を付けた名前になっています。
%\else
% As shown above, the files in \upLaTeX\ is named after \pLaTeX\ ones,
% prefixed with ``u.''
%\fi
%
%
%\ifJAPANESE
% \subsubsection{バージョン}
% \upLaTeXe{}のバージョンやフォーマットファイル名は、
% \file{uplvers.dtx}で定義しています。これは、\pLaTeXe{}のバージョンや
% フォーマットファイル名が\file{plvers.dtx}で定義されているのと同じです。
%\else
% \subsubsection{Version}
% The version (like ``\pfmtversion'') and the format name
% (``\pfmtname'') of \upLaTeXe\ are defined in \file{uplvers.dtx}.
% This is similar to \pLaTeXe, which defines those in \file{plvers.dtx}.
%\fi
%
%
%\ifJAPANESE
% \subsubsection{\NFSS2コマンド}
% \upLaTeXe{}は\pLaTeXe{}と共通の\file{plcore.ltx}を使用していますので、
% \NFSS2の和文フォント選択への拡張が有効になっています。
%\else
% \subsubsection{\NFSS2 Commands}
% \upLaTeXe\ shares \file{plcore.dtx} with \pLaTeXe, so
% the extensions of \NFSS2 for selecting Japanese fonts are available.
%\fi
%
%
%\ifJAPANESE
% \subsubsection{出力ルーチンとフロート}
% \upLaTeXe{}は\pLaTeXe{}と共通の\file{plcore.ltx}を使用していますので、
% 出力ルーチンや脚注マクロなどは\pLaTeXe{}と同じように動作します。
%\else
% \subsubsection{Output Routine and Floats}
% \upLaTeXe\ shares \file{plcore.dtx} with \pLaTeXe, so
% the output routine and footnote macros will behave similar to \pLaTeXe.
%\fi
%
%
%\ifJAPANESE
% \subsection{クラスファイルとパッケージファイル}
%
% \upLaTeXe{}が提供をするクラスファイルやパッケージファイルは、
% \pLaTeXe{}に含まれるファイルを基にしています。
%
% \upLaTeXe{}に付属のクラスファイルは、次のとおりです。
%
% \begin{itemize}
% \item ujarticle.cls, ujbook.cls, ujreport.cls\par
%   横組用の標準クラスファイル。
%   \file{ujclasses.dtx}から作成される。
%   それぞれjarticle.cls, jbook.cls, jreport.clsの\upLaTeX{}版。
%
% \item utarticle.cls, utbook.cls, utreport.cls\par
%   縦組用の標準クラスファイル。
%   \file{ujclasses.dtx}から作成される。
%   それぞれtarticle.cls, tbook.cls, treport.clsの\upLaTeX{}版。
% \end{itemize}
%
% なおjltxdoc.clsの\upLaTeX{}版はありませんが、これは\pLaTeX{}のものが
% \upLaTeX{}でもそのまま使えます。
%\else
% \subsection{Classes and Packages}
%
% Classes and packages bundled with \upLaTeXe\ are based on
% those in original \pLaTeXe, and modified some parameters.
%
% \upLaTeXe\ classes:
%
% \begin{itemize}
% \item ujarticle.cls, ujbook.cls, ujreport.cls\par
%   Standard \emph{yoko-kumi} (horizontal writing) classes;
%   stripped from \file{ujclasses.dtx}.
%   \upLaTeX\ edition of jarticle.cls, jbook.cls and jreport.cls.
%
% \item utarticle.cls, utbook.cls, utreport.cls\par
%   Standard \emph{tate-kumi} (vertical writing) classes;
%   stripped from \file{ujclasses.dtx}.
%   \upLaTeX\ edition of tarticle.cls, tbook.cls and treport.cls.
% \end{itemize}
%
% We don't provide \upLaTeX\ edition of jltxdoc.cls, but the one
% from \pLaTeX\ can be used also on \upLaTeX\ without problem.
%\fi
%
%\ifJAPANESE
% また、\upLaTeXe{}に付属のパッケージファイルは、次のとおりです。
%
% \begin{itemize}
% \item uptrace.sty\par
%   ptrace.styの\upLaTeX{}版。
%   \LaTeX{}でフォント選択コマンドのトレースに使う\file{tracefnt.sty}が
%   再定義してしまう\NFSS2コマンドを、\upLaTeXe{}用に再々定義するための
%   パッケージ。
%   \file{uplfonts.dtx}から作成される。
% \end{itemize}
%
% 他の\pLaTeX{}のパッケージは、\upLaTeX{}でもそのまま動作します。
%\else
% \upLaTeXe\ packages:
%
% \begin{itemize}
% \item uptrace.sty\par
%   \upLaTeXe\ version of \file{tracefnt.sty};
%   the package \file{tracefnt.sty} overwrites \upLaTeXe-style \NFSS2
%   commands, so \file{uptrace.sty} provides redefinitions to recover
%   \upLaTeXe\ extensions. Stripped from \file{uplfonts.dtx}.
% \end{itemize}
%
% Other \pLaTeX\ packages work also on \upLaTeX.
%\fi
%
%
%\ifJAPANESE
% \section{他のフォーマット・旧バージョンとの互換性}
% \label{platex:compatibility}
% ここでは、この\upLaTeXe{}のバージョンと以前のバージョン、あるいは
% \pLaTeXe{}/\LaTeXe{}との互換性について説明をしています。
%
% \subsection{\pLaTeXe{}および\LaTeXe{}との互換性}
% \upLaTeXe{}は、\pLaTeXe{}の上位互換という形を取っていますので、
% クラスファイルやいくつかのコマンドを置き換えるだけで、
% たいていの\pLaTeXe{}文書を簡単に\upLaTeXe{}文書に変更することができます。
% ただし、\upLaTeXe{}のデフォルトの日本語フォントメトリックは\pLaTeXe{}の
% それと異なりますので、レイアウトが変化することがあります。
% また、\LaTeXe{}のいくつかの命令の定義も変更していますので、
% \LaTeXe{}で処理できるファイルを\upLaTeXe{}で処理した場合に
% 完全に同じ結果になるとは限りません。
%
% また、\upLaTeXe{}は新しいマクロパッケージですので、2.09互換モードを
% サポートしていません。\LaTeXe{}の仕様に従ってドキュメントを作成して
% ください。
%
% \pLaTeXe{}向けあるいは\LaTeXe{}向けに作られた多くのクラスファイルや
% パッケージファイルはそのまま使えると思います。
% ただし、例えばクラスファイルが\pLaTeX{}標準の
% 漢字エンコーディング(JY1, JT1)を前提としている場合は、
% \upLaTeX{}で採用した漢字エンコーディング(JY2, JT2)と合致せずに
% エラーが発生してしまいます。この場合は、そのクラスファイルが
% \upLaTeX{}に対応していないことになります。このような場合は、
% \pLaTeX{}を使い続けるか、その作者に連絡して\upLaTeX{}に対応して
% もらうなどの対応をとってください。
%\else
% \section{Compatibility with Other Formats and Older Versions}
% \label{platex:compatibility}
% Here we provide some information about the compatibility between
% current \upLaTeXe\ and older versions or original \pLaTeXe/\LaTeXe.
%
% \subsection{Compatibility with \pLaTeXe/\LaTeXe}
% \upLaTeXe\ is in most part upward compatible with \pLaTeXe,
% so you can move from \pLaTeXe\ to \upLaTeXe\ by simply replacing
% the document class and some macros. However, the default Japanese
% font metrics in \upLaTeXe\ is different from those in \pLaTeXe;
% therefore, you should not expect identical output from both
% \pLaTeXe\ and \upLaTeXe.
%
% Note that \upLaTeX\ is a new format, so we do \emph{not} provide support
% for 2.09 compatibility mode. Follow the standard \LaTeXe\ convention!
%
% We hope that most classes and packages meant for \LaTeXe/\pLaTeXe\ works
% also for \upLaTeXe\ without any modification. However for example,
% if a class or a package uses Kanji encoding `JY1' or `JT1' (default on
% \pLaTeXe), an error complaining the mismatch of Kanji encoding might
% happen on \upLaTeX, in which the default is `JY2' and `JT2.'
% In this case, we have to say that the class or package does not support
% \upLaTeXe; you should use \pLaTeX, or report to the author of the
% package or class.
%\fi
%
%\ifJAPANESE
% \subsection{latexreleaseパッケージへの対応}
% \LaTeX\ \texttt{<2015/01/01>}で導入されたlatexreleaseパッケージを
% もとに、新しい\pLaTeX{}ではplatexreleaseパッケージが用意されました。
% 本来は\upLaTeX{}でも同様のパッケージを用意するのがよいのですが、
% 現在は\pLaTeX{}から\upLaTeX{}への変更点が含まれていませんので、
% 幸いplatexreleaseパッケージをそのまま用いることができます。
% このため、\upLaTeX{}で独自のパッケージを用意することはしていません。
% platexreleaseパッケージを用いると、過去の\upLaTeX{}をエミュレート
% したり、フォーマットを作り直すことなく新しい\upLaTeX{}を試したりする
% ことができます。詳細はplatexreleaseのドキュメントを参照してください。
%\else
% \subsection{Support for Package `latexrelease'}
% \pLaTeX\ provides `platexrelease' package, which is based on
% `latexrelease' package (introduced in \LaTeX\ \texttt{<2015/01/01>}).
% It could be better if we also provide a similar package on \upLaTeX,
% but currently we don't need it; \upLaTeX\ does not have any recent
% \upLaTeX-specific changes. So, you can safely use `platexrelease'
% package for emulating the specified format date.
%\fi
%
%
%
% \appendix
%
%\ifJAPANESE
% \section{\dst{}プログラムのためのオプション}\label{app:dst}
% この文書のソース(\file{uplatex.dtx})を\dst{}プログラムで
% 処理することによって、
% いくつかの異なるファイルを生成することができます。
% \dst{}プログラムの詳細は、\file{docstrip.dtx}を参照してください。
%
% この文書の\dst{}プログラムのためのオプションは、次のとおりです。
%
% \DeleteShortVerb{\|}
% \begin{center}
% \begin{tabular}{l|p{.8\linewidth}}
% \emph{オプション} & \emph{意味}\\\hline
% plcore & フォーマットファイルを作るためのファイルを生成\\
% pldoc  & \upLaTeXe{}のソースファイルをまとめて組版するための
%          文書ファイル(upldoc.tex)を生成\\[2mm]
% shprog & 上記のファイルを作成するためのshスクリプトを生成\\
% Xins   & 上記のshスクリプトやperlスクリプトを取り出すための
%          \dst{}バッチファイル(Xins.ins)を生成\\
% \end{tabular}
% \end{center}
% \MakeShortVerb{\|}
%\else
% \section{\dst\ Options}\label{app:dst}
% By processing \file{uplatex.dtx} with \dst\ program,
% different files can be generated.
% Here are the \dst\ options for this document:
%
% \DeleteShortVerb{\|}
% \begin{center}
% \begin{tabular}{l|p{.8\linewidth}}
% \emph{Option} & \emph{Function}\\\hline
% plcore & Generates a fragment of format sources\\
% pldoc  & Generates `upldoc.tex' for typesetting
%          \upLaTeXe\ sources\\[2mm]
% shprog & Generates a shell script to process `upldoc.tex'\\
% Xins   & Generates a \dst\ batch file `Xins.ins' for
%          generating the above shell/perl scripts\\
% \end{tabular}
% \end{center}
% \MakeShortVerb{\|}
%\fi
%
%
%\ifJAPANESE
% \section{文書ファイル}\label{app:pldoc}
% ここでは、このパッケージに含まれているdtxファイルをまとめて組版し、
% ソースコード説明書を得るための文書ファイル\file{upldoc.tex}について
% 説明をしています。個別に処理した場合と異なり、
% 変更履歴や索引も付きます。
%
% デフォルトではソースコードの説明が日本語で書かれます。
% もし英語の説明書を読みたい場合は、\par\medskip
% \begin{minipage}{.5\textwidth}\ttfamily
% | |\cs{newif}\cs{ifJAPANESE}
% \end{minipage}\par\medskip\noindent
% という内容の\file{uplatex.cfg}を予め用意してから\file{upldoc.tex}を
% 処理してください(2016年7月1日以降の\upLaTeXe{}が必要)。
%
% コードは\pLaTeXe{}のものと(ファイル名を除き)ほぼ同一なので、
% ここでは違っている部分だけ説明します。
%\else
% \section{Documentation of \upLaTeXe\ sources}\label{app:pldoc}
% The contents of `upldoc.tex' for typesetting \upLaTeXe\ sources
% is described here. Compared to individual processings,
% batch processing using `upldoc.tex' prints also changes and an index.
%
% By default, the description of \upLaTeXe\ sources is written in
% Japanese. If you need English version, first save\par\medskip
% \begin{minipage}{.5\textwidth}\ttfamily
% | |\cs{newif}\cs{ifJAPANESE}
% \end{minipage}\par\medskip\noindent
% as \file{uplatex.cfg}, and process \file{upldoc.tex}
% (\upLaTeXe\ newer than July 2016 is required).
%
% Here we explain only difference between \file{pldoc.tex} (\pLaTeXe)
% and \file{upldoc.tex} (\upLaTeXe).
%\fi
%
%    \begin{macrocode}
%<*pldoc>
\begin{filecontents}{upldoc.dic}
西暦    せいれき
和暦    われき
\end{filecontents}
%    \end{macrocode}
%\ifJAPANESE
% \pLaTeXe{}のドキュメントでは、\file{plext.dtx}の中で組み立てるサンプル
% のために\file{plext}パッケージが必要ですが、\upLaTeXe{}のファイル
% にはそのようなサンプルが含まれないので除外しています。
%\else
% The document of \pLaTeXe\ requires \file{plext} package,
% since \file{plext.dtx} contains several examples of partial
% vertical writing. However, we don't have such examples in
% \upLaTeXe\ files, so no need for it.
%\fi
%    \begin{macrocode}
\documentclass{jltxdoc}
%\usepackage{plext} %% comment out for upLaTeX
\listfiles

\DoNotIndex{\def,\long,\edef,\xdef,\gdef,\let,\global}
\DoNotIndex{\if,\ifnum,\ifdim,\ifcat,\ifmmode,\ifvmode,\ifhmode,%
            \iftrue,\iffalse,\ifvoid,\ifx,\ifeof,\ifcase,\else,\or,\fi}
\DoNotIndex{\box,\copy,\setbox,\unvbox,\unhbox,\hbox,%
            \vbox,\vtop,\vcenter}
\DoNotIndex{\@empty,\immediate,\write}
\DoNotIndex{\egroup,\bgroup,\expandafter,\begingroup,\endgroup}
\DoNotIndex{\divide,\advance,\multiply,\count,\dimen}
\DoNotIndex{\relax,\space,\string}
\DoNotIndex{\csname,\endcsname,\@spaces,\openin,\openout,%
            \closein,\closeout}
\DoNotIndex{\catcode,\endinput}
\DoNotIndex{\jobname,\message,\read,\the,\m@ne,\noexpand}
\DoNotIndex{\hsize,\vsize,\hskip,\vskip,\kern,\hfil,\hfill,\hss,\vss,\unskip}
\DoNotIndex{\m@ne,\z@,\z@skip,\@ne,\tw@,\p@,\@minus,\@plus}
\DoNotIndex{\dp,\wd,\ht,\setlength,\addtolength}
\DoNotIndex{\newcommand, \renewcommand}

\ifJAPANESE
\IndexPrologue{\part*{索 引}%
                 \markboth{索 引}{索 引}%
                 \addcontentsline{toc}{part}{索 引}%
イタリック体の数字は、その項目が説明されているページを示しています。
下線の引かれた数字は、定義されているページを示しています。
その他の数字は、その項目が使われているページを示しています。}
\else
\IndexPrologue{\part*{Index}%
                 \markboth{Index}{Index}%
                 \addcontentsline{toc}{part}{Index}%
The italic numbers denote the pages where the corresponding entry
is described, numbers underlined point to the definition,
all others indicate the places where it is used.}
\fi
%
\ifJAPANESE
\GlossaryPrologue{\part*{変更履歴}%
                 \markboth{変更履歴}{変更履歴}%
                 \addcontentsline{toc}{part}{変更履歴}}
\else
\GlossaryPrologue{\part*{Change History}%
                 \markboth{Change History}{Change History}%
                 \addcontentsline{toc}{part}{Change History}}
\fi

\makeatletter
\def\changes@#1#2#3{%
  \let\protect\@unexpandable@protect
  \edef\@tempa{\noexpand\glossary{#2\space
               \currentfile\space#1\levelchar
               \ifx\saved@macroname\@empty
                  \space\actualchar\generalname
               \else
                  \expandafter\@gobble
                  \saved@macroname\actualchar
                  \string\verb\quotechar*%
                  \verbatimchar\saved@macroname
                  \verbatimchar
               \fi
               :\levelchar #3}}%
  \@tempa\endgroup\@esphack}
\renewcommand*\MacroFont{\fontencoding\encodingdefault
                   \fontfamily\ttdefault
                   \fontseries\mddefault
                   \fontshape\updefault
                   \small
                   \hfuzz 6pt\relax}
\renewcommand*\l@subsection{\@dottedtocline{2}{1.5em}{2.8em}}
\renewcommand*\l@subsubsection{\@dottedtocline{3}{3.8em}{3.4em}}
\makeatother
\RecordChanges
\CodelineIndex
\EnableCrossrefs
\setcounter{IndexColumns}{2}
\settowidth\MacroIndent{\ttfamily\scriptsize 000\ }
%    \end{macrocode}
%\ifJAPANESE
% この文書のタイトル・著者・日付を設定します。
% \changes{v1.0h-u00}{2016/05/08}{ドキュメントから\file{uplpatch.ltx}を除外
%     (based on platex.dtx 2016/05/08 v1.0h)}
% \changes{v1.0l-u01}{2016/06/19}{パッチレベルを\file{uplvers.dtx}から取得
%     (based on platex.dtx 2016/06/19 v1.0l)}
% \changes{v1.0y-u02}{2018/09/22}{最終更新日を\file{upldoc.pdf}に表示
%     (based on platex.dtx 2018/09/22 v1.0y)}
%\else
% Set the title, authors and the date for this document.
% \changes{v1.0h-u00}{2016/05/08}{Exclude \file{uplpatch.ltx} from the document
%     (based on platex.dtx 2016/05/08 v1.0h)}
% \changes{v1.0l-u01}{2016/06/19}{Get the patch level from \file{uplvers.dtx}
%     (based on platex.dtx 2016/06/19 v1.0l)}
% \changes{v1.0y-u02}{2018/09/22}{Show last update info on \file{upldoc.pdf}
%     (based on platex.dtx 2018/09/22 v1.0y)}
%\fi
%    \begin{macrocode}
 \title{The \upLaTeXe\ Sources}
 \author{Ken Nakano \& Japanese \TeX\ Development Community \& TTK}

% Get the (temporary) date and up-patch level from uplvers.dtx
\makeatletter
\let\patchdate=\@empty
\begingroup
   \def\ProvidesFile#1[#2 #3]#4\def\uppatch@level#5{%
      \date{#2}\xdef\patchdate{#5}\endinput}
   % \iffalse meta-comment
%% File: uplvers.dtx
%
%    pLaTeX version setting file:
%       Copyright 1995-2006 ASCII Corporation.
%    and modified for upLaTeX
%
%  Copyright (c) 2010 ASCII MEDIA WORKS
%  Copyright (c) 2016 Takuji Tanaka
%  Copyright (c) 2016-2017 Japanese TeX Development Community
%
%  This file is part of the upLaTeX2e system (community edition).
%  --------------------------------------------------------------
%
% \fi
%
%
% \setcounter{StandardModuleDepth}{1}
% \StopEventually{}
%
% \iffalse
% \changes{v1.0}{1995/05/16}{p\LaTeXe\ 用に\file{ltvers.dtx}を修正}
% \changes{v1.0a}{1995/08/30}{\LaTeX\ \texttt{!<1995/06/01!>}版用に修正}
% \changes{v1.0b}{1996/01/31}{\LaTeX\ \texttt{!<1995/12/01!>}版用に修正}
% \changes{v1.0c}{1997/01/11}{\LaTeX\ \texttt{!<1996/06/01!>}版用に修正}
% \changes{v1.0d}{1997/01/23}{\LaTeX\ \texttt{!<1996/12/01!>}版用に修正}
% \changes{v1.0e}{1997/07/02}{\LaTeX\ \texttt{!<1997/06/01!>}版用に修正}
% \changes{v1.0f}{1998/02/17}{\LaTeX\ \texttt{!<1997/12/01!>}版用に修正}
% \changes{v1.0g}{1998/09/01}{\LaTeX\ \texttt{!<1998/06/01!>}版用に修正}
% \changes{v1.0h}{1999/04/05}{\LaTeX\ \texttt{!<1998/12/01!>}版用に修正}
% \changes{v1.0i}{1999/08/09}{\LaTeX\ \texttt{!<1999/06/01!>}版用に修正}
% \changes{v1.0j}{2000/02/29}{\LaTeX\ \texttt{!<1999/12/01!>}版用に修正}
% \changes{v1.0k}{2000/11/03}{\LaTeX\ \texttt{!<2000/06/01!>}版用に修正}
% \changes{v1.0l}{2001/09/04}{\LaTeX\ \texttt{!<2001/06/01!>}版用に修正}
% \changes{v1.0m}{2004/08/10}{\LaTeX\ \texttt{!<2003/12/01!>}版対応確認}
% \changes{v1.0n}{2005/01/04}{plfonts.dtxバグ修正}
% \changes{v1.0o}{2006/01/04}{plfonts.dtxバグ修正}
% \changes{v1.0p}{2006/06/27}{plfonts.dtx \LaTeX\ \texttt{!<2005/12/01!>}対応}
% \changes{v1.0q}{2006/11/10}{plfonts.dtxバグ修正}
% \changes{v1.0q-u00}{2011/05/07}{p\LaTeX{}用からup\LaTeX{}用に修正。}
% \changes{v1.0r}{2016/01/26}{plcore.dtx p\TeX\ (r28720)対応}
% \changes{v1.0s}{2016/02/01}{\LaTeX\ \texttt{!<2015/01/01!>}のlatexreleaseに
%    対応するためのコードを導入}
% \changes{v1.0t}{2016/02/03}{\cs{plIncludeInRelease}と
%    \cs{plEndIncludeInRelease}を新設。}
% \changes{v1.0u}{2016/04/17}{\LaTeX\ \texttt{!<2016/03/31!>}版対応確認}
% \changes{v1.0u-u00}{2016/04/17}{p\LaTeX{}の変更に追随。}
% \changes{v1.0v}{2016/05/07}{パッチファイルをロードするのをやめた。}
% \changes{v1.0v}{2016/05/07}{起動時の文字列を最新の\LaTeX{}に合わせた。}
% \changes{v1.0w}{2016/05/12}{起動時の文字列に入れる\LaTeX{}のバージョンを
%    元の\LaTeX{}のバナーから引き継ぐように改良}
% \changes{v1.0w-u00}{2016/05/12}{起動時の文字列に入れるBabelのバージョンを
%    元の\LaTeX{}のバナーから取得するコードを\file{uplatex.ini}から取り入れた}
% \changes{v1.0w-u01}{2016/05/21}{サポート外の\LaTeX~2.09互換モードが
%    使われた場合に明確なエラーを出すようにした。}
% \changes{v1.0x}{2016/06/19}{パッチレベルを\file{plvers.dtx}で設定}
% \changes{v1.0x-u01}{2016/06/19}{p\LaTeX{}の変更に追随。}
% \changes{v1.0y-u01}{2016/06/29}{\file{uplatex.cfg}の読み込みを追加}
% \changes{v1.0z-u01}{2016/08/26}{\file{uplatex.cfg}の読み込みを
%    \file{uplcore.ltx}から\file{uplatex.ltx}へ移動}
% \changes{v1.1}{2016/09/14}{起動時のバナーを取得するコードを改良}
% \changes{v1.1-u01}{2016/09/14}{p\LaTeX{}の変更に追随。}
% \changes{v1.1a}{2017/02/20}{\LaTeX\ \texttt{!<2017/01/01!>}版対応確認}
% \changes{v1.1a-u01}{2017/03/05}{p\LaTeX{}の変更に追随。}
% \changes{v1.1b}{2017/03/19}{\cs{l@nohyphenation}の定義を保証
%    (sync with ltfinal 2017/03/09 v2.0t)}
% \changes{v1.1b}{2017/03/19}{\cs{document@default@language}の定義を保証
%    (sync with ltfinal 2017/03/09 v2.0t)}
% \changes{v1.1b-u01}{2017/03/19}{p\LaTeX{}の変更に追随。}
% \changes{v1.1c}{2017/04/23}{\LaTeX\ \texttt{!<2017/04/15!>}版対応確認}
% \changes{v1.1c-u01}{2017/05/04}{p\LaTeX{}の変更に追随。}
% \changes{v1.1d}{2017/09/24}{パッチレベルが負の数の場合をpre-release扱いへ}
% \changes{v1.1d-u01}{2017/09/24}{p\LaTeX{}の変更に追随。}
% \fi
%
% \iffalse
%<*driver>
% \fi
\ProvidesFile{uplvers.dtx}[2017/09/24 v1.1d-u01 upLaTeX Kernel (Version Info)]
% \iffalse
\documentclass{jltxdoc}
\GetFileInfo{uplvers.dtx}
\author{Ken Nakano \& Hideaki Togashi \& TTK}
\title{\filename}
\date{作成日:\filedate}
\begin{document}
  \maketitle
  \DocInput{\filename}
\end{document}
%</driver>
% \fi
%
% \section{バージョンの設定}
% まず、このディストリビューションでのup\LaTeXe{}の日付とバージョン番号
% を定義します。また、up\LaTeXe{}が起動されたときに表示される文字列の
% 設定もします。
%
% \changes{v1.0}{1995/05/16}{p\LaTeXe\ 用に\file{ltvers.dtx}を修正}
% \changes{v1.0a}{1995/08/30}{\LaTeX\ \texttt{!<1995/06/01!>}版用に修正}
% \changes{v1.0b}{1996/01/31}{\LaTeX\ \texttt{!<1995/12/01!>}版用に修正}
% \changes{v1.0c}{1997/01/11}{\LaTeX\ \texttt{!<1996/06/01!>}版用に修正}
% \changes{v1.0d}{1997/01/23}{\LaTeX\ \texttt{!<1996/12/01!>}版用に修正}
% \changes{v1.0e}{1997/07/02}{\LaTeX\ \texttt{!<1997/06/01!>}版用に修正}
% \changes{v1.0f}{1998/02/17}{\LaTeX\ \texttt{!<1997/12/01!>}版用に修正}
% \changes{v1.0g}{1998/09/01}{\LaTeX\ \texttt{!<1998/06/01!>}版用に修正}
% \changes{v1.0h}{1999/04/05}{\LaTeX\ \texttt{!<1998/12/01!>}版用に修正}
% \changes{v1.0i}{1999/08/09}{\LaTeX\ \texttt{!<1999/06/01!>}版用に修正}
% \changes{v1.0j}{2000/02/29}{\LaTeX\ \texttt{!<1999/12/01!>}版用に修正}
% \changes{v1.0k}{2000/11/03}{\LaTeX\ \texttt{!<2000/06/01!>}版用に修正}
% \changes{v1.0l}{2001/09/04}{\LaTeX\ \texttt{!<2001/06/01!>}版用に修正}
% \changes{v1.0m}{2004/08/10}{\LaTeX\ \texttt{!<2003/12/01!>}版対応確認}
% \changes{v1.0s}{2016/02/01}{\LaTeX\ \texttt{!<2015/01/01!>}版用に修正}
% \changes{v1.0u}{2016/04/17}{\LaTeX\ \texttt{!<2016/03/31!>}版対応確認}
% \changes{v1.1a}{2017/02/20}{\LaTeX\ \texttt{!<2017/01/01!>}版対応確認}
% \changes{v1.1c}{2017/04/23}{\LaTeX\ \texttt{!<2017/04/15!>}版対応確認}
%
% このバージョンのup\LaTeXe{}は、次のバージョンの\LaTeX{}\footnote{%
% \LaTeX\ authors: Johannes Braams, David Carlisle, Alan Jeffrey,
%   Leslie Lamport, Frank Mittelbach, Chris Rowley, Rainer Sch\"opf}を
% もとにしています。
%    \begin{macrocode}
%<*2ekernel>
%\def\fmtname{LaTeX2e}
%\edef\fmtversion
%</2ekernel>
%<latexrelease>\edef\latexreleaseversion
%<platexrelease>\edef\p@known@latexreleaseversion
%<*2ekernel|latexrelease|platexrelease>
   {2017/04/15}
%</2ekernel|latexrelease|platexrelease>
%    \end{macrocode}
%
% \begin{macro}{\pfmtname}
% \begin{macro}{\pfmtversion}
% \begin{macro}{\ppatch@level}
% up\LaTeXe{}のフォーマットファイル名とバージョンです。
% \changes{v1.0x}{2016/06/19}{パッチレベルを\file{plvers.dtx}で設定}
%    \begin{macrocode}
%<*plcore>
\def\pfmtname{pLaTeX2e}
\def\pfmtversion
%</plcore>
%<platexrelease>\edef\platexreleaseversion
%<*plcore|platexrelease>
   {2017/10/28u01}
%</plcore|platexrelease>
%<*plcore>
\def\ppatch@level{1}
%</plcore>
%    \end{macrocode}
% \end{macro}
% \end{macro}
% \end{macro}
%
% \subsection{\LaTeX~2.09互換モードの抑制}
%
% \begin{macro}{\documentstyle}
% p\LaTeX{}は、|\documentclass|の代わりに|\documentstyle|が使われると
% \LaTeX~2.09互換モードに入ります。しかし、up\LaTeX{}は新しいマクロ
% パッケージですので、\LaTeX~2.09互換モードをサポートしません。
% このため、\file{plcore.dtx}の定義を上書きして明確なエラーを出します。
% \changes{v1.0w-u01}{2016/05/21}{サポート外の\LaTeX~2.09互換モードが
%    使われた場合に明確なエラーを出すようにした。}
%    \begin{macrocode}
%<*plfinal>
\def\documentstyle{%
  \@latex@error{upLaTeX does NOT support LaTeX 2.09 compatibility
    mode.\MessageBreak Use \noexpand\documentclass instead}{%
    \noexpand\documentstyle is an old convention of LaTeX 2.09,
    which has been\MessageBreak obsolete since 1995. upLaTeX is
    first released in 2007, so we do\MessageBreak not provide any
    emulation of the LaTeX 2.09 author environment.\MessageBreak
    New documents should use Standard LaTeX conventions, and
    start\MessageBreak with the \noexpand\documentclass command.}%
  \documentclass}
%    \end{macrocode}
% \end{macro}
%
% \subsection{パッチファイルのロード}
%
% 次の部分は、up\LaTeXe{}のパッチファイルをロードするためのコードです。
% バグを修正するためのパッチを配布するかもしれません。
%
% パッチファイルをロードするコードはコメントアウトしました。
% \changes{v1.0v}{2016/05/07}{パッチファイルをロードするのをやめた。}
%    \begin{macrocode}
%\IfFileExists{uplpatch.ltx}
%  {\typeout{************************************^^J%
%            * Appliying patch file uplpatch.ltx *^^J%
%            ************************************}
%  \def\pfmtversion@topatch{unknown}
%  \input{uplpatch.ltx}
%  \ifx\pfmtversion\pfmtversion@topatch
%    \ifx\ppatch@level\@undefined
%      \typeout{^^J^^J^^J%
%   !!!!!!!!!!!!!!!!!!!!!!!!!!!!!!!!!!!!!!!!!!!!!!!!!!!!!!!^^J%
%   !! Patch file `uplpatch.ltx' (for version <\pfmtversion@topatch>)^^J%
%   !! is not suitable for version <\pfmtversion> of upLaTeX.^^J^^J%
%   !! Please check if iniptex found an old patch file:^^J%
%   !! --- if so, rename it or delete it, and redo the^^J%
%   !!     iniptex run.^^J%
%   !!!!!!!!!!!!!!!!!!!!!!!!!!!!!!!!!!!!!!!!!!!!!!!!!!!!!!!^^J}%
%      \batchmode \@@end
%    \fi
%  \else
%      \typeout{^^J^^J^^J%
%   !!!!!!!!!!!!!!!!!!!!!!!!!!!!!!!!!!!!!!!!!!!!!!!!!!!!!!!^^J%
%   !! Patch file `uplpatch.ltx' (for version <\pfmtversion@topatch>)^^J%
%   !! is not suitable for version <\pfmtversion> of upLaTeX.^^J%
%   !!^^J%
%   !! Please check if iniptex found an old patch file:^^J%
%   !! --- if so, rename it or delete it, and redo the^^J%
%   !!     iniptex run.^^J%
%   !!!!!!!!!!!!!!!!!!!!!!!!!!!!!!!!!!!!!!!!!!!!!!!!!!!!!!!^^J}%
%      \batchmode \@@end
%  \fi
%  \let\pfmtversion@topatch\relax
%  }{}
%    \end{macrocode}
%
% \subsection{起動時に表示するバナー}
%
% \begin{macro}{\everyjob}
% 起動時に表示される文字列です。
% \LaTeX{}にパッチがあてられている場合は、それも表示します。
%
%\iffalse
% この実装については\file{uplatex.dtx}のコメントを参照。(2016/09/14)
%\fi
%
% \changes{v1.0v}{2016/05/07}{起動時の文字列を最新の\LaTeX{}に合わせた。}
% \changes{v1.0w}{2016/05/12}{起動時の文字列に入れる\LaTeX{}のバージョンを
%    元の\LaTeX{}のバナーから引き継ぐように改良}
% \changes{v1.1}{2016/09/14}{起動時のバナーを取得するコードを改良}
% \changes{v1.1d}{2017/09/24}{パッチレベルが負の数の場合をpre-release扱いへ}
%    \begin{macrocode}
\ifx\patch@level\@undefined % fallback if undefined in LaTeX
  \def\patch@level{0}\fi
\ifx\ppatch@level\@undefined % fallback if undefined in upLaTeX
  \def\ppatch@level{0}\fi
\begingroup
  \def\parse@@BANNER\typeout#1\typeout#2#3\relax{#1}
  \edef\platexTMP{%
    \ifnum\ppatch@level=0
      \everyjob{\noexpand\typeout{%
        \pfmtname\space<\pfmtversion>\space
          (based on \expandafter\parse@@BANNER\platexBANNER)}}%
    \else\ifnum\ppatch@level>0
      \everyjob{\noexpand\typeout{%
        \pfmtname\space<\pfmtversion>+\ppatch@level\space
          (based on \expandafter\parse@@BANNER\platexBANNER)}}%
    \else
      \everyjob{\noexpand\typeout{%
        \pfmtname\space<\pfmtversion>-pre\ppatch@level\space
          (based on \expandafter\parse@@BANNER\platexBANNER)}}%
    \fi\fi
  }
\expandafter
\endgroup \platexTMP
%    \end{macrocode}
%
% p\LaTeX{}やup\LaTeX{}は、独自のハイフネーション・パターンを定義していません。
% \TeX\ Liveの標準的インストールでは、代わりに\LaTeX{}が読み込んでいる
% Babelパッケージのものが適用されるはずですから、起動時の文字列にも
% \file{hyphen.cfg}のバージョンを反映します(Babelパッケージの
% \file{hyphen.cfg}でない場合は、何も表示されず空行になるはずです)。
%
%\iffalse
% この実装については\file{uplatex.dtx}のコメントを参照。(2016/09/14)
%\fi
%
% \changes{v1.0w-u00}{2016/05/12}{起動時の文字列に入れるBabelのバージョンを
%    元の\LaTeX{}のバナーから取得するコードを\file{uplatex.ini}から取り入れた}
%    \begin{macrocode}
\begingroup
  \def\parse@@BANNER\typeout#1\typeout#2#3\relax{#2}
  \edef\platexTMP{%
    \the\everyjob\noexpand\typeout{\expandafter\parse@@BANNER\platexBANNER}%
  }
  \everyjob=\expandafter{\platexTMP}%
  \edef\platexTMP{%
    \noexpand\let\noexpand\platexBANNER=\noexpand\@undefined
    \noexpand\everyjob={\the\everyjob}%
  }
  \expandafter
\endgroup \platexTMP
%</plfinal>
%    \end{macrocode}
% \end{macro}
%
% ^^A 起動時に\file{uplatex.cfg}がある場合、それを読み込むようにする
% ^^A コードは、\file{uplcore.ltx}から\file{uplatex.ltx}へ移動しました。
% \changes{v1.0y-u01}{2016/06/29}{\file{uplatex.cfg}の読み込みを追加}
% \changes{v1.0z-u01}{2016/08/26}{\file{uplatex.cfg}の読み込みを
%    \file{uplcore.ltx}から\file{uplatex.ltx}へ移動}
%
% \subsection{ハイフネーション関連}
%
% \begin{macro}{\l@nohyphenation}
% \LaTeXe\ 2017-04-15で、|\verb|の途中でハイフネーションが起きないように
% する修正が入りました。この修正には|\l@nohyphenation|が定義済みでなければ
% なりませんが、通常はBabelの定義ファイルによって提供されています。
% \LaTeXe{}は特殊な状況も想定してltfinalで対策しているようですので、
% p\LaTeXe{}も念のためplfinalで対策します(参考:latex2e svn r1405)。
% \changes{v1.1b}{2017/03/19}{\cs{l@nohyphenation}の定義を保証
%    (sync with ltfinal 2017/03/09 v2.0t)}
%    \begin{macrocode}
%<*plfinal>
\ifx\l@nohyphenation \@undefined
  \newlanguage\l@nohyphenation
\fi
%    \end{macrocode}
% \end{macro}
%
% \begin{macro}{\document@default@language}
% \LaTeXe\ 2017-04-15で導入されたパラメータです。更新タイミングのずれの
% 可能性を考慮し、p\LaTeXe{}でも準備しておきます。verbatim環境の途中で
% 改ページが起きた場合にヘッダでハイフネーションが抑制されないように、
% |\@outputpage|で|\language|をリセットするときに使われます
% (参考:latex2e svn r1407)。
% \changes{v1.1b}{2017/03/19}{\cs{document@default@language}の定義を保証
%    (sync with ltfinal 2017/03/09 v2.0t)}
%    \begin{macrocode}
\ifx\document@default@language \@undefined
  \let\document@default@language\m@ne
\fi
%</plfinal>
%    \end{macrocode}
% \end{macro}
%
% \subsection{latexreleaseパッケージへの対応}
%
% 最後に、latexreleaseパッケージへの対応です。
% \begin{macro}{\plIncludeInRelease}
% \changes{v1.0t}{2016/02/03}{\cs{plIncludeInRelease}と
%    \cs{plEndIncludeInRelease}を新設。}
%    \begin{macrocode}
%<*plcore|platexrelease>
\def\plIncludeInRelease#1{\kernel@ifnextchar[%
  {\@plIncludeInRelease{#1}}
  {\@plIncludeInRelease{#1}[#1]}}
%    \end{macrocode}
%
%    \begin{macrocode}
\def\@plIncludeInRelease#1[#2]{\@plIncludeInRele@se{#2}}
%    \end{macrocode}
%
%    \begin{macrocode}
\def\@plIncludeInRele@se#1#2#3{%
  \toks@{[#1] #3}%
  \expandafter\ifx\csname\string#2+\@currname+IIR\endcsname\relax
    \ifnum\expandafter\@parse@version#1//00\@nil
          >\expandafter\@parse@version\pfmtversion//00\@nil
      \GenericInfo{}{Skipping: \the\toks@}%
     \expandafter\expandafter\expandafter\@gobble@plIncludeInRelease
    \else
      \GenericInfo{}{Applying: \the\toks@}%
      \expandafter\let\csname\string#2+\@currname+IIR\endcsname\@empty
    \fi
  \else
    \GenericInfo{}{Already applied: \the\toks@}%
    \expandafter\@gobble@plIncludeInRelease
  \fi
}
%    \end{macrocode}
%
%    \begin{macrocode}
\long\def\@gobble@plIncludeInRelease#1\plEndIncludeInRelease{}
\let\plEndIncludeInRelease\relax
%</plcore|platexrelease>
%    \end{macrocode}
% \end{macro}
%
% \LaTeXe{}が提供するlatexreleaseパッケージが読み込まれていて、
% かつp\LaTeXe{}が提供するplatexreleaseパッケージが読み込まれていない
% 場合は、警告を出します。
% \changes{v1.0s}{2016/02/01}{latexrelease利用時に警告を出すようにした}
%    \begin{macrocode}
%<*plfinal>
\AtBeginDocument{%
  \@ifpackageloaded{latexrelease}{%
    \@ifpackageloaded{platexrelease}{}{%
      \@latex@warning@no@line{%
        Package latexrelease is loaded.\MessageBreak
        Some patches in pLaTeX2e core may be overwritten.\MessageBreak
        Consider using platexrelease.\MessageBreak
        See platex.pdf for detail}%
    }%
  }{}%
}
%</plfinal>
%    \end{macrocode}
%
% \Finale
%
\endinput

\endgroup

% Add the patch version if available.
\def\Xpatch{}
\ifx\patchdate\Xpatch\else
  \edef\@date{\@date\space version \patchdate}
\fi

% Obtain the last update info, as upLaTeX does not change format date
% -> if successful, reconstruct the date completely
\def\lastupd@te{0000/00/00}
\begingroup
   \def\ProvidesFile#1[#2 #3]{%
      \def\@tempd@te{#2}\endinput
      \@ifl@t@r{\@tempd@te}{\lastupd@te}{%
         \global\let\lastupd@te\@tempd@te
      }{}}
   \let\ProvidesClass\ProvidesFile
   \let\ProvidesPackage\ProvidesFile
   % \iffalse meta-comment
%% File: uplvers.dtx
%
%    pLaTeX version setting file:
%       Copyright 1995-2006 ASCII Corporation.
%    and modified for upLaTeX
%
%  Copyright (c) 2010 ASCII MEDIA WORKS
%  Copyright (c) 2016 Takuji Tanaka
%  Copyright (c) 2016-2017 Japanese TeX Development Community
%
%  This file is part of the upLaTeX2e system (community edition).
%  --------------------------------------------------------------
%
% \fi
%
%
% \setcounter{StandardModuleDepth}{1}
% \StopEventually{}
%
% \iffalse
% \changes{v1.0}{1995/05/16}{p\LaTeXe\ 用に\file{ltvers.dtx}を修正}
% \changes{v1.0a}{1995/08/30}{\LaTeX\ \texttt{!<1995/06/01!>}版用に修正}
% \changes{v1.0b}{1996/01/31}{\LaTeX\ \texttt{!<1995/12/01!>}版用に修正}
% \changes{v1.0c}{1997/01/11}{\LaTeX\ \texttt{!<1996/06/01!>}版用に修正}
% \changes{v1.0d}{1997/01/23}{\LaTeX\ \texttt{!<1996/12/01!>}版用に修正}
% \changes{v1.0e}{1997/07/02}{\LaTeX\ \texttt{!<1997/06/01!>}版用に修正}
% \changes{v1.0f}{1998/02/17}{\LaTeX\ \texttt{!<1997/12/01!>}版用に修正}
% \changes{v1.0g}{1998/09/01}{\LaTeX\ \texttt{!<1998/06/01!>}版用に修正}
% \changes{v1.0h}{1999/04/05}{\LaTeX\ \texttt{!<1998/12/01!>}版用に修正}
% \changes{v1.0i}{1999/08/09}{\LaTeX\ \texttt{!<1999/06/01!>}版用に修正}
% \changes{v1.0j}{2000/02/29}{\LaTeX\ \texttt{!<1999/12/01!>}版用に修正}
% \changes{v1.0k}{2000/11/03}{\LaTeX\ \texttt{!<2000/06/01!>}版用に修正}
% \changes{v1.0l}{2001/09/04}{\LaTeX\ \texttt{!<2001/06/01!>}版用に修正}
% \changes{v1.0m}{2004/08/10}{\LaTeX\ \texttt{!<2003/12/01!>}版対応確認}
% \changes{v1.0n}{2005/01/04}{plfonts.dtxバグ修正}
% \changes{v1.0o}{2006/01/04}{plfonts.dtxバグ修正}
% \changes{v1.0p}{2006/06/27}{plfonts.dtx \LaTeX\ \texttt{!<2005/12/01!>}対応}
% \changes{v1.0q}{2006/11/10}{plfonts.dtxバグ修正}
% \changes{v1.0q-u00}{2011/05/07}{p\LaTeX{}用からup\LaTeX{}用に修正。}
% \changes{v1.0r}{2016/01/26}{plcore.dtx p\TeX\ (r28720)対応}
% \changes{v1.0s}{2016/02/01}{\LaTeX\ \texttt{!<2015/01/01!>}のlatexreleaseに
%    対応するためのコードを導入}
% \changes{v1.0t}{2016/02/03}{\cs{plIncludeInRelease}と
%    \cs{plEndIncludeInRelease}を新設。}
% \changes{v1.0u}{2016/04/17}{\LaTeX\ \texttt{!<2016/03/31!>}版対応確認}
% \changes{v1.0u-u00}{2016/04/17}{p\LaTeX{}の変更に追随。}
% \changes{v1.0v}{2016/05/07}{パッチファイルをロードするのをやめた。}
% \changes{v1.0v}{2016/05/07}{起動時の文字列を最新の\LaTeX{}に合わせた。}
% \changes{v1.0w}{2016/05/12}{起動時の文字列に入れる\LaTeX{}のバージョンを
%    元の\LaTeX{}のバナーから引き継ぐように改良}
% \changes{v1.0w-u00}{2016/05/12}{起動時の文字列に入れるBabelのバージョンを
%    元の\LaTeX{}のバナーから取得するコードを\file{uplatex.ini}から取り入れた}
% \changes{v1.0w-u01}{2016/05/21}{サポート外の\LaTeX~2.09互換モードが
%    使われた場合に明確なエラーを出すようにした。}
% \changes{v1.0x}{2016/06/19}{パッチレベルを\file{plvers.dtx}で設定}
% \changes{v1.0x-u01}{2016/06/19}{p\LaTeX{}の変更に追随。}
% \changes{v1.0y-u01}{2016/06/29}{\file{uplatex.cfg}の読み込みを追加}
% \changes{v1.0z-u01}{2016/08/26}{\file{uplatex.cfg}の読み込みを
%    \file{uplcore.ltx}から\file{uplatex.ltx}へ移動}
% \changes{v1.1}{2016/09/14}{起動時のバナーを取得するコードを改良}
% \changes{v1.1-u01}{2016/09/14}{p\LaTeX{}の変更に追随。}
% \changes{v1.1a}{2017/02/20}{\LaTeX\ \texttt{!<2017/01/01!>}版対応確認}
% \changes{v1.1a-u01}{2017/03/05}{p\LaTeX{}の変更に追随。}
% \changes{v1.1b}{2017/03/19}{\cs{l@nohyphenation}の定義を保証
%    (sync with ltfinal 2017/03/09 v2.0t)}
% \changes{v1.1b}{2017/03/19}{\cs{document@default@language}の定義を保証
%    (sync with ltfinal 2017/03/09 v2.0t)}
% \changes{v1.1b-u01}{2017/03/19}{p\LaTeX{}の変更に追随。}
% \changes{v1.1c}{2017/04/23}{\LaTeX\ \texttt{!<2017/04/15!>}版対応確認}
% \changes{v1.1c-u01}{2017/05/04}{p\LaTeX{}の変更に追随。}
% \changes{v1.1d}{2017/09/24}{パッチレベルが負の数の場合をpre-release扱いへ}
% \changes{v1.1d-u01}{2017/09/24}{p\LaTeX{}の変更に追随。}
% \fi
%
% \iffalse
%<*driver>
% \fi
\ProvidesFile{uplvers.dtx}[2017/09/24 v1.1d-u01 upLaTeX Kernel (Version Info)]
% \iffalse
\documentclass{jltxdoc}
\GetFileInfo{uplvers.dtx}
\author{Ken Nakano \& Hideaki Togashi \& TTK}
\title{\filename}
\date{作成日:\filedate}
\begin{document}
  \maketitle
  \DocInput{\filename}
\end{document}
%</driver>
% \fi
%
% \section{バージョンの設定}
% まず、このディストリビューションでのup\LaTeXe{}の日付とバージョン番号
% を定義します。また、up\LaTeXe{}が起動されたときに表示される文字列の
% 設定もします。
%
% \changes{v1.0}{1995/05/16}{p\LaTeXe\ 用に\file{ltvers.dtx}を修正}
% \changes{v1.0a}{1995/08/30}{\LaTeX\ \texttt{!<1995/06/01!>}版用に修正}
% \changes{v1.0b}{1996/01/31}{\LaTeX\ \texttt{!<1995/12/01!>}版用に修正}
% \changes{v1.0c}{1997/01/11}{\LaTeX\ \texttt{!<1996/06/01!>}版用に修正}
% \changes{v1.0d}{1997/01/23}{\LaTeX\ \texttt{!<1996/12/01!>}版用に修正}
% \changes{v1.0e}{1997/07/02}{\LaTeX\ \texttt{!<1997/06/01!>}版用に修正}
% \changes{v1.0f}{1998/02/17}{\LaTeX\ \texttt{!<1997/12/01!>}版用に修正}
% \changes{v1.0g}{1998/09/01}{\LaTeX\ \texttt{!<1998/06/01!>}版用に修正}
% \changes{v1.0h}{1999/04/05}{\LaTeX\ \texttt{!<1998/12/01!>}版用に修正}
% \changes{v1.0i}{1999/08/09}{\LaTeX\ \texttt{!<1999/06/01!>}版用に修正}
% \changes{v1.0j}{2000/02/29}{\LaTeX\ \texttt{!<1999/12/01!>}版用に修正}
% \changes{v1.0k}{2000/11/03}{\LaTeX\ \texttt{!<2000/06/01!>}版用に修正}
% \changes{v1.0l}{2001/09/04}{\LaTeX\ \texttt{!<2001/06/01!>}版用に修正}
% \changes{v1.0m}{2004/08/10}{\LaTeX\ \texttt{!<2003/12/01!>}版対応確認}
% \changes{v1.0s}{2016/02/01}{\LaTeX\ \texttt{!<2015/01/01!>}版用に修正}
% \changes{v1.0u}{2016/04/17}{\LaTeX\ \texttt{!<2016/03/31!>}版対応確認}
% \changes{v1.1a}{2017/02/20}{\LaTeX\ \texttt{!<2017/01/01!>}版対応確認}
% \changes{v1.1c}{2017/04/23}{\LaTeX\ \texttt{!<2017/04/15!>}版対応確認}
%
% このバージョンのup\LaTeXe{}は、次のバージョンの\LaTeX{}\footnote{%
% \LaTeX\ authors: Johannes Braams, David Carlisle, Alan Jeffrey,
%   Leslie Lamport, Frank Mittelbach, Chris Rowley, Rainer Sch\"opf}を
% もとにしています。
%    \begin{macrocode}
%<*2ekernel>
%\def\fmtname{LaTeX2e}
%\edef\fmtversion
%</2ekernel>
%<latexrelease>\edef\latexreleaseversion
%<platexrelease>\edef\p@known@latexreleaseversion
%<*2ekernel|latexrelease|platexrelease>
   {2017/04/15}
%</2ekernel|latexrelease|platexrelease>
%    \end{macrocode}
%
% \begin{macro}{\pfmtname}
% \begin{macro}{\pfmtversion}
% \begin{macro}{\ppatch@level}
% up\LaTeXe{}のフォーマットファイル名とバージョンです。
% \changes{v1.0x}{2016/06/19}{パッチレベルを\file{plvers.dtx}で設定}
%    \begin{macrocode}
%<*plcore>
\def\pfmtname{pLaTeX2e}
\def\pfmtversion
%</plcore>
%<platexrelease>\edef\platexreleaseversion
%<*plcore|platexrelease>
   {2017/10/28u01}
%</plcore|platexrelease>
%<*plcore>
\def\ppatch@level{1}
%</plcore>
%    \end{macrocode}
% \end{macro}
% \end{macro}
% \end{macro}
%
% \subsection{\LaTeX~2.09互換モードの抑制}
%
% \begin{macro}{\documentstyle}
% p\LaTeX{}は、|\documentclass|の代わりに|\documentstyle|が使われると
% \LaTeX~2.09互換モードに入ります。しかし、up\LaTeX{}は新しいマクロ
% パッケージですので、\LaTeX~2.09互換モードをサポートしません。
% このため、\file{plcore.dtx}の定義を上書きして明確なエラーを出します。
% \changes{v1.0w-u01}{2016/05/21}{サポート外の\LaTeX~2.09互換モードが
%    使われた場合に明確なエラーを出すようにした。}
%    \begin{macrocode}
%<*plfinal>
\def\documentstyle{%
  \@latex@error{upLaTeX does NOT support LaTeX 2.09 compatibility
    mode.\MessageBreak Use \noexpand\documentclass instead}{%
    \noexpand\documentstyle is an old convention of LaTeX 2.09,
    which has been\MessageBreak obsolete since 1995. upLaTeX is
    first released in 2007, so we do\MessageBreak not provide any
    emulation of the LaTeX 2.09 author environment.\MessageBreak
    New documents should use Standard LaTeX conventions, and
    start\MessageBreak with the \noexpand\documentclass command.}%
  \documentclass}
%    \end{macrocode}
% \end{macro}
%
% \subsection{パッチファイルのロード}
%
% 次の部分は、up\LaTeXe{}のパッチファイルをロードするためのコードです。
% バグを修正するためのパッチを配布するかもしれません。
%
% パッチファイルをロードするコードはコメントアウトしました。
% \changes{v1.0v}{2016/05/07}{パッチファイルをロードするのをやめた。}
%    \begin{macrocode}
%\IfFileExists{uplpatch.ltx}
%  {\typeout{************************************^^J%
%            * Appliying patch file uplpatch.ltx *^^J%
%            ************************************}
%  \def\pfmtversion@topatch{unknown}
%  \input{uplpatch.ltx}
%  \ifx\pfmtversion\pfmtversion@topatch
%    \ifx\ppatch@level\@undefined
%      \typeout{^^J^^J^^J%
%   !!!!!!!!!!!!!!!!!!!!!!!!!!!!!!!!!!!!!!!!!!!!!!!!!!!!!!!^^J%
%   !! Patch file `uplpatch.ltx' (for version <\pfmtversion@topatch>)^^J%
%   !! is not suitable for version <\pfmtversion> of upLaTeX.^^J^^J%
%   !! Please check if iniptex found an old patch file:^^J%
%   !! --- if so, rename it or delete it, and redo the^^J%
%   !!     iniptex run.^^J%
%   !!!!!!!!!!!!!!!!!!!!!!!!!!!!!!!!!!!!!!!!!!!!!!!!!!!!!!!^^J}%
%      \batchmode \@@end
%    \fi
%  \else
%      \typeout{^^J^^J^^J%
%   !!!!!!!!!!!!!!!!!!!!!!!!!!!!!!!!!!!!!!!!!!!!!!!!!!!!!!!^^J%
%   !! Patch file `uplpatch.ltx' (for version <\pfmtversion@topatch>)^^J%
%   !! is not suitable for version <\pfmtversion> of upLaTeX.^^J%
%   !!^^J%
%   !! Please check if iniptex found an old patch file:^^J%
%   !! --- if so, rename it or delete it, and redo the^^J%
%   !!     iniptex run.^^J%
%   !!!!!!!!!!!!!!!!!!!!!!!!!!!!!!!!!!!!!!!!!!!!!!!!!!!!!!!^^J}%
%      \batchmode \@@end
%  \fi
%  \let\pfmtversion@topatch\relax
%  }{}
%    \end{macrocode}
%
% \subsection{起動時に表示するバナー}
%
% \begin{macro}{\everyjob}
% 起動時に表示される文字列です。
% \LaTeX{}にパッチがあてられている場合は、それも表示します。
%
%\iffalse
% この実装については\file{uplatex.dtx}のコメントを参照。(2016/09/14)
%\fi
%
% \changes{v1.0v}{2016/05/07}{起動時の文字列を最新の\LaTeX{}に合わせた。}
% \changes{v1.0w}{2016/05/12}{起動時の文字列に入れる\LaTeX{}のバージョンを
%    元の\LaTeX{}のバナーから引き継ぐように改良}
% \changes{v1.1}{2016/09/14}{起動時のバナーを取得するコードを改良}
% \changes{v1.1d}{2017/09/24}{パッチレベルが負の数の場合をpre-release扱いへ}
%    \begin{macrocode}
\ifx\patch@level\@undefined % fallback if undefined in LaTeX
  \def\patch@level{0}\fi
\ifx\ppatch@level\@undefined % fallback if undefined in upLaTeX
  \def\ppatch@level{0}\fi
\begingroup
  \def\parse@@BANNER\typeout#1\typeout#2#3\relax{#1}
  \edef\platexTMP{%
    \ifnum\ppatch@level=0
      \everyjob{\noexpand\typeout{%
        \pfmtname\space<\pfmtversion>\space
          (based on \expandafter\parse@@BANNER\platexBANNER)}}%
    \else\ifnum\ppatch@level>0
      \everyjob{\noexpand\typeout{%
        \pfmtname\space<\pfmtversion>+\ppatch@level\space
          (based on \expandafter\parse@@BANNER\platexBANNER)}}%
    \else
      \everyjob{\noexpand\typeout{%
        \pfmtname\space<\pfmtversion>-pre\ppatch@level\space
          (based on \expandafter\parse@@BANNER\platexBANNER)}}%
    \fi\fi
  }
\expandafter
\endgroup \platexTMP
%    \end{macrocode}
%
% p\LaTeX{}やup\LaTeX{}は、独自のハイフネーション・パターンを定義していません。
% \TeX\ Liveの標準的インストールでは、代わりに\LaTeX{}が読み込んでいる
% Babelパッケージのものが適用されるはずですから、起動時の文字列にも
% \file{hyphen.cfg}のバージョンを反映します(Babelパッケージの
% \file{hyphen.cfg}でない場合は、何も表示されず空行になるはずです)。
%
%\iffalse
% この実装については\file{uplatex.dtx}のコメントを参照。(2016/09/14)
%\fi
%
% \changes{v1.0w-u00}{2016/05/12}{起動時の文字列に入れるBabelのバージョンを
%    元の\LaTeX{}のバナーから取得するコードを\file{uplatex.ini}から取り入れた}
%    \begin{macrocode}
\begingroup
  \def\parse@@BANNER\typeout#1\typeout#2#3\relax{#2}
  \edef\platexTMP{%
    \the\everyjob\noexpand\typeout{\expandafter\parse@@BANNER\platexBANNER}%
  }
  \everyjob=\expandafter{\platexTMP}%
  \edef\platexTMP{%
    \noexpand\let\noexpand\platexBANNER=\noexpand\@undefined
    \noexpand\everyjob={\the\everyjob}%
  }
  \expandafter
\endgroup \platexTMP
%</plfinal>
%    \end{macrocode}
% \end{macro}
%
% ^^A 起動時に\file{uplatex.cfg}がある場合、それを読み込むようにする
% ^^A コードは、\file{uplcore.ltx}から\file{uplatex.ltx}へ移動しました。
% \changes{v1.0y-u01}{2016/06/29}{\file{uplatex.cfg}の読み込みを追加}
% \changes{v1.0z-u01}{2016/08/26}{\file{uplatex.cfg}の読み込みを
%    \file{uplcore.ltx}から\file{uplatex.ltx}へ移動}
%
% \subsection{ハイフネーション関連}
%
% \begin{macro}{\l@nohyphenation}
% \LaTeXe\ 2017-04-15で、|\verb|の途中でハイフネーションが起きないように
% する修正が入りました。この修正には|\l@nohyphenation|が定義済みでなければ
% なりませんが、通常はBabelの定義ファイルによって提供されています。
% \LaTeXe{}は特殊な状況も想定してltfinalで対策しているようですので、
% p\LaTeXe{}も念のためplfinalで対策します(参考:latex2e svn r1405)。
% \changes{v1.1b}{2017/03/19}{\cs{l@nohyphenation}の定義を保証
%    (sync with ltfinal 2017/03/09 v2.0t)}
%    \begin{macrocode}
%<*plfinal>
\ifx\l@nohyphenation \@undefined
  \newlanguage\l@nohyphenation
\fi
%    \end{macrocode}
% \end{macro}
%
% \begin{macro}{\document@default@language}
% \LaTeXe\ 2017-04-15で導入されたパラメータです。更新タイミングのずれの
% 可能性を考慮し、p\LaTeXe{}でも準備しておきます。verbatim環境の途中で
% 改ページが起きた場合にヘッダでハイフネーションが抑制されないように、
% |\@outputpage|で|\language|をリセットするときに使われます
% (参考:latex2e svn r1407)。
% \changes{v1.1b}{2017/03/19}{\cs{document@default@language}の定義を保証
%    (sync with ltfinal 2017/03/09 v2.0t)}
%    \begin{macrocode}
\ifx\document@default@language \@undefined
  \let\document@default@language\m@ne
\fi
%</plfinal>
%    \end{macrocode}
% \end{macro}
%
% \subsection{latexreleaseパッケージへの対応}
%
% 最後に、latexreleaseパッケージへの対応です。
% \begin{macro}{\plIncludeInRelease}
% \changes{v1.0t}{2016/02/03}{\cs{plIncludeInRelease}と
%    \cs{plEndIncludeInRelease}を新設。}
%    \begin{macrocode}
%<*plcore|platexrelease>
\def\plIncludeInRelease#1{\kernel@ifnextchar[%
  {\@plIncludeInRelease{#1}}
  {\@plIncludeInRelease{#1}[#1]}}
%    \end{macrocode}
%
%    \begin{macrocode}
\def\@plIncludeInRelease#1[#2]{\@plIncludeInRele@se{#2}}
%    \end{macrocode}
%
%    \begin{macrocode}
\def\@plIncludeInRele@se#1#2#3{%
  \toks@{[#1] #3}%
  \expandafter\ifx\csname\string#2+\@currname+IIR\endcsname\relax
    \ifnum\expandafter\@parse@version#1//00\@nil
          >\expandafter\@parse@version\pfmtversion//00\@nil
      \GenericInfo{}{Skipping: \the\toks@}%
     \expandafter\expandafter\expandafter\@gobble@plIncludeInRelease
    \else
      \GenericInfo{}{Applying: \the\toks@}%
      \expandafter\let\csname\string#2+\@currname+IIR\endcsname\@empty
    \fi
  \else
    \GenericInfo{}{Already applied: \the\toks@}%
    \expandafter\@gobble@plIncludeInRelease
  \fi
}
%    \end{macrocode}
%
%    \begin{macrocode}
\long\def\@gobble@plIncludeInRelease#1\plEndIncludeInRelease{}
\let\plEndIncludeInRelease\relax
%</plcore|platexrelease>
%    \end{macrocode}
% \end{macro}
%
% \LaTeXe{}が提供するlatexreleaseパッケージが読み込まれていて、
% かつp\LaTeXe{}が提供するplatexreleaseパッケージが読み込まれていない
% 場合は、警告を出します。
% \changes{v1.0s}{2016/02/01}{latexrelease利用時に警告を出すようにした}
%    \begin{macrocode}
%<*plfinal>
\AtBeginDocument{%
  \@ifpackageloaded{latexrelease}{%
    \@ifpackageloaded{platexrelease}{}{%
      \@latex@warning@no@line{%
        Package latexrelease is loaded.\MessageBreak
        Some patches in pLaTeX2e core may be overwritten.\MessageBreak
        Consider using platexrelease.\MessageBreak
        See platex.pdf for detail}%
    }%
  }{}%
}
%</plfinal>
%    \end{macrocode}
%
% \Finale
%
\endinput

   % \iffalse meta-comment
%% File: uplfonts.dtx
%
%    pLaTeX fonts files:
%       Copyright 1994-2006 ASCII Corporation.
%    and modified for upLaTeX
%
%  Copyright (c) 2010 ASCII MEDIA WORKS
%  Copyright (c) 2016 Takuji Tanaka
%  Copyright (c) 2016-2020 Japanese TeX Development Community
%
%  This file is part of the upLaTeX2e system (community edition).
%  --------------------------------------------------------------
%
% \fi
%
% \iffalse
%<*driver>
\ifx\JAPANESEtrue\undefined
  \expandafter\newif\csname ifJAPANESE\endcsname
  \JAPANESEtrue
\fi
\def\eTeX{$\varepsilon$-\TeX}
\def\pTeX{p\kern-.15em\TeX}
\def\epTeX{$\varepsilon$-\pTeX}
\def\pLaTeX{p\kern-.05em\LaTeX}
\def\pLaTeXe{p\kern-.05em\LaTeXe}
\def\upTeX{u\pTeX}
\def\eupTeX{$\varepsilon$-\upTeX}
\def\upLaTeX{u\pLaTeX}
\def\upLaTeXe{u\pLaTeXe}
%</driver>
% \fi
%
% \setcounter{StandardModuleDepth}{1}
% \StopEventually{}
%
% \iffalse
% \changes{v1.5-u00}{2011/05/07}{p\LaTeX{}用からup\LaTeX{}用に修正。
%     (based on plfonts.dtx 2006/11/10 v1.5)}
% \changes{v1.6a-u00}{2016/04/06}{p\LaTeX{}の変更に追随。
%     (based on plfonts.dtx 2016/04/01 v1.6a)}
% \changes{v1.6b-u00}{2016/04/30}{uptrace.styの冒頭でtracefnt.styを
%    \cs{RequirePackageWithOptions}するようにした
%     (based on plfonts.dtx 2016/04/30 v1.6b)}
% \changes{v1.6c-u00}{2016/06/06}{p\LaTeX{}の変更に追随。
%     (based on plfonts.dtx 2016/06/06 v1.6c)}
% \changes{v1.6d-u00}{2016/06/19}{p\LaTeX{}の変更に追随。
%     (based on plfonts.dtx 2016/06/19 v1.6d)}
% \changes{v1.6e-u00}{2016/06/29}{p\LaTeX{}の変更に追随。
%     (based on plfonts.dtx 2016/06/26 v1.6e)}
% \changes{v1.6f-u00}{2017/03/05}{uptrace.styのplatexrelease対応
%     (based on plfonts.dtx 2017/02/20 v1.6f)}
% \changes{v1.6g-u00}{2017/03/08}{p\LaTeX{}の変更に追随。
%     (based on plfonts.dtx 2017/03/07 v1.6g)}
% \changes{v1.6h-u00}{2017/08/05}{p\LaTeX{}の変更に追随。
%     (based on plfonts.dtx 2017/08/05 v1.6h)}
% \changes{v1.6i-u00}{2017/09/24}{p\LaTeX{}の変更に追随。
%     (based on plfonts.dtx 2017/09/24 v1.6i)}
% \changes{v1.6j-u00}{2017/11/06}{p\LaTeX{}の変更に追随。
%     (based on plfonts.dtx 2017/11/06 v1.6j)}
% \changes{v1.6k-u00}{2017/12/05}{デフォルト設定ファイルの読み込みを
%    \file{uplcore.ltx}から\file{uplatex.ltx}へ移動
%     (based on plfonts.dtx 2017/12/05 v1.6k)}
% \changes{v1.6k-u01}{2017/12/10}{uptraceパッケージは
%    ptraceパッケージを読み込むだけとした}
% \changes{v1.6k-u02}{2017/12/10}{p\LaTeX{}との統合のため、
%    up\LaTeX{}用の最小限の変更だけを定義するようにした}
% \changes{v1.6l-u02}{2018/02/04}{p\LaTeX{}の変更に追随。
%     (based on plfonts.dtx 2018/02/04 v1.6l)}
% \changes{v1.6q-u02}{2018/07/03}{p\LaTeX{}の変更に追随。
%     (based on plfonts.dtx 2018/07/03 v1.6q)}
% \changes{v1.6t-u02}{2019/09/22}{p\LaTeX{}の変更に追随。
%     (based on plfonts.dtx 2019/09/16 v1.6t)}
% \changes{v1.6v-u02}{2020/02/01}{p\LaTeX{}の変更に追随。
%     (based on plfonts.dtx 2020/02/01 v1.6v)}
% \fi
%
% \iffalse
%<*driver>
\NeedsTeXFormat{pLaTeX2e}
% \fi
\ProvidesFile{uplfonts.dtx}[2020/02/01 v1.6v-u02 upLaTeX New Font Selection Scheme]
% \iffalse
\documentclass{jltxdoc}
\GetFileInfo{uplfonts.dtx}
\title{up\LaTeXe{}のフォントコマンド\space\fileversion}
\author{Ken Nakano \& Hideaki Togashi \& TTK}
\date{作成日:\filedate}
\begin{document}
   \maketitle
   \tableofcontents
   \DocInput{\filename}
\end{document}
%</driver>
% \fi
%
% \section{概要}\label{plfonts:intro}
% ここでは、和文書体を\NFSS2のインターフェイスで選択するための
% コマンドやマクロについて説明をしています。
% また、フォント定義ファイルや初期設定ファイルなどの説明もしています。
% 新しいフォント選択コマンドの使い方については、\file{fntguide.tex}や
% \file{usrguide.tex}を参照してください。
% \changes{v1.5-u00}{2011/05/07}{p\LaTeX{}用からup\LaTeX{}用に修正。
%     (based on plfonts.dtx 2006/11/10 v1.5)}
% \changes{v1.6k-u02}{2017/12/10}{p\LaTeX{}との統合のため、
%    up\LaTeX{}用の最小限の変更だけを定義するようにした}
%
% \begin{description}
% \item[第\ref{plfonts:intro}節] この節です。このファイルの概要と
%    \dst{}プログラムのためのオプションを示しています。
% \item[第\ref{plfonts:codes}節] 実際のコードの部分です。
% \item[第\ref{plfonts:pldefs}節] プリロードフォントやエラーフォントなどの
%  初期設定について説明をしています。
% \item[第\ref{plfonts:fontdef}節] フォント定義ファイルについて
%    説明をしています。
% \end{description}
%
%
% \subsection{\dst{}プログラムのためのオプション}
% \dst{}プログラムのためのオプションを次に示します。
%
% \DeleteShortVerb{\|}
% \begin{center}
% \begin{tabular}{l|p{0.7\linewidth}}
% \emph{オプション} & \emph{意味}\\\hline
% plcore & \file{uplcore.ltx}の断片を生成するオプションでしたが、削除。\\
% trace  & \file{uptrace.sty}を生成します。\\
% JY2mc  & 横組用、明朝体のフォント定義ファイルを生成します。\\
% JY2gt  & 横組用、ゴシック体のフォント定義ファイルを生成します。\\
% JT2mc  & 縦組用、明朝体のフォント定義ファイルを生成します。\\
% JT2gt  & 縦組用、ゴシック体のフォント定義ファイルを生成します。\\
% pldefs & \file{upldefs.ltx}を生成します。次の4つのオプションを付加する
%          ことで、プリロードするフォントを選択することができます。
%          デフォルトは10ptです。\\
% xpt    & 10pt プリロード\\
% xipt   & 11pt プリロード\\
% xiipt  & 12pt プリロード\\
% ori    & \file{plfonts.tex}に似たプリロード\\
% \end{tabular}
% \end{center}
% \MakeShortVerb{\|}
%
%
%
% \section{コード}\label{plfonts:codes}
% \NFSS2の拡張は、p\LaTeX{}において\file{plfonts.dtx}から生成される
% \file{plcore.ltx}をそのまま利用するので、up\LaTeX{}では定義しません。
% 後方互換性のため、\file{uptrace.sty}を提供しますが、
% これも単に\file{ptrace.sty}を読み込むだけとします。
%
% \changes{v1.6b-u00}{2016/04/30}{uptrace.styの冒頭でtracefnt.styを
%    \cs{RequirePackageWithOptions}するようにした}
% \changes{v1.6k-u01}{2017/12/10}{uptraceパッケージは
%    ptraceパッケージを読み込むだけとした}
%    \begin{macrocode}
%<*trace>
\NeedsTeXFormat{pLaTeX2e}
\ProvidesPackage{uptrace}
     [2019/09/22 v1.6t-u02 Standard upLaTeX package (font tracing)]
\RequirePackageWithOptions{ptrace}
%</trace>
%    \end{macrocode}
%
% デフォルト設定ファイル\file{upldefs.ltx}は、もともと\file{uplcore.ltx}の途中で
% 読み込んでいましたが、2018年以降の新しいコミュニティ版\upLaTeX{}では
% \file{uplatex.ltx}から読み込むことにしました。
% 実際の中身については、第\ref{plfonts:pldefs}節を参照してください。
% \changes{v1.6k-u00}{2017/12/05}{デフォルト設定ファイルの読み込みを
%    \file{uplcore.ltx}から\file{uplatex.ltx}へ移動
%     (based on plfonts.dtx 2017/12/05 v1.6k)}
%
%
% \section{デフォルト設定ファイル}\label{plfonts:pldefs}
% ここでは、フォーマットファイルに読み込まれるデフォルト値を設定しています。
% この節での内容は\file{upldefs.ltx}に出力されます。
% このファイルの内容を\file{uplcore.ltx}に含めてもよいのですが、
% デフォルトの設定を参照しやすいように、別ファイルにしてあります。
%
% プリロードサイズは、\dst{}プログラムのオプションで変更することができます。
% これ以外の設定を変更したい場合は、\file{upldefs.ltx}を
% 直接、修正するのではなく、このファイルを\file{upldefs.cfg}という名前で
% コピーをして、そのファイルに対して修正を加えるようにしてください。
%    \begin{macrocode}
%<*pldefs>
\ProvidesFile{upldefs.ltx}
      [2020/02/01 v1.6v-u02 upLaTeX Kernel (Default settings)]
%</pldefs>
%    \end{macrocode}
%
% \subsection{テキストフォント}
% テキストフォントのための属性やエラー書体などの宣言です。
% p\LaTeX{}のデフォルトの横組エンコードはJY1、縦組エンコードはJT1ですが、
% up\LaTeX{}では横組エンコードはJY2、縦組エンコードはJT2とします。
%
% \changes{v1.6s}{2019/08/13}{Explicitly set some defaults
%    after \cs{DeclareErrorKanjiFont} change
%    (sync with ltfssini.dtx 2019/07/09 v3.1c)}
% \noindent
% 縦横エンコード共通:
%    \begin{macrocode}
%<*pldefs>
\DeclareKanjiEncodingDefaults{}{}
\DeclareErrorKanjiFont{JY2}{mc}{m}{n}{10}
\kanjifamily{mc}
\kanjiseries{m}
\kanjishape{n}
\fontsize{10}{10}
%    \end{macrocode}
% 横組エンコード:
%    \begin{macrocode}
\DeclareYokoKanjiEncoding{JY2}{}{}
\DeclareKanjiSubstitution{JY2}{mc}{m}{n}
%    \end{macrocode}
% 縦組エンコード:
%    \begin{macrocode}
\DeclareTateKanjiEncoding{JT2}{}{}
\DeclareKanjiSubstitution{JT2}{mc}{m}{n}
%    \end{macrocode}
% 縦横のエンコーディングのセット化:
% \changes{v1.6j}{2017/11/06}{縦横のエンコーディングのセット化を
%    plcoreからpldefsへ移動}
%    \begin{macrocode}
\KanjiEncodingPair{JY2}{JT2}
%    \end{macrocode}
% フォント属性のデフォルト値:
% \LaTeXe~2019-10-01までは|\shapedefault|は|\updefault|でしたが、
% \LaTeXe~2020-02-02で|\updefault|が``n''から``up''へと修正されたことに
% 伴い、|\shapedefault|は明示的に``n''に設定されました。
% \changes{v1.6v}{2020/02/01}{Set \cs{kanjishapedefault} explicitly to ``n''
%    (sync with fontdef.dtx 2019/12/17 v3.0e)}
%    \begin{macrocode}
\newcommand\mcdefault{mc}
\newcommand\gtdefault{gt}
\newcommand\kanjiencodingdefault{JY2}
\newcommand\kanjifamilydefault{\mcdefault}
\newcommand\kanjiseriesdefault{\mddefault}
\newcommand\kanjishapedefault{n}% formerly \updefault
%    \end{macrocode}
% 和文エンコードの指定:
%    \begin{macrocode}
\kanjiencoding{JY2}
%    \end{macrocode}
% フォント定義:
% これらの具体的な内容は第\ref{plfonts:fontdef}節を参照してください。
% \changes{v1.3}{1997/01/24}{Rename font definition filename.}
%    \begin{macrocode}
%%
%% This is file `jy2mc.fd',
%% generated with the docstrip utility.
%%
%% The original source files were:
%%
%% uplfonts.dtx  (with options: `JY2mc')
%% 
%% Copyright (c) 2010 ASCII MEDIA WORKS
%% Copyright (c) 2016 Takuji Tanaka
%% Copyright (c) 2016-2018 Japanese TeX Development Community
%% 
%% This file is part of the upLaTeX2e system (community edition).
%% --------------------------------------------------------------
%% 
%% File: uplfonts.dtx
\ProvidesFile{jy2mc.fd}
       [2018/07/03 v1.6q-u02 KANJI font defines]
\DeclareKanjiFamily{JY2}{mc}{}
\DeclareRelationFont{JY2}{mc}{m}{}{T1}{cmr}{m}{}
\DeclareRelationFont{JY2}{mc}{bx}{}{T1}{cmr}{bx}{}
\DeclareFontShape{JY2}{mc}{m}{n}{<->s*[0.962216]upjisr-h}{}
\DeclareFontShape{JY2}{mc}{bx}{n}{<->ssub*gt/m/n}{}
\DeclareFontShape{JY2}{mc}{b}{n}{<->ssub*mc/bx/n}{}
\endinput
%%
%% End of file `jy2mc.fd'.

%%
%% This is file `jy2gt.fd',
%% generated with the docstrip utility.
%%
%% The original source files were:
%%
%% uplfonts.dtx  (with options: `JY2gt')
%% 
%% Copyright (c) 2010 ASCII MEDIA WORKS
%% Copyright (c) 2016 Takuji Tanaka
%% Copyright (c) 2016-2018 Japanese TeX Development Community
%% 
%% This file is part of the upLaTeX2e system (community edition).
%% --------------------------------------------------------------
%% 
%% File: uplfonts.dtx
\ProvidesFile{jy2gt.fd}
       [2018/07/03 v1.6q-u02 KANJI font defines]
\DeclareKanjiFamily{JY2}{gt}{}
\DeclareRelationFont{JY2}{gt}{m}{}{T1}{cmr}{bx}{}
\DeclareFontShape{JY2}{gt}{m}{n}{<->s*[0.962216]upjisg-h}{}
\DeclareFontShape{JY2}{gt}{bx}{n}{<->ssub*gt/m/n}{}
\DeclareFontShape{JY2}{gt}{b}{n}{<->ssub*gt/bx/n}{}
\endinput
%%
%% End of file `jy2gt.fd'.

%%
%% This is file `jt2mc.fd',
%% generated with the docstrip utility.
%%
%% The original source files were:
%%
%% uplfonts.dtx  (with options: `JT2mc')
%% 
%% Copyright (c) 2010 ASCII MEDIA WORKS
%% Copyright (c) 2016 Takuji Tanaka
%% Copyright (c) 2016-2018 Japanese TeX Development Community
%% 
%% This file is part of the upLaTeX2e system (community edition).
%% --------------------------------------------------------------
%% 
%% File: uplfonts.dtx
\ProvidesFile{jt2mc.fd}
       [2018/07/03 v1.6q-u02 KANJI font defines]
\DeclareKanjiFamily{JT2}{mc}{}
\DeclareRelationFont{JT2}{mc}{m}{}{T1}{cmr}{m}{}
\DeclareRelationFont{JT2}{mc}{bx}{}{T1}{cmr}{bx}{}
\DeclareFontShape{JT2}{mc}{m}{n}{<->s*[0.962216]upjisr-v}{}
\DeclareFontShape{JT2}{mc}{bx}{n}{<->ssub*gt/m/n}{}
\DeclareFontShape{JT2}{mc}{b}{n}{<->ssub*mc/bx/n}{}
\endinput
%%
%% End of file `jt2mc.fd'.

%%
%% This is file `jt2gt.fd',
%% generated with the docstrip utility.
%%
%% The original source files were:
%%
%% uplfonts.dtx  (with options: `JT2gt')
%% 
%% Copyright (c) 2010 ASCII MEDIA WORKS
%% Copyright (c) 2016 Takuji Tanaka
%% Copyright (c) 2016 Japanese TeX Development Community
%% 
%% This file is part of the upLaTeX2e system (community edition).
%% --------------------------------------------------------------
%% 
%% File: uplfonts.dtx
\ProvidesFile{jt2gt.fd}
       [1997/01/24 v1.3 KANJI font defines]
\DeclareKanjiFamily{JT2}{gt}{}
\DeclareRelationFont{JT2}{gt}{m}{}{T1}{cmr}{bx}{}
\DeclareFontShape{JT2}{gt}{m}{n}{<->s*[0.962216]upjisg-v}{}
\DeclareFontShape{JT2}{gt}{bx}{n}{<->ssub*gt/m/n}{}
\endinput
%%
%% End of file `jt2gt.fd'.

%    \end{macrocode}
% フォントを有効にします。
%    \begin{macrocode}
\fontencoding{JT2}\selectfont
\fontencoding{JY2}\selectfont
%    \end{macrocode}
%
% \changes{v1.3b}{1997/01/30}{数式用フォントの宣言をクラスファイルに移動した}
%
%
% \subsection{プリロードフォント}
% あらかじめフォーマットファイルにロードされるフォントの宣言です。
% \dst{}プログラムのオプションでロードされるフォントのサイズを
% 変更することができます。\file{uplfmt.ins}では|xpt|を指定しています。
%    \begin{macrocode}
%<*xpt>
\DeclarePreloadSizes{JY2}{mc}{m}{n}{5,7,10,12}
\DeclarePreloadSizes{JY2}{gt}{m}{n}{5,7,10,12}
\DeclarePreloadSizes{JT2}{mc}{m}{n}{5,7,10,12}
\DeclarePreloadSizes{JT2}{gt}{m}{n}{5,7,10,12}
%</xpt>
%<*xipt>
\DeclarePreloadSizes{JY2}{mc}{m}{n}{5,7,10.95,12}
\DeclarePreloadSizes{JY2}{gt}{m}{n}{5,7,10.95,12}
\DeclarePreloadSizes{JT2}{mc}{m}{n}{5,7,10.95,12}
\DeclarePreloadSizes{JT2}{gt}{m}{n}{5,7,10.95,12}
%</xipt>
%<*xiipt>
\DeclarePreloadSizes{JY2}{mc}{m}{n}{7,9,12,14.4}
\DeclarePreloadSizes{JY2}{gt}{m}{n}{7,9,12,14.4}
\DeclarePreloadSizes{JT2}{mc}{m}{n}{7,9,12,14.4}
\DeclarePreloadSizes{JT2}{gt}{m}{n}{7,9,12,14.4}
%</xiipt>
%<*ori>
\DeclarePreloadSizes{JY2}{mc}{m}{n}
        {5,6,7,8,9,10,10.95,12,14.4,17.28,20.74,24.88}
\DeclarePreloadSizes{JY2}{gt}{m}{n}
        {5,6,7,8,9,10,10.95,12,14.4,17.28,20.74,24.88}
\DeclarePreloadSizes{JT2}{mc}{m}{n}
        {5,6,7,8,9,10,10.95,12,14.4,17.28,20.74,24.88}
\DeclarePreloadSizes{JT2}{gt}{m}{n}
        {5,6,7,8,9,10,10.95,12,14.4,17.28,20.74,24.88}
%</ori>
%    \end{macrocode}
%
%
% \subsection{組版パラメータ}
% 禁則パラメータや文字間へ挿入するスペースの設定などです。
% 実際の各文字への禁則パラメータおよびスペースの挿入の許可設定などは、
% \file{ukinsoku.tex}で行なっています。
% 具体的な設定については、\file{ukinsoku.dtx}を参照してください。
%    \begin{macrocode}
\InputIfFileExists{ukinsoku.tex}%
  {\message{Loading kinsoku patterns for japanese.}}
  {\errhelp{The configuration for kinsoku is incorrectly installed.^^J%
            If you don't understand this error message you need
            to seek^^Jexpert advice.}%
   \errmessage{OOPS! I can't find any kinsoku patterns for japanese^^J%
               \space Think of getting some or the
               uplatex2e setup will never succeed}\@@end}
%    \end{macrocode}
%
% 組版パラメータの設定をします。
% |\kanjiskip|は、漢字と漢字の間に挿入されるグルーです。
% |\noautospacing|で、挿入を中止することができます。
% デフォルトは|\autospacing|です。
%    \begin{macrocode}
\kanjiskip=0pt plus .4pt minus .5pt
\autospacing
%    \end{macrocode}
% |\xkanjiskip|は、和欧文間に自動的に挿入されるグルーです。
% |\noautoxspacing|で、挿入を中止することができます。
% デフォルトは|\autoxspacing|です。
% \changes{v1.1c}{1995/09/12}{\cs{xkanjiskip}のデフォルト値}
%    \begin{macrocode}
\xkanjiskip=.25zw plus1pt minus1pt
\autoxspacing
%    \end{macrocode}
% |\jcharwidowpenalty|は、パラグラフに対する禁則です。
% パラグラフの最後の行が1文字だけにならないように調整するために使われます。
%    \begin{macrocode}
\jcharwidowpenalty=500
%    \end{macrocode}
%
% ここまでが、\file{pldefs.ltx}の内容です。
%    \begin{macrocode}
%</pldefs>
%    \end{macrocode}
%
%
%
% \section{フォント定義ファイル}\label{plfonts:fontdef}
% \changes{v1.3}{1997/01/24}{Rename provided font definition filename.}
% ここでは、フォント定義ファイルの設定をしています。フォント定義ファイルは、
% \LaTeX{}のフォント属性を\TeX{}フォントに置き換えるためのファイルです。
% 記述方法についての詳細は、|fntguide.tex|を参照してください。
%
% 欧文書体の設定については、
% \file{cmfonts.fdd}や\file{slides.fdd}などを参照してください。
% \file{skfonts.fdd}には、写研代用書体を使うためのパッケージと
% フォント定義が記述されています。
%    \begin{macrocode}
%<JY2mc>\ProvidesFile{jy2mc.fd}
%<JY2gt>\ProvidesFile{jy2gt.fd}
%<JT2mc>\ProvidesFile{jt2mc.fd}
%<JT2gt>\ProvidesFile{jt2gt.fd}
%<JY2mc,JY2gt,JT2mc,JT2gt>       [2018/07/03 v1.6q-u02 KANJI font defines]
%    \end{macrocode}
% 横組用、縦組用ともに、
% 明朝体のシリーズ|bx|がゴシック体となるように宣言しています。
% \changes{v1.2}{1995/11/24}{it, sl, scの宣言を外した}
% \changes{v1.3b}{1997/01/29}{フォント定義ファイルのサイズ指定の調整}
% \changes{v1.3b}{1997/03/11}{すべてのサイズをロード可能にした}
% また、シリーズ|b|は同じ書体の|bx|と等価になるように宣言します。
% \changes{v1.6q}{2018/07/03}{シリーズbがbxと等価になるように宣言}
%
% p\LaTeX{}では従属書体にOT1エンコーディングを指定していましたが、
% up\LaTeX{}ではT1エンコーディングを用いるように変更しました。
% また、要求サイズ(指定されたフォントサイズ)が10ptのとき、
% 全角幅の実寸が9.62216ptとなるようにしますので、
% 和文スケール値($1\,\mathrm{zw} \div \textmc{要求サイズ}$)は
% $9.62216\,\mathrm{pt}/10\,\mathrm{pt}=0.962216$です。
% upjis系のメトリックは全角幅が10ptでデザインされているので、
% これを0.962216倍で読込みます。
% \changes{v1.6l}{2018/02/04}{和文スケール値を明文化}
%    \begin{macrocode}
%<*JY2mc>
\DeclareKanjiFamily{JY2}{mc}{}
\DeclareRelationFont{JY2}{mc}{m}{}{T1}{cmr}{m}{}
\DeclareRelationFont{JY2}{mc}{bx}{}{T1}{cmr}{bx}{}
\DeclareFontShape{JY2}{mc}{m}{n}{<->s*[0.962216]upjisr-h}{}
\DeclareFontShape{JY2}{mc}{bx}{n}{<->ssub*gt/m/n}{}
\DeclareFontShape{JY2}{mc}{b}{n}{<->ssub*mc/bx/n}{}
%</JY2mc>
%<*JT2mc>
\DeclareKanjiFamily{JT2}{mc}{}
\DeclareRelationFont{JT2}{mc}{m}{}{T1}{cmr}{m}{}
\DeclareRelationFont{JT2}{mc}{bx}{}{T1}{cmr}{bx}{}
\DeclareFontShape{JT2}{mc}{m}{n}{<->s*[0.962216]upjisr-v}{}
\DeclareFontShape{JT2}{mc}{bx}{n}{<->ssub*gt/m/n}{}
\DeclareFontShape{JT2}{mc}{b}{n}{<->ssub*mc/bx/n}{}
%</JT2mc>
%<*JY2gt>
\DeclareKanjiFamily{JY2}{gt}{}
\DeclareRelationFont{JY2}{gt}{m}{}{T1}{cmr}{bx}{}
\DeclareFontShape{JY2}{gt}{m}{n}{<->s*[0.962216]upjisg-h}{}
\DeclareFontShape{JY2}{gt}{bx}{n}{<->ssub*gt/m/n}{}
\DeclareFontShape{JY2}{gt}{b}{n}{<->ssub*gt/bx/n}{}
%</JY2gt>
%<*JT2gt>
\DeclareKanjiFamily{JT2}{gt}{}
\DeclareRelationFont{JT2}{gt}{m}{}{T1}{cmr}{bx}{}
\DeclareFontShape{JT2}{gt}{m}{n}{<->s*[0.962216]upjisg-v}{}
\DeclareFontShape{JT2}{gt}{bx}{n}{<->ssub*gt/m/n}{}
\DeclareFontShape{JT2}{gt}{b}{n}{<->ssub*gt/bx/n}{}
%</JT2gt>
%    \end{macrocode}
%
%
% \Finale
%
\endinput

   % \iffalse meta-comment
%% File: ukinsoku.dtx
%
%    pLaTeX kinsoku file:
%       Copyright 1995 ASCII Corporation.
%    and modified for upLaTeX
%
%  Copyright (c) 2010 ASCII MEDIA WORKS
%  Copyright (c) 2016 Takuji Tanaka
%  Copyright (c) 2016-2021 Japanese TeX Development Community
%
%  This file is part of the upLaTeX2e system (community edition).
%  --------------------------------------------------------------
%
% \fi
%
%
% \setcounter{StandardModuleDepth}{1}
% \StopEventually{}
%
% \iffalse
% \changes{v1.0-u00}{2011/05/07}{p\LaTeX{}用からup\LaTeX{}用に修正。}
% \changes{v1.0-u01}{2017/08/02}{U+00B7 (MIDDLE DOT; JIS X 0213)の
%    前禁則ペナルティをU+30FBと同じ値に設定、注意点を明文化}
% \changes{v1.0b}{2017/08/05}{%、&、\%、\&の禁則ペナルティが
%      誤っていたのを修正(post $\rightarrow$ pre)}
% \changes{v1.0b-u01}{2017/08/05}{p\LaTeX{}の変更に追随}
% \changes{v1.0b-u02}{2018/01/27}{up\TeX{}の将来の変更に備え、
%      Latin-1 Supplementのうち属性がLatinのもの
%      (Latin-1 letters)をコードポイントで指定}
% \changes{v1.0b-u03}{2018/04/08}{\LaTeX\ 2018-04-01対策}
% \changes{v1.0b-u04}{2019/01/29}{内部Unicode化されていることを確認}
% \changes{v1.0b-u05}{2019/05/19}{up\TeX~v1.24の\cs{kcatcode}の既定値のバグ回避}
% \changes{v1.0b-u06}{2019/09/22}{バグ回避コードがかえって有害なため除去}
% \changes{v1.0c}{2020/09/28}{!の\cs{inhibitxspcode}を設定}
% \changes{v1.0c-u06}{2020/09/28}{p\LaTeX{}の変更に追随}
% \changes{v1.0d}{2021/03/04}{:の\cs{inhibitxspcode}と:の\cs{xspcode}を設定}
% \changes{v1.0d-u06}{2021/03/04}{p\LaTeX{}の変更に追随}
% \fi
%
% \iffalse
%<*driver>
\NeedsTeXFormat{pLaTeX2e}
% \fi
\ProvidesFile{ukinsoku.dtx}[2021/03/04 v1.0d-u06 upLaTeX Kernel]
% \iffalse
\documentclass{jltxdoc}
\GetFileInfo{ukinsoku.dtx}
\title{禁則パラメータ\space\fileversion}
\author{Ken Nakano \& TTK}
\date{作成日:\filedate}
\begin{document}
   \maketitle
   \DocInput{\filename}
\end{document}
%</driver>
% \fi
%
% このファイルは、禁則と文字間スペースの設定について説明をしています。
% 日本語\TeX{}の機能についての詳細は、『日本語\TeX テクニカルブックI』を
% 参照してください。
%
% なお、このファイルのコード部分は、
% p\TeX{}やp\LaTeX{}で配布されている\file{kinsoku.tex}に、
% JIS X 0213の定義文字などの設定を追加したものです。
% このファイルは内部コードUnicode (|uptex|)なup\TeX{}エンジンで読まれる
% 必要があります。
% \changes{v1.0-u00}{2011/05/07}{p\LaTeX{}用からup\LaTeX{}用に修正。}
% \changes{v1.0b-u04}{2019/01/29}{内部コードがUnicodeであることを確認}
%
%    \begin{macrocode}
%<*plcore>
\ifnum\ucs"3000="3000 \else
    \errhelp{Please try to run (e)uptex with option
             `-kanji-internal=uptex'.}%
    \errmessage{This file should be read with
                internal Kanji encoding Unicode}\@@end
\fi
%    \end{macrocode}
%
% \changes{v1.0b-u05}{2019/05/19}{up\TeX~v1.24の\cs{kcatcode}の既定値のバグ回避}
% \changes{v1.0b-u06}{2019/09/22}{バグ回避コードがかえって有害なため除去}
%
% \section{禁則}
%
% ある文字を行頭禁則の対象にするには、|\prebreakpenalty|に正の値を指定します。
% ある文字を行末禁則の対象にするには、|\postbreakpenalty|に正の値を指定します。
% 数値が大きいほど、行頭、あるいは行末で改行されにくくなります。
%
% \subsection{半角文字に対する禁則}
% ここでは、半角文字に対する禁則の設定を行なっています。
% \changes{v1.0b}{2017/08/05}{%、&、\%、\&の禁則ペナルティが
%      誤っていたのを修正(post $\rightarrow$ pre)}
%
%    \begin{macrocode}
%%
%% 行頭、行末禁則パラメータ
%%
%% 1byte characters
\prebreakpenalty`!=10000
\prebreakpenalty`"=10000
\postbreakpenalty`\#=500
\postbreakpenalty`\$=500
\prebreakpenalty`\%=500
\prebreakpenalty`\&=500
\postbreakpenalty`\`=10000
\prebreakpenalty`'=10000
\prebreakpenalty`)=10000
\postbreakpenalty`(=10000
\prebreakpenalty`*=500
\prebreakpenalty`+=500
\prebreakpenalty`-=10000
\prebreakpenalty`.=10000
\prebreakpenalty`,=10000
\prebreakpenalty`/=500
\prebreakpenalty`;=10000
\prebreakpenalty`?=10000
\prebreakpenalty`:=10000
\prebreakpenalty`]=10000
\postbreakpenalty`[=10000
%    \end{macrocode}
%
% \subsection{全角文字に対する禁則}
% ここでは、全角文字に対する禁則の設定を行なっています。
%
% up\TeX{}/up\LaTeX{}の場合、JIS X 0213(日本)・KS C 5601(韓国)・
% GB2312(中国)・Big5(台湾)などの文字集合に含まれる、
% いわゆる全角文字の一部が、8-bit Latinと同じコードポイントを
% 共有します。すなわち、同じコードポイントが、CJKトークンとしても
% non-CJKトークンとしても有効に扱われることがあります。
% 以下に例を示します\footnote{ここで表示しているnon-CJKトークンと
% して扱われた結果は、up\LaTeX{}のデフォルト従属欧文エンコーディング
% であるT1の場合のものです。}。
% {\font\lmr=rm-lmr10\lmr
% \begin{itemize}
% \item \texttt{0xA1}: \kchar"A1 (CJK) vs. \char"A1\ (non-CJK)
% \item \texttt{0xAB}: \kchar"AB (CJK) vs. \char"AB\ (non-CJK)
% \item \texttt{0xB7}: \kchar"B7 (CJK) vs. \char"B7\ (non-CJK)
% \item \texttt{0xB9}: \kchar"B9 (CJK) vs. \char"B9\ (non-CJK)
% \item …
% \end{itemize}}
% \file{ukinsoku.tex}ではCJKトークンを優先した禁則設定を行っています。
% この設定により、同じコードポイントをnon-CJKトークンとして扱う場合に
% 予期せずLatin-1の文字が禁則対象になってしまいます。
% 問題が起きた場合は禁則の設定を調整してください。
% \changes{v1.0-u01}{2017/08/02}{U+00B7 (MIDDLE DOT; JIS X 0213)の
%    前禁則ペナルティをU+30FBと同じ値に設定、注意点を明文化}
% \changes{v1.0b-u02}{2018/01/27}{up\TeX{}の将来の変更に備え、
%      Latin-1 Supplementのうち属性がLatinのもの
%      (Latin-1 letters)をコードポイントで指定}
%
% なお、以下で複数回登場する |"AA| と |"BA| はそれぞれªとºですが、
% \LaTeXe\ 2018-04-01でUTF-8入力になった影響で、これらの文字は
% |macrocode| 環境内のコードに(たとえ |%| に続くコメントであっても)
% 書けなくなってしまったようです。これらの文字で
% docstrip処理中にエラー
%\begin{verbatim}
%   ! Argument of \@font@info has an extra }.
%\end{verbatim}
% が出ないように、コメントからも削除しました。
% \changes{v1.0b-u03}{2018/04/08}{\LaTeX\ 2018-04-01対策}
%
% \changes{v1.0d}{2021/03/04}{:の\cs{xspcode}を設定}
%    \begin{macrocode}
%%全角文字
\prebreakpenalty`、=10000
\prebreakpenalty`。=10000
\prebreakpenalty`,=10000
\prebreakpenalty`.=10000
\prebreakpenalty`・=10000
\prebreakpenalty`:=10000
\prebreakpenalty`;=10000
\prebreakpenalty`?=10000
\prebreakpenalty`!=10000
\prebreakpenalty`゛=10000%\jis"212B
\prebreakpenalty`゜=10000%\jis"212C
\prebreakpenalty`´=10000%\jis"212D
\postbreakpenalty``=10000%\jis"212E
\prebreakpenalty`々=10000%\jis"2139
\prebreakpenalty`…=250%\jis"2144
\prebreakpenalty`‥=250%\jis"2145
\postbreakpenalty`‘=10000%\jis"2146
\prebreakpenalty`’=10000%\jis"2147
\postbreakpenalty`“=10000%\jis"2148
\prebreakpenalty`”=10000%\jis"2149
\prebreakpenalty`)=10000
\postbreakpenalty`(=10000
\prebreakpenalty`}=10000
\postbreakpenalty`{=10000
\prebreakpenalty`]=10000
\postbreakpenalty`[=10000
%%\postbreakpenalty`‘=10000
%%\prebreakpenalty`’=10000
\postbreakpenalty`〔=10000%\jis"214C
\prebreakpenalty`〕=10000%\jis"214D
\postbreakpenalty`〈=10000%\jis"2152
\prebreakpenalty`〉=10000%\jis"2153
\postbreakpenalty`《=10000%\jis"2154
\prebreakpenalty`》=10000%\jis"2155
\postbreakpenalty`「=10000%\jis"2156
\prebreakpenalty`」=10000%\jis"2157
\postbreakpenalty`『=10000%\jis"2158
\prebreakpenalty`』=10000%\jis"2159
\postbreakpenalty`【=10000%\jis"215A
\prebreakpenalty`】=10000%\jis"215B
\prebreakpenalty`ー=10000
\prebreakpenalty`+=200
\prebreakpenalty`−=200% U+2212 MINUS SIGN
\prebreakpenalty`-=200% U+FF0D FULLWIDTH HYPHEN-MINUS
\prebreakpenalty`==200
\postbreakpenalty`#=200
\postbreakpenalty`$=200
\prebreakpenalty`%=200
\prebreakpenalty`&=200
\prebreakpenalty`ぁ=150
\prebreakpenalty`ぃ=150
\prebreakpenalty`ぅ=150
\prebreakpenalty`ぇ=150
\prebreakpenalty`ぉ=150
\prebreakpenalty`っ=150
\prebreakpenalty`ゃ=150
\prebreakpenalty`ゅ=150
\prebreakpenalty`ょ=150
\prebreakpenalty`ゎ=150%\jis"246E
\prebreakpenalty`ァ=150
\prebreakpenalty`ィ=150
\prebreakpenalty`ゥ=150
\prebreakpenalty`ェ=150
\prebreakpenalty`ォ=150
\prebreakpenalty`ッ=150
\prebreakpenalty`ャ=150
\prebreakpenalty`ュ=150
\prebreakpenalty`ョ=150
\prebreakpenalty`ヮ=150%\jis"256E
\prebreakpenalty`ヵ=150%\jis"2575
\prebreakpenalty`ヶ=150%\jis"2576
%% kinsoku  JIS X 0208 additional
\prebreakpenalty`ヽ=10000
\prebreakpenalty`ヾ=10000
\prebreakpenalty`ゝ=10000
\prebreakpenalty`ゞ=10000
%%
%% kinsoku  JIS X 0213
%%
\prebreakpenalty`〳=10000
\prebreakpenalty`〴=10000
\prebreakpenalty`〵=10000
\prebreakpenalty`〻=10000
\postbreakpenalty`⦅=10000
\prebreakpenalty`⦆=10000
\postbreakpenalty`⦅=10000
\prebreakpenalty`⦆=10000
\postbreakpenalty`〘=10000
\prebreakpenalty`〙=10000
\postbreakpenalty`〖=10000
\prebreakpenalty`〗=10000
\postbreakpenalty`«=10000
\prebreakpenalty`»=10000
\postbreakpenalty`〝=10000
\prebreakpenalty`〟=10000
\prebreakpenalty`‼=10000
\prebreakpenalty`⁇=10000
\prebreakpenalty`⁈=10000
\prebreakpenalty`⁉=10000
\postbreakpenalty`¡=10000
\postbreakpenalty`¿=10000
\prebreakpenalty`ː=10000
\prebreakpenalty`·=10000
\prebreakpenalty"AA=10000
\prebreakpenalty"BA=10000
\prebreakpenalty`¹=10000
\prebreakpenalty`²=10000
\prebreakpenalty`³=10000
\postbreakpenalty`€=10000
\prebreakpenalty`ゕ=150
\prebreakpenalty`ゖ=150
\prebreakpenalty`ㇰ=150
\prebreakpenalty`ㇱ=150
\prebreakpenalty`ㇲ=150
\prebreakpenalty`ㇳ=150
\prebreakpenalty`ㇴ=150
\prebreakpenalty`ㇵ=150
\prebreakpenalty`ㇶ=150
\prebreakpenalty`ㇷ=150
\prebreakpenalty`ㇸ=150
\prebreakpenalty`ㇹ=150
%%\prebreakpenalty`ㇷ゚=150
\prebreakpenalty`ㇺ=150
\prebreakpenalty`ㇻ=150
\prebreakpenalty`ㇼ=150
\prebreakpenalty`ㇽ=150
\prebreakpenalty`ㇾ=150
\prebreakpenalty`ㇿ=150
%%
%% kinsoku  JIS X 0212
%%
%%\postbreakpenalty`¡=10000
%%\postbreakpenalty`¿=10000
%%\prebreakpenalty"BA=10000
%%\prebreakpenalty"AA=10000
\prebreakpenalty`™=10000
%%
%% kinsoku  半角片仮名
%%
\prebreakpenalty`。=10000
\prebreakpenalty`、=10000
\prebreakpenalty`゙=10000
\prebreakpenalty`゚=10000
\prebreakpenalty`」=10000
\postbreakpenalty`「=10000
%    \end{macrocode}
%
% \section{文字間のスペース}
%
% ある英字の前後と、その文字に隣合う漢字に挿入されるスペースを制御するには、
% |\xspcode|を用います。
%
% ある漢字の前後と、その文字に隣合う英字に挿入されるスペースを制御するには、
% |\inhibitxspcode|を用います。
%
% \subsection{ある英字と前後の漢字の間の制御}
% ここでは、英字に対する設定を行なっています。
%
% 指定する数値とその意味は次のとおりです。
%
% \begin{center}
% \begin{tabular}{ll}
% 0 & 前後の漢字の間での処理を禁止する。\\
% 1 & 直前の漢字との間にのみ、スペースの挿入を許可する。\\
% 2 & 直後の漢字との間にのみ、スペースの挿入を許可する。\\
% 3 & 前後の漢字との間でのスペースの挿入を許可する。\\
% \end{tabular}
% \end{center}
%
%    \begin{macrocode}
%%
%% xspcode
\xspcode`(=1
\xspcode`)=2
\xspcode`[=1
\xspcode`]=2
\xspcode``=1
\xspcode`'=2
\xspcode`:=2
\xspcode`;=2
\xspcode`,=2
\xspcode`.=2
%%  for 8bit Latin
\xspcode"80=3
\xspcode"81=3
\xspcode"82=3
\xspcode"83=3
\xspcode"84=3
\xspcode"85=3
\xspcode"86=3
\xspcode"87=3
\xspcode"88=3
\xspcode"89=3
\xspcode"8A=3
\xspcode"8B=3
\xspcode"8C=3
\xspcode"8D=3
\xspcode"8E=3
\xspcode"8F=3
\xspcode"90=3
\xspcode"91=3
\xspcode"92=3
\xspcode"93=3
\xspcode"94=3
\xspcode"95=3
\xspcode"96=3
\xspcode"97=3
\xspcode"98=3
\xspcode"99=3
\xspcode"9A=3
\xspcode"9B=3
\xspcode"9C=3
\xspcode"9D=3
\xspcode"9E=3
\xspcode"9F=3
\xspcode"A0=3
\xspcode"A1=3
\xspcode"A2=3
\xspcode"A3=3
\xspcode"A4=3
\xspcode"A5=3
\xspcode"A6=3
\xspcode"A7=3
\xspcode"A8=3
\xspcode"A9=3
\xspcode"AA=3
\xspcode"AB=3
\xspcode"AC=3
\xspcode"AD=3
\xspcode"AE=3
\xspcode"AF=3
\xspcode"B0=3
\xspcode"B1=3
\xspcode"B2=3
\xspcode"B3=3
\xspcode"B4=3
\xspcode"B5=3
\xspcode"B6=3
\xspcode"B7=3
\xspcode"B8=3
\xspcode"B9=3
\xspcode"BA=3
\xspcode"BB=3
\xspcode"BC=3
\xspcode"BD=3
\xspcode"BE=3
\xspcode"BF=3
\xspcode"C0=3
\xspcode"C1=3
\xspcode"C2=3
\xspcode"C3=3
\xspcode"C4=3
\xspcode"C5=3
\xspcode"C6=3
\xspcode"C7=3
\xspcode"C8=3
\xspcode"C9=3
\xspcode"CA=3
\xspcode"CB=3
\xspcode"CC=3
\xspcode"CD=3
\xspcode"CE=3
\xspcode"CF=3
\xspcode"D0=3
\xspcode"D1=3
\xspcode"D2=3
\xspcode"D3=3
\xspcode"D4=3
\xspcode"D5=3
\xspcode"D6=3
\xspcode"D7=3
\xspcode"D8=3
\xspcode"D9=3
\xspcode"DA=3
\xspcode"DB=3
\xspcode"DC=3
\xspcode"DD=3
\xspcode"DE=3
\xspcode"DF=3
\xspcode"E0=3
\xspcode"E1=3
\xspcode"E2=3
\xspcode"E3=3
\xspcode"E4=3
\xspcode"E5=3
\xspcode"E6=3
\xspcode"E7=3
\xspcode"E8=3
\xspcode"E9=3
\xspcode"EA=3
\xspcode"EB=3
\xspcode"EC=3
\xspcode"ED=3
\xspcode"EE=3
\xspcode"EF=3
\xspcode"F0=3
\xspcode"F1=3
\xspcode"F2=3
\xspcode"F3=3
\xspcode"F4=3
\xspcode"F5=3
\xspcode"F6=3
\xspcode"F7=3
\xspcode"F8=3
\xspcode"F9=3
\xspcode"FA=3
\xspcode"FB=3
\xspcode"FC=3
\xspcode"FD=3
\xspcode"FE=3
\xspcode"FF=3
%    \end{macrocode}
%
% \subsection{ある漢字と前後の英字の間の制御}
% ここでは、漢字に対する設定を行なっています。
%
% 指定する数値とその意味は次のとおりです。
%
% \begin{center}
% \begin{tabular}{ll}
% 0 & 前後の英字との間にスペースを挿入することを禁止する。\\
% 1 & 直前の英字との間にスペースを挿入することを禁止する。\\
% 2 & 直後の英字との間にスペースを挿入することを禁止する。\\
% 3 & 前後の英字との間でのスペースの挿入を許可する。\\
% \end{tabular}
% \end{center}
%
% \changes{v1.0c}{2020/09/28}{!の\cs{inhibitxspcode}を設定}
% \changes{v1.0d}{2021/03/04}{:の\cs{inhibitxspcode}を設定}
%    \begin{macrocode}
%%
%% inhibitxspcode
\inhibitxspcode`、=1
\inhibitxspcode`。=1
\inhibitxspcode`,=1
\inhibitxspcode`.=1
\inhibitxspcode`:=1
\inhibitxspcode`;=1
\inhibitxspcode`?=1
\inhibitxspcode`!=1
\inhibitxspcode`)=1
\inhibitxspcode`(=2
\inhibitxspcode`]=1
\inhibitxspcode`[=2
\inhibitxspcode`}=1
\inhibitxspcode`{=2
\inhibitxspcode`‘=2
\inhibitxspcode`’=1
\inhibitxspcode`“=2
\inhibitxspcode`”=1
\inhibitxspcode`〔=2
\inhibitxspcode`〕=1
\inhibitxspcode`〈=2
\inhibitxspcode`〉=1
\inhibitxspcode`《=2
\inhibitxspcode`》=1
\inhibitxspcode`「=2
\inhibitxspcode`」=1
\inhibitxspcode`『=2
\inhibitxspcode`』=1
\inhibitxspcode`【=2
\inhibitxspcode`】=1
\inhibitxspcode`—=0% U+2014 EM DASH
\inhibitxspcode`―=0% U+2015 HORIZONTAL BAR
\inhibitxspcode`〜=0% U+301C WAVE DASH
\inhibitxspcode`~=0% U+FF5E FULLWIDTH TILDE
\inhibitxspcode`…=0
\inhibitxspcode`¥=0% U+00A5 YEN SIGN
\inhibitxspcode`¥=0% U+FFE5 FULLWIDTH YEN SIGN
\inhibitxspcode`°=1
\inhibitxspcode`′=1
\inhibitxspcode`″=1
%%
%% inhibitxspcode  JIS X 0213
%%
\inhibitxspcode`⦅=2
\inhibitxspcode`⦆=1
\inhibitxspcode`⦅=2
\inhibitxspcode`⦆=1
\inhibitxspcode`〘=2
\inhibitxspcode`〙=1
\inhibitxspcode`〖=2
\inhibitxspcode`〗=1
\inhibitxspcode`«=2
\inhibitxspcode`»=1
\inhibitxspcode`〝=2
\inhibitxspcode`〟=1
\inhibitxspcode`‼=1
\inhibitxspcode`⁇=1
\inhibitxspcode`⁈=1
\inhibitxspcode`⁉=1
\inhibitxspcode`¡=2
\inhibitxspcode`¿=2
\inhibitxspcode"AA=1
\inhibitxspcode"BA=1
\inhibitxspcode`¹=1
\inhibitxspcode`²=1
\inhibitxspcode`³=1
\inhibitxspcode`€=2
%%
%% inhibitxspcode  JIS X 0212
%%
%%\inhibitxspcode`¡=2
%%\inhibitxspcode`¿=2
%%\inhibitxspcode"BA=1
%%\inhibitxspcode"AA=1
\inhibitxspcode`™=1
%%
%% inhibitxspcode  半角片仮名
%%
\inhibitxspcode`。=1
\inhibitxspcode`、=1
\inhibitxspcode`「=2
\inhibitxspcode`」=1
%    \end{macrocode}
%
%    \begin{macrocode}
%</plcore>
%    \end{macrocode}
%
% \Finale
%
\endinput

   \input{ujclasses.dtx}
\endgroup
\@ifl@t@r{\lastupd@te}{0000/00/00}{%
  \date{Version \patchdate\break (last updated: \lastupd@te)}%
}{}
\makeatother
%    \end{macrocode}
%\ifJAPANESE
% ここからが本文ページとなります。
%\else
% Here starts the document body.
%\fi
%    \begin{macrocode}
\begin{document}
\pagenumbering{roman}
\maketitle
\renewcommand\maketitle{}
\tableofcontents
\clearpage
\pagenumbering{arabic}

\DocInclude{uplvers}   % upLaTeX version

\DocInclude{uplfonts}  % NFSS2 commands

\DocInclude{ukinsoku}  % kinsoku parameter

\DocInclude{ujclasses} % Standard class

\StopEventually{\end{document}}

\clearpage
\pagestyle{headings}
% Make TeX shut up.
\hbadness=10000
\newcount\hbadness
\hfuzz=\maxdimen
%
\PrintChanges
\clearpage
%
\begingroup
  \def\endash{--}
  \catcode`\-\active
  \def-{\futurelet\temp\indexdash}
  \def\indexdash{\ifx\temp-\endash\fi}

  \PrintIndex
\endgroup
\let\PrintChanges\relax
\let\PrintIndex\relax
\end{document}
%</pldoc>
%    \end{macrocode}
%
%
%
%\ifJAPANESE
% \section{おまけプログラム}\label{app:omake}
%
% \subsection{シェルスクリプト\file{mkpldoc.sh}}\label{app:shprog}
% \upLaTeXe{}のマクロ定義ファイルをまとめて組版し、変更履歴と索引も
% 付けるときに便利なシェルスクリプトです。
% このシェルスクリプトの使用方法は次のとおりです。
%\begin{verbatim}
%    sh mkpldoc.sh
%\end{verbatim}
%
% コードは\pLaTeXe{}のものと(ファイル名を除き)ほぼ同一なので、
% ここでは違っている部分だけ説明します。
%\else
% \section{Additional Utility Programs}\label{app:omake}
%
% \subsection{Shell Script \file{mkpldoc.sh}}\label{app:shprog}
% A shell script to process `pldoc.tex' and produce a fully indexed
% source code description. Run |sh mkpldoc.sh| to use it.
%
% The script is almost identical to that in \pLaTeXe, so
% here we describe only the difference.
%\fi
%
%    \begin{macrocode}
%<*shprog>
%<ja>rm -f upldoc.toc upldoc.idx upldoc.glo
%<en>rm -f upldoc-en.toc upldoc-en.idx upldoc-en.glo
echo "" > ltxdoc.cfg
%<ja>uplatex upldoc.tex
%<en>uplatex -jobname=upldoc-en upldoc.tex
%    \end{macrocode}
%\ifJAPANESE
% 変更履歴や索引の生成にはmendexを用いますが、
% \upLaTeX{}の場合はUTF-8モードで実行する必要がありますので、
% |-U|というオプションを付けます\footnote{uplatexコマンドも
% 実際にはUTF-8モードで実行する必要がありますが、デフォルトの内部漢字
% コードがUTF-8に設定されているはずですので、\texttt{-kanji=utf8}を
% 付けなくても処理できると思います。}。
% makeindexコマンドには、このオプションがありません。
%\else
% To make the Change log and Glossary (Change History) for
% \upLaTeX\ using `mendex,' we need to run it in UTF-8 mode.
% So, option |-U| is important.\footnote{The command `uplatex'
% should be also in UTF-8 mode, but it defaults to UTF-8 mode;
% therefore, we don't need to add \texttt{-kanji=utf8} explicitly.}
%\fi
%    \begin{macrocode}
%<ja>mendex -U -s gind.ist -d upldoc.dic -o upldoc.ind upldoc.idx
%<en>mendex -U -s gind.ist -d upldoc.dic -o upldoc-en.ind upldoc-en.idx
%<ja>mendex -U -f -s gglo.ist -o upldoc.gls upldoc.glo
%<en>mendex -U -f -s gglo.ist -o upldoc-en.gls upldoc-en.glo
echo "\includeonly{}" > ltxdoc.cfg
%<ja>uplatex upldoc.tex
%<en>uplatex -jobname=upldoc-en upldoc.tex
echo "" > ltxdoc.cfg
%<ja>uplatex upldoc.tex
%<en>uplatex -jobname=upldoc-en upldoc.tex
# EOT
%</shprog>
%    \end{macrocode}
%
%
%\ifJAPANESE
% \subsection{perlスクリプト\file{dstcheck.pl}}\label{app:plprog}
% \pLaTeXe{}のものがそのまま使えるので、\upLaTeXe{}では省略します。
%\else
% \subsection{Perl Script \file{dstcheck.pl}}\label{app:plprog}
% The one from \pLaTeXe\ can be use without any change, so
% omitted here in \upLaTeXe.
%\fi
%
%
%\ifJAPANESE
% \subsection{\dst{}バッチファイル}
% 付録\ref{app:shprog}で説明をしたスクリプトを、このファイルから
% 取り出すための\dst{}バッチファイルです。コードは\pLaTeXe{}の
% ものと(ファイル名を除き)ほぼ同一なので、説明は割愛します。
%\else
% \subsection{\dst{} Batch file}
% Here we introduce a \dst\ batch file `Xins.ins,' which generates the
% script described in Appendix \ref{app:shprog}.
% The code is almost identical to that in \pLaTeXe.
%\fi
%
%    \begin{macrocode}
%<*Xins>
\input docstrip
\keepsilent
%    \end{macrocode}
%
%    \begin{macrocode}
{\catcode`#=12 \gdef\MetaPrefix{## }}
%    \end{macrocode}
%
%    \begin{macrocode}
\declarepreamble\thispre
\endpreamble
\usepreamble\thispre
%    \end{macrocode}
%
%    \begin{macrocode}
\declarepostamble\thispost
\endpostamble
\usepostamble\thispost
%    \end{macrocode}
%
%    \begin{macrocode}
\generate{
   \file{mkpldoc.sh}{\from{uplatex.dtx}{shprog,ja}}
   \file{mkpldoc-en.sh}{\from{uplatex.dtx}{shprog,en}}
}
\endbatchfile
%</Xins>
%    \end{macrocode}
%
% \newpage
% \begin{thebibliography}{9}
% \bibitem{tb108tanaka}
% Takuji Tanaka,
% \newblock Up\TeX\ --- Unicode version of \pTeX\ with CJK extensions.
% \newblock TUGboat issue 34:3, 2013.\\
%   (\texttt{http://tug.org/TUGboat/tb34-3/tb108tanaka.pdf})
% \end{thebibliography}
%
% \iffalse
% ここで、このあとに組版されるかもしれない文書のために、
% 節見出しの番号を算用数字に戻します。
% \fi
%
% \renewcommand{\thesection}{\arabic{section}}
%
% \Finale
%
\endinput
